\documentclass[12pt]{report}

\input{preamble}
\input{macros}
\input{letterfonts}
\usepackage{fancyhdr}
\pagestyle{fancy}

\lhead{\bf Rohan Jain}
\cfoot{}
\rhead{\bf
Abstract Algebra \\
Assignment 14}

\begin{document}

\qs{4}{Prove or disprove: Any subring of a field $F$ containing 1 is an integral domain.}
\sol Let $R \subseteq F$. Suppose $x,y \in R$ such that $xy = 0$. Since the 0 element is the same in $R$ and $F$, either $x=0$ or $y=0$ and as such, $R$ has no zero divisors and therefore, is an integral domain. $\qed$

\qs{6}{Let $F$ be a field of characteristic zero. Prove that $F$ contains a subfield isomorphic to $\QQ$.}
\sol 

\qs{10}{A field $F$ is called a \emph{\textbf{prime field}} if it has no proper subfields. If $E$ is a subfield of $F$ and $E$ is a prime subfield of $F$:
\begin{enumerate}[label=\alph*.]
    \item Prove that every field contains a unique prime subfield.
    \item If $F$ is a field of characteristic 0, prove that the prime subfield of $F$ is isomorphic to the field of rational numbers, $\QQ$.
    \item If $F$ is a field of characteristic $p$, prove that the prime subfield of $F$ is isomorphic to the field of integers modulo $p$, $\ZZ_p$.
\end{enumerate}}
\sol
\begin{enumerate}[label=\alph*.]
    \item To convince ourselves that $E$ is nonempty, we realize that $0,1 \in E$. For any $a,b \in E$, $a,b \in L$, so $ab$, $a+b$, $a-b$, and $a/b$ are all in $L$, and thus all in $E$. As such, $E$ is a subfield.
    \item Ok
    \item Ok
\end{enumerate}
\end{document}