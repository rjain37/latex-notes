\documentclass{report}

\input{../preamble}
\input{../macros}
\input{../letterfonts}

\usepackage{fancyhdr}
\pagestyle{fancy}

\lhead{\bf Rohan Jain}
\cfoot{}
\rhead{\bf
Putnam Seminar\\
Homework 1}

\begin{document}

\qs{6}{Prove that for every positive integer $n$, there exists an $n$-digit number divisible by $5^n$ all of whose digits are odd.}
\sol We prove this by induction. For the base case, $n=1$, we have $5$ as our number. For our inductive step, we will show that if there exists an $(n-1)$-digit number in the form $a \cdot 5^{n-1}$, then there exists an odd digit, $b$, such that $b \cdot 10^{n-1} + a \cdot 5^{n-1}$ is divisible by $5^n$. We can rewrite this as $5^{n-1}(a + b\cdot 2^{n-1})$. This just means that we have to prove that there exists an odd $b$ that makes $a + b\cdot 2^{n-1}$ divisible by $5$. By modular arithmetic, we just need $b \equiv -3^{n-1}a \pmod{5}$. Since there are $5$ odd digits, we can fill the residue class which means at least one digit will work. $\qed$

\qs{10}{A license plate has six digits from 0 to 9 and may have leading zeroes. If two plates must differ in at least two places, what is the largest number of plates possible.}
\sol We first start by noting that we can set an initial maximum on the number of license plates at $10^5$. If there are $\geq 10^5 + 1$, then there are $\geq 2$ license plates with the same first $5$ digits WLOG, therefore they only differ in one place.

I conjecture that the answer is indeed $10^5$ and will prove so by construction. WLOG, construct all $10^5$ combinations of the $5$ digits. With the $6$th digit, set it equal to the sum of the previous $5$ but$\mod{10}$. Therefore, we have $10^5$ license plates that are unique because if the first $5$ digits contain different permutations of the same numbers, then they will trivially differ in at least 2 of the first 5 spots. Additionally, if you have a number that agrees with another number on the first $5$ digits, then the $6$th digit will be the same and they are therefore the same number. Finally, if a number disagrees with another number at one of the first $5$ spots, then the $6$th digit will be different and they will therefore differ in at least $2$ spots. Therefore, we have constructed $10^5$ license plates that differ in at least $2$ spots. $\qed$


\end{document}