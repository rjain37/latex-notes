\documentclass{report}

\input{../preamble}
\input{../macros}
\input{../letterfonts}

\usepackage{fancyhdr}
\pagestyle{fancy}

\lhead{\bf Rohan Jain}
\cfoot{}
\rhead{\bf
Putnam Seminar\\
Homework 5}

\begin{document}
\qs{1}{Let $f: \RR \to \RR$ be a continuous function such that $f(x) = f(x^2)$. Prove that $f$ is constant.}
\sol We will start by contradiction. That is, assume $f$ is non constant and that $\exists a, b > 1: f(a) \neq f(b)$. Then take a very large amount of square roots of $a$, which tends to 1. Then by definition of the limit, the limit of $f(x)$ as $x$ tends to 1 is $f(a)$. But by definition of continuity, it's also $f(1)$. We can apply this to any positive $a \geq 1$ 

If $0 \leq a < 1$, then the case is the same except everything tends to 0.

Negatives follow trivially. $\qed$


\qs{11}{Prove that there is no function $f$ from the set of non negative integers to itself such that $f(f(n)) = n + 1987$ for every $n$.}
\sol First, we analyze the fact that $f$ is injective. This is because, if $f(m) = f(n)$, then $f(f(n)) = f(f(m)) = n + 1987 = m + 1987 \Longrightarrow n = m$. 

Consider the set of natural numbers that are not in the image of $f$. Namely, $A = \NN \setminus f(\NN)$. Now consider $f(A)$. It is not hard to see that $f(A) = f(\NN) \setminus f(f(\NN))$. This is because $f(\NN) \in f(A)$ and if any value $x$ is in $f(\NN)$, then it can't be in $f(f(\NN))$ because of injectivity.

Since $A$ subtracts $f(\NN)$, $A \cap f(A) = \emptyset$. On the other hand, $A \cup f(A) = \NN \setminus f(f(\NN))$, which is the set of integers from 0 to 1986 inclusive. But $f$ is injective so it must have the same number of elements, which doesn't work because the previous union has an odd number of elements. $\qed$


\end{document}