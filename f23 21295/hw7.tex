\documentclass{report}

\input{../preamble}
\input{../macros}
\input{../letterfonts}

\usepackage{fancyhdr}
\pagestyle{fancy}

\lhead{\bf Rohan Jain}
\cfoot{}
\rhead{\bf
Putnam Seminar\\
Homework 7}

\begin{document}
\qs{1}{ }
\sol 


\qs{4}{ Show that there is a rearrangement of the fractions $-\frac{1}{1}, \frac{1}{2}, -\frac{1}{3}, \frac{1}{4}, \dots$ such that the sum of the rearranged series is $\pi$.}
\sol By the Riemann series theorem, this is possible. I will now devise an algorithm to make this apparent.

\qs{8}{Evaluate $$\lim_{x \to \infty} (2x)^{1 + 1/2x} - (x)^{1+1/x} - x$$}
\sol We start by factoring out an $x$ and realizing that $x^{1/x} = e^{\ln(x)/x}$. Doing this, we get:
\begin{align*}
    &\lim_{x \to \infty} x (2(2x)^{1/2x} - (x)^{1/x} - 1) \\
    &\lim_{x \to \infty} x(2e^{\ln(2x)/2x} - e^{\ln(x)/x} - 1) \\
    &\lim_{x \to \infty} x \left(2 + \frac{2\ln(2x)}{2x} + O\left(\frac{\ln(x)^2}{x^2}\right) - \left(1 + \frac{\ln(x)}{x} + O\left(\frac{\ln(x)^2}{x^2}\right)\right) - 1\right) \\
    &\lim_{x \to \infty} x \left((2 - 1 - 1) + \left(\frac{\ln(2x)}{x} - \frac{\ln(x)}{x}\right) + \left(O\left(\frac{\ln(x)^2}{x^2}\right) - O\left(\frac{\ln(x)^2}{x^2}\right)\right)\right) \\
    &\lim_{x \to \infty} x\left(\frac{\ln(2)}{x}\right) = \ln(2)
\end{align*}



\end{document}