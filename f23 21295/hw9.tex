\documentclass{report}

\input{../preamble}
\input{../macros}
\input{../letterfonts}

\usepackage{fancyhdr}
\pagestyle{fancy}

\lhead{\bf Rohan Jain}
\cfoot{}
\rhead{\bf
21295\
Homework 9}

\begin{document}

\qs{4}{ i dont know how to type set this. }
\sol First realization is that the determinant is a function of the different independent $x_i$. And we also see that since if $x_i = x_j$, then the determinant is zero, we can write the determinant as a product of linear factors of the form $(x_i - x_j)$ for $1 \leq i < j \leq n$. But since all the $k_i$ are positive, this determinant is divisible by $x_1x_2\ldots x_n$. So, it suffices to show that $n!$ divides the following value:
\begin{align*}
    x_1x_2\ldots x_n \prod_{1 \leq i < j \leq n} (x_i - x_j)
\end{align*}
A proof with induction trivially shows this.

\qs{7}{Let $A$ be a square matrix. Prove that there exists $B$ such that $ABA=A$. }
\sol Rephrasing the problem for invertible matrices, we have to prove that there exists $B$ such that either $AB = I$ or $BA = I$. We constructively find the solutions for $B$ here, trivially showing that $B$ can either be the right or left inverse of $A$.

For the noninvertible case, we have the trivial result of the Moore-Penrose inverse.



\end{document}