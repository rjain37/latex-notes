\documentclass{report}

\input{../preamble}
\input{../macros}
\input{../letterfonts}

\usepackage{fancyhdr}
\pagestyle{fancy}

\lhead{\bf Rohan Jain}
\cfoot{}
\rhead{\bf
21-295\
Homework 13}

\begin{document}

\qs{1}{The radius of the base of a right circular cone is 1. The vertex of the cone is $V$, and $P$ is a point on
the circumference of the base. The length of $PV$ is 6 and the midpoint of $P V$ is $M$. A piece of string
is attached to $M$ and wound tightly twice round the cone finishing at $P$. What is the length of the string?}
\sol 

\includegraphics[width=0.5\textwidth]{hw13_1.jpg}


\qs{2}{Take a $5 \times 5$ square, and put $3-4-5$ right triangles on its top and bottom sides, oriented such that they
stick out of the square, and are 180-degree rotations of each other. Determine the distance between the right-angled corners of the $3-4-5$ triangles.}
\sol Start with the following diagram:

\includegraphics[width=0.5\textwidth]{hw13_2.jpg}

This let's us coord-bash the rest of the problem. The top corner can be expressed as the vector $v$ that I wrote out, while the bottom right corner can be expressed as $\displaystyle \begin{bmatrix} 5 \\ -5 \end{bmatrix} - v$. Now we can just find the vector that describes one corner to the other, which is $v - \left(\begin{bmatrix} 5 \\ -5 \end{bmatrix} - v\right) = 2v - \begin{bmatrix} 5 \\ -5 \end{bmatrix}$. Calculating the magnitude of this vector, we get $7\sqrt{2}$ as the answer. 

\end{document}