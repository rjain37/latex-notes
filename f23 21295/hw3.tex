\documentclass{report}

\input{../preamble}
\input{../macros}
\input{../letterfonts}

\usepackage{fancyhdr}
\pagestyle{fancy}

\lhead{\bf Rohan Jain}
\cfoot{}
\rhead{\bf
Putnam Seminar\\
Homework 3}

\begin{document}

\qs{1}{}
\sol 

\qs{3}{Let $f(1) = 3$ and $f(n+1) = 3^{f(n)}$. Find the last two digits of $f(2012)$.}
\sol We claim that $3^{20} = 1 \pmod{100}$, which is trivially shown by Fermat's or Euler's. As such, we take the exponent $\pmod{20}$ to find the last two digits in this cyclic 



\end{document}