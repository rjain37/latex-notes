\documentclass{report}

\input{../preamble}
\input{../macros}
\input{../letterfonts}

\usepackage{fancyhdr}
\pagestyle{fancy}

\lhead{\bf Rohan Jain}
\cfoot{}
\rhead{\bf
21-295\
Homework 11}

\begin{document}

\qs{1}{What is the largest positive integer that is a factor of $P(1) - 2P(7) + P(13)$, for every polynomial $P$
with integer coefficients?}
\sol It is a result that for any polynomial $P$ with integer coefficients, $a - b$ divides $P(a) - P(b)$. So, we rewrite this as $P(13) - P(7) + P(1) - P(7)$. Now, we know that $13 - 7 = 6$ and $1 - 7 = -6$. So, we can factor out a $6$ from the expression, meaning that this is the largest integer we can guarantee to be a factor of the expression.

\qs{3}{Prove that for every prime number $p$, the polynomial $$P(x) = \sum_{i = 0}^{p-1}x^{i}$$ cannot be expressed as the product of two non-constant polynomials with integer coefficients.}
\sol This is equivalent to showing that it is irreducible. We will use the idea that $f(x)$ is irreducible iff $f(x+1)$ is irreducible. So, 
$$f(x+1) = \frac{(x+1)^p - 1}{(x+1) - 1} = x^{p-1} + \binom{p}{1}x^{p-2} + \cdots + p$$ 
This clearly fails Eisenstein's for $p$. So, $f(x+1)$ is irreducible, and thus $f(x)$ is irreducible.

\end{document}