\documentclass{report}

\input{../preamble}
\input{../macros}
\input{../letterfonts}

\title{\Huge{21-235 Math Studies Analysis I}}
\author{\huge{Rohan Jain}}
\date{}

\begin{document}

\maketitle
\newpage% or \cleardoublepage
% \pdfbookmark[<level>]{<title>}{<dest>}
\pdfbookmark[section]{\contentsname}{toc}
\tableofcontents

\pagebreak

\chapter{}
\section{Ordered Fields (Review)}

\dfn{Order}{Let $E$ be a set. An \emph{order} on $E$ is a relation $<$ on $E$ such that for all $x, y, z \in E$:
\begin{enumerate}
    \item (Trichotomy) Exactly one of the following holds: $x < y$, $x = y$, or $x > y$.
    \item (Transitivity) If $x < y$ and $y < z$, then $x < z$.
\end{enumerate}}

\ex{Examples of Ordered Sets}{\begin{enumerate}
    \item This definition develops orders on basic number systems: e.g. $\ZZ$, $\QQ$, and $\RR$.
    \item Define $\lesssim$ on $\ZZ$ as follows: We say that $m \lesssim n$ for $m, n \in \ZZ$ if:
\begin{enumerate}
    \item $m$ is even and $n$ is odd
    \item $m,n$ are even and $m < n$
    \item $m,n$ are odd and $m < n$. 
\end{enumerate}
\end{enumerate}}
\noindent Key Concepts: 
\begin{itemize}
    \renewcommand\labelitemi{--}
    \item upper/lower bounds of sets
    \item bounded sets
    \item max/min
    \item supremum/infimum
    \item supremum/infimum property: An ordered set $E$ satisfies such a property if every nonempty set $A \subseteq E$ that's bounded above/below has a supremum/infimum in $E$.
    \item Fact: sup prop $\implies$ inf prop
\end{itemize}

\dfn{Ordered Field}{Let $\FF$ be a field with order $<$. We say that $\FF$ is an \emph{ordered field} provided that:
\begin{enumerate}
    \item For all $x, y, z \in \FF$, if $x < y$, then $x + z < y + z$.
    \item For all $x, y \in \FF$, if $0 < x$ and $0 < y$, then $0 < x \cdot y$.
\end{enumerate}}

\ex{}{$\QQ$ is a field.}
\noindent Facts of any ordered field:
\begin{enumerate}
    \item $0 < 1$
    \item $\nexists x \in \FF$ such that $x^2 = -1$.
\end{enumerate}

\dfn{Ordered Subfield, Homomorphism, Isomorphism}{Let $\FF$ be an ordered field. 
\begin{enumerate}
    \item A set $\mathbb K \subseteq \FF$ is called an \emph{ordered subfield} if $mathbb K$ is an algeraic subfield and $\mathbb K$ is an ordered field equipped with $<$ from $\FF$.
    \item Let $\mathbb G$ be an ordered field and let $f : \mathbb F \to \mathbb G$. We say that $f$ is an \emph{ordered field homomorphism} if it's a field homomorphism and $f(x) < f(y)$ whenever $x < y$.
    \item $f$ is an ordered field isomorphism if $f$ is an ordered field homomorphism and $f$ is bijective.
\end{enumerate}}
\nt{
\begin{enumerate}
    \item If $f : \mathbb F \to \mathbb G$ is an ordered field homomorphism, $f(\mathbb F)$ is an ordered subfield of $\mathbb G$.
    \item OF property $\implies$ $f$ is injective.
    \item $\therefore$ every ordered field homomorphism $f : \mathbb F \to \mathbb G$ is such that $f$ induces a bijection $f: \mathbb F \to f(\mathbb F) \subseteq \mathbb G$.
\end{enumerate}
}

\thm{$\mathbb Q$ is the smallest ordered field. More precisely, if $\FF$ is an ordered field, then there exists a canonical ordered field homomorphism $f : \QQ \to \FF$.}

\noindent Upshot/notation abuse: We identify $f(\QQ) = \QQ$ to view $\QQ \subseteq \FF$. 
In turn, $\NN \subset \ZZ \subset \QQ \subseteq \FF$. 


\section{Types of Ordered Fields}
\dfn{Archimedean, Dedekind complete}{Let $\FF$ be an ordered field.
\begin{enumerate}
    \item We say that $\FF$ is Archimedean if $\forall 0 < x \in \FF$, $\exists n \in \NN$ such that $n > x$.
    \item We say that $\FF$ is Dedekind complete if it satisfies the supremum property.
\end{enumerate}}

\noindent Facts:
\begin{enumerate}
    \item $\QQ$ is Archimedean.
    \item If $\FF$ is Dedekind complete, then $\forall 0 < x \in \FF$ and $\forall 0< n \in \NN$, $\exists !$  $0 < y \in \FF$ such that $y^n = x$.
    \item $\QQ$ is not Dedekind complete. ($\sqrt 2$ is a counterexample.)
\end{enumerate}

\thm{}{Suppose $\FF$ is a Dedekind complete ordered field. Then $\FF$ is Archimedean.}
\begin{proof}
    If not, then $\NN \subset \FF$ is bounded above, and so the supremum property provides $x \in \FF$ such that $x = \sup \NN$. But then $x - 1$ is an upper bound for $\NN$, so there exists $n \in \NN$ such that $x-1 < n$. Hence $x < n + 1$, which contradicts the definition of $x$ as an upper bound. Therefore, $\FF$ is Archimedean.
\end{proof}

\section{Dedekind Completion}
Throughout this section, let $\FF$ be an Archimedean ordered field.

\dfn{Dedekind cut}{We say a set $\mathcal C \subseteq \FF$ is \emph{Dedekind cut} if:
\begin{enumerate}
    \item $\mathcal C \neq \emptyset$ and $\mathcal C \neq \FF$.
    \item If $p \in \mathcal C$ and $q \in \FF$ such that $q < p$, then $q \in \mathcal C$.
    \item If $p \in \mathcal C$, then $\exists r \in \mathcal C$ such that $p < r$.
\end{enumerate}
We will write $\FF^*$ for the set of all Dedekind cuts in $\FF$. It is called the \emph{Dedekind completion} of $\FF$.}

\nt{Let $\mathcal C \subseteq \FF$ be a cut. Then:
\begin{enumerate}
    \item If $p \in \mathcal C$, then $q \notin \mathcal C$, then $p < q$.
    \item If $r \notin \mathcal C$, and $r < s \in \FF$, then $s \notin \mathcal C$.
\end{enumerate}}

\ex{Cut examples}{\begin{enumerate}
    \item Let $q \in \FF$ and define $\mathcal C_q = \{ p \in \FF \mid p < q\}$. Then $\mathcal C_q$ is a cut.
    \begin{proof}
        \begin{enumerate}
            \item $q - 1 < q \implies q - 1 \in \mathcal C_q$. $q \not < q \implies q \notin \mathcal C_q \implies \mathcal C_q \neq \FF$.
            \item Let $p \in \mathcal C_q$. Suppose $s \in \FF$ such that $s < p$. Then $s < q \implies s \in \mathcal C_q$.
            \item Let $p \in \mathcal C_q$. Then $p < \frac{p + q}{2} < q \implies \frac{p + q}{2} \in \mathcal C_q$.
        \end{enumerate}
    \end{proof}
    \item Suppose $\FF$ is such that $\nexists x \in \FF$ such that $x^2 = 2$. Let $\mathcal C = \{ p \in \FF \mid p \leq 0 \text{ or } 0 < p^2 < 2\}$. Then $\mathcal C$ is a cut.
    \begin{proof}
        \begin{enumerate}
            \item $1 \in \mathcal C$ and $1^2 = 1 < 2$. $2 \notin \mathcal C$ and $2^2 = 4 > 2$. 
            \item Let $p \in \mathcal C$ and $q \in \FF$ such that $q < p$. If $q \leq 0$, then $q \in \mathcal C$ trivially. Suppose $0 < q < p$. Then $0 < q^2 < p^2 < 2$, so $q \in \mathcal C$.
            \item Let $p \in \mathcal C$. If $p \leq 0$, then $1 \in \mathcal C$ and $p < 1$, so we're done. Suppose $0 < p^2 < 2$. Note, $0 < 2 - p^2$, so $\frac{2p + 1}{2 - p^2} > 0$. Then we can define $r = 1 + \frac{2p + 1}{2 - p^2} \geq \max(1, \frac{2p + 1}{2 - p^2})$. Then $(p + 1/r)^2 = p^2 + \frac{2p}{r} + \frac{1}{r^2}$. We have:
            \begin{align*}
                p^2 + \frac{2p}{r} + \frac{1}{r^2} &< p^2 + \frac{2p}{r} + \frac 1r \\
                &= p^2 + \frac{2p+1}{r} \\
                &\leq p^2 + 2-p^2 \\
                &= 2.
            \end{align*}
            So, $p < p + 1/r < 2$ and $p + 1/r \in \mathcal C$.
        \end{enumerate}
    \end{proof}
\end{enumerate}}

\subsection{Ordering $\FF^*$}

\mlenma{}{The following hold:
\begin{enumerate}
    \item If $\mathcal A, \mathcal B \in \FF^*$, then exactly one holds:
    \begin{itemize}
        \item $\mathcal A \subset \mathcal B$
        \item $\mathcal A = \mathcal B$
        \item $\mathcal B \subset \mathcal A$
    \end{itemize}
    \item If $\mathcal A, \mathcal B, \mathcal C \in \FF^*$ and $\mathcal A \subset \mathcal B$ and $\mathcal B \subset \mathcal C$, then $\mathcal A \subset \mathcal C$.
\end{enumerate}}
\begin{proof}
    Proof of $2$ is trivial, as well as the equality part for $1$. 
    \begin{itemize}
        \item If $\mathcal A = \mathcal B$, we're done.
        \item Suppose $\exists b \in \mathcal B \setminus\mathcal A$. If $a \in \mathcal A$, then $a < b$, but $\mathcal B$ is a cut so $a \in \mathcal B$, so $\mathcal A \subset \mathcal B$.
        \item Suppose $\exists a \in \mathcal A \setminus \mathcal B$. Then $a < b$ for all $b \in \mathcal B$, so $a \in \mathcal B$, so $\mathcal B \subset \mathcal A$.
    \end{itemize}
\end{proof}


\dfn{Order on cuts}{Given $\mathcal A, \mathcal B \in \FF^*$, we say that $\mathcal A < \mathcal B$ if $\mathcal A \subset \mathcal B$. The lemma above shows that this is infact an order.}


\mlenma{}{Let $E \subseteq \FF^*$ be nonempty and bounded above. Then $\mathcal B = \bigcup_{\mathcal A \in E} \mathcal A$ is a cut.}
\begin{proof}
    \begin{enumerate}
        \item Since $E \neq \emptyset$, there exists $\mathcal A \in E$. So $\mathcal A \neq \emptyset$, hence $\mathcal B \neq \emptyset$. 
        
        Since $E$ is bounded above, there exists $\mathcal C \in \FF^*$ such that $\mathcal A \subset \mathcal C$ for all $\mathcal A \in E$. Since $\mathcal C$ is a cut, there is $q \in \FF$ such that $q \notin \mathcal C$. Then $q \notin \mathcal A$ for all $\mathcal A \in E$, so $q \notin \mathcal B$.
        \item Let $p \in \mathcal B$ and $q \in \FF$ such that $q < p$. Since $\mathcal B$ is a union of cuts, it follows that $p \in \mathcal A$ for some $\mathcal A \in E$. Since $\mathcal A$ is a cut, $q \in \mathcal A \subseteq \mathcal B$.
        \item Let $p \in \mathcal B$. Then $p \in \mathcal A$ for some $\mathcal A \in E$. Since $\mathcal A$ is a cut, there exists $r \in \mathcal A$ such that $p < r$. Since $\mathcal A \subset \mathcal B$, we have $r \in \mathcal B$.
    \end{enumerate}
\end{proof}

\thm{}{$\FF^*$ equipped with the order $<$ satisfies the supremum property.}
\begin{proof}
    Let $E \subseteq \FF$ be a nonempty set that is bounded above.  From last time, we know that $\mathcal B = \cup_{\mathcal A \in E} \mathcal A$ is a cut. We claim that $\mathcal B = \sup E$. 

    If $\mathcal A \in E$, then $\mathcal A \subseteq \mathcal B$. And so $\mathcal A \leq \mathcal B$, so $\mathcal B$ is an upper bound for $E$.

    Next, suppose that $\mathcal C \in \FF^*$ is an upper bound of $E$. This means that $\mathcal A \leq \mathcal C$ for every $\mathcal A \in E$, meaning $\mathcal A \subseteq \mathcal C \forall \mathcal A \in E$. So $\mathcal B \subseteq \mathcal C$. As such, $\mathcal B \leq \mathcal C$, so $\mathcal B = \sup E$.
\end{proof}\

\noindent Remark: In none of the results leading up to this theorem did we use that $\FF$ is anything other than an ordered set. This shows that the cut construction of Dedekind works in general for ordered sets and yields $\FF^*$ that satisfies the supremum property. Also, $\{ \mathcal C_p \mid p \in \FF\} \subseteq \FF^*$. 

\subsection{Addition}
Idea: $\FF \cong \{ \mathcal C_p \mid p \in \FF\}$. 

\mlenma{}{Let $\mathcal A, \mathcal B \in \FF^*$. Then $\mathcal C = \{a  + b \mid a \in \mathcal A, b \in \mathcal B\}$ is a cut.}
\begin{proof}
    Claim: $\mathcal A, \mathcal B \neq \emptyset \implies \mathcal C \neq \emptyset$.

    $\mathcal A, \mathcal B$ are cuts, so $\exists M_1, M_2 \in \FF$ such that $a < M_1$ for all $a \in \mathcal A$ and $b < M_2$ for all $b \in \mathcal B$. Then $a + b < M_1 + M_2$ for all $a \in \mathcal A, b \in \mathcal B$, so $a + b < M_1 + M_2$, meaning $M_1 + M_2 \notin \mathcal C$.

    Also, let $c = a + b \in \mathcal C$ for $a \in \mathcal A, b \in \mathcal B$. Let $q < c \implies q-a < b \implies q - a \in \mathcal B$. Hence, $q = a + (q-a) \in \mathcal C$. 

    Thirdly, let $c = a + b \in\mathcal C$ for $a \in \mathcal A, b \in \mathcal B$. Since $\mathcal A, \mathcal B \in \FF^*$, $\exists r_a, r_b$ such that $a < r_a \in \mathcal A$, $b < r_b \in \mathcal B$. Then $c = a + b < r_a + r_b$, so $r_a + r_b \in \mathcal C$.

    As such, $\mathcal C$ is a cut.
\end{proof}
\noindent Before we define addition, we need to define the negative of a cut.


Heuristic: What we want is that $-\mathcal C_1 = \mathcal C_{-1}$. The way we do this is by defining $\mathcal C_{-p} = \{ q \in \FF \mid \exists p > q : p \in -\mathcal C_p^C\}$. This is the same as $\{ q\in \FF \mid \exists p > q : -p \notin \mathcal C_p\}$. 

Now we study $\{ q \in \FF \mid \exists p > q : -p \notin \mathcal C\}$.

\mlenma{}{Let $\mathcal C \in \FF^*$. Then $\{ q \in \FF \mid \exists p > q : -p \notin \mathcal C\}$ is a cut.}

\dfn{Addition}{For $\mathcal A, \mathcal B \in \FF^*$, we define $\mathcal A + \mathcal B = \{ a + b \mid a \in \mathcal A, b \in \mathcal B\}$ and $-\mathcal A = \{ q \in \FF \mid \exists p > q : -p \notin \mathcal A\}$.}

\thm{}{Define $0 = \mathcal C_0 \in \FF^*$. The following hold:
\begin{enumerate}
    \item $\mathcal A, \mathcal B \in \FF^* \implies \mathcal A + \mathcal B \in \FF^*$.
    \item $\mathcal A, \mathcal B \in \FF^* \implies \mathcal A + \mathcal B = \mathcal B + \mathcal A$.
    \item $\mathcal A, \mathcal B, \mathcal C \in \FF^* \implies (\mathcal A + \mathcal B) + \mathcal C = \mathcal A + (\mathcal B + \mathcal C)$.
    \item $\mathcal A \in \FF^* \implies \mathcal A + 0 = \mathcal A$.
    \item $\mathcal A \in \FF^* \implies \mathcal A + (-\mathcal A) = 0$.
\end{enumerate}
}
\begin{proof}
    Easy proof, too lazy to write out.
\end{proof}
Also: $\mathcal A, \mathcal B, \mathcal C \in \FF^*$ and $\mathcal A < \mathcal B \implies \mathcal A + \mathcal C < \mathcal B + \mathcal C$.

Important Remark: The Archimedean property is actually needed for the above theorem in orer to prove the 5th condition. 

\newpage
\subsection{Multiplication}
\mlenma{}{Let $\mathcal A, \mathcal B \in \FF^*$ such that $\mathcal A, \mathcal B > 0$. Then $\mathcal C = \{ p \in \FF \mid p \leq 0\} \cup \{ ab \mid a \in \mathcal A, b \in \mathcal B, a, b > 0\}$ is a cut.}

\mlenma{}{Let $\mathcal A \in \FF^*$ be such that $\mathcal A > 0$. Then $\mathcal C = \{ p \in \FF^* \mid p \leq 0\}\cup \{ 0 < q \in \FF \mid \exists p > q : p^{-1} \notin \mathcal A\}$ is a cut.}

\dfn{Multiplication}{Let $\mathcal A, \mathcal B \in \FF^*$. We define multiplication as:
\begin{enumerate}
    \item If $\mathcal A , \mathcal B > 0$, then $\mathcal A \cdot \mathcal B = \{ ab \mid 0 < a \in \mathcal A, 0 < b \in \mathcal B\}$.
    \item If $\mathcal A = 0$  or $\mathcal B = 0$, then $\mathcal A \cdot \mathcal B = 0$.
    \item If $\mathcal A > 0$ and $\mathcal B < 0$, then $\mathcal A \cdot \mathcal B = -(\mathcal A \cdot (-\mathcal B))$.
    \item If $\mathcal A < 0$ and $\mathcal B > 0$, then $\mathcal A \cdot \mathcal B = -((-\mathcal A) \cdot \mathcal B)$.
    \item If $\mathcal A, \mathcal B < 0$, then $\mathcal A \cdot \mathcal B = (-\mathcal A) \cdot (-\mathcal B)$.
\end{enumerate}
We define multiplication inversion via:
\begin{enumerate}
    \item If $\mathcal A > 0$, then $\mathcal A^{-1} = \{ q \in \FF \mid \exists p > q : p^{-1} \notin \mathcal A\}$.
    \item If $\mathcal A < 0$, then $\mathcal A^{-1} = -(-\mathcal A)^{-1}$.
\end{enumerate}}
\thm{}{Set $1 = \mathcal C_1$. The following hold:
\begin{enumerate}
    \item If $\mathcal A, \mathcal B \in \FF^*$, then $\mathcal A \cdot \mathcal B \in \FF^*$.
    \item If $\mathcal A, \mathcal B \in \FF^*$, then $\mathcal A \cdot \mathcal B = \mathcal B \cdot \mathcal A$.
    \item If $\mathcal A, \mathcal B, \mathcal C \in \FF^*$, then $(\mathcal A \cdot \mathcal B) \cdot \mathcal C = \mathcal A \cdot (\mathcal B \cdot \mathcal C)$.
    \item If $\mathcal A \in \FF^*$, then $\mathcal A \cdot 1 = \mathcal A$.
    \item If $\mathcal A \in \FF^*$, then $\mathcal A \cdot \mathcal A^{-1} = 1$.
\end{enumerate}}

Also if $\mathcal A, \mathcal B \in \FF^*$ and $\mathcal A, \mathcal B > 0$, then $\mathcal A \cdot \mathcal B > 0$. 

\thm{}{If $\mathcal A, \mathcal B, \mathcal C \in \FF^*$, then $\mathcal A \cdot(\mathcal B + \mathcal C) = \mathcal A \cdot \mathcal B + \mathcal A \cdot \mathcal C$.}

We now know that $\FF^*$ is an ordered field.


\subsection{Robert Reci}

\thm{}{$\QQ$ is the smallest ordered field.}
\begin{proof}
    Let $\FF$ be any ordered field. Let $1 \in \FF$. Let $\iota : \NN \to \FF$, $n \mapsto 1 + \cdots + 1$ $n$ times. Then $\iota(-n) = -\iota(n)$ for $n \in \NN_0$ and $-n \in \ZZ^-$. 

    Then we say $\iota(p/q) = \iota(p)\iota(q)^{-1}$ for $p/q \in \QQ$.
\end{proof}

\cor{Every ordered field is infinite}{$\iota[\QQ] \subseteq \FF$ is infinite.}

\subsubsection{Roots}
Let $\FF$ be a Dedekind complete ordered field, $0 < x \in \FF$, $n \in \NN$. Then $\exists ! y \in \FF$ such that $y > 0$ and $y^n = x$.

\begin{proof}
    $n = 1$ is silly. Assume $n \geq 2$. Let $E = \{ z \in \FF \mid z > 0 \text{ and } z^n < x\}$. Then $E$ is nonempty and bounded above by $x$. Let $y = \sup E$. We claim that $y^n = x$.

    We want to show that $y^n \ngtr > x$ and $y^n \nless < x$. 

    \mlenma{}{In any commutative ring R, $b^n - a^n = (b - a)(b^{n-1} + b^{n-2}a + \cdots + ba^{n-2} + a^{n-1})$.}

    And hence for $0 < a < b$ in $\FF$, we have $0 < b^n - a^n = (b-a)nb^{n-1}$. 

    Suppose $y^n < x$, so $x-y^n > 0$. We define $h = \frac 12 \min\left(1, \frac{x-y^n}{n(y+1)^{n-1}}\right)$. $0 < h < 1$, also $0 < h < \frac{x-y^n}{n(y+1)^{n-1}}$. 

    Then, by the inequality below the lemma, we have \begin{align*}
        0 &< (y+h)^n - y^n \\
        &< hn(y + h)^{n-1} \\
        &< hn(y + 1)^{n-1} \\
        &< {x-y^n},
    \end{align*}
    so $(y+h)^n < x$, which contradicts the definition of $y$ as the supremum.
\end{proof}

\dfn{Ring*}{A ring is a field where actually we don't care about inverses anymore. }

\dfn{Domain}{$R$ is a domain when $xy = 0 \implies x = 0 \wedge y = 0$.}

Let $R$ be a ring. For $(r, s) \in R \times R \setminus \{0\}$, we say $(r, s) \sim (r', s')$ if $rs' = r's$.

The field of fractions, Frac$(R)$ is the set of equivalence classes of $R \times R \setminus \{0\}$ under $\sim$ equipped with the operations $[(r, s)] + [(r', s')] = [(rs' + r's, ss')]$ and $[(r, s)] \cdot [(r', s')] = (rr', ss')$.

We check that $[(r , s)] \cdot [(s, r)] = [(rs, sr)] = [(1, 1)]$.

Let $\FF$ a field, $\FF[x]$ its polynomial ring. Let $\FF(x)$ be the field of fractions of $\FF[x]$. Then $\FF(x) := \text{Frac}(\FF[x])$ is the field of rational functions in $x$ with coefficients in $\FF$.

Given $p, q \in \FF[x]$, say $p/q > 0$ if $p$ and $q$ have the same sign. Say $f, g \in \FF(x)$, that $f > g$ when $f- g > 0$. 

\thm{}{$\FF(x)$ is never Archimedean.}
\begin{proof}
    $x$ is an upper bound for all $n \in \NN$. 
\end{proof}

\nt{If $\FF$ is Archimedean, $|\FF| \leq 2^{\aleph_0}$.}

\thm{}{Let $\lambda$ be an infinite cardinal. Then there is an ordered field of cardinality $\lambda$.}

\cor{}{The Archimedean property is not a first-order property.}
\end{document}