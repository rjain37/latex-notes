\documentclass{report}

%%%%%%%%%%%%%%%%%%%%%%%%%%%%%%%%%
% PACKAGE IMPORTS
%%%%%%%%%%%%%%%%%%%%%%%%%%%%%%%%%


\usepackage[tmargin=2cm,rmargin=1in,lmargin=1in,margin=0.85in,bmargin=2cm,footskip=.2in]{geometry}
\usepackage{amsmath,amsfonts,amsthm,amssymb,mathtools}
\usepackage[varbb]{newpxmath}
\usepackage{xfrac}
\usepackage[makeroom]{cancel}
\usepackage{mathtools}
\usepackage{bookmark}
\usepackage{enumitem}
\usepackage{hyperref,theoremref}
\hypersetup{
	pdftitle={Assignment},
	colorlinks=true, linkcolor=doc!90,
	bookmarksnumbered=true,
	bookmarksopen=true
}
\usepackage[most,many,breakable]{tcolorbox}
\usepackage{xcolor}
\usepackage{varwidth}
\usepackage{varwidth}
\usepackage{etoolbox}
%\usepackage{authblk}
\usepackage{nameref}
\usepackage{multicol,array}
\usepackage{tikz-cd}
\usepackage[ruled,vlined,linesnumbered]{algorithm2e}
\usepackage{comment} % enables the use of multi-line comments (\ifx \fi) 
\usepackage{import}
\usepackage{xifthen}
\usepackage{pdfpages}
\usepackage{transparent}

% \usepackage{extsizes}

\newcommand\mycommfont[1]{\footnotesize\ttfamily\textcolor{blue}{#1}}
\SetCommentSty{mycommfont}
\newcommand{\incfig}[1]{%
    \def\svgwidth{\columnwidth}
    \import{./figures/}{#1.pdf_tex}
}

\usepackage{tikzsymbols}
\renewcommand\qedsymbol{$\Laughey$}


%\usepackage{import}
%\usepackage{xifthen}
%\usepackage{pdfpages}
%\usepackage{transparent}


%%%%%%%%%%%%%%%%%%%%%%%%%%%%%%
% SELF MADE COLORS
%%%%%%%%%%%%%%%%%%%%%%%%%%%%%%



\definecolor{myg}{RGB}{56, 140, 70}
\definecolor{myb}{RGB}{45, 111, 177}
\definecolor{myr}{RGB}{199, 68, 64}
\definecolor{mytheorembg}{HTML}{F2F2F9}
\definecolor{mytheoremfr}{HTML}{00007B}
\definecolor{mylenmabg}{HTML}{FFFAF8}
\definecolor{mylenmafr}{HTML}{983b0f}
\definecolor{mypropbg}{HTML}{f2fbfc}
\definecolor{mypropfr}{HTML}{191971}
\definecolor{myexamplebg}{HTML}{F2FBF8}
\definecolor{myexamplefr}{HTML}{88D6D1}
\definecolor{myexampleti}{HTML}{2A7F7F}
\definecolor{mydefinitbg}{HTML}{E5E5FF}
\definecolor{mydefinitfr}{HTML}{3F3FA3}
\definecolor{notesgreen}{RGB}{0,162,0}
\definecolor{myp}{RGB}{197, 92, 212}
\definecolor{mygr}{HTML}{2C3338}
\definecolor{myred}{RGB}{127,0,0}
\definecolor{myyellow}{RGB}{169,121,69}
\definecolor{myexercisebg}{HTML}{F2FBF8}
\definecolor{myexercisefg}{HTML}{88D6D1}


%%%%%%%%%%%%%%%%%%%%%%%%%%%%
% TCOLORBOX SETUPS
%%%%%%%%%%%%%%%%%%%%%%%%%%%%

\setlength{\parindent}{1cm}
%================================
% THEOREM BOX
%================================

\tcbuselibrary{theorems,skins,hooks}
\newtcbtheorem[number within=section]{Theorem}{Theorem}
{%
	enhanced,
	breakable,
	colback = mytheorembg,
	frame hidden,
	boxrule = 0sp,
	borderline west = {2pt}{0pt}{mytheoremfr},
	sharp corners,
	detach title,
	before upper = \tcbtitle\par\smallskip,
	coltitle = mytheoremfr,
	fonttitle = \bfseries\sffamily,
	description font = \mdseries,
	separator sign none,
	segmentation style={solid, mytheoremfr},
}
{th}

\tcbuselibrary{theorems,skins,hooks}
\newtcbtheorem[number within=chapter]{theorem}{Theorem}
{%
	enhanced,
	breakable,
	colback = mytheorembg,
	frame hidden,
	boxrule = 0sp,
	borderline west = {2pt}{0pt}{mytheoremfr},
	sharp corners,
	detach title,
	before upper = \tcbtitle\par\smallskip,
	coltitle = mytheoremfr,
	fonttitle = \bfseries\sffamily,
	description font = \mdseries,
	separator sign none,
	segmentation style={solid, mytheoremfr},
}
{th}


\tcbuselibrary{theorems,skins,hooks}
\newtcolorbox{Theoremcon}
{%
	enhanced
	,breakable
	,colback = mytheorembg
	,frame hidden
	,boxrule = 0sp
	,borderline west = {2pt}{0pt}{mytheoremfr}
	,sharp corners
	,description font = \mdseries
	,separator sign none
}

%================================
% Corollery
%================================
\tcbuselibrary{theorems,skins,hooks}
\newtcbtheorem[number within=section]{Corollary}{Corollary}
{%
	enhanced
	,breakable
	,colback = myp!10
	,frame hidden
	,boxrule = 0sp
	,borderline west = {2pt}{0pt}{myp!85!black}
	,sharp corners
	,detach title
	,before upper = \tcbtitle\par\smallskip
	,coltitle = myp!85!black
	,fonttitle = \bfseries\sffamily
	,description font = \mdseries
	,separator sign none
	,segmentation style={solid, myp!85!black}
}
{th}
\tcbuselibrary{theorems,skins,hooks}
\newtcbtheorem[number within=chapter]{corollary}{Corollary}
{%
	enhanced
	,breakable
	,colback = myp!10
	,frame hidden
	,boxrule = 0sp
	,borderline west = {2pt}{0pt}{myp!85!black}
	,sharp corners
	,detach title
	,before upper = \tcbtitle\par\smallskip
	,coltitle = myp!85!black
	,fonttitle = \bfseries\sffamily
	,description font = \mdseries
	,separator sign none
	,segmentation style={solid, myp!85!black}
}
{th}


%================================
% LENMA
%================================

\tcbuselibrary{theorems,skins,hooks}
\newtcbtheorem[number within=section]{Lenma}{Lenma}
{%
	enhanced,
	breakable,
	colback = mylenmabg,
	frame hidden,
	boxrule = 0sp,
	borderline west = {2pt}{0pt}{mylenmafr},
	sharp corners,
	detach title,
	before upper = \tcbtitle\par\smallskip,
	coltitle = mylenmafr,
	fonttitle = \bfseries\sffamily,
	description font = \mdseries,
	separator sign none,
	segmentation style={solid, mylenmafr},
}
{th}

\tcbuselibrary{theorems,skins,hooks}
\newtcbtheorem[number within=chapter]{lenma}{Lenma}
{%
	enhanced,
	breakable,
	colback = mylenmabg,
	frame hidden,
	boxrule = 0sp,
	borderline west = {2pt}{0pt}{mylenmafr},
	sharp corners,
	detach title,
	before upper = \tcbtitle\par\smallskip,
	coltitle = mylenmafr,
	fonttitle = \bfseries\sffamily,
	description font = \mdseries,
	separator sign none,
	segmentation style={solid, mylenmafr},
}
{th}


%================================
% PROPOSITION
%================================

\tcbuselibrary{theorems,skins,hooks}
\newtcbtheorem[number within=section]{Prop}{Proposition}
{%
	enhanced,
	breakable,
	colback = mypropbg,
	frame hidden,
	boxrule = 0sp,
	borderline west = {2pt}{0pt}{mypropfr},
	sharp corners,
	detach title,
	before upper = \tcbtitle\par\smallskip,
	coltitle = mypropfr,
	fonttitle = \bfseries\sffamily,
	description font = \mdseries,
	separator sign none,
	segmentation style={solid, mypropfr},
}
{th}

\tcbuselibrary{theorems,skins,hooks}
\newtcbtheorem[number within=chapter]{prop}{Proposition}
{%
	enhanced,
	breakable,
	colback = mypropbg,
	frame hidden,
	boxrule = 0sp,
	borderline west = {2pt}{0pt}{mypropfr},
	sharp corners,
	detach title,
	before upper = \tcbtitle\par\smallskip,
	coltitle = mypropfr,
	fonttitle = \bfseries\sffamily,
	description font = \mdseries,
	separator sign none,
	segmentation style={solid, mypropfr},
}
{th}


%================================
% CLAIM
%================================

\tcbuselibrary{theorems,skins,hooks}
\newtcbtheorem[number within=section]{claim}{Claim}
{%
	enhanced
	,breakable
	,colback = myg!10
	,frame hidden
	,boxrule = 0sp
	,borderline west = {2pt}{0pt}{myg}
	,sharp corners
	,detach title
	,before upper = \tcbtitle\par\smallskip
	,coltitle = myg!85!black
	,fonttitle = \bfseries\sffamily
	,description font = \mdseries
	,separator sign none
	,segmentation style={solid, myg!85!black}
}
{th}



%================================
% Exercise
%================================

\tcbuselibrary{theorems,skins,hooks}
\newtcbtheorem[number within=section]{Exercise}{Exercise}
{%
	enhanced,
	breakable,
	colback = myexercisebg,
	frame hidden,
	boxrule = 0sp,
	borderline west = {2pt}{0pt}{myexercisefg},
	sharp corners,
	detach title,
	before upper = \tcbtitle\par\smallskip,
	coltitle = myexercisefg,
	fonttitle = \bfseries\sffamily,
	description font = \mdseries,
	separator sign none,
	segmentation style={solid, myexercisefg},
}
{th}

\tcbuselibrary{theorems,skins,hooks}
\newtcbtheorem[number within=chapter]{exercise}{Exercise}
{%
	enhanced,
	breakable,
	colback = myexercisebg,
	frame hidden,
	boxrule = 0sp,
	borderline west = {2pt}{0pt}{myexercisefg},
	sharp corners,
	detach title,
	before upper = \tcbtitle\par\smallskip,
	coltitle = myexercisefg,
	fonttitle = \bfseries\sffamily,
	description font = \mdseries,
	separator sign none,
	segmentation style={solid, myexercisefg},
}
{th}

%================================
% EXAMPLE BOX
%================================

\newtcbtheorem[number within=section]{Example}{Example}
{%
	colback = myexamplebg
	,breakable
	,colframe = myexamplefr
	,coltitle = myexampleti
	,boxrule = 1pt
	,sharp corners
	,detach title
	,before upper=\tcbtitle\par\smallskip
	,fonttitle = \bfseries
	,description font = \mdseries
	,separator sign none
	,description delimiters parenthesis
}
{ex}

\newtcbtheorem[number within=chapter]{example}{Example}
{%
	colback = myexamplebg
	,breakable
	,colframe = myexamplefr
	,coltitle = myexampleti
	,boxrule = 1pt
	,sharp corners
	,detach title
	,before upper=\tcbtitle\par\smallskip
	,fonttitle = \bfseries
	,description font = \mdseries
	,separator sign none
	,description delimiters parenthesis
}
{ex}

%================================
% DEFINITION BOX
%================================

\newtcbtheorem[number within=section]{Definition}{Definition}{enhanced,
	before skip=2mm,after skip=2mm, colback=red!5,colframe=red!80!black,boxrule=0.5mm,
	attach boxed title to top left={xshift=1cm,yshift*=1mm-\tcboxedtitleheight}, varwidth boxed title*=-3cm,
	boxed title style={frame code={
					\path[fill=tcbcolback]
					([yshift=-1mm,xshift=-1mm]frame.north west)
					arc[start angle=0,end angle=180,radius=1mm]
					([yshift=-1mm,xshift=1mm]frame.north east)
					arc[start angle=180,end angle=0,radius=1mm];
					\path[left color=tcbcolback!60!black,right color=tcbcolback!60!black,
						middle color=tcbcolback!80!black]
					([xshift=-2mm]frame.north west) -- ([xshift=2mm]frame.north east)
					[rounded corners=1mm]-- ([xshift=1mm,yshift=-1mm]frame.north east)
					-- (frame.south east) -- (frame.south west)
					-- ([xshift=-1mm,yshift=-1mm]frame.north west)
					[sharp corners]-- cycle;
				},interior engine=empty,
		},
	fonttitle=\bfseries,
	title={#2},#1}{def}
\newtcbtheorem[number within=chapter]{definition}{Definition}{enhanced,
	before skip=2mm,after skip=2mm, colback=red!5,colframe=red!80!black,boxrule=0.5mm,
	attach boxed title to top left={xshift=1cm,yshift*=1mm-\tcboxedtitleheight}, varwidth boxed title*=-3cm,
	boxed title style={frame code={
					\path[fill=tcbcolback]
					([yshift=-1mm,xshift=-1mm]frame.north west)
					arc[start angle=0,end angle=180,radius=1mm]
					([yshift=-1mm,xshift=1mm]frame.north east)
					arc[start angle=180,end angle=0,radius=1mm];
					\path[left color=tcbcolback!60!black,right color=tcbcolback!60!black,
						middle color=tcbcolback!80!black]
					([xshift=-2mm]frame.north west) -- ([xshift=2mm]frame.north east)
					[rounded corners=1mm]-- ([xshift=1mm,yshift=-1mm]frame.north east)
					-- (frame.south east) -- (frame.south west)
					-- ([xshift=-1mm,yshift=-1mm]frame.north west)
					[sharp corners]-- cycle;
				},interior engine=empty,
		},
	fonttitle=\bfseries,
	title={#2},#1}{def}



%================================
% Solution BOX
%================================

\makeatletter
\newtcbtheorem{question}{Question}{enhanced,
	breakable,
	colback=white,
	colframe=myb!80!black,
	attach boxed title to top left={yshift*=-\tcboxedtitleheight},
	fonttitle=\bfseries,
	title={#2},
	boxed title size=title,
	boxed title style={%
			sharp corners,
			rounded corners=northwest,
			colback=tcbcolframe,
			boxrule=0pt,
		},
	underlay boxed title={%
			\path[fill=tcbcolframe] (title.south west)--(title.south east)
			to[out=0, in=180] ([xshift=5mm]title.east)--
			(title.center-|frame.east)
			[rounded corners=\kvtcb@arc] |-
			(frame.north) -| cycle;
		},
	#1
}{def}
\makeatother

%================================
% SOLUTION BOX
%================================

\makeatletter
\newtcolorbox{solution}{enhanced,
	breakable,
	colback=white,
	colframe=myg!80!black,
	attach boxed title to top left={yshift*=-\tcboxedtitleheight},
	title=Solution,
	boxed title size=title,
	boxed title style={%
			sharp corners,
			rounded corners=northwest,
			colback=tcbcolframe,
			boxrule=0pt,
		},
	underlay boxed title={%
			\path[fill=tcbcolframe] (title.south west)--(title.south east)
			to[out=0, in=180] ([xshift=5mm]title.east)--
			(title.center-|frame.east)
			[rounded corners=\kvtcb@arc] |-
			(frame.north) -| cycle;
		},
}
\makeatother

%================================
% Question BOX
%================================

\makeatletter
\newtcbtheorem{qstion}{Question}{enhanced,
	breakable,
	colback=white,
	colframe=mygr,
	attach boxed title to top left={yshift*=-\tcboxedtitleheight},
	fonttitle=\bfseries,
	title={#2},
	boxed title size=title,
	boxed title style={%
			sharp corners,
			rounded corners=northwest,
			colback=tcbcolframe,
			boxrule=0pt,
		},
	underlay boxed title={%
			\path[fill=tcbcolframe] (title.south west)--(title.south east)
			to[out=0, in=180] ([xshift=5mm]title.east)--
			(title.center-|frame.east)
			[rounded corners=\kvtcb@arc] |-
			(frame.north) -| cycle;
		},
	#1
}{def}
\makeatother

\newtcbtheorem[number within=chapter]{wconc}{Wrong Concept}{
	breakable,
	enhanced,
	colback=white,
	colframe=myr,
	arc=0pt,
	outer arc=0pt,
	fonttitle=\bfseries\sffamily\large,
	colbacktitle=myr,
	attach boxed title to top left={},
	boxed title style={
			enhanced,
			skin=enhancedfirst jigsaw,
			arc=3pt,
			bottom=0pt,
			interior style={fill=myr}
		},
	#1
}{def}



%================================
% NOTE BOX
%================================

\usetikzlibrary{arrows,calc,shadows.blur}
\tcbuselibrary{skins}
\newtcolorbox{note}[1][]{%
	enhanced jigsaw,
	colback=gray!20!white,%
	colframe=gray!80!black,
	size=small,
	boxrule=1pt,
	title=\textbf{Note:},
	halign title=flush center,
	coltitle=black,
	breakable,
	drop shadow=black!50!white,
	attach boxed title to top left={xshift=1cm,yshift=-\tcboxedtitleheight/2,yshifttext=-\tcboxedtitleheight/2},
	minipage boxed title=1.5cm,
	boxed title style={%
			colback=white,
			size=fbox,
			boxrule=1pt,
			boxsep=2pt,
			underlay={%
					\coordinate (dotA) at ($(interior.west) + (-0.5pt,0)$);
					\coordinate (dotB) at ($(interior.east) + (0.5pt,0)$);
					\begin{scope}
						\clip (interior.north west) rectangle ([xshift=3ex]interior.east);
						\filldraw [white, blur shadow={shadow opacity=60, shadow yshift=-.75ex}, rounded corners=2pt] (interior.north west) rectangle (interior.south east);
					\end{scope}
					\begin{scope}[gray!80!black]
						\fill (dotA) circle (2pt);
						\fill (dotB) circle (2pt);
					\end{scope}
				},
		},
	#1,
}

%%%%%%%%%%%%%%%%%%%%%%%%%%%%%%
% SELF MADE COMMANDS
%%%%%%%%%%%%%%%%%%%%%%%%%%%%%%


\newcommand{\thm}[2]{\begin{Theorem}{#1}{}#2\end{Theorem}}
\newcommand{\cor}[2]{\begin{Corollary}{#1}{}#2\end{Corollary}}
\newcommand{\mlenma}[2]{\begin{Lenma}{#1}{}#2\end{Lenma}}
\newcommand{\mprop}[2]{\begin{Prop}{#1}{}#2\end{Prop}}
\newcommand{\clm}[3]{\begin{claim}{#1}{#2}#3\end{claim}}
\newcommand{\wc}[2]{\begin{wconc}{#1}{}\setlength{\parindent}{1cm}#2\end{wconc}}
\newcommand{\thmcon}[1]{\begin{Theoremcon}{#1}\end{Theoremcon}}
\newcommand{\ex}[2]{\begin{Example}{#1}{}#2\end{Example}}
\newcommand{\dfn}[2]{\begin{Definition}[colbacktitle=red!75!black]{#1}{}#2\end{Definition}}
\newcommand{\dfnc}[2]{\begin{definition}[colbacktitle=red!75!black]{#1}{}#2\end{definition}}
\newcommand{\qs}[2]{\begin{question*}{#1}{}#2\end{question*}}
\newcommand{\mpf}[2]{\begin{myproof}[#1]#2\end{myproof}}
\newcommand{\nt}[1]{\begin{note}#1\end{note}}

\newcommand*\circled[1]{\tikz[baseline=(char.base)]{
		\node[shape=circle,draw,inner sep=1pt] (char) {#1};}}
\newcommand\getcurrentref[1]{%
	\ifnumequal{\value{#1}}{0}
	{??}
	{\the\value{#1}}%
}
\newcommand{\getCurrentSectionNumber}{\getcurrentref{section}}
\newenvironment{myproof}[1][\proofname]{%
	\proof[\bfseries #1: ]%
}{\endproof}

\newcommand{\mclm}[2]{\begin{myclaim}[#1]#2\end{myclaim}}
\newenvironment{myclaim}[1][\claimname]{\proof[\bfseries #1: ]}{}

\newcounter{mylabelcounter}

\makeatletter
\newcommand{\setword}[2]{%
	\phantomsection
	#1\def\@currentlabel{\unexpanded{#1}}\label{#2}%
}
\makeatother




\tikzset{
	symbol/.style={
			draw=none,
			every to/.append style={
					edge node={node [sloped, allow upside down, auto=false]{$#1$}}}
		}
}


% deliminators
\DeclarePairedDelimiter{\abs}{\lvert}{\rvert}
\DeclarePairedDelimiter{\norm}{\lVert}{\rVert}

\DeclarePairedDelimiter{\ceil}{\lceil}{\rceil}
\DeclarePairedDelimiter{\floor}{\lfloor}{\rfloor}
\DeclarePairedDelimiter{\round}{\lfloor}{\rceil}

\newsavebox\diffdbox
\newcommand{\slantedromand}{{\mathpalette\makesl{d}}}
\newcommand{\makesl}[2]{%
\begingroup
\sbox{\diffdbox}{$\mathsurround=0pt#1\mathrm{#2}$}%
\pdfsave
\pdfsetmatrix{1 0 0.2 1}%
\rlap{\usebox{\diffdbox}}%
\pdfrestore
\hskip\wd\diffdbox
\endgroup
}
\newcommand{\dd}[1][]{\ensuremath{\mathop{}\!\ifstrempty{#1}{%
\slantedromand\@ifnextchar^{\hspace{0.2ex}}{\hspace{0.1ex}}}%
{\slantedromand\hspace{0.2ex}^{#1}}}}
\ProvideDocumentCommand\dv{o m g}{%
  \ensuremath{%
    \IfValueTF{#3}{%
      \IfNoValueTF{#1}{%
        \frac{\dd #2}{\dd #3}%
      }{%
        \frac{\dd^{#1} #2}{\dd #3^{#1}}%
      }%
    }{%
      \IfNoValueTF{#1}{%
        \frac{\dd}{\dd #2}%
      }{%
        \frac{\dd^{#1}}{\dd #2^{#1}}%
      }%
    }%
  }%
}
\providecommand*{\pdv}[3][]{\frac{\partial^{#1}#2}{\partial#3^{#1}}}
%  - others
\DeclareMathOperator{\Lap}{\mathcal{L}}
\DeclareMathOperator{\Var}{Var} % varience
\DeclareMathOperator{\Cov}{Cov} % covarience
\DeclareMathOperator{\E}{E} % expected

% Since the amsthm package isn't loaded

% I prefer the slanted \leq
\let\oldleq\leq % save them in case they're every wanted
\let\oldgeq\geq
\renewcommand{\leq}{\leqslant}
\renewcommand{\geq}{\geqslant}

% % redefine matrix env to allow for alignment, use r as default
% \renewcommand*\env@matrix[1][r]{\hskip -\arraycolsep
%     \let\@ifnextchar\new@ifnextchar
%     \array{*\c@MaxMatrixCols #1}}


%\usepackage{framed}
%\usepackage{titletoc}
%\usepackage{etoolbox}
%\usepackage{lmodern}


%\patchcmd{\tableofcontents}{\contentsname}{\sffamily\contentsname}{}{}

%\renewenvironment{leftbar}
%{\def\FrameCommand{\hspace{6em}%
%		{\color{myyellow}\vrule width 2pt depth 6pt}\hspace{1em}}%
%	\MakeFramed{\parshape 1 0cm \dimexpr\textwidth-6em\relax\FrameRestore}\vskip2pt%
%}
%{\endMakeFramed}

%\titlecontents{chapter}
%[0em]{\vspace*{2\baselineskip}}
%{\parbox{4.5em}{%
%		\hfill\Huge\sffamily\bfseries\color{myred}\thecontentspage}%
%	\vspace*{-2.3\baselineskip}\leftbar\textsc{\small\chaptername~\thecontentslabel}\\\sffamily}
%{}{\endleftbar}
%\titlecontents{section}
%[8.4em]
%{\sffamily\contentslabel{3em}}{}{}
%{\hspace{0.5em}\nobreak\itshape\color{myred}\contentspage}
%\titlecontents{subsection}
%[8.4em]
%{\sffamily\contentslabel{3em}}{}{}  
%{\hspace{0.5em}\nobreak\itshape\color{myred}\contentspage}



%%%%%%%%%%%%%%%%%%%%%%%%%%%%%%%%%%%%%%%%%%%
% TABLE OF CONTENTS
%%%%%%%%%%%%%%%%%%%%%%%%%%%%%%%%%%%%%%%%%%%

\usepackage{tikz}
\definecolor{doc}{RGB}{0,60,110}
\usepackage{titletoc}
\contentsmargin{0cm}
\titlecontents{chapter}[3.7pc]
{\addvspace{30pt}%
	\begin{tikzpicture}[remember picture, overlay]%
		\draw[fill=doc!60,draw=doc!60] (-7,-.1) rectangle (-0.9,.5);%
		\pgftext[left,x=-3.5cm,y=0.2cm]{\color{white}\Large\sc\bfseries Chapter\ \thecontentslabel};%
	\end{tikzpicture}\color{doc!60}\large\sc\bfseries}%
{}
{}
{\;\titlerule\;\large\sc\bfseries Page \thecontentspage
	\begin{tikzpicture}[remember picture, overlay]
		\draw[fill=doc!60,draw=doc!60] (2pt,0) rectangle (4,0.1pt);
	\end{tikzpicture}}%
\titlecontents{section}[3.7pc]
{\addvspace{2pt}}
{\contentslabel[\thecontentslabel]{2pc}}
{}
{\hfill\small \thecontentspage}
[]
\titlecontents*{subsection}[3.7pc]
{\addvspace{-1pt}\small}
{}
{}
{\ --- \small\thecontentspage}
[ \textbullet\ ][]

\makeatletter
\renewcommand{\tableofcontents}{%
	\chapter*{%
	  \vspace*{-20\p@}%
	  \begin{tikzpicture}[remember picture, overlay]%
		  \pgftext[right,x=15cm,y=0.2cm]{\color{doc!60}\Huge\sc\bfseries \contentsname};%
		  \draw[fill=doc!60,draw=doc!60] (13,-.75) rectangle (20,1);%
		  \clip (13,-.75) rectangle (20,1);
		  \pgftext[right,x=15cm,y=0.2cm]{\color{white}\Huge\sc\bfseries \contentsname};%
	  \end{tikzpicture}}%
	\@starttoc{toc}}
\makeatother
\newcommand{\id}{\mathrm{id}}
\newcommand{\taking}[1]{\xrightarrow{#1}}
\newcommand{\inv}{^{-1}}

%From M170 "Introduction to Graph Theory" at SJSU
\DeclareMathOperator{\diam}{diam}
\DeclareMathOperator{\ord}{ord}
\newcommand{\defeq}{\overset{\mathrm{def}}{=}}

%From the USAMO .tex files
\newcommand{\ts}{\textsuperscript}
\newcommand{\dg}{^\circ}
\newcommand{\ii}{\item}

% % From Math 55 and Math 145 at Harvard
% \newenvironment{subproof}[1][Proof]{%
% \begin{proof}[#1] \renewcommand{\qedsymbol}{$\blacksquare$}}%
% {\end{proof}}

\newcommand{\liff}{\leftrightarrow}
\newcommand{\lthen}{\rightarrow}
\newcommand{\opname}{\operatorname}
\newcommand{\surjto}{\twoheadrightarrow}
\newcommand{\injto}{\hookrightarrow}
\newcommand{\On}{\mathrm{On}} % ordinals
\DeclareMathOperator{\img}{im} % Image
\DeclareMathOperator{\Img}{Im} % Image
\DeclareMathOperator{\coker}{coker} % Cokernel
\DeclareMathOperator{\Coker}{Coker} % Cokernel
\DeclareMathOperator{\Ker}{Ker} % Kernel
\DeclareMathOperator{\rank}{rank}
\DeclareMathOperator{\Spec}{Spec} % spectrum
\DeclareMathOperator{\Tr}{Tr} % trace
\DeclareMathOperator{\pr}{pr} % projection
\DeclareMathOperator{\ext}{ext} % extension
\DeclareMathOperator{\pred}{pred} % predecessor
\DeclareMathOperator{\dom}{dom} % domain
\DeclareMathOperator{\ran}{ran} % range
\DeclareMathOperator{\Hom}{Hom} % homomorphism
\DeclareMathOperator{\Mor}{Mor} % morphisms
\DeclareMathOperator{\End}{End} % endomorphism

\newcommand{\eps}{\epsilon}
\newcommand{\veps}{\varepsilon}
\newcommand{\ol}{\overline}
\newcommand{\ul}{\underline}
\newcommand{\wt}{\widetilde}
\newcommand{\wh}{\widehat}
\newcommand{\vocab}[1]{\textbf{\color{blue} #1}}
\providecommand{\half}{\frac{1}{2}}
\newcommand{\dang}{\measuredangle} %% Directed angle
\newcommand{\ray}[1]{\overrightarrow{#1}}
\newcommand{\seg}[1]{\overline{#1}}
\newcommand{\arc}[1]{\wideparen{#1}}
\DeclareMathOperator{\cis}{cis}
\DeclareMathOperator*{\lcm}{lcm}
\DeclareMathOperator*{\argmin}{arg min}
\DeclareMathOperator*{\argmax}{arg max}
\newcommand{\cycsum}{\sum_{\mathrm{cyc}}}
\newcommand{\symsum}{\sum_{\mathrm{sym}}}
\newcommand{\cycprod}{\prod_{\mathrm{cyc}}}
\newcommand{\symprod}{\prod_{\mathrm{sym}}}
\newcommand{\Qed}{\begin{flushright}\qed\end{flushright}}
\newcommand{\parinn}{\setlength{\parindent}{1cm}}
\newcommand{\parinf}{\setlength{\parindent}{0cm}}
% \newcommand{\norm}{\|\cdot\|}
\newcommand{\inorm}{\norm_{\infty}}
\newcommand{\opensets}{\{V_{\alpha}\}_{\alpha\in I}}
\newcommand{\oset}{V_{\alpha}}
\newcommand{\opset}[1]{V_{\alpha_{#1}}}
\newcommand{\lub}{\text{lub}}
\newcommand{\del}[2]{\frac{\partial #1}{\partial #2}}
\newcommand{\Del}[3]{\frac{\partial^{#1} #2}{\partial^{#1} #3}}
\newcommand{\deld}[2]{\dfrac{\partial #1}{\partial #2}}
\newcommand{\Deld}[3]{\dfrac{\partial^{#1} #2}{\partial^{#1} #3}}
\newcommand{\lm}{\lambda}
\newcommand{\uin}{\mathbin{\rotatebox[origin=c]{90}{$\in$}}}
\newcommand{\usubset}{\mathbin{\rotatebox[origin=c]{90}{$\subset$}}}
\newcommand{\lt}{\left}
\newcommand{\rt}{\right}
\newcommand{\bs}[1]{\boldsymbol{#1}}
\newcommand{\exs}{\exists}
\newcommand{\st}{\strut}
\newcommand{\dps}[1]{\displaystyle{#1}}

\newcommand{\sol}{\setlength{\parindent}{0cm}\textbf{\textit{Solution:}}\setlength{\parindent}{1cm} }
\newcommand{\solve}[1]{\setlength{\parindent}{0cm}\textbf{\textit{Solution: }}\setlength{\parindent}{1cm}#1 \Qed}
% Things Lie
\newcommand{\kb}{\mathfrak b}
\newcommand{\kg}{\mathfrak g}
\newcommand{\kh}{\mathfrak h}
\newcommand{\kn}{\mathfrak n}
\newcommand{\ku}{\mathfrak u}
\newcommand{\kz}{\mathfrak z}
\DeclareMathOperator{\Ext}{Ext} % Ext functor
\DeclareMathOperator{\Tor}{Tor} % Tor functor
\newcommand{\gl}{\opname{\mathfrak{gl}}} % frak gl group
\renewcommand{\sl}{\opname{\mathfrak{sl}}} % frak sl group chktex 6

% More script letters etc.
\newcommand{\SA}{\mathcal A}
\newcommand{\SB}{\mathcal B}
\newcommand{\SC}{\mathcal C}
\newcommand{\SF}{\mathcal F}
\newcommand{\SG}{\mathcal G}
\newcommand{\SH}{\mathcal H}
\newcommand{\OO}{\mathcal O}

\newcommand{\SCA}{\mathscr A}
\newcommand{\SCB}{\mathscr B}
\newcommand{\SCC}{\mathscr C}
\newcommand{\SCD}{\mathscr D}
\newcommand{\SCE}{\mathscr E}
\newcommand{\SCF}{\mathscr F}
\newcommand{\SCG}{\mathscr G}
\newcommand{\SCH}{\mathscr H}

% Mathfrak primes
\newcommand{\km}{\mathfrak m}
\newcommand{\kp}{\mathfrak p}
\newcommand{\kq}{\mathfrak q}

% number sets
\newcommand{\RR}[1][]{\ensuremath{\ifstrempty{#1}{\mathbb{R}}{\mathbb{R}^{#1}}}}
\newcommand{\NN}[1][]{\ensuremath{\ifstrempty{#1}{\mathbb{N}}{\mathbb{N}^{#1}}}}
\newcommand{\ZZ}[1][]{\ensuremath{\ifstrempty{#1}{\mathbb{Z}}{\mathbb{Z}^{#1}}}}
\newcommand{\QQ}[1][]{\ensuremath{\ifstrempty{#1}{\mathbb{Q}}{\mathbb{Q}^{#1}}}}
\newcommand{\CC}[1][]{\ensuremath{\ifstrempty{#1}{\mathbb{C}}{\mathbb{C}^{#1}}}}
\newcommand{\PP}[1][]{\ensuremath{\ifstrempty{#1}{\mathbb{P}}{\mathbb{P}^{#1}}}}
\newcommand{\HH}[1][]{\ensuremath{\ifstrempty{#1}{\mathbb{H}}{\mathbb{H}^{#1}}}}
\newcommand{\FF}[1][]{\ensuremath{\ifstrempty{#1}{\mathbb{F}}{\mathbb{F}^{#1}}}}

% number sets without arguments
\newcommand{\R}{\ensuremath{\mathbb{R}}}
\newcommand{\N}{\ensuremath{\mathbb{N}}}
\newcommand{\Z}{\ensuremath{\mathbb{Z}}}
\newcommand{\Q}{\ensuremath{\mathbb{Q}}}
\newcommand{\C}{\ensuremath{\mathbb{C}}}
\newcommand{\F}{\ensuremath{\mathbb{F}}}

% expected value
\newcommand{\EE}{\ensuremath{\mathbb{E}}}
\newcommand{\charin}{\text{ char }}
\DeclareMathOperator{\sign}{sign}
\DeclareMathOperator{\Aut}{Aut}
\DeclareMathOperator{\Inn}{Inn}
\DeclareMathOperator{\Syl}{Syl}
\DeclareMathOperator{\Gal}{Gal}
\DeclareMathOperator{\GL}{GL} % General linear group
\DeclareMathOperator{\SL}{SL} % Special linear group

%---------------------------------------
% BlackBoard Math Fonts :-
%---------------------------------------

%Captital Letters
\newcommand{\bbA}{\mathbb{A}}	\newcommand{\bbB}{\mathbb{B}}
\newcommand{\bbC}{\mathbb{C}}	\newcommand{\bbD}{\mathbb{D}}
\newcommand{\bbE}{\mathbb{E}}	\newcommand{\bbF}{\mathbb{F}}
\newcommand{\bbG}{\mathbb{G}}	\newcommand{\bbH}{\mathbb{H}}
\newcommand{\bbI}{\mathbb{I}}	\newcommand{\bbJ}{\mathbb{J}}
\newcommand{\bbK}{\mathbb{K}}	\newcommand{\bbL}{\mathbb{L}}
\newcommand{\bbM}{\mathbb{M}}	\newcommand{\bbN}{\mathbb{N}}
\newcommand{\bbO}{\mathbb{O}}	\newcommand{\bbP}{\mathbb{P}}
\newcommand{\bbQ}{\mathbb{Q}}	\newcommand{\bbR}{\mathbb{R}}
\newcommand{\bbS}{\mathbb{S}}	\newcommand{\bbT}{\mathbb{T}}
\newcommand{\bbU}{\mathbb{U}}	\newcommand{\bbV}{\mathbb{V}}
\newcommand{\bbW}{\mathbb{W}}	\newcommand{\bbX}{\mathbb{X}}
\newcommand{\bbY}{\mathbb{Y}}	\newcommand{\bbZ}{\mathbb{Z}}

%---------------------------------------
% MathCal Fonts :-
%---------------------------------------

%Captital Letters
\newcommand{\mcA}{\mathcal{A}}	\newcommand{\mcB}{\mathcal{B}}
\newcommand{\mcC}{\mathcal{C}}	\newcommand{\mcD}{\mathcal{D}}
\newcommand{\mcE}{\mathcal{E}}	\newcommand{\mcF}{\mathcal{F}}
\newcommand{\mcG}{\mathcal{G}}	\newcommand{\mcH}{\mathcal{H}}
\newcommand{\mcI}{\mathcal{I}}	\newcommand{\mcJ}{\mathcal{J}}
\newcommand{\mcK}{\mathcal{K}}	\newcommand{\mcL}{\mathcal{L}}
\newcommand{\mcM}{\mathcal{M}}	\newcommand{\mcN}{\mathcal{N}}
\newcommand{\mcO}{\mathcal{O}}	\newcommand{\mcP}{\mathcal{P}}
\newcommand{\mcQ}{\mathcal{Q}}	\newcommand{\mcR}{\mathcal{R}}
\newcommand{\mcS}{\mathcal{S}}	\newcommand{\mcT}{\mathcal{T}}
\newcommand{\mcU}{\mathcal{U}}	\newcommand{\mcV}{\mathcal{V}}
\newcommand{\mcW}{\mathcal{W}}	\newcommand{\mcX}{\mathcal{X}}
\newcommand{\mcY}{\mathcal{Y}}	\newcommand{\mcZ}{\mathcal{Z}}


%---------------------------------------
% Bold Math Fonts :-
%---------------------------------------

%Captital Letters
\newcommand{\bmA}{\boldsymbol{A}}	\newcommand{\bmB}{\boldsymbol{B}}
\newcommand{\bmC}{\boldsymbol{C}}	\newcommand{\bmD}{\boldsymbol{D}}
\newcommand{\bmE}{\boldsymbol{E}}	\newcommand{\bmF}{\boldsymbol{F}}
\newcommand{\bmG}{\boldsymbol{G}}	\newcommand{\bmH}{\boldsymbol{H}}
\newcommand{\bmI}{\boldsymbol{I}}	\newcommand{\bmJ}{\boldsymbol{J}}
\newcommand{\bmK}{\boldsymbol{K}}	\newcommand{\bmL}{\boldsymbol{L}}
\newcommand{\bmM}{\boldsymbol{M}}	\newcommand{\bmN}{\boldsymbol{N}}
\newcommand{\bmO}{\boldsymbol{O}}	\newcommand{\bmP}{\boldsymbol{P}}
\newcommand{\bmQ}{\boldsymbol{Q}}	\newcommand{\bmR}{\boldsymbol{R}}
\newcommand{\bmS}{\boldsymbol{S}}	\newcommand{\bmT}{\boldsymbol{T}}
\newcommand{\bmU}{\boldsymbol{U}}	\newcommand{\bmV}{\boldsymbol{V}}
\newcommand{\bmW}{\boldsymbol{W}}	\newcommand{\bmX}{\boldsymbol{X}}
\newcommand{\bmY}{\boldsymbol{Y}}	\newcommand{\bmZ}{\boldsymbol{Z}}
%Small Letters
\newcommand{\bma}{\boldsymbol{a}}	\newcommand{\bmb}{\boldsymbol{b}}
\newcommand{\bmc}{\boldsymbol{c}}	\newcommand{\bmd}{\boldsymbol{d}}
\newcommand{\bme}{\boldsymbol{e}}	\newcommand{\bmf}{\boldsymbol{f}}
\newcommand{\bmg}{\boldsymbol{g}}	\newcommand{\bmh}{\boldsymbol{h}}
\newcommand{\bmi}{\boldsymbol{i}}	\newcommand{\bmj}{\boldsymbol{j}}
\newcommand{\bmk}{\boldsymbol{k}}	\newcommand{\bml}{\boldsymbol{l}}
\newcommand{\bmm}{\boldsymbol{m}}	\newcommand{\bmn}{\boldsymbol{n}}
\newcommand{\bmo}{\boldsymbol{o}}	\newcommand{\bmp}{\boldsymbol{p}}
\newcommand{\bmq}{\boldsymbol{q}}	\newcommand{\bmr}{\boldsymbol{r}}
\newcommand{\bms}{\boldsymbol{s}}	\newcommand{\bmt}{\boldsymbol{t}}
\newcommand{\bmu}{\boldsymbol{u}}	\newcommand{\bmv}{\boldsymbol{v}}
\newcommand{\bmw}{\boldsymbol{w}}	\newcommand{\bmx}{\boldsymbol{x}}
\newcommand{\bmy}{\boldsymbol{y}}	\newcommand{\bmz}{\boldsymbol{z}}

%---------------------------------------
% Scr Math Fonts :-
%---------------------------------------

\newcommand{\sA}{{\mathscr{A}}}   \newcommand{\sB}{{\mathscr{B}}}
\newcommand{\sC}{{\mathscr{C}}}   \newcommand{\sD}{{\mathscr{D}}}
\newcommand{\sE}{{\mathscr{E}}}   \newcommand{\sF}{{\mathscr{F}}}
\newcommand{\sG}{{\mathscr{G}}}   \newcommand{\sH}{{\mathscr{H}}}
\newcommand{\sI}{{\mathscr{I}}}   \newcommand{\sJ}{{\mathscr{J}}}
\newcommand{\sK}{{\mathscr{K}}}   \newcommand{\sL}{{\mathscr{L}}}
\newcommand{\sM}{{\mathscr{M}}}   \newcommand{\sN}{{\mathscr{N}}}
\newcommand{\sO}{{\mathscr{O}}}   \newcommand{\sP}{{\mathscr{P}}}
\newcommand{\sQ}{{\mathscr{Q}}}   \newcommand{\sR}{{\mathscr{R}}}
\newcommand{\sS}{{\mathscr{S}}}   \newcommand{\sT}{{\mathscr{T}}}
\newcommand{\sU}{{\mathscr{U}}}   \newcommand{\sV}{{\mathscr{V}}}
\newcommand{\sW}{{\mathscr{W}}}   \newcommand{\sX}{{\mathscr{X}}}
\newcommand{\sY}{{\mathscr{Y}}}   \newcommand{\sZ}{{\mathscr{Z}}}


%---------------------------------------
% Math Fraktur Font
%---------------------------------------

%Captital Letters
\newcommand{\mfA}{\mathfrak{A}}	\newcommand{\mfB}{\mathfrak{B}}
\newcommand{\mfC}{\mathfrak{C}}	\newcommand{\mfD}{\mathfrak{D}}
\newcommand{\mfE}{\mathfrak{E}}	\newcommand{\mfF}{\mathfrak{F}}
\newcommand{\mfG}{\mathfrak{G}}	\newcommand{\mfH}{\mathfrak{H}}
\newcommand{\mfI}{\mathfrak{I}}	\newcommand{\mfJ}{\mathfrak{J}}
\newcommand{\mfK}{\mathfrak{K}}	\newcommand{\mfL}{\mathfrak{L}}
\newcommand{\mfM}{\mathfrak{M}}	\newcommand{\mfN}{\mathfrak{N}}
\newcommand{\mfO}{\mathfrak{O}}	\newcommand{\mfP}{\mathfrak{P}}
\newcommand{\mfQ}{\mathfrak{Q}}	\newcommand{\mfR}{\mathfrak{R}}
\newcommand{\mfS}{\mathfrak{S}}	\newcommand{\mfT}{\mathfrak{T}}
\newcommand{\mfU}{\mathfrak{U}}	\newcommand{\mfV}{\mathfrak{V}}
\newcommand{\mfW}{\mathfrak{W}}	\newcommand{\mfX}{\mathfrak{X}}
\newcommand{\mfY}{\mathfrak{Y}}	\newcommand{\mfZ}{\mathfrak{Z}}
%Small Letters
\newcommand{\mfa}{\mathfrak{a}}	\newcommand{\mfb}{\mathfrak{b}}
\newcommand{\mfc}{\mathfrak{c}}	\newcommand{\mfd}{\mathfrak{d}}
\newcommand{\mfe}{\mathfrak{e}}	\newcommand{\mff}{\mathfrak{f}}
\newcommand{\mfg}{\mathfrak{g}}	\newcommand{\mfh}{\mathfrak{h}}
\newcommand{\mfi}{\mathfrak{i}}	\newcommand{\mfj}{\mathfrak{j}}
\newcommand{\mfk}{\mathfrak{k}}	\newcommand{\mfl}{\mathfrak{l}}
\newcommand{\mfm}{\mathfrak{m}}	\newcommand{\mfn}{\mathfrak{n}}
\newcommand{\mfo}{\mathfrak{o}}	\newcommand{\mfp}{\mathfrak{p}}
\newcommand{\mfq}{\mathfrak{q}}	\newcommand{\mfr}{\mathfrak{r}}
\newcommand{\mfs}{\mathfrak{s}}	\newcommand{\mft}{\mathfrak{t}}
\newcommand{\mfu}{\mathfrak{u}}	\newcommand{\mfv}{\mathfrak{v}}
\newcommand{\mfw}{\mathfrak{w}}	\newcommand{\mfx}{\mathfrak{x}}
\newcommand{\mfy}{\mathfrak{y}}	\newcommand{\mfz}{\mathfrak{z}}

\title{\Huge{21-235 Math Studies Analysis I}}
\author{\huge{Rohan Jain}}
\date{}

\begin{document}

\maketitle
\newpage% or \cleardoublepage
% \pdfbookmark[<level>]{<title>}{<dest>}
\pdfbookmark[section]{\contentsname}{toc}
\tableofcontents

\pagebreak

\chapter{}
\section{Ordered Fields (Review)}

\dfn{Order}{Let $E$ be a set. An \emph{order} on $E$ is a relation $<$ on $E$ such that for all $x, y, z \in E$:
\begin{enumerate}
    \item (Trichotomy) Exactly one of the following holds: $x < y$, $x = y$, or $x > y$.
    \item (Transitivity) If $x < y$ and $y < z$, then $x < z$.
\end{enumerate}}

\ex{Examples of Ordered Sets}{\begin{enumerate}
    \item This definition develops orders on basic number systems: e.g. $\ZZ$, $\QQ$, and $\RR$.
    \item Define $\lesssim$ on $\ZZ$ as follows: We say that $m \lesssim n$ for $m, n \in \ZZ$ if:
\begin{enumerate}
    \item $m$ is even and $n$ is odd
    \item $m,n$ are even and $m < n$
    \item $m,n$ are odd and $m < n$. 
\end{enumerate}
\end{enumerate}}
\noindent Key Concepts: 
\begin{itemize}
    \renewcommand\labelitemi{--}
    \item upper/lower bounds of sets
    \item bounded sets
    \item max/min
    \item supremum/infimum
    \item supremum/infimum property: An ordered set $E$ satisfies such a property if every nonempty set $A \subseteq E$ that's bounded above/below has a supremum/infimum in $E$.
    \item Fact: sup prop $\implies$ inf prop
\end{itemize}

\dfn{Ordered Field}{Let $\FF$ be a field with order $<$. We say that $\FF$ is an \emph{ordered field} provided that:
\begin{enumerate}
    \item For all $x, y, z \in \FF$, if $x < y$, then $x + z < y + z$.
    \item For all $x, y \in \FF$, if $0 < x$ and $0 < y$, then $0 < x \cdot y$.
\end{enumerate}}

\ex{}{$\QQ$ is a field.}
\noindent Facts of any ordered field:
\begin{enumerate}
    \item $0 < 1$
    \item $\nexists x \in \FF$ such that $x^2 = -1$.
\end{enumerate}

\dfn{Ordered Subfield, Homomorphism, Isomorphism}{Let $\FF$ be an ordered field. 
\begin{enumerate}
    \item A set $\mathbb K \subseteq \FF$ is called an \emph{ordered subfield} if $mathbb K$ is an algeraic subfield and $\mathbb K$ is an ordered field equipped with $<$ from $\FF$.
    \item Let $\mathbb G$ be an ordered field and let $f : \mathbb F \to \mathbb G$. We say that $f$ is an \emph{ordered field homomorphism} if it's a field homomorphism and $f(x) < f(y)$ whenever $x < y$.
    \item $f$ is an ordered field isomorphism if $f$ is an ordered field homomorphism and $f$ is bijective.
\end{enumerate}}
\nt{
\begin{enumerate}
    \item If $f : \mathbb F \to \mathbb G$ is an ordered field homomorphism, $f(\mathbb F)$ is an ordered subfield of $\mathbb G$.
    \item OF property $\implies$ $f$ is injective.
    \item $\therefore$ every ordered field homomorphism $f : \mathbb F \to \mathbb G$ is such that $f$ induces a bijection $f: \mathbb F \to f(\mathbb F) \subseteq \mathbb G$.
\end{enumerate}
}

\thm{$\mathbb Q$ is the smallest ordered field. More precisely, if $\FF$ is an ordered field, then there exists a canonical ordered field homomorphism $f : \QQ \to \FF$.}

\noindent Upshot/notation abuse: We identify $f(\QQ) = \QQ$ to view $\QQ \subseteq \FF$. 
In turn, $\NN \subset \ZZ \subset \QQ \subseteq \FF$. 


\section{Types of Ordered Fields}
\dfn{Archimedean, Dedekind complete}{Let $\FF$ be an ordered field.
\begin{enumerate}
    \item We say that $\FF$ is Archimedean if $\forall 0 < x \in \FF$, $\exists n \in \NN$ such that $n > x$.
    \item We say that $\FF$ is Dedekind complete if it satisfies the supremum property.
\end{enumerate}}

\noindent Facts:
\begin{enumerate}
    \item $\QQ$ is Archimedean.
    \item If $\FF$ is Dedekind complete, then $\forall 0 < x \in \FF$ and $\forall 0< n \in \NN$, $\exists !$  $0 < y \in \FF$ such that $y^n = x$.
    \item $\QQ$ is not Dedekind complete. ($\sqrt 2$ is a counterexample.)
\end{enumerate}

\thm{}{Suppose $\FF$ is a Dedekind complete ordered field. Then $\FF$ is Archimedean.}
\begin{proof}
    If not, then $\NN \subset \FF$ is bounded above, and so the supremum property provides $x \in \FF$ such that $x = \sup \NN$. But then $x - 1$ is an upper bound for $\NN$, so there exists $n \in \NN$ such that $x-1 < n$. Hence $x < n + 1$, which contradicts the definition of $x$ as an upper bound. Therefore, $\FF$ is Archimedean.
\end{proof}

\section{Dedekind Completion}
Throughout this section, let $\FF$ be an Archimedean ordered field.

\dfn{Dedekind cut}{We say a set $\mathcal C \subseteq \FF$ is \emph{Dedekind cut} if:
\begin{enumerate}
    \item $\mathcal C \neq \emptyset$ and $\mathcal C \neq \FF$.
    \item If $p \in \mathcal C$ and $q \in \FF$ such that $q < p$, then $q \in \mathcal C$.
    \item If $p \in \mathcal C$, then $\exists r \in \mathcal C$ such that $p < r$.
\end{enumerate}
We will write $\FF^*$ for the set of all Dedekind cuts in $\FF$. It is called the \emph{Dedekind completion} of $\FF$.}

\nt{Let $\mathcal C \subseteq \FF$ be a cut. Then:
\begin{enumerate}
    \item If $p \in \mathcal C$, then $q \notin \mathcal C$, then $p < q$.
    \item If $r \notin \mathcal C$, and $r < s \in \FF$, then $s \notin \mathcal C$.
\end{enumerate}}

\ex{Cut examples}{\begin{enumerate}
    \item Let $q \in \FF$ and define $\mathcal C_q = \{ p \in \FF \mid p < q\}$. Then $\mathcal C_q$ is a cut.
    \begin{proof}
        \begin{enumerate}
            \item $q - 1 < q \implies q - 1 \in \mathcal C_q$. $q \not < q \implies q \notin \mathcal C_q \implies \mathcal C_q \neq \FF$.
            \item Let $p \in \mathcal C_q$. Suppose $s \in \FF$ such that $s < p$. Then $s < q \implies s \in \mathcal C_q$.
            \item Let $p \in \mathcal C_q$. Then $p < \frac{p + q}{2} < q \implies \frac{p + q}{2} \in \mathcal C_q$.
        \end{enumerate}
    \end{proof}
    \item Suppose $\FF$ is such that $\nexists x \in \FF$ such that $x^2 = 2$. Let $\mathcal C = \{ p \in \FF \mid p \leq 0 \text{ or } 0 < p^2 < 2\}$. Then $\mathcal C$ is a cut.
    \begin{proof}
        \begin{enumerate}
            \item $1 \in \mathcal C$ and $1^2 = 1 < 2$. $2 \notin \mathcal C$ and $2^2 = 4 > 2$. 
            \item Let $p \in \mathcal C$ and $q \in \FF$ such that $q < p$. If $q \leq 0$, then $q \in \mathcal C$ trivially. Suppose $0 < q < p$. Then $0 < q^2 < p^2 < 2$, so $q \in \mathcal C$.
            \item Let $p \in \mathcal C$. If $p \leq 0$, then $1 \in \mathcal C$ and $p < 1$, so we're done. Suppose $0 < p^2 < 2$. Note, $0 < 2 - p^2$, so $\frac{2p + 1}{2 - p^2} > 0$. Then we can define $r = 1 + \frac{2p + 1}{2 - p^2} \geq \max(1, \frac{2p + 1}{2 - p^2})$. Then $(p + 1/r)^2 = p^2 + \frac{2p}{r} + \frac{1}{r^2}$. We have:
            \begin{align*}
                p^2 + \frac{2p}{r} + \frac{1}{r^2} &< p^2 + \frac{2p}{r} + \frac 1r \\
                &= p^2 + \frac{2p+1}{r} \\
                &\leq p^2 + 2-p^2 \\
                &= 2.
            \end{align*}
            So, $p < p + 1/r < 2$ and $p + 1/r \in \mathcal C$.
        \end{enumerate}
    \end{proof}
\end{enumerate}}

\subsection{Ordering $\FF^*$}

\mlenma{}{The following hold:
\begin{enumerate}
    \item If $\mathcal A, \mathcal B \in \FF^*$, then exactly one holds:
    \begin{itemize}
        \item $\mathcal A \subset \mathcal B$
        \item $\mathcal A = \mathcal B$
        \item $\mathcal B \subset \mathcal A$
    \end{itemize}
    \item If $\mathcal A, \mathcal B, \mathcal C \in \FF^*$ and $\mathcal A \subset \mathcal B$ and $\mathcal B \subset \mathcal C$, then $\mathcal A \subset \mathcal C$.
\end{enumerate}}
\begin{proof}
    Proof of $2$ is trivial, as well as the equality part for $1$. 
    \begin{itemize}
        \item If $\mathcal A = \mathcal B$, we're done.
        \item Suppose $\exists b \in \mathcal B \setminus\mathcal A$. If $a \in \mathcal A$, then $a < b$, but $\mathcal B$ is a cut so $a \in \mathcal B$, so $\mathcal A \subset \mathcal B$.
        \item Suppose $\exists a \in \mathcal A \setminus \mathcal B$. Then $a < b$ for all $b \in \mathcal B$, so $a \in \mathcal B$, so $\mathcal B \subset \mathcal A$.
    \end{itemize}
\end{proof}


\dfn{Order on cuts}{Given $\mathcal A, \mathcal B \in \FF^*$, we say that $\mathcal A < \mathcal B$ if $\mathcal A \subset \mathcal B$. The lemma above shows that this is infact an order.}


\mlenma{}{Let $E \subseteq \FF^*$ be nonempty and bounded above. Then $\mathcal B = \bigcup_{\mathcal A \in E} \mathcal A$ is a cut.}
\begin{proof}
    \begin{enumerate}
        \item Since $E \neq \emptyset$, there exists $\mathcal A \in E$. So $\mathcal A \neq \emptyset$, hence $\mathcal B \neq \emptyset$. 
        
        Since $E$ is bounded above, there exists $\mathcal C \in \FF^*$ such that $\mathcal A \subset \mathcal C$ for all $\mathcal A \in E$. Since $\mathcal C$ is a cut, there is $q \in \FF$ such that $q \notin \mathcal C$. Then $q \notin \mathcal A$ for all $\mathcal A \in E$, so $q \notin \mathcal B$.
        \item Let $p \in \mathcal B$ and $q \in \FF$ such that $q < p$. Since $\mathcal B$ is a union of cuts, it follows that $p \in \mathcal A$ for some $\mathcal A \in E$. Since $\mathcal A$ is a cut, $q \in \mathcal A \subseteq \mathcal B$.
        \item Let $p \in \mathcal B$. Then $p \in \mathcal A$ for some $\mathcal A \in E$. Since $\mathcal A$ is a cut, there exists $r \in \mathcal A$ such that $p < r$. Since $\mathcal A \subset \mathcal B$, we have $r \in \mathcal B$.
    \end{enumerate}
\end{proof}

\thm{}{$\FF^*$ equipped with the order $<$ satisfies the supremum property.}
\begin{proof}
    Let $E \subseteq \FF$ be a nonempty set that is bounded above.  From last time, we know that $\mathcal B = \cup_{\mathcal A \in E} \mathcal A$ is a cut. We claim that $\mathcal B = \sup E$. 

    If $\mathcal A \in E$, then $\mathcal A \subseteq \mathcal B$. And so $\mathcal A \leq \mathcal B$, so $\mathcal B$ is an upper bound for $E$.

    Next, suppose that $\mathcal C \in \FF^*$ is an upper bound of $E$. This means that $\mathcal A \leq \mathcal C$ for every $\mathcal A \in E$, meaning $\mathcal A \subseteq \mathcal C \forall \mathcal A \in E$. So $\mathcal B \subseteq \mathcal C$. As such, $\mathcal B \leq \mathcal C$, so $\mathcal B = \sup E$.
\end{proof}\

\noindent Remark: In none of the results leading up to this theorem did we use that $\FF$ is anything other than an ordered set. This shows that the cut construction of Dedekind works in general for ordered sets and yields $\FF^*$ that satisfies the supremum property. Also, $\{ \mathcal C_p \mid p \in \FF\} \subseteq \FF^*$. 

\subsection{Addition}
Idea: $\FF \cong \{ \mathcal C_p \mid p \in \FF\}$. 

\mlenma{}{Let $\mathcal A, \mathcal B \in \FF^*$. Then $\mathcal C = \{a  + b \mid a \in \mathcal A, b \in \mathcal B\}$ is a cut.}
\begin{proof}
    Claim: $\mathcal A, \mathcal B \neq \emptyset \implies \mathcal C \neq \emptyset$.

    $\mathcal A, \mathcal B$ are cuts, so $\exists M_1, M_2 \in \FF$ such that $a < M_1$ for all $a \in \mathcal A$ and $b < M_2$ for all $b \in \mathcal B$. Then $a + b < M_1 + M_2$ for all $a \in \mathcal A, b \in \mathcal B$, so $a + b < M_1 + M_2$, meaning $M_1 + M_2 \notin \mathcal C$.

    Also, let $c = a + b \in \mathcal C$ for $a \in \mathcal A, b \in \mathcal B$. Let $q < c \implies q-a < b \implies q - a \in \mathcal B$. Hence, $q = a + (q-a) \in \mathcal C$. 

    Thirdly, let $c = a + b \in\mathcal C$ for $a \in \mathcal A, b \in \mathcal B$. Since $\mathcal A, \mathcal B \in \FF^*$, $\exists r_a, r_b$ such that $a < r_a \in \mathcal A$, $b < r_b \in \mathcal B$. Then $c = a + b < r_a + r_b$, so $r_a + r_b \in \mathcal C$.

    As such, $\mathcal C$ is a cut.
\end{proof}
\noindent Before we define addition, we need to define the negative of a cut.


Heuristic: What we want is that $-\mathcal C_1 = \mathcal C_{-1}$. The way we do this is by defining $\mathcal C_{-p} = \{ q \in \FF \mid \exists p > q : p \in -\mathcal C_p^C\}$. This is the same as $\{ q\in \FF \mid \exists p > q : -p \notin \mathcal C_p\}$. 

Now we study $\{ q \in \FF \mid \exists p > q : -p \notin \mathcal C\}$.

\mlenma{}{Let $\mathcal C \in \FF^*$. Then $\{ q \in \FF \mid \exists p > q : -p \notin \mathcal C\}$ is a cut.}

\dfn{Addition}{For $\mathcal A, \mathcal B \in \FF^*$, we define $\mathcal A + \mathcal B = \{ a + b \mid a \in \mathcal A, b \in \mathcal B\}$ and $-\mathcal A = \{ q \in \FF \mid \exists p > q : -p \notin \mathcal A\}$.}

\thm{}{Define $0 = \mathcal C_0 \in \FF^*$. The following hold:
\begin{enumerate}
    \item $\mathcal A, \mathcal B \in \FF^* \implies \mathcal A + \mathcal B \in \FF^*$.
    \item $\mathcal A, \mathcal B \in \FF^* \implies \mathcal A + \mathcal B = \mathcal B + \mathcal A$.
    \item $\mathcal A, \mathcal B, \mathcal C \in \FF^* \implies (\mathcal A + \mathcal B) + \mathcal C = \mathcal A + (\mathcal B + \mathcal C)$.
    \item $\mathcal A \in \FF^* \implies \mathcal A + 0 = \mathcal A$.
    \item $\mathcal A \in \FF^* \implies \mathcal A + (-\mathcal A) = 0$.
\end{enumerate}
}
\begin{proof}
    Easy proof, too lazy to write out.
\end{proof}
Also: $\mathcal A, \mathcal B, \mathcal C \in \FF^*$ and $\mathcal A < \mathcal B \implies \mathcal A + \mathcal C < \mathcal B + \mathcal C$.

Important Remark: The Archimedean property is actually needed for the above theorem in orer to prove the 5th condition. 

\newpage
\subsection{Multiplication}
\mlenma{}{Let $\mathcal A, \mathcal B \in \FF^*$ such that $\mathcal A, \mathcal B > 0$. Then $\mathcal C = \{ p \in \FF \mid p \leq 0\} \cup \{ ab \mid a \in \mathcal A, b \in \mathcal B, a, b > 0\}$ is a cut.}

\mlenma{}{Let $\mathcal A \in \FF^*$ be such that $\mathcal A > 0$. Then $\mathcal C = \{ p \in \FF^* \mid p \leq 0\}\cup \{ 0 < q \in \FF \mid \exists p > q : p^{-1} \notin \mathcal A\}$ is a cut.}

\dfn{Multiplication}{Let $\mathcal A, \mathcal B \in \FF^*$. We define multiplication as:
\begin{enumerate}
    \item If $\mathcal A , \mathcal B > 0$, then $\mathcal A \cdot \mathcal B = \{ ab \mid 0 < a \in \mathcal A, 0 < b \in \mathcal B\}$.
    \item If $\mathcal A = 0$  or $\mathcal B = 0$, then $\mathcal A \cdot \mathcal B = 0$.
    \item If $\mathcal A > 0$ and $\mathcal B < 0$, then $\mathcal A \cdot \mathcal B = -(\mathcal A \cdot (-\mathcal B))$.
    \item If $\mathcal A < 0$ and $\mathcal B > 0$, then $\mathcal A \cdot \mathcal B = -((-\mathcal A) \cdot \mathcal B)$.
    \item If $\mathcal A, \mathcal B < 0$, then $\mathcal A \cdot \mathcal B = (-\mathcal A) \cdot (-\mathcal B)$.
\end{enumerate}
We define multiplication inversion via:
\begin{enumerate}
    \item If $\mathcal A > 0$, then $\mathcal A^{-1} = \{ q \in \FF \mid \exists p > q : p^{-1} \notin \mathcal A\}$.
    \item If $\mathcal A < 0$, then $\mathcal A^{-1} = -(-\mathcal A)^{-1}$.
\end{enumerate}}
\thm{}{Set $1 = \mathcal C_1$. The following hold:
\begin{enumerate}
    \item If $\mathcal A, \mathcal B \in \FF^*$, then $\mathcal A \cdot \mathcal B \in \FF^*$.
    \item If $\mathcal A, \mathcal B \in \FF^*$, then $\mathcal A \cdot \mathcal B = \mathcal B \cdot \mathcal A$.
    \item If $\mathcal A, \mathcal B, \mathcal C \in \FF^*$, then $(\mathcal A \cdot \mathcal B) \cdot \mathcal C = \mathcal A \cdot (\mathcal B \cdot \mathcal C)$.
    \item If $\mathcal A \in \FF^*$, then $\mathcal A \cdot 1 = \mathcal A$.
    \item If $\mathcal A \in \FF^*$, then $\mathcal A \cdot \mathcal A^{-1} = 1$.
\end{enumerate}}

Also if $\mathcal A, \mathcal B \in \FF^*$ and $\mathcal A, \mathcal B > 0$, then $\mathcal A \cdot \mathcal B > 0$. 

\thm{}{If $\mathcal A, \mathcal B, \mathcal C \in \FF^*$, then $\mathcal A \cdot(\mathcal B + \mathcal C) = \mathcal A \cdot \mathcal B + \mathcal A \cdot \mathcal C$.}

We now know that $\FF^*$ is an ordered field.


\section{Robert Reci}

\thm{}{$\QQ$ is the smallest ordered field.}
\begin{proof}
    Let $\FF$ be any ordered field. Let $1 \in \FF$. Let $\iota : \NN \to \FF$, $n \mapsto 1 + \cdots + 1$ $n$ times. Then $\iota(-n) = -\iota(n)$ for $n \in \NN_0$ and $-n \in \ZZ^-$. 

    Then we say $\iota(p/q) = \iota(p)\iota(q)^{-1}$ for $p/q \in \QQ$.
\end{proof}

\cor{Every ordered field is infinite}{$\iota[\QQ] \subseteq \FF$ is infinite.}

\subsubsection{Roots}
Let $\FF$ be a Dedekind complete ordered field, $0 < x \in \FF$, $n \in \NN$. Then $\exists ! y \in \FF$ such that $y > 0$ and $y^n = x$.

\begin{proof}
    $n = 1$ is silly. Assume $n \geq 2$. Let $E = \{ z \in \FF \mid z > 0 \text{ and } z^n < x\}$. Then $E$ is nonempty and bounded above by $x$. Let $y = \sup E$. We claim that $y^n = x$.

    We want to show that $y^n \ngtr > x$ and $y^n \nless < x$. 

    \mlenma{}{In any commutative ring R, $b^n - a^n = (b - a)(b^{n-1} + b^{n-2}a + \cdots + ba^{n-2} + a^{n-1})$.}

    And hence for $0 < a < b$ in $\FF$, we have $0 < b^n - a^n = (b-a)nb^{n-1}$. 

    Suppose $y^n < x$, so $x-y^n > 0$. We define $h = \frac 12 \min\left(1, \frac{x-y^n}{n(y+1)^{n-1}}\right)$. $0 < h < 1$, also $0 < h < \frac{x-y^n}{n(y+1)^{n-1}}$. 

    Then, by the inequality below the lemma, we have \begin{align*}
        0 &< (y+h)^n - y^n \\
        &< hn(y + h)^{n-1} \\
        &< hn(y + 1)^{n-1} \\
        &< {x-y^n},
    \end{align*}
    so $(y+h)^n < x$, which contradicts the definition of $y$ as the supremum.
\end{proof}

\dfn{Ring*}{A ring is a field where actually we don't care about inverses anymore. }

\dfn{Domain}{$R$ is a domain when $xy = 0 \implies x = 0 \wedge y = 0$.}

Let $R$ be a ring. For $(r, s) \in R \times R \setminus \{0\}$, we say $(r, s) \sim (r', s')$ if $rs' = r's$.

The field of fractions, Frac$(R)$ is the set of equivalence classes of $R \times R \setminus \{0\}$ under $\sim$ equipped with the operations $[(r, s)] + [(r', s')] = [(rs' + r's, ss')]$ and $[(r, s)] \cdot [(r', s')] = (rr', ss')$.

We check that $[(r , s)] \cdot [(s, r)] = [(rs, sr)] = [(1, 1)]$.

Let $\FF$ a field, $\FF[x]$ its polynomial ring. Let $\FF(x)$ be the field of fractions of $\FF[x]$. Then $\FF(x) := \text{Frac}(\FF[x])$ is the field of rational functions in $x$ with coefficients in $\FF$.

Given $p, q \in \FF[x]$, say $p/q > 0$ if $p$ and $q$ have the same sign. Say $f, g \in \FF(x)$, that $f > g$ when $f- g > 0$. 

\thm{}{$\FF(x)$ is never Archimedean.}
\begin{proof}
    $x$ is an upper bound for all $n \in \NN$. 
\end{proof}

\nt{If $\FF$ is Archimedean, $|\FF| \leq 2^{\aleph_0}$.}

\thm{}{Let $\lambda$ be an infinite cardinal. Then there is an ordered field of cardinality $\lambda$.}

\cor{}{The Archimedean property is not a first-order property.}

\section{Completeness}
\mlenma{}{Suppose $\FF$ is an ordered field that is not Dedekind complete. Then $\exists$ and infinite $E \subseteq \FF$ such that:
\begin{enumerate}
    \item $E$ bounded above, $\emptyset \neq U(E)$ is open, $\emptyset \neq U(E)^C$ is open.
    \item $a \in U(E)^C$, $b \in U(E) \implies a < b$.
    \item $f: \FF \to \FF$ with $f(x) = \begin{cases} 1 & x \in U(E) \\ 0 & x \in U(E)^C \end{cases}$ is differentiable with $f' = 0$. 
\end{enumerate}}

\thm{Characteristics of Dedekind Completeness}{
    Let $\FF$ be an ordered field. The following are equivalent:
    \begin{enumerate}
        \item $\FF$ is Dedekind complete.
        \item $\FF$ has the intermediate value property: If $f: [a, b] \to \FF$ is continuous and $\min(f(a), f(b)) < c < \max(f(a), f(b))$, then $\exists x \in [a, b]$ such that $f(x) = c$.
        \item $\FF$ satisfies the mean value property: If $f: [a, b] \to \FF$ is continuous and differentiable on $(a, b)$, then $\exists x \in (a, b)$ such that $f'(x) = \frac{f(b) - f(a)}{b-a}$.
        \item $\FF$ satisfies Cauchy mean value property: If $f,g: [a, b] \to \FF$ are both continuous and differentiable on $(a, b)$, then $\exists x \in (a, b)$ such that $\frac{f'(x)}{g'(x)} = \frac{f(b) - f(a)}{g(b) - g(a)}$.
        \item $\FF$ satisfies the extreme value property: If $f: [a, b] \to \FF$ is continuous, then $f$ attains a maximum and minimum on $[a, b]$.
    \end{enumerate}
}
\begin{proof}
    $1 \implies 2$: Let $f: [a, b] \to \FF$ and continuous. WLOG, assume $f(a) < c < f(b)$. Define $E = \{x \in [a,b] \mid f(x) < c\}$. $E$ is nonempty and bounded above by $b$. Let $x = \sup E$. We claim that $f(x) = c$. Since $f$ is continuous, $\exists \kappa > 0$ such that $f(t) < c \, \forall t \in [a, a + \kappa]$ and $f(t) > c \, \forall t \in [b - \kappa, b]$. So, $a + \frac \kappa 2 < x < b - \frac \kappa 2$. 

    Suppose BWOC $f(x) < c$. Again by continuity, $\exists \delta > 0$ such that $f(t) < c$ for all $t \in B(x, \delta) \subseteq [a, b]$. Then $x + \frac \delta 2 \in E$, contradiction. 

    Then suppose BWOC $f(x) > c$. Again, $\exists \delta > 0$ such that $f(t) > c$ for all $t \in B(x, \delta) \subseteq [a, b]$. Then $\exists z \in E$ such that $x - \frac \delta 2 < z  \leq x$ and $f(z) < c$. But then $c < f(z) < c$, contradiction. 

    So $f(x) = c$ by trichotomy.

    $2 \implies 1$: We'll show $\neg 1 \implies \neg 2$. Suppose $\FF$ is not Dedekind complete. Then we can let $f : \FF \to \FF$ be the strange function from the lemma, and we can pick $a < b$ with $a \in U(E)^C$ and $b \in U(E)$. Then $f$ is continuous on $[a, b]$, $f(a) - < 1 = f(b)$, but there is not $x \in [a,b]$ with $f(x) = \frac 12$, by construction.

    $1 \implies 5$: First we claim that if $\FF$ is Dedekind and $f: [a, b]\to \FF$ is continuous, then $f([a, b]) \subseteq \FF$ is a bounded set. We prove the claim.

    Consider $E = \{x \in [a, b] \mid f([a, x]) \text{ is bounded}\}$. $a \in E$ and $E$ is bounded, so we can let $s = \sup E$. Next note that by continuity, if $[c,d] \subseteq [a, b]$ such that $f([c, d])$ is bounded, then $\exists \delta  > 0$ such that $f([a,b] \cap [c - \delta, d + \delta])$ is bounded. Using this, deduce in turn that $a < s$, $s = \max E$, and $s = b$. 

    So now suppose $\FF$ is Dedekind complete and let $f: [a,b] \to \FF$ be continuous. The claim establishes that $f([a, b]) \subseteq \FF$ is a bounded set, so we can let$
    \begin{cases} \mu = \inf f([a, b]) \\ \lambda = \sup f([a, b]) \end{cases}$. Suppose BWOC that $f(x) < \lambda$ for all $x \in [a, b]$. Then teh function $g: [a,b] \to \FF$ defined by $g(x) = \frac{1}{\lambda - f(x)}$ is continuous and positive. So by the claim, there is $k > 0$ such that $g(x) \leq k$ for all $x \in [a, b]$. But then 
    \begin{align*}
        \frac{1}{\lambda - f(x)} \leq k \implies \frac 1k \leq \lambda - f(x) \implies f(x) \leq \lambda - \frac 1k,
    \end{align*}
    for all $x \in [a,b]$. But this contradicts the definition of $\lambda$, as we just found a better upper bound.

    Therefore, there does exists $M \in [a,b]$ such that $f(M) = \lambda$, which is $\max f([a,b])$. 

    The min follows from a similar argument. 

    $5 \implies 4$: Let $f, g: [a,b] \to \FF$ be continuous and differentiable on $(a, b)$. Let $h:[a,b]\to \FF$ via $h(x) = f(x)(g(b) - g(a)) - g(x)(f(b) - f(a))$. It suffices to show $\exists x \in (a, b)$ such that $h'(x) = 0$.

    By construction, $h(a) = h(b)$. If $h(x) = h(a)$ for all $x \in [a,b]$, then $h' = 0$ and we're done. Suppose then that $h$ is not constant. Then EVT shows that $f$ attains its maximal/minimum values, and at least one must occur at the point $x \in (a, b)$, therefore $h'(x) = 0$.

    $4 \implies 3$: Let $g(x) = x$. Done.

    $3 \implies 1$. We'll show $\neg 1 \implies \neg 3$. Suppose $\FF$ is not Dedekind complete. Then we can let $f: \FF \to \FF$ be the function from the lemma, and we can pick $a < b$ with $a \in U(E)^C$ and $b \in U(E)$. Then consider the restriction $f:[a,b] \to \FF$. Then $1 = 1 - 0 = f(b) - f(a)$. Then, $f'(x) (b-a) = 0 \cdot(b-a) = 0$ for all $x \in \FF$. $0 \neq 1$ so $\neg 3$ as desired.
\end{proof}

\chapter{$\RR, \CC, \bar \RR$}
\thm{}{$\RR$ is uncountable.}
\begin{proof}
    $\QQ \subseteq \RR$, so $\RR$ is definitely infinite. Suppose BWOC that there was a bijection $f: \NN \to \RR$. Set $I_0 = [f(0) + 1, f(0) + 2]$ and not that $f(0) \notin I_0$. Suppose we are given closed, nested, non-singleton intervals $I_n \subseteq I_{n-1} \subseteq \cdots \subseteq I_0$ such that $f(k) \notin I_k$ for $0 \leq k \leq n$. If $f(n+1) \notin I_n$, then set $I_{n+1} = I_n$. Otherwise, set $I_{n+1}$ to some non-singleton closed interval contained in $I_n$ such that $f(n+1) \notin I_{n+1}$.

    Since $\RR$ is Dedekind complete, we have that $\bigcap_{n=0}^\infty I_n \neq \emptyset$. So, there is an $x$ such that $x \in I_n$ for all $n \in \NN$. But then $x \neq f(n)$ for all $n \in \NN$, contradiction since $f$ is a bijection.
\end{proof}
\nt{Upshot: Most of $\RR$ is transcendental over $\QQ$.}

\section{Extended Reals: $\bar \RR$}
\dfn{Extended Reals}{$\bar \RR = \RR \cup \{-\infty, \infty\}$. We endow $\bar \RR$ with the following order: We write $x< y$ for $x, y\in \bar \RR$ if:
\begin{enumerate}
    \item $x, y \in \RR$ and $x < y$.
    \item $x = -\infty$ and $y \in \bar \RR \setminus \{-\infty\}$.
    \item $x \in \bar\RR \setminus \{\infty\}$ and $y = \infty$.
\end{enumerate}}

\noindent Facts:
\begin{itemize}
    \item $(\bar \RR, <)$ is an ordered set that satisfies the supremum property.
    \item All sets in $\bar \RR$ are bounded above.
    \item All sets in $\bar \RR$ admit a sup/inf, i.e.
    \begin{itemize}
        \item $\sup: \mathcal P(\bar \RR) \to \bar \RR$.
        \item $\inf: \mathcal P(\bar \RR) \to \bar \RR$.
    \end{itemize}
    Note: $\sup \emptyset = -\infty$ and $\inf \emptyset = \infty$. Also, $A \subseteq B \subseteq \bar \RR$ implies $\sup A \leq \sup B$ and $\inf A \geq \inf B$. And if $E \neq \emptyset$, then $\inf E \leq \sup E$. 
\end{itemize}
\nt{$\bar \RR$ isn't an OF because if it were, then it would be Dedekind complete and then there would exists an ordered field isomorphism $f: \RR \to \RR$ such that $f(x) = \infty$ for some $x \in \RR$. but then $f(x+1) = f(x) + f(1) = \infty + 1 = \infty$, which is not a true statement. }

\dfn{}{We endow $\bar \RR$ with the following ``algebra.''
\begin{enumerate}
    \item If $x \in \RR$, we set $x + \infty = \infty + x = \infty$.
    \item If $x \in \RR$, we set $x + (-\infty) = (-\infty) + x = -\infty$.
    \item $\infty + \infty = \infty$.
    \item $-\infty + (-\infty) = -\infty$.
    \item If $0 < x \in \bar \RR$, we set $x \cdot \infty = \infty \cdot x = \infty$.
    \item If $0 < x \in \bar \RR$, we set $x \cdot (-\infty) = (-\infty) \cdot x = -\infty$.
    \item If $0 > x \in \bar \RR$, we set $x \cdot \infty = \infty \cdot x = -\infty$.
    \item If $0 > x \in \bar \RR$, we set $x \cdot (-\infty) = (-\infty) \cdot x = \infty$.
    \item If $x \in \RR$, we set $\frac{x}{\infty} = \frac{x}{-\infty} = 0$.
    \item $\infty^{-1} = 0 = (-\infty)^{-1}$.
    \item If $0 < x \in \bar \RR$, we set $\frac x0 = \infty$.
    \item If $0 > x \in \bar \RR$, we set $\frac x0 = -\infty$.
\end{enumerate}}
\noindent Forbidden/undefined: $\infty + (-\infty)$, $\infty \cdot 0$, $\frac 00$, $\frac{\pm \infty}{\pm \infty}$, $\frac{\pm \infty}{\mp \infty}$.

\subsection{Sequences in $\bar \RR$}
\dfn{Sequence}{A sequence in $\bar \RR$ is $\{x_n \}^\infty_{n = \ell} \subseteq \bar \RR$ for $\ell \in \ZZ$.}

\noindent In turn, we define new sequences $\{a_N\}^\infty_{N = \ell}, \{b_N\}^\infty_{N = \ell}  \subseteq \bar \RR$:
\begin{itemize}
    \item $a_N = \inf \{x_n \mid n \geq N\}$.
    \item $b_N = \sup \{x_n \mid n \geq N\}$.
\end{itemize}
We then set $\liminf_{n \to \infty} x_n = \sup_{N \geq \ell} \inf_{n \geq N} x_n = \sup_{N \geq \ell} a_N$ and $\limsup_{n \to \infty} x_n = \inf_{N \geq \ell} \sup_{n \geq N} x_n = \inf_{N \geq \ell} b_N$.


\ex{}{Let $x_n = \begin{cases} (-1)^n & n \equiv 0 \mod 2 \\ n & n \equiv 1 \mod 2\end{cases}$. Then, $\limsup_{n \to \infty} x_n = \infty$ and $\liminf_{n \to \infty} x_n = 1$.}

\mprop{}{Let $\{x_n\}_{n = \ell}^\infty \subseteq \bar \RR$. Then $\liminf_{n \to \infty} x_n \leq \limsup_{n \to \infty} x_n$.}
\begin{proof}
    Let $M,N \geq \ell$ and $K = \max(M,N)$. Then, $\inf_{n > N} x_n \leq \inf_{n > K} x_n \leq \sup_{n \geq K} x_n \leq \sup_{n \geq M} x_n$.

    Thus, $\liminf_{n \to \infty} x_n  = \sup_{N \geq \ell} \inf_{n \geq N} x_n \leq \sup_{n \geq M} x_n$ for all $M \geq \ell$. So, $\liminf_{n \to \infty} x_n \leq \limsup_{n \to \infty} x_n$.
\end{proof}


\mprop{}{Let $a_n, b_n \in \bar \RR$ and suppose $\exists K \geq \ell$ such that $a_n \leq b_n$ for all $n \geq K$. Then, $\liminf_{n \to \infty} a_n \leq \liminf_{n\to \infty} b_n$ and $\limsup_{n \to \infty} a_n \leq \limsup_{n \to \infty} b_n$.}
\begin{proof}
    We can claim that if $k \geq K$, then 
    \begin{align*}
        \inf \{ a_n \mid n \geq k\} \leq \inf \{ b_n \mid n \geq k\} \\ 
        \sup \{ b_n \mid n \geq k\} \leq \sup \{ a_n \mid n \geq k\}.
    \end{align*}
    Indeed, if $\exists k \geq K$ such that $\inf \{ a_n \mid n \geq k\} > \inf \{ b_n \mid n \geq k\}$, then $\exists m \geq k$ such that $b_m < \inf\{a_n \mid n \geq k\} \leq a_m \leq b_m$, contradiction. Ditto for $\sup$. 

    Now define for $N \geq \ell$, $C_N = \inf_{n \geq N} a_n$, $D_N = \inf_{n \geq N} b_N$, $E_N = \sup_{n \geq N} a_n$, and $F_N = \sup_{n \geq N} b_n$. 

    The above claims show that $N \geq K$ then $C_N \leq D_N$ and $E_N \leq F_N$. Then we iterate to learn:
    \begin{align*}
        \liminf_{n \to \infty} a_n = \sup_{N \geq \ell} C_N \leq \sup_{N \geq \ell} D_N = \liminf_{n \to \infty} b_n \\
        \limsup_{n \to \infty} a_n = \inf_{N \geq \ell} E_N \leq \inf_{N \geq \ell} F_N = \limsup_{n \to \infty} b_n.
    \end{align*}
\end{proof}

\thm{}{Suppose $a_n, b_n \in \bar \RR$. The following hold:
\begin{enumerate}
    \item If $\limsup_{n \to \infty} a_n < x \in \bar \RR$, then $\exists N \geq \ell$ such that $a_n < x$ for all $n \geq N$.
    \item If $\liminf_{n \to \infty} a_n > x \in \bar \RR$, then $\exists N \geq \ell$ such that $a_n > x$ for all $n \geq N$.
    \item $\liminf_{n \to \infty} a_n = - \limsup_{n \to \infty} -a_n$.
    \item $\limsup_{n \to \infty} a_n = - \liminf_{n \to \infty} -a_n$.
    \item $\limsup_{n \to \infty} a_n + b_n \leq \limsup_{n \to \infty} a_n + \limsup_{n \to \infty} b_n$, provided that all arithmetic operations are well-defined. 
    \item $\liminf_{n \to \infty} a_n + \liminf_{n \to \infty} b_n \leq \liminf_{n \to \infty} a_n + b_n$, provided that all arithmetic operations are well-defined.
\end{enumerate}}

\begin{proof}
    \begin{enumerate}
        \item Suppose $\limsup_{n\to\infty}a_n = \inf_{N \geq \ell} \sup_{n\geq N} a_n < x$. This implies that $\exists N \geq \ell$ such that $\sup_{n \geq N} a_n < x$, meaning $a_n < x$ for all $n \geq N$.
        \item Similar as above.
        \item For any $\emptyset \neq X \subseteq \FF$, we have that $-\sup(-X) = \inf X$ and $-\inf(-X) = \sup X$. So the result follows.
        \item Same as above. 
        \item We break into cases:
        \begin{enumerate}
            \item $\limsup a_n = \infty$ or $\limsup b_n = \infty$. Then $\limsup a_n + b_n = \infty  \geq \limsup a_n + \limsup b_n$.
            \item Suppose either $\limsup a_n = -\infty$ or $\limsup b_n = -\infty$. WLOG consider the first option. Since $\limsup b_n < \infty$, then there eixsts $N_1 \geq \ell$ and $K \geq \RR$ such that $b_n < K$ for $n \geq N_1$. Now let $m \in \NN$ and note that $-\infty < -m - K$. We can use the first result of the theorem to pick $N_2 \geq \ell$ such that $n \geq N_2 \implies a_n < -m -K$. Then, if $n \geq \max(N_1, N_2)$, we have $a_n + b_n < -m$, so $\limsup a_n + b_n = -\infty \leq \limsup a_n + \limsup b_n$.
            \item $\limsup a_n, \limsup b_n \in \RR$. Let $\epsilon > 0$, then $\exists N_1, N_2 \geq \ell$ such that $n \geq N_1 \implies a_n < \limsup a_n + \frac{\epsilon}{2}$ and $n \geq N_2 \implies b_n < \limsup b_n + \frac{\epsilon}{2}$. Then, $n \geq \max(N_1, N_2) \implies a_n + b_n < \limsup a_n + \limsup b_n + \epsilon$, so $\limsup a_n + b_n \leq \limsup a_n + \limsup b_n + \epsilon$ for all $\epsilon$.
        \end{enumerate}
        \item Same as above.
    \end{enumerate}
\end{proof}


\mlenma{}{Let $x_n \subseteq\RR$. The following are equivalent for $x \in \RR$:
\begin{enumerate}
    \item $x_n \to x$ as $n \to \infty$.
    \item $\liminf_{n \to \infty} x_n = \limsup_{n \to \infty} x_n = x$.
\end{enumerate}}
\begin{proof}
    Let $\epsilon > 0$. Then $\exists N \geq \ell$ such that $n \geq N \implies x - \epsilon < x_n < x + \epsilon$. Thus, $x - \epsilon \leq \liminf_{n \to \infty} x_n \leq \limsup_{n \to \infty} x_n \leq x + \epsilon$ for all $\epsilon > 0$. This implies that $\liminf_{n \to \infty} x_n = \limsup_{n \to \infty} x_n = x$.

    Now let $\epsilon > 0$. Then by the previous theorem, there exists $N_1 , N_2 \geq \ell$ such that $\begin{cases}
        x - \epsilon < x_n & n \geq N_1 \\
        x_n < x + \epsilon & n \geq N_2
    \end{cases}$. Thus, $n \geq \max(N_1, N_2) \implies x - \epsilon < x_n < x + \epsilon$, so $x_n \to x$ as $n \to \infty$.
\end{proof}

\dfn{}{Let $x_n \in \bar \RR$ and $x \in \bar \RR$. We say that $x_n \to x$ as $n \to \infty$ if $\liminf_{n \to \infty} x_n = \limsup_{n \to \infty} x$.}
\noindent Remarks:
\begin{enumerate}
    \item The lemma shows this extends the notion of convergence in $\RR$.
    \item Limits are unique, when they exist.
\end{enumerate}

\ex{}{
    \begin{enumerate}
        \item $\lim_{n \to \infty} n = \infty$ ($n \to \infty$ as $n \to \infty$).
        \item Version of squeeze lemma
        \item TFAE:
        \begin{itemize}
            \item $x_n \to \infty$ as $n \to \infty$. 
            \item $\liminf_{n \to \infty} x_n = \infty$.
            \item $\forall M \in \NN$, there exists $N \geq \ell$ such that $n \geq N \implies M \leq x_N$.
        \end{itemize}
    \end{enumerate}
}

\chapter{Metric Spaces}
\dfn{Metric}{Let $X$ be a nonempty set. A metric on $X$ is a function $d: X \times X \to \RR$ such that:
\begin{enumerate}
    \item $d(x,y) \geq 0$ for all $x,y \in X$, and $d(x,y) = 0 \iff x = y$.
    \item $d(x,y) = d(y,x)$ for all $x,y \in X$.
    \item $d(x,y) \leq d(x,z) + d(z,y)$ for all $x,y,z \in X$.
\end{enumerate}}
\dfn{}{A metric space is $(X, d)$ for $X \neq \emptyset$ and $d$ a metric on $X$.}

\ex{}{\begin{enumerate}
    \item $\RR$ with $d(x, y) = |x - y|$.
    \item $\CC$ with $d(x, y) = |x - y|$.
    \item (Discrete Metric) Let $X \neq \emptyset$ be any set. Then $d: X \times X \to \{0, 1\}$ defined by $d(x, y) = \begin{cases} 0 & x = y \\ 1 & x \neq y \end{cases}$ is a metric on $X$.
    \item Let $V$ be a normed metric space with norm $\Vert \cdot \Vert$. Then $d(x, y) = \Vert x - y \Vert$ is a metric on $V$.
    \item Suppose $(Y, d)$ is a metric space and suppose $f: X \to Y$ is an injection where $X \neq \emptyset$ is a set. Then $\sigma: X \times X \to \RR$ defined by $\sigma(x, y) = d(f(x), f(y))$ is a metric on $X$.
    \begin{proof}
        We need to show that $\sigma$ satisfies the three properties of a metric. 
        \begin{enumerate}
            \item $\sigma(x, y) \geq 0$ because $d \geq 0$ and $\sigma(x, y) = 0 \iff d(f(x), f(y)) = 0 \iff f(x) = f(y) \iff x = y$.
            \item The other two are very trivial.
        \end{enumerate}
    \end{proof}
    \item Let $Y$ be a metric space and $\emptyset \neq X \subseteq Y$. Then $d: X \times X \to \RR$ defined by $d(x, y) = d_Y(x, y)$ is a metric on $X$.
    \item Consider $f:(0, \infty) \to \RR$ and $g: (0, \infty) \to \RR$ with $f(x) = \log x$ and $g(x) = \frac 1x$. Then $d_f(x, y) = \left| \log \frac xy \right|$ and $d_g(x, y) = \left| \frac 1x - \frac 1y \right| = \frac{|x - y|}{|x||y|}$ are metrics on $(0, \infty)$.
    \item Let $V,W$ be finite dimensional vector spaces over $\FF \in \{ \RR, \CC\}$. Let $L(V, W) = \{T: V \to W: T \text{ linear}\}$. Then define $\text{rk}(T) = \dim \ran T$ for $T \in L(V, W)$. Note that $\ran (T + S) = \{Tx + Sx \mid  x \in \FF\} \subseteq \{Tx + Sy \mid x, y \in \FF\} = \ran T + \ran S$. Then, $\text{rk}(T + S) \leq \text{rk}(T) + \text{rk}(S)$.
    
    Define $d(T, S)  = \text{rk}(T-S) \in \NN \subseteq [0, \infty]$. 
    \begin{itemize}
        \item $d(T, S) = 0 \iff \text{rk}(T-S) = 0 \iff T -S = 0$.
        \item Has symmetry.
        \item Triangle inequality: $d(T-S) = \text{rk}(T - R + R - S) \leq \text{rk}(T - R) + \text{rk}(R - S) = d(T, R) + d(R, S)$.
    \end{itemize}
    \item Let $f: \bar RR \to [-1, 1]$ via $f(x) = \begin{cases} 1 & x = \infty \\ -1 & x = -\infty \\ \frac{x}{\sqrt{1 + x^2}} & x \in \RR \end{cases}$. Then $d(x, y) = |f(x) - f(y)|$ is a metric on $\bar \RR$.
\end{enumerate}}

\dfn{}{Let $X$ be a metric space.
\begin{enumerate}
    \item For $x \in X$ and $r \geq 0$, we define $B(x, r) = \{y \in X \mid d(x, y) < r \}$. And $B[x, r] = \{y \in X \mid d(x, y) \leq r \}$.
    \item A set $E \subseteq XX$ is bounded if $\exists (R \geq 0)$ such that $E \subseteq B(x, R)$ for some $x \in X$.
    \item Let $Y$ be any set and $f:  Y \to X$. We say $f$ is a bounded function if $f(Y) \subseteq X$ is bounded. We write $\mathcal B(Y; X) = \{g : Y \to X \mid g \text{ is bounded}\}$.
\end{enumerate}

}

\ex{}{
    \begin{enumerate}
        \item $f : \RR \to \CC$ via $f(t) = e^{it} \implies f(t) = 1 \implies f(\RR) \subseteq B[0, 1]$ is bounded. So, $f \in \mathcal B(\RR; \CC)$.
        \item $f:(0, \infty) \to \RR$ via $f(t) = \frac{\log t}{\sqrt{1 + (\log t)^2}}$. So, $f \in \mathcal B((0, \infty); \RR)$.
        \item Let $X$ be a metric space and $Y$ a nonempty set. Consider $\mathcal B(X ; Y)$. If $f \in \mathcal B(X; Y)$, then $\exists y \in Y$ and $R \geq 0$ such that $d(f(x), y) \leq R$ for all $x$. Thus, $\sup_{x \in X} d(f(x), y) := \sup \{d(f(x), y) \mid x \in X\} \in [0, R]$.
        
        Similarly, if $f, g \in \mathcal B(X; Y)$, then exists $R \geq 0$ and $y_1, y_2 \in Y$ such that $d(f(x), y_1) \leq R$ and $d(g(x), y_2) \leq R$ for all $x \in X$. Then, $d(f(x), g(x)) \leq d(f(x), y_1) + d(y_1, y_2) + d(y_2, g(x)) \leq 2R + d(y_1, y_2) < \infty$ for all $x \in X$. So, $\sup_{x \in X} d(f(x), g(x)) < \infty$. We now define
        \begin{align*}
            d: \mathcal B(X; Y) \times \mathcal B(X; Y) &\to [0, \infty) \\
            (f, g) &\mapsto \sup_{x \in X} d(f(x), g(x)).
        \end{align*}
        \begin{proof}
            Consider the properties of a metric:
            \begin{itemize}
                \item $d(f, g) = 0 \iff \sup_{x \in X} d(f(x), g(x)) = 0 \iff d(f(x), g(x)) = 0 \iff f(x) = g(x)$ for all $x \in X \iff f = g$.
                \item Symmetry is trivial.
                \item Let $f, g, h\in \mathcal B(X; Y)$. Then, $d(f, h) = \sup_{x \in X} d(f(x), h(x)) \leq \sup_{x \in X} d(f(x), g(x)) + d(g(x), h(x)) \leq d(f, g) + d(g, h)$.
            \end{itemize}
        \end{proof}
    \end{enumerate}
}


\dfn{}{Let $X$ and $Y$ be metric spaces:
\begin{enumerate}
    \item A map $f : X \to Y$ is an isometric embedding if $d_Y(f(x), f(y)) = d_X(x, y)$ for all $x, y \in X$. Note, such an $f$ is injective.
    \item $f$ is an isometry if it's an isometric embedding and surjective.
    \item $X$ and $Y$ are isometric if there exists an isometry $f: X \to Y$.
\end{enumerate}}

\ex{}{
    \begin{enumerate}
        \item Consider $\RR^n$ with $|\cdot| = \Vert \cdot \Vert_2$, that is, 2-norm. 
        \item Recall $O(n) = \{\mathcal M \in \RR^{n \times n} \mid \mathcal M^T \mathcal M = I \}$ and $R \in O(n) \implies |Rx| = |x|$. 
        
        Let $a \in \RR^n$, $R \in O(n)$, and set $f: \RR^n \to \RR^n$ via $f(x) = a + Rx$. Then, 
        \begin{align*}
            |f(x) - f(y)| = |a + Rx - (a + Ry)| = |Rx - Ry| = |R(x - y)|.
        \end{align*}
        Also, $y = f(x) = a + Rx \iff y - a = Rx$.  So, $f$ is an isometry.
        \item Consider $x \mapsto ix \in \CC$ for $x \in \RR$. This is an isometric embedding but obviously not an isometry for it is not surjective.
    \end{enumerate}
}
The next example is so important that we call it a theorem. Recall $\mathcal B(X) = \mathcal B(X; \RR)$ for $X \neq \emptyset$ is a set. Note that if $V$ is a normed vector space, then $\mathcal B(X; V)$ is too: $\Vert f \Vert_{\mathcal B} = \sup_{x \in X}\Vert f(x) \Vert_V$ is a norm (exercise) and $d_{\mathcal B}(f, g) = \Vert f -g \Vert_{\mathcal B}$. 

\thm{}{
    Let $X$ be a metric space and fix an arbitrary element $a \in X$. For $x \in X$, we'll define $\varphi_x : X \to \RR$ via $\varphi_x(y) = d(x, y) - d(y, a)$. The following hold:
    \begin{enumerate}
        \item $\varphi_x \in \mathcal B(X)$ for all $x \in X$.
        \item Define $\Phi : X \to \mathcal B(X)$ via $\Phi(x) = \varphi_x$. Then, $\Phi$ is an isometric embedding.
    \end{enumerate}
}
\begin{proof}
    First note, $|\varphi_x(y)| = |d(x, y) - d(y, a)| \leq d(x, a)$ by the triangle inequality. So, $\Vert \varphi_X \Vert_{\mathcal B} = \sup_{y \in X} |\varphi_x(y)| \leq d(x, a) < \infty$. This shows the first result.

    Next, fix $x, z \in X$ and consider $\varphi_x(y) - \varphi_z(y) = d(x, y) - d(y, a) - d(z, y) + d(y, a)$. So,
    \begin{align*}
        |\varphi_x(y) - \varphi_z(y)| = |d(x, y) - d(y, z)| \leq d(x, z).
    \end{align*}
    Thus, $d_{\mathcal B}(\varphi_x, \varphi_y) = \Vert \varphi_x - \varphi_y \Vert_{\mathcal B} = \sup_{y \in X} |\varphi_x(y) - \varphi_z(y)| \leq d(x, z)$.

    On the other hand, $|\varphi_x(z) - \varphi_z(z)| = |d(x, z) - \cancelto{0}{d(z, z)}| = d(x, z)$. So, $d_{\mathcal B}(\varphi_x, \varphi_z) = d(x, z)$. 
\end{proof}

\chapter{Basic Metric Space Topology}


FILL IN LATER 


\mprop{}{
    Let $Y_1, \ldots, Y_n$ be metric spaces and consider $Y = \prod_{i=1}^n Y_i$, endowed with a $p$-metric from Homework 3. That is,
    \begin{align*}
        d_p(x, y) = \begin{cases}
        \left( \sum_{i=1}^n d^p_{Y_i}(x_i, y_i)\right)^{1/p} & 1 \leq p < \infty \\
        \max_{1 \leq i \leq n} d^p_{Y_i}(x_i, y_i) & p = \infty
        \end{cases}.
    \end{align*}
    Suppose $\{y_k\}_{k = \ell}^\infty \subseteq Y$ is given by $y_k = (y_{k, 1}, \ldots, y_{k, n})$. The following hold:
    \begin{enumerate}
        \item Let $y = (y_1, \ldots, y_n) \in Y$. Then $y_k \to y$ in $Y$ as $n \to \infty \iff y_{k, i} \to y_i$ in $Y_i$ as $k \to \infty$ for all $1 \leq i \leq n$. 
        \item $\{y_k\}_{k = \ell}^\infty$ is Cauchy in $Y$ if and only if $\{y_{k, i}\}_{k = \ell}^\infty$ is Cauchy in $Y_i$ for all $1 \leq i \leq n$.
    \end{enumerate}
}
\begin{proof}
    We'll only prove 1. as 2. is very similar. Suppose $y_k \to y$ as $k \to \infty$. Note that for $ 1 \leq i \leq n$, $d_i(y_{k, i}, y_i) \leq d_Y(y_k, y)$. Thus, for $\epsilon > 0$, we pick $K \geq \ell$ such that if $k \geq K$, then $d_Y(y_k, y) \leq \epsilon$. But then $k \geq K \implies d_i(y_{k ,i}) \leq d_Y(y_k, y) \leq \epsilon$ for all $1 \leq i \leq n$, meaning $y_{k ,i} \to y_i$ as $k \to \infty$ for $1 \leq i \leq n$. 

    Now suppose $y_{k ,i} \to y_i$ as $k \to \infty$ for all $1 \leq i \leq n$. Let $\epsilon > 0$ and pick $K_i \geq \ell$ such that $k \geq K_i \implies d_i(y_{k ,i}, y_i) < \frac{\epsilon}{n^{1/p}}$. Let $K = \max K_i \geq \ell$, and note $k \geq K \implies d_i(y_{k ,i}, y_i) < \frac{\epsilon}{n^{1/p}}$ for all $1 \leq i \leq n$. This means 
    \begin{align*}
        \begin{cases}
            \left(\sum_{i=1}^n d_i^p(y_{k, i}, y_i)\right)^{1/p}  \leq \left(\sum_{i=1}^n \frac{\epsilon^p}{n}\right)^{1/p} = \epsilon & 1 \leq p < \infty \\
            \max_i d_i(y_{k, i}, y_i)  < \epsilon & p = \infty
        \end{cases}
    \end{align*}
    So, $y_k \to y$ as $k \to \infty$.
\end{proof}

\dfn{}{Let $X \neq \emptyset$ be a set and $d_1, d_2$ be metrics on $X$. We say $d_1$ and $d_2$ are equivalent if $\exists c_1, c_2 >0$ such that $c_1d_1(x, y) \leq d_2(x, y) \leq c_2d_1(x, y)$ for all $x, y \in X$.}
The point is that equivalent metrics give the same notions of convergence, Cauchyness, and boundedness. 
\newpage
\ex{Equivalent Norms}{
    \begin{enumerate}
        \item All norms on $\FF^n$ are equivalent. 
        \item From recitation, $\Vert \cdot \Vert_p$ are all equivalent on $\FF^n$ for $1 \leq p \leq \infty$. 
        \item Let $Y_1, \ldots, Y_n$ be metric spaces and form $Y = \prod_{i=1}^n Y_i$. Then
        \begin{align*}
            d_p(x, y) = \Vert (d_1(x, y), \ldots, d_n(x, y)) \Vert_p \asymp \Vert (d_1(x, y), \ldots, d_n(x, y)) \Vert_q = d_q(x, y)
        \end{align*}
        Therefore, $d_p \asymp d_q$ in $Y$. 
        
        Note: This does not mean all metrics on $Y$ are equivalent. 
    \end{enumerate}
}
\ex{}{
    Let $V_1, \ldots, V_n, W$ be normed vector sapces over $\FF$. We define $\mathcal L(V_1, \ldots, V_n; W)$ is the set of $\{ T \in L(V_1, \ldots, V_n ; W) \mid \Vert T \Vert_{\mathcal L}\ < \infty\}$ where $\Vert T \Vert_{\mathcal L} := \sup \{\Vert T(v_1, \ldots, v_n) \Vert_W \mid v_i \in V_i : \Vert v_i \Vert_{V_i} < 1\} \in [0, \infty]$. Facts:
    \begin{enumerate}
        \item This is indeed a norm.
        \item $T \in \mathcal L \iff \Vert T(v_1, \ldots, v_n) \Vert_W \leq c\prod_{i=1}^n \Vert v_i \Vert_{V_i}$ for all $v_i \in V_i$ for some $0 \leq c < \infty$. $c = \Vert T \Vert_{\mathcal L}$ is the best constant. 
    \end{enumerate}
}

\thm{Algebra of Sequences}{
    Let $V_1, \ldots, V_n, W$ be normed vector spaces over a common field $\FF$. The following hold:
    \begin{enumerate}
        \item Let $\{v_{k ,i}\}^\infty_{k = \ell} \subseteq V_i$ for $1 \leq i \leq n$ be such that $v_{k, i} \to v_i$ in $V_i$ as $k \to \infty$. Let $\{T_k\}_{k = \ell}^\infty \subseteq \mathcal L(V_1, \ldots, V_n; W)$ be such that $T_k \to T$ as $k \to \infty$. Then $T_k(v_{k, 1}, \ldots, v_{k, n}) \to T(v_1, \ldots, v_n)$ in $W$ as $k \to \infty$.
        \item If $\{u_k\}, \{v_k\} \subseteq V_1$ are such that $u_k \to u$, $v_k \to v$ then $u_k + v_k \to u + v$ as $k \to \infty$. 
    \end{enumerate}
}
\begin{proof}
    We'll only do $1$ because $2$ is easy. We start with $n=2$ for simplicity. Suppose $\{x_k\} \subseteq V_1$, $\{y_k\} \subseteq V_w$ such taht $x_k \to x$ and $y_k \to y$ as $k \to \infty$. Then let $\sup_{k \geq \ell} \max \{\Vert x_k \Vert_{V_1}, \Vert y_k \Vert_{V_2}$, $\Vert T_k \Vert_{\mathcal L} \} = M < \infty$. Then,
    \begin{align*}
        T_k(x_k, y_k) - T(x, y)&= T_k(x_k, y_k - y) + T_k(x_k, y) - T(x, y) \\ 
        T_k(x_k, y_k + y) + T(x_k - x, y) + T_k(x, y) - T(x, y).
    \end{align*}
    This shows that 
    \begin{align*}
        \Vert T_k(x_k, y_k) - T(x, y) \Vert_W &\leq \Vert T_k \Vert_{\mathcal L} \Vert x_k \Vert_{V_1} \Vert y - y_k \Vert_{V_2}  + \Vert T_k \Vert_{\mathcal L} \Vert x - x_k \Vert_{V_1} \Vert y_k \Vert_{V_2}  + \Vert T - T_k \Vert_{\mathcal L} \Vert x_k \Vert_{V_1} \Vert y_k \Vert_{V_2}  \\
        \leq M^2 \Vert y - y_k \Vert_{V_2} + M^2 \Vert x - x_k \Vert_{V_1}  + M^2 \Vert T - T_k \Vert_{\mathcal L} \to 0
    \end{align*}
    as $k \to \infty$. 
\end{proof}
\dfn{}{\begin{enumerate}
    \item We say a metric space $X$ is complete if every Cauchy sequence in $X$ is convergent in $X$.
    \item We say a normed vector space is Banach if it's complete.
    \item We say an inner product space is a Hilbert space if it's Banach.
\end{enumerate}}
\ex{}{
    \begin{enumerate}
        \item $(\RR, | \cdot|)$ is complete.
        \item $X = \prod_{i=1}^n X_i$ with $p$-metric is complete if and only if each $X_i$ is complete. In particular, $(\RR^n, \Vert\cdot\Vert)$ is complete. 
        \item $\FF^n$ is complete with any more. 
        \item $\RR \setminus \{0\}$ is not complete with $|\cdot|$ as the metric.
        \item $\QQ^n$ with $|\cdot|$ is not complete. 
    \end{enumerate}
}

\ex{}{
    \begin{enumerate}
        \item $V$ is a finite dimensional normed vector spaces. $\varphi : \FF^n \to V$ isomorphism. Then $\FF^n \ni x \mapsto \Vert \varphi(x) \Vert_V \in [0, \infty)$ defines a norm on $\FF^N$, which we call $\Vert | x |\Vert$. Then $(\FF^n, \Vert | \cdot | \Vert)$ is isometric to $(V, \Vert \cdot \Vert_V)$, and hence $V$ is complete. 
        \item Let $\emptyset \neq X$ be a set endowed with the discrete metric. Suppose $\{x_n \}^\infty_{n = \ell} \subseteq X$ is Cauchy and pick $N \geq \ell$ such that $n,m \geq N \implies d(x_n, x_m) < 1$. Then $x_n = x_m = x_N$. So $x_n \to x_N$ as $n \to \infty$. Therefore $X$ is complete. 
    \end{enumerate}
}
Note that $Y = \prod Y_i$ is complete iff each individual $Y_i$ is complete. 
\thm{}{
    Let $V_1, \ldots, V_k, W$ be normed vector spaces over $\FF$. If $W$ is Banach, then so is $\mathcal L(V_1, \ldots, V_k)$.
}
\begin{proof}
    Suppose $\{T_n\}^\infty_{n = \ell} \subseteq \mathcal L(V_1, \ldots, V_k ; W$ is Cauchy. For fixed $v_1, \ldots, v_k) \in \prod_{i=1}^k V_i$, we bound
    \begin{align*}
        \Vert T_n(v_1, \ldots, v_k) - T_m(v_1, \ldots, v_k)\Vert_W \leq \Vert T_n - T_m \Vert_{\mathcal L}\prod_{i=1}^k \Vert v_i \Vert_{V_i}.
    \end{align*}
    Therefore, $\{T_n(v_1, \ldots, v_k)\}^\infty_{n = \ell} \subseteq  W$ is Cauchy and hence convergent. We may thus define $T: V_1 \times \cdots \times V_k \to W$ via $T(v_1, \ldots, v_k) = \lim_{n \to \infty} T_n(v_1, \ldots, v_k)$.
    \begin{enumerate}
        \item $T \in L(V_1, \ldots, V_k; W)$:
        
        \begin{align*}
            T_n(\alpha x  + \beta y, v_2,\ldots, v_k) = \alpha T_n(x, v_2, \ldots, v_k) + \beta T_n(y, v_2, \ldots, v_k)
        \end{align*}
        

        As $n \to \infty$, we get:

        \begin{align*}
            T(\alpha x + \beta y, v_2, \ldots, v_k) = \alpha T(x, v_2, \ldots, v_k) + \beta T(y, v_2, \ldots, v_k).
        \end{align*}
        Repeat in other slots if $k \geq 2$. As such, it is multilinear.

        \item $T \in \mathcal L(V_1, \ldots, V_k; W)$: Fix $v_i \in V_i$ with $\Vert v_i \Vert_{V_i} \leq 1$. Then 
        \begin{align*}
            \Vert T(v_1, \ldots, v_k) \Vert_W &= \lim_{n \to \infty} \Vert T_n(v_1, \ldots, v_k) \Vert_W \\
            &\leq \left(\limsup_{n \to \infty} \Vert T_n \Vert_{\mathcal L} \right) \prod_{n=1}^\infty \Vert v_i \Vert_{V_i} \leq \limsup_{n\to \infty} \Vert T_n \Vert_{\mathcal L} < \infty.
        \end{align*}
        \item $T_n \to T$ in $\mathcal L$ as $n \to \infty$: Let $\epsilon > 0$ and pick $N \geq \ell$ such that $n , m \geq N \implies \Vert T_n - T_m \Vert_{\mathcal L} < \frac \epsilon 2$. Then let $v_i \in V_i$ with $\Vert v_i \Vert_{V_i} \leq 1$. Then,
        \begin{align*}
            \Vert T(v_1, \ldots, v_k) - T_n (v_1, \ldots, v_k) \Vert_W = \lim_{m \to \infty} \Vert T_m(v_1, \ldots, v_k) - T_n (v_1, \ldots, v_k) \Vert_W \leq \lim_{m \to \infty} \Vert T_m - T_n \Vert_{\mathcal L} < \frac \epsilon 2.
        \end{align*}
        But this implies 
        \begin{align*}
            \Vert T(v_1, \ldots, v_k) - T_n(v_1, \ldots, v_k)\Vert_W \leq \frac \epsilon 2.
        \end{align*}
        By taking the supremum, we get that $\Vert T - T_n \Vert_{\mathcal L} \leq \frac \epsilon 2 < \epsilon$. 
    \end{enumerate}
\end{proof}

\cor{}{
    $V^* = \mathcal L (V; \FF)$ is always Banach.
}

\dfn{}{
    Let $X$ be a metric space, $E \subseteq X$. 
    \begin{enumerate}
        \item $x \in E$ is an interior point if $\exists \epsilon > 0$ such that $B(x, \epsilon) \subseteq E$. $E^\circ = \{ x \in E \mid x \text{ is an interior point}\}$. $E$ is open iff $E = E^\circ$. $E$ is closed iff $E^c$ is open.
        \item $x \in X$ is a boundary point of $E$ if $\forall \epsilon > 0$, $B(x, \epsilon) \cap E \neq \emptyset$ and $B(x, \epsilon) \cap E^c = \emptyset$. We write $\partial E = \{ x \in X \mid x \text{ is a boundary point of } E\}$. $\bar E = E^\circ \cup \partial E$.
        \item We say $x \in X$ is a limit point (accumulation point) of $E$ if $\forall \epsilon > 0$ $(B(x, \epsilon) \cap E) \setminus \{ x \} \neq \emptyset$. We write $E' = \{ x \in X \mid x \text{ is a limit point of } E\}$. If $x \in E \setminus E'$, then $x$ is an isolated point. 
    \end{enumerate}
}

\ex{}{
    Let $(X, disc)$ be given. Claim: all subsets of $X$ are both open and closed. 
    \begin{proof}
        $B(x, 1) = \{x\} \implies E \subseteq X$ can be written as 
        \begin{align*}
            E = \cup_{x \in E} B(x, 1),
        \end{align*}
        which is open. Therefore $E = (E^c)^c$ is also closed.
    \end{proof}
    Any metric space in which all sets are open and closed is called a discrete space.
}

\thm{}{
    Let $X$ be a metric space and $C \subseteq X$. The following are equivalent:
    \begin{enumerate}
        \item $C$ is closed.
        \item $C$ is sequentially closed; If $\{x_n \}_{n = \ell}^\infty \subseteq C$ is such that $x_n \to x$ in $X$ as $n \to \infty$, then $x \in C$. 
    \end{enumerate}
}
\begin{proof}
    $1 \to 2$. Let $\{x_n\} \subseteq C$ be such that $x_n \to x \in X$. Suppose BWOC that $x \in C^c$, which is open. Then $\exists N \geq \ell$ such that $n \geq N \implies x_n \in C^c \cup C$, which is a contradiction.

    $2 \to 1$. BWOC, suppose that $C$ is not closed, which emans $C^c$ is not open. Then $\exists x \in C^c$ such that we can pick $\{x_n \}_{n = 0}^\infty \subseteq C$ such that $x_n \in B(x, 2^{-n}) \cap C$. This means that $\{x_n \}_{n = 0}^\infty \subseteq C$ and $x_n \to x$ as $n \to \infty$. But $x \notin C$, so we have a contradiction. 
\end{proof}

\cor{}{
    Let $X$ be a complete metric space, and $\emptyset \neq C \subseteq X$. Then $C$ is closed in $X$ iff $C$ is a complete metric space with the metric from $X$. 
}
\begin{proof}
    $\implies$: Let $\{x_n\}_{n = \ell}^\infty \subseteq C$ be Cauchy. Then $x_n \to x \in X$ as $n \to \infty$ because $X$ is complete. By since $C$ is closed, $x \in C$.

    $\impliedby$: Let $\{x_n \} \subseteq C$ be such that $x_n \to x$ in $X$ as $n \to \infty$. Then $\{x _n \}$ is cauchy in $C$, meaning it's convergent in $C$, so $x \in C$, so $C$ is sequentially closed. 
\end{proof}

\dfn{}{Let $X$ be a metric space and $A \subseteq B \subseteq X$. We say $A$ is dense in $B$ if $\forall b \in B$, $\exists \{a_n\} \subseteq A$ such that $a_n \to b$ as $n \to \infty$.}

\ex{}{
    \begin{enumerate}
        \item $\QQ$ is dense in $\RR$. $\QQ^n$ is dense in $\RR^n$. $(\QQ^n + i \QQ^n) \subseteq \CC^n$ is dense. 
        \item $B(x, r) \subseteq \RR^n$ is dense in $B[x, r]$. 
        \item Let $X$ be given the discrete metric. $B(x, 1) = \{x\}$, but $B[x, 1] = X$, so as long as $X \neq \{x\}$, we do not have $B(x, 1) \subseteq B[x, 1]$ is dense.
    \end{enumerate}
}

\mprop{}{Let $X$ be a metrid space, $A \subseteq B \subseteq X$. The following are equivalent: 
\begin{enumerate}
    \item $A$ is dense in $B$.
    \item $ B \subseteq \bar A$.
    \item $\forall x \in B$ and $\epsilon > 0$, $\exists a \in A$ such that $d(x, a) < \epsilon$.
    \item $\forall x \in B$ and $\epsilon > 0$, $B(x, \epsilon) \cap A \neq \emptyset$.
\end{enumerate}
}
\begin{proof}
    Recall $\bar A = A \cup A'$. 

    $1 \implies 2$. Let $b \in B$. If $b \in A$, we're done. Otherwise $b \notin A$, but by density $\exists \{ a_n\}_{n = \ell}^\infty \subseteq A \setminus \{b\}$ such that $a_n \to b$ as $n \to \infty$. Thus, $b \in A'$. 

    $2 \implies 1$. Suppose $B \subseteq A \cup A' = \bar A$. Let $b \in B$. If $b \in A$, let $\{ a \}_{n = \ell}^\infty = b$ then we're done. 

    So suppose $b \in A' \setminus A$. By definition of limit point, we can pick a sequence $\{a_n\}$ such that $a_n \to b$ as $n \to \infty$. So $A$ is dense in $B$.

    $3 \iff 4$ is trivial. 

    $2 \iff 3$. Again, use $\bar A = A \cup A'$. 
\end{proof}


\cor{}{
    Let $X$ be a metric space and $A \subseteq B \subseteq X$. If $A$ is dense in $B$, then $A$ is also dense in $\bar B$. 
}
\begin{proof}
    $A \subseteq B$ is dense $\implies A \subseteq B \subseteq \bar A$. So $\bar B \subseteq \bar A$, meaning $A$ is dense is $\bar B$ as desired. 
\end{proof}


\dfn{}{
    Let $X$ be a metric space. We say $X$ is separable if $X$ has a countable dense subset.
}

\ex{Separable Vector Spaces}{
    \begin{enumerate}
        \item $\RR^n$ is separable, ditto for $\CC^n$. 
        \item Let $V$ be a finite dimensional normed vector space. Let $\varphi: \FF^n \to V$ be an isomorphism. Endow $\FF^n$ with norm $\Vert | x | \Vert = \Vert \varphi(x) \Vert_V$, which is equiavalent to $|\cdot|$ on $\FF^n$. Then $V$ is separable with $\varphi(\QQ^n)$ as a countable dense subset. 
        \item $\ell^\infty(\NN; \FF)$ is not separable, but $\ell^p(\NN; \FF)$ is for $1 \leq p < \infty$. 
    \end{enumerate}
}


\dfn{}{Let $X, X^*$ be metric spaces. We say that $X^*$ completes $X$ if:
\begin{enumerate}
    \item $X^*$ is complete.
    \item $\exists f : X \to X^*$ an isometric embedding. 
    \item $f(x) \subseteq X^*$ is dense.
\end{enumerate}}

\thm{Uniqueness of completions}{
    Let $X,Y,Z$ be metric spaces. Suppose $Y$ and $Z$ both complete $X$. Then $Y$ and $Z$ are isometric.
}
\begin{proof}
    Let $g: X \to Y$ and $h: X \to Z$ be isometric embeddings. We will construct an isometric $f: Y \to Z$. Let $y \in Y$. Since $g(X) \subseteq Y$ is dense, $\exists \{y_n\}_{n = \ell}^\infty \subseteq g(X)$ such that $y_n \to y$ as $n \to \infty$.

    Then $\exists! \{x_n \}_{n = \ell}^\infty \subseteq X$ such that $g(x_n) = y_n$ for all $n \geq \ell$. Then upon setting $z_n  = h(x_n) = h \circ g^{-1}(y_n)$, we have 
    \begin{align*}
        d_Z(z_n, z_m) = d_X(x_n, x_m) = d_Y(y_n, y_m).
    \end{align*}
    This means $\{z_n\}$ is Cauchy, and therefore convergent as $Z$ is complete. 

    Suppose $\{y_n'\}_{n = \ell}^\infty$ is another sequence such that $y_n' \to y$ as $n \to \infty$. Note 
    \begin{align*}
        d_Y(y_n, y_n') = d_X(g^{-1}(y_n), g^{-1}(y_n')) = d_Z(h(g^{-1}(y_n)), h(g^{-1}(y_n'))) = d_Z(z_n, z_n').
    \end{align*}
    Therefore, $\lim_{n \to \infty} z_n = \lim_{n \to \infty} z_n'$. So, we can define $f: Y \to Z$ as $f(y) = \lim_{n \to \infty} h(g^{-1}(y_n))$ for any sequence $\{y_n\} \subseteq g(X)$ such that $y_n \to y$ as $n \to \infty$.

    We claim that $f$ is an isometric embedding. Let $y , y' \in Y$ and pick $\{y_n\}_{n = \ell}^\infty$ and $\{y'_n\}_{n = \ell}^\infty$ such that $y_n \to y$ and $y'_n \to y'$ as $n \to \infty$. Then,
    \begin{align*}
        d_Y(y_n, y_n') = d_X(g^{-1}(y_n), g^{-1}(y_n')) = d_Z(h(g^{-1}(y_n)), h(g^{-1}(y_n'))) \to d_Z(f(y), f(y')) = d_Y(y, y'),
    \end{align*}
    so $f$ is an isometric embedding.

    We claim that $f$ is surjective. Let $z \in Z$ and pick $\{x_n\}_{n = \ell}^\infty$ such that $h(x_n) = z_n \to z$ as $n \to \infty$. Then let $y_n = g(x_n)$. Then $\{y_n\}_{n = \ell}^\infty \subseteq Y$ are Cauchy and hence convergent to $y \in Y$. Then $f(y) =\lim_{n \to \infty} h \circ g^{-1}(y_n) = \lim_{n \to \infty} z_n = z$. So $f: Y \to Z$ is an isometry!
\end{proof}
\nt{This is analogous to the uniqueness of Dedekind complete ordered fields. In principal, there can be different techniques for finding /constructing completions of a given metric space, but in the end they're isometric.}


\end{document}