\documentclass{report}

%%%%%%%%%%%%%%%%%%%%%%%%%%%%%%%%%
% PACKAGE IMPORTS
%%%%%%%%%%%%%%%%%%%%%%%%%%%%%%%%%


\usepackage[tmargin=2cm,rmargin=1in,lmargin=1in,margin=0.85in,bmargin=2cm,footskip=.2in]{geometry}
\usepackage{amsmath,amsfonts,amsthm,amssymb,mathtools}
\usepackage[varbb]{newpxmath}
\usepackage{xfrac}
\usepackage[makeroom]{cancel}
\usepackage{mathtools}
\usepackage{bookmark}
\usepackage{enumitem}
\usepackage{hyperref,theoremref}
\hypersetup{
	pdftitle={Assignment},
	colorlinks=true, linkcolor=doc!90,
	bookmarksnumbered=true,
	bookmarksopen=true
}
\usepackage[most,many,breakable]{tcolorbox}
\usepackage{xcolor}
\usepackage{varwidth}
\usepackage{varwidth}
\usepackage{etoolbox}
%\usepackage{authblk}
\usepackage{nameref}
\usepackage{multicol,array}
\usepackage{tikz-cd}
\usepackage[ruled,vlined,linesnumbered]{algorithm2e}
\usepackage{comment} % enables the use of multi-line comments (\ifx \fi) 
\usepackage{import}
\usepackage{xifthen}
\usepackage{pdfpages}
\usepackage{transparent}

% \usepackage{extsizes}

\newcommand\mycommfont[1]{\footnotesize\ttfamily\textcolor{blue}{#1}}
\SetCommentSty{mycommfont}
\newcommand{\incfig}[1]{%
    \def\svgwidth{\columnwidth}
    \import{./figures/}{#1.pdf_tex}
}

\usepackage{tikzsymbols}
\renewcommand\qedsymbol{$\Laughey$}


%\usepackage{import}
%\usepackage{xifthen}
%\usepackage{pdfpages}
%\usepackage{transparent}


%%%%%%%%%%%%%%%%%%%%%%%%%%%%%%
% SELF MADE COLORS
%%%%%%%%%%%%%%%%%%%%%%%%%%%%%%



\definecolor{myg}{RGB}{56, 140, 70}
\definecolor{myb}{RGB}{45, 111, 177}
\definecolor{myr}{RGB}{199, 68, 64}
\definecolor{mytheorembg}{HTML}{F2F2F9}
\definecolor{mytheoremfr}{HTML}{00007B}
\definecolor{mylenmabg}{HTML}{FFFAF8}
\definecolor{mylenmafr}{HTML}{983b0f}
\definecolor{mypropbg}{HTML}{f2fbfc}
\definecolor{mypropfr}{HTML}{191971}
\definecolor{myexamplebg}{HTML}{F2FBF8}
\definecolor{myexamplefr}{HTML}{88D6D1}
\definecolor{myexampleti}{HTML}{2A7F7F}
\definecolor{mydefinitbg}{HTML}{E5E5FF}
\definecolor{mydefinitfr}{HTML}{3F3FA3}
\definecolor{notesgreen}{RGB}{0,162,0}
\definecolor{myp}{RGB}{197, 92, 212}
\definecolor{mygr}{HTML}{2C3338}
\definecolor{myred}{RGB}{127,0,0}
\definecolor{myyellow}{RGB}{169,121,69}
\definecolor{myexercisebg}{HTML}{F2FBF8}
\definecolor{myexercisefg}{HTML}{88D6D1}


%%%%%%%%%%%%%%%%%%%%%%%%%%%%
% TCOLORBOX SETUPS
%%%%%%%%%%%%%%%%%%%%%%%%%%%%

\setlength{\parindent}{1cm}
%================================
% THEOREM BOX
%================================

\tcbuselibrary{theorems,skins,hooks}
\newtcbtheorem[number within=section]{Theorem}{Theorem}
{%
	enhanced,
	breakable,
	colback = mytheorembg,
	frame hidden,
	boxrule = 0sp,
	borderline west = {2pt}{0pt}{mytheoremfr},
	sharp corners,
	detach title,
	before upper = \tcbtitle\par\smallskip,
	coltitle = mytheoremfr,
	fonttitle = \bfseries\sffamily,
	description font = \mdseries,
	separator sign none,
	segmentation style={solid, mytheoremfr},
}
{th}

\tcbuselibrary{theorems,skins,hooks}
\newtcbtheorem[number within=chapter]{theorem}{Theorem}
{%
	enhanced,
	breakable,
	colback = mytheorembg,
	frame hidden,
	boxrule = 0sp,
	borderline west = {2pt}{0pt}{mytheoremfr},
	sharp corners,
	detach title,
	before upper = \tcbtitle\par\smallskip,
	coltitle = mytheoremfr,
	fonttitle = \bfseries\sffamily,
	description font = \mdseries,
	separator sign none,
	segmentation style={solid, mytheoremfr},
}
{th}


\tcbuselibrary{theorems,skins,hooks}
\newtcolorbox{Theoremcon}
{%
	enhanced
	,breakable
	,colback = mytheorembg
	,frame hidden
	,boxrule = 0sp
	,borderline west = {2pt}{0pt}{mytheoremfr}
	,sharp corners
	,description font = \mdseries
	,separator sign none
}

%================================
% Corollery
%================================
\tcbuselibrary{theorems,skins,hooks}
\newtcbtheorem[number within=section]{Corollary}{Corollary}
{%
	enhanced
	,breakable
	,colback = myp!10
	,frame hidden
	,boxrule = 0sp
	,borderline west = {2pt}{0pt}{myp!85!black}
	,sharp corners
	,detach title
	,before upper = \tcbtitle\par\smallskip
	,coltitle = myp!85!black
	,fonttitle = \bfseries\sffamily
	,description font = \mdseries
	,separator sign none
	,segmentation style={solid, myp!85!black}
}
{th}
\tcbuselibrary{theorems,skins,hooks}
\newtcbtheorem[number within=chapter]{corollary}{Corollary}
{%
	enhanced
	,breakable
	,colback = myp!10
	,frame hidden
	,boxrule = 0sp
	,borderline west = {2pt}{0pt}{myp!85!black}
	,sharp corners
	,detach title
	,before upper = \tcbtitle\par\smallskip
	,coltitle = myp!85!black
	,fonttitle = \bfseries\sffamily
	,description font = \mdseries
	,separator sign none
	,segmentation style={solid, myp!85!black}
}
{th}


%================================
% LENMA
%================================

\tcbuselibrary{theorems,skins,hooks}
\newtcbtheorem[number within=section]{Lenma}{Lenma}
{%
	enhanced,
	breakable,
	colback = mylenmabg,
	frame hidden,
	boxrule = 0sp,
	borderline west = {2pt}{0pt}{mylenmafr},
	sharp corners,
	detach title,
	before upper = \tcbtitle\par\smallskip,
	coltitle = mylenmafr,
	fonttitle = \bfseries\sffamily,
	description font = \mdseries,
	separator sign none,
	segmentation style={solid, mylenmafr},
}
{th}

\tcbuselibrary{theorems,skins,hooks}
\newtcbtheorem[number within=chapter]{lenma}{Lenma}
{%
	enhanced,
	breakable,
	colback = mylenmabg,
	frame hidden,
	boxrule = 0sp,
	borderline west = {2pt}{0pt}{mylenmafr},
	sharp corners,
	detach title,
	before upper = \tcbtitle\par\smallskip,
	coltitle = mylenmafr,
	fonttitle = \bfseries\sffamily,
	description font = \mdseries,
	separator sign none,
	segmentation style={solid, mylenmafr},
}
{th}


%================================
% PROPOSITION
%================================

\tcbuselibrary{theorems,skins,hooks}
\newtcbtheorem[number within=section]{Prop}{Proposition}
{%
	enhanced,
	breakable,
	colback = mypropbg,
	frame hidden,
	boxrule = 0sp,
	borderline west = {2pt}{0pt}{mypropfr},
	sharp corners,
	detach title,
	before upper = \tcbtitle\par\smallskip,
	coltitle = mypropfr,
	fonttitle = \bfseries\sffamily,
	description font = \mdseries,
	separator sign none,
	segmentation style={solid, mypropfr},
}
{th}

\tcbuselibrary{theorems,skins,hooks}
\newtcbtheorem[number within=chapter]{prop}{Proposition}
{%
	enhanced,
	breakable,
	colback = mypropbg,
	frame hidden,
	boxrule = 0sp,
	borderline west = {2pt}{0pt}{mypropfr},
	sharp corners,
	detach title,
	before upper = \tcbtitle\par\smallskip,
	coltitle = mypropfr,
	fonttitle = \bfseries\sffamily,
	description font = \mdseries,
	separator sign none,
	segmentation style={solid, mypropfr},
}
{th}


%================================
% CLAIM
%================================

\tcbuselibrary{theorems,skins,hooks}
\newtcbtheorem[number within=section]{claim}{Claim}
{%
	enhanced
	,breakable
	,colback = myg!10
	,frame hidden
	,boxrule = 0sp
	,borderline west = {2pt}{0pt}{myg}
	,sharp corners
	,detach title
	,before upper = \tcbtitle\par\smallskip
	,coltitle = myg!85!black
	,fonttitle = \bfseries\sffamily
	,description font = \mdseries
	,separator sign none
	,segmentation style={solid, myg!85!black}
}
{th}



%================================
% Exercise
%================================

\tcbuselibrary{theorems,skins,hooks}
\newtcbtheorem[number within=section]{Exercise}{Exercise}
{%
	enhanced,
	breakable,
	colback = myexercisebg,
	frame hidden,
	boxrule = 0sp,
	borderline west = {2pt}{0pt}{myexercisefg},
	sharp corners,
	detach title,
	before upper = \tcbtitle\par\smallskip,
	coltitle = myexercisefg,
	fonttitle = \bfseries\sffamily,
	description font = \mdseries,
	separator sign none,
	segmentation style={solid, myexercisefg},
}
{th}

\tcbuselibrary{theorems,skins,hooks}
\newtcbtheorem[number within=chapter]{exercise}{Exercise}
{%
	enhanced,
	breakable,
	colback = myexercisebg,
	frame hidden,
	boxrule = 0sp,
	borderline west = {2pt}{0pt}{myexercisefg},
	sharp corners,
	detach title,
	before upper = \tcbtitle\par\smallskip,
	coltitle = myexercisefg,
	fonttitle = \bfseries\sffamily,
	description font = \mdseries,
	separator sign none,
	segmentation style={solid, myexercisefg},
}
{th}

%================================
% EXAMPLE BOX
%================================

\newtcbtheorem[number within=section]{Example}{Example}
{%
	colback = myexamplebg
	,breakable
	,colframe = myexamplefr
	,coltitle = myexampleti
	,boxrule = 1pt
	,sharp corners
	,detach title
	,before upper=\tcbtitle\par\smallskip
	,fonttitle = \bfseries
	,description font = \mdseries
	,separator sign none
	,description delimiters parenthesis
}
{ex}

\newtcbtheorem[number within=chapter]{example}{Example}
{%
	colback = myexamplebg
	,breakable
	,colframe = myexamplefr
	,coltitle = myexampleti
	,boxrule = 1pt
	,sharp corners
	,detach title
	,before upper=\tcbtitle\par\smallskip
	,fonttitle = \bfseries
	,description font = \mdseries
	,separator sign none
	,description delimiters parenthesis
}
{ex}

%================================
% DEFINITION BOX
%================================

\newtcbtheorem[number within=section]{Definition}{Definition}{enhanced,
	before skip=2mm,after skip=2mm, colback=red!5,colframe=red!80!black,boxrule=0.5mm,
	attach boxed title to top left={xshift=1cm,yshift*=1mm-\tcboxedtitleheight}, varwidth boxed title*=-3cm,
	boxed title style={frame code={
					\path[fill=tcbcolback]
					([yshift=-1mm,xshift=-1mm]frame.north west)
					arc[start angle=0,end angle=180,radius=1mm]
					([yshift=-1mm,xshift=1mm]frame.north east)
					arc[start angle=180,end angle=0,radius=1mm];
					\path[left color=tcbcolback!60!black,right color=tcbcolback!60!black,
						middle color=tcbcolback!80!black]
					([xshift=-2mm]frame.north west) -- ([xshift=2mm]frame.north east)
					[rounded corners=1mm]-- ([xshift=1mm,yshift=-1mm]frame.north east)
					-- (frame.south east) -- (frame.south west)
					-- ([xshift=-1mm,yshift=-1mm]frame.north west)
					[sharp corners]-- cycle;
				},interior engine=empty,
		},
	fonttitle=\bfseries,
	title={#2},#1}{def}
\newtcbtheorem[number within=chapter]{definition}{Definition}{enhanced,
	before skip=2mm,after skip=2mm, colback=red!5,colframe=red!80!black,boxrule=0.5mm,
	attach boxed title to top left={xshift=1cm,yshift*=1mm-\tcboxedtitleheight}, varwidth boxed title*=-3cm,
	boxed title style={frame code={
					\path[fill=tcbcolback]
					([yshift=-1mm,xshift=-1mm]frame.north west)
					arc[start angle=0,end angle=180,radius=1mm]
					([yshift=-1mm,xshift=1mm]frame.north east)
					arc[start angle=180,end angle=0,radius=1mm];
					\path[left color=tcbcolback!60!black,right color=tcbcolback!60!black,
						middle color=tcbcolback!80!black]
					([xshift=-2mm]frame.north west) -- ([xshift=2mm]frame.north east)
					[rounded corners=1mm]-- ([xshift=1mm,yshift=-1mm]frame.north east)
					-- (frame.south east) -- (frame.south west)
					-- ([xshift=-1mm,yshift=-1mm]frame.north west)
					[sharp corners]-- cycle;
				},interior engine=empty,
		},
	fonttitle=\bfseries,
	title={#2},#1}{def}



%================================
% Solution BOX
%================================

\makeatletter
\newtcbtheorem{question}{Question}{enhanced,
	breakable,
	colback=white,
	colframe=myb!80!black,
	attach boxed title to top left={yshift*=-\tcboxedtitleheight},
	fonttitle=\bfseries,
	title={#2},
	boxed title size=title,
	boxed title style={%
			sharp corners,
			rounded corners=northwest,
			colback=tcbcolframe,
			boxrule=0pt,
		},
	underlay boxed title={%
			\path[fill=tcbcolframe] (title.south west)--(title.south east)
			to[out=0, in=180] ([xshift=5mm]title.east)--
			(title.center-|frame.east)
			[rounded corners=\kvtcb@arc] |-
			(frame.north) -| cycle;
		},
	#1
}{def}
\makeatother

%================================
% SOLUTION BOX
%================================

\makeatletter
\newtcolorbox{solution}{enhanced,
	breakable,
	colback=white,
	colframe=myg!80!black,
	attach boxed title to top left={yshift*=-\tcboxedtitleheight},
	title=Solution,
	boxed title size=title,
	boxed title style={%
			sharp corners,
			rounded corners=northwest,
			colback=tcbcolframe,
			boxrule=0pt,
		},
	underlay boxed title={%
			\path[fill=tcbcolframe] (title.south west)--(title.south east)
			to[out=0, in=180] ([xshift=5mm]title.east)--
			(title.center-|frame.east)
			[rounded corners=\kvtcb@arc] |-
			(frame.north) -| cycle;
		},
}
\makeatother

%================================
% Question BOX
%================================

\makeatletter
\newtcbtheorem{qstion}{Question}{enhanced,
	breakable,
	colback=white,
	colframe=mygr,
	attach boxed title to top left={yshift*=-\tcboxedtitleheight},
	fonttitle=\bfseries,
	title={#2},
	boxed title size=title,
	boxed title style={%
			sharp corners,
			rounded corners=northwest,
			colback=tcbcolframe,
			boxrule=0pt,
		},
	underlay boxed title={%
			\path[fill=tcbcolframe] (title.south west)--(title.south east)
			to[out=0, in=180] ([xshift=5mm]title.east)--
			(title.center-|frame.east)
			[rounded corners=\kvtcb@arc] |-
			(frame.north) -| cycle;
		},
	#1
}{def}
\makeatother

\newtcbtheorem[number within=chapter]{wconc}{Wrong Concept}{
	breakable,
	enhanced,
	colback=white,
	colframe=myr,
	arc=0pt,
	outer arc=0pt,
	fonttitle=\bfseries\sffamily\large,
	colbacktitle=myr,
	attach boxed title to top left={},
	boxed title style={
			enhanced,
			skin=enhancedfirst jigsaw,
			arc=3pt,
			bottom=0pt,
			interior style={fill=myr}
		},
	#1
}{def}



%================================
% NOTE BOX
%================================

\usetikzlibrary{arrows,calc,shadows.blur}
\tcbuselibrary{skins}
\newtcolorbox{note}[1][]{%
	enhanced jigsaw,
	colback=gray!20!white,%
	colframe=gray!80!black,
	size=small,
	boxrule=1pt,
	title=\textbf{Note:},
	halign title=flush center,
	coltitle=black,
	breakable,
	drop shadow=black!50!white,
	attach boxed title to top left={xshift=1cm,yshift=-\tcboxedtitleheight/2,yshifttext=-\tcboxedtitleheight/2},
	minipage boxed title=1.5cm,
	boxed title style={%
			colback=white,
			size=fbox,
			boxrule=1pt,
			boxsep=2pt,
			underlay={%
					\coordinate (dotA) at ($(interior.west) + (-0.5pt,0)$);
					\coordinate (dotB) at ($(interior.east) + (0.5pt,0)$);
					\begin{scope}
						\clip (interior.north west) rectangle ([xshift=3ex]interior.east);
						\filldraw [white, blur shadow={shadow opacity=60, shadow yshift=-.75ex}, rounded corners=2pt] (interior.north west) rectangle (interior.south east);
					\end{scope}
					\begin{scope}[gray!80!black]
						\fill (dotA) circle (2pt);
						\fill (dotB) circle (2pt);
					\end{scope}
				},
		},
	#1,
}

%%%%%%%%%%%%%%%%%%%%%%%%%%%%%%
% SELF MADE COMMANDS
%%%%%%%%%%%%%%%%%%%%%%%%%%%%%%


\newcommand{\thm}[2]{\begin{Theorem}{#1}{}#2\end{Theorem}}
\newcommand{\cor}[2]{\begin{Corollary}{#1}{}#2\end{Corollary}}
\newcommand{\mlenma}[2]{\begin{Lenma}{#1}{}#2\end{Lenma}}
\newcommand{\mprop}[2]{\begin{Prop}{#1}{}#2\end{Prop}}
\newcommand{\clm}[3]{\begin{claim}{#1}{#2}#3\end{claim}}
\newcommand{\wc}[2]{\begin{wconc}{#1}{}\setlength{\parindent}{1cm}#2\end{wconc}}
\newcommand{\thmcon}[1]{\begin{Theoremcon}{#1}\end{Theoremcon}}
\newcommand{\ex}[2]{\begin{Example}{#1}{}#2\end{Example}}
\newcommand{\dfn}[2]{\begin{Definition}[colbacktitle=red!75!black]{#1}{}#2\end{Definition}}
\newcommand{\dfnc}[2]{\begin{definition}[colbacktitle=red!75!black]{#1}{}#2\end{definition}}
\newcommand{\qs}[2]{\begin{question*}{#1}{}#2\end{question*}}
\newcommand{\mpf}[2]{\begin{myproof}[#1]#2\end{myproof}}
\newcommand{\nt}[1]{\begin{note}#1\end{note}}

\newcommand*\circled[1]{\tikz[baseline=(char.base)]{
		\node[shape=circle,draw,inner sep=1pt] (char) {#1};}}
\newcommand\getcurrentref[1]{%
	\ifnumequal{\value{#1}}{0}
	{??}
	{\the\value{#1}}%
}
\newcommand{\getCurrentSectionNumber}{\getcurrentref{section}}
\newenvironment{myproof}[1][\proofname]{%
	\proof[\bfseries #1: ]%
}{\endproof}

\newcommand{\mclm}[2]{\begin{myclaim}[#1]#2\end{myclaim}}
\newenvironment{myclaim}[1][\claimname]{\proof[\bfseries #1: ]}{}

\newcounter{mylabelcounter}

\makeatletter
\newcommand{\setword}[2]{%
	\phantomsection
	#1\def\@currentlabel{\unexpanded{#1}}\label{#2}%
}
\makeatother




\tikzset{
	symbol/.style={
			draw=none,
			every to/.append style={
					edge node={node [sloped, allow upside down, auto=false]{$#1$}}}
		}
}


% deliminators
\DeclarePairedDelimiter{\abs}{\lvert}{\rvert}
\DeclarePairedDelimiter{\norm}{\lVert}{\rVert}

\DeclarePairedDelimiter{\ceil}{\lceil}{\rceil}
\DeclarePairedDelimiter{\floor}{\lfloor}{\rfloor}
\DeclarePairedDelimiter{\round}{\lfloor}{\rceil}

\newsavebox\diffdbox
\newcommand{\slantedromand}{{\mathpalette\makesl{d}}}
\newcommand{\makesl}[2]{%
\begingroup
\sbox{\diffdbox}{$\mathsurround=0pt#1\mathrm{#2}$}%
\pdfsave
\pdfsetmatrix{1 0 0.2 1}%
\rlap{\usebox{\diffdbox}}%
\pdfrestore
\hskip\wd\diffdbox
\endgroup
}
\newcommand{\dd}[1][]{\ensuremath{\mathop{}\!\ifstrempty{#1}{%
\slantedromand\@ifnextchar^{\hspace{0.2ex}}{\hspace{0.1ex}}}%
{\slantedromand\hspace{0.2ex}^{#1}}}}
\ProvideDocumentCommand\dv{o m g}{%
  \ensuremath{%
    \IfValueTF{#3}{%
      \IfNoValueTF{#1}{%
        \frac{\dd #2}{\dd #3}%
      }{%
        \frac{\dd^{#1} #2}{\dd #3^{#1}}%
      }%
    }{%
      \IfNoValueTF{#1}{%
        \frac{\dd}{\dd #2}%
      }{%
        \frac{\dd^{#1}}{\dd #2^{#1}}%
      }%
    }%
  }%
}
\providecommand*{\pdv}[3][]{\frac{\partial^{#1}#2}{\partial#3^{#1}}}
%  - others
\DeclareMathOperator{\Lap}{\mathcal{L}}
\DeclareMathOperator{\Var}{Var} % varience
\DeclareMathOperator{\Cov}{Cov} % covarience
\DeclareMathOperator{\E}{E} % expected

% Since the amsthm package isn't loaded

% I prefer the slanted \leq
\let\oldleq\leq % save them in case they're every wanted
\let\oldgeq\geq
\renewcommand{\leq}{\leqslant}
\renewcommand{\geq}{\geqslant}

% % redefine matrix env to allow for alignment, use r as default
% \renewcommand*\env@matrix[1][r]{\hskip -\arraycolsep
%     \let\@ifnextchar\new@ifnextchar
%     \array{*\c@MaxMatrixCols #1}}


%\usepackage{framed}
%\usepackage{titletoc}
%\usepackage{etoolbox}
%\usepackage{lmodern}


%\patchcmd{\tableofcontents}{\contentsname}{\sffamily\contentsname}{}{}

%\renewenvironment{leftbar}
%{\def\FrameCommand{\hspace{6em}%
%		{\color{myyellow}\vrule width 2pt depth 6pt}\hspace{1em}}%
%	\MakeFramed{\parshape 1 0cm \dimexpr\textwidth-6em\relax\FrameRestore}\vskip2pt%
%}
%{\endMakeFramed}

%\titlecontents{chapter}
%[0em]{\vspace*{2\baselineskip}}
%{\parbox{4.5em}{%
%		\hfill\Huge\sffamily\bfseries\color{myred}\thecontentspage}%
%	\vspace*{-2.3\baselineskip}\leftbar\textsc{\small\chaptername~\thecontentslabel}\\\sffamily}
%{}{\endleftbar}
%\titlecontents{section}
%[8.4em]
%{\sffamily\contentslabel{3em}}{}{}
%{\hspace{0.5em}\nobreak\itshape\color{myred}\contentspage}
%\titlecontents{subsection}
%[8.4em]
%{\sffamily\contentslabel{3em}}{}{}  
%{\hspace{0.5em}\nobreak\itshape\color{myred}\contentspage}



%%%%%%%%%%%%%%%%%%%%%%%%%%%%%%%%%%%%%%%%%%%
% TABLE OF CONTENTS
%%%%%%%%%%%%%%%%%%%%%%%%%%%%%%%%%%%%%%%%%%%

\usepackage{tikz}
\definecolor{doc}{RGB}{0,60,110}
\usepackage{titletoc}
\contentsmargin{0cm}
\titlecontents{chapter}[3.7pc]
{\addvspace{30pt}%
	\begin{tikzpicture}[remember picture, overlay]%
		\draw[fill=doc!60,draw=doc!60] (-7,-.1) rectangle (-0.9,.5);%
		\pgftext[left,x=-3.5cm,y=0.2cm]{\color{white}\Large\sc\bfseries Chapter\ \thecontentslabel};%
	\end{tikzpicture}\color{doc!60}\large\sc\bfseries}%
{}
{}
{\;\titlerule\;\large\sc\bfseries Page \thecontentspage
	\begin{tikzpicture}[remember picture, overlay]
		\draw[fill=doc!60,draw=doc!60] (2pt,0) rectangle (4,0.1pt);
	\end{tikzpicture}}%
\titlecontents{section}[3.7pc]
{\addvspace{2pt}}
{\contentslabel[\thecontentslabel]{2pc}}
{}
{\hfill\small \thecontentspage}
[]
\titlecontents*{subsection}[3.7pc]
{\addvspace{-1pt}\small}
{}
{}
{\ --- \small\thecontentspage}
[ \textbullet\ ][]

\makeatletter
\renewcommand{\tableofcontents}{%
	\chapter*{%
	  \vspace*{-20\p@}%
	  \begin{tikzpicture}[remember picture, overlay]%
		  \pgftext[right,x=15cm,y=0.2cm]{\color{doc!60}\Huge\sc\bfseries \contentsname};%
		  \draw[fill=doc!60,draw=doc!60] (13,-.75) rectangle (20,1);%
		  \clip (13,-.75) rectangle (20,1);
		  \pgftext[right,x=15cm,y=0.2cm]{\color{white}\Huge\sc\bfseries \contentsname};%
	  \end{tikzpicture}}%
	\@starttoc{toc}}
\makeatother
\newcommand{\id}{\mathrm{id}}
\newcommand{\taking}[1]{\xrightarrow{#1}}
\newcommand{\inv}{^{-1}}

%From M170 "Introduction to Graph Theory" at SJSU
\DeclareMathOperator{\diam}{diam}
\DeclareMathOperator{\ord}{ord}
\newcommand{\defeq}{\overset{\mathrm{def}}{=}}

%From the USAMO .tex files
\newcommand{\ts}{\textsuperscript}
\newcommand{\dg}{^\circ}
\newcommand{\ii}{\item}

% % From Math 55 and Math 145 at Harvard
% \newenvironment{subproof}[1][Proof]{%
% \begin{proof}[#1] \renewcommand{\qedsymbol}{$\blacksquare$}}%
% {\end{proof}}

\newcommand{\liff}{\leftrightarrow}
\newcommand{\lthen}{\rightarrow}
\newcommand{\opname}{\operatorname}
\newcommand{\surjto}{\twoheadrightarrow}
\newcommand{\injto}{\hookrightarrow}
\newcommand{\On}{\mathrm{On}} % ordinals
\DeclareMathOperator{\img}{im} % Image
\DeclareMathOperator{\Img}{Im} % Image
\DeclareMathOperator{\coker}{coker} % Cokernel
\DeclareMathOperator{\Coker}{Coker} % Cokernel
\DeclareMathOperator{\Ker}{Ker} % Kernel
\DeclareMathOperator{\rank}{rank}
\DeclareMathOperator{\Spec}{Spec} % spectrum
\DeclareMathOperator{\Tr}{Tr} % trace
\DeclareMathOperator{\pr}{pr} % projection
\DeclareMathOperator{\ext}{ext} % extension
\DeclareMathOperator{\pred}{pred} % predecessor
\DeclareMathOperator{\dom}{dom} % domain
\DeclareMathOperator{\ran}{ran} % range
\DeclareMathOperator{\Hom}{Hom} % homomorphism
\DeclareMathOperator{\Mor}{Mor} % morphisms
\DeclareMathOperator{\End}{End} % endomorphism

\newcommand{\eps}{\epsilon}
\newcommand{\veps}{\varepsilon}
\newcommand{\ol}{\overline}
\newcommand{\ul}{\underline}
\newcommand{\wt}{\widetilde}
\newcommand{\wh}{\widehat}
\newcommand{\vocab}[1]{\textbf{\color{blue} #1}}
\providecommand{\half}{\frac{1}{2}}
\newcommand{\dang}{\measuredangle} %% Directed angle
\newcommand{\ray}[1]{\overrightarrow{#1}}
\newcommand{\seg}[1]{\overline{#1}}
\newcommand{\arc}[1]{\wideparen{#1}}
\DeclareMathOperator{\cis}{cis}
\DeclareMathOperator*{\lcm}{lcm}
\DeclareMathOperator*{\argmin}{arg min}
\DeclareMathOperator*{\argmax}{arg max}
\newcommand{\cycsum}{\sum_{\mathrm{cyc}}}
\newcommand{\symsum}{\sum_{\mathrm{sym}}}
\newcommand{\cycprod}{\prod_{\mathrm{cyc}}}
\newcommand{\symprod}{\prod_{\mathrm{sym}}}
\newcommand{\Qed}{\begin{flushright}\qed\end{flushright}}
\newcommand{\parinn}{\setlength{\parindent}{1cm}}
\newcommand{\parinf}{\setlength{\parindent}{0cm}}
% \newcommand{\norm}{\|\cdot\|}
\newcommand{\inorm}{\norm_{\infty}}
\newcommand{\opensets}{\{V_{\alpha}\}_{\alpha\in I}}
\newcommand{\oset}{V_{\alpha}}
\newcommand{\opset}[1]{V_{\alpha_{#1}}}
\newcommand{\lub}{\text{lub}}
\newcommand{\del}[2]{\frac{\partial #1}{\partial #2}}
\newcommand{\Del}[3]{\frac{\partial^{#1} #2}{\partial^{#1} #3}}
\newcommand{\deld}[2]{\dfrac{\partial #1}{\partial #2}}
\newcommand{\Deld}[3]{\dfrac{\partial^{#1} #2}{\partial^{#1} #3}}
\newcommand{\lm}{\lambda}
\newcommand{\uin}{\mathbin{\rotatebox[origin=c]{90}{$\in$}}}
\newcommand{\usubset}{\mathbin{\rotatebox[origin=c]{90}{$\subset$}}}
\newcommand{\lt}{\left}
\newcommand{\rt}{\right}
\newcommand{\bs}[1]{\boldsymbol{#1}}
\newcommand{\exs}{\exists}
\newcommand{\st}{\strut}
\newcommand{\dps}[1]{\displaystyle{#1}}

\newcommand{\sol}{\setlength{\parindent}{0cm}\textbf{\textit{Solution:}}\setlength{\parindent}{1cm} }
\newcommand{\solve}[1]{\setlength{\parindent}{0cm}\textbf{\textit{Solution: }}\setlength{\parindent}{1cm}#1 \Qed}
% Things Lie
\newcommand{\kb}{\mathfrak b}
\newcommand{\kg}{\mathfrak g}
\newcommand{\kh}{\mathfrak h}
\newcommand{\kn}{\mathfrak n}
\newcommand{\ku}{\mathfrak u}
\newcommand{\kz}{\mathfrak z}
\DeclareMathOperator{\Ext}{Ext} % Ext functor
\DeclareMathOperator{\Tor}{Tor} % Tor functor
\newcommand{\gl}{\opname{\mathfrak{gl}}} % frak gl group
\renewcommand{\sl}{\opname{\mathfrak{sl}}} % frak sl group chktex 6

% More script letters etc.
\newcommand{\SA}{\mathcal A}
\newcommand{\SB}{\mathcal B}
\newcommand{\SC}{\mathcal C}
\newcommand{\SF}{\mathcal F}
\newcommand{\SG}{\mathcal G}
\newcommand{\SH}{\mathcal H}
\newcommand{\OO}{\mathcal O}

\newcommand{\SCA}{\mathscr A}
\newcommand{\SCB}{\mathscr B}
\newcommand{\SCC}{\mathscr C}
\newcommand{\SCD}{\mathscr D}
\newcommand{\SCE}{\mathscr E}
\newcommand{\SCF}{\mathscr F}
\newcommand{\SCG}{\mathscr G}
\newcommand{\SCH}{\mathscr H}

% Mathfrak primes
\newcommand{\km}{\mathfrak m}
\newcommand{\kp}{\mathfrak p}
\newcommand{\kq}{\mathfrak q}

% number sets
\newcommand{\RR}[1][]{\ensuremath{\ifstrempty{#1}{\mathbb{R}}{\mathbb{R}^{#1}}}}
\newcommand{\NN}[1][]{\ensuremath{\ifstrempty{#1}{\mathbb{N}}{\mathbb{N}^{#1}}}}
\newcommand{\ZZ}[1][]{\ensuremath{\ifstrempty{#1}{\mathbb{Z}}{\mathbb{Z}^{#1}}}}
\newcommand{\QQ}[1][]{\ensuremath{\ifstrempty{#1}{\mathbb{Q}}{\mathbb{Q}^{#1}}}}
\newcommand{\CC}[1][]{\ensuremath{\ifstrempty{#1}{\mathbb{C}}{\mathbb{C}^{#1}}}}
\newcommand{\PP}[1][]{\ensuremath{\ifstrempty{#1}{\mathbb{P}}{\mathbb{P}^{#1}}}}
\newcommand{\HH}[1][]{\ensuremath{\ifstrempty{#1}{\mathbb{H}}{\mathbb{H}^{#1}}}}
\newcommand{\FF}[1][]{\ensuremath{\ifstrempty{#1}{\mathbb{F}}{\mathbb{F}^{#1}}}}

% number sets without arguments
\newcommand{\R}{\ensuremath{\mathbb{R}}}
\newcommand{\N}{\ensuremath{\mathbb{N}}}
\newcommand{\Z}{\ensuremath{\mathbb{Z}}}
\newcommand{\Q}{\ensuremath{\mathbb{Q}}}
\newcommand{\C}{\ensuremath{\mathbb{C}}}
\newcommand{\F}{\ensuremath{\mathbb{F}}}

% expected value
\newcommand{\EE}{\ensuremath{\mathbb{E}}}
\newcommand{\charin}{\text{ char }}
\DeclareMathOperator{\sign}{sign}
\DeclareMathOperator{\Aut}{Aut}
\DeclareMathOperator{\Inn}{Inn}
\DeclareMathOperator{\Syl}{Syl}
\DeclareMathOperator{\Gal}{Gal}
\DeclareMathOperator{\GL}{GL} % General linear group
\DeclareMathOperator{\SL}{SL} % Special linear group

%---------------------------------------
% BlackBoard Math Fonts :-
%---------------------------------------

%Captital Letters
\newcommand{\bbA}{\mathbb{A}}	\newcommand{\bbB}{\mathbb{B}}
\newcommand{\bbC}{\mathbb{C}}	\newcommand{\bbD}{\mathbb{D}}
\newcommand{\bbE}{\mathbb{E}}	\newcommand{\bbF}{\mathbb{F}}
\newcommand{\bbG}{\mathbb{G}}	\newcommand{\bbH}{\mathbb{H}}
\newcommand{\bbI}{\mathbb{I}}	\newcommand{\bbJ}{\mathbb{J}}
\newcommand{\bbK}{\mathbb{K}}	\newcommand{\bbL}{\mathbb{L}}
\newcommand{\bbM}{\mathbb{M}}	\newcommand{\bbN}{\mathbb{N}}
\newcommand{\bbO}{\mathbb{O}}	\newcommand{\bbP}{\mathbb{P}}
\newcommand{\bbQ}{\mathbb{Q}}	\newcommand{\bbR}{\mathbb{R}}
\newcommand{\bbS}{\mathbb{S}}	\newcommand{\bbT}{\mathbb{T}}
\newcommand{\bbU}{\mathbb{U}}	\newcommand{\bbV}{\mathbb{V}}
\newcommand{\bbW}{\mathbb{W}}	\newcommand{\bbX}{\mathbb{X}}
\newcommand{\bbY}{\mathbb{Y}}	\newcommand{\bbZ}{\mathbb{Z}}

%---------------------------------------
% MathCal Fonts :-
%---------------------------------------

%Captital Letters
\newcommand{\mcA}{\mathcal{A}}	\newcommand{\mcB}{\mathcal{B}}
\newcommand{\mcC}{\mathcal{C}}	\newcommand{\mcD}{\mathcal{D}}
\newcommand{\mcE}{\mathcal{E}}	\newcommand{\mcF}{\mathcal{F}}
\newcommand{\mcG}{\mathcal{G}}	\newcommand{\mcH}{\mathcal{H}}
\newcommand{\mcI}{\mathcal{I}}	\newcommand{\mcJ}{\mathcal{J}}
\newcommand{\mcK}{\mathcal{K}}	\newcommand{\mcL}{\mathcal{L}}
\newcommand{\mcM}{\mathcal{M}}	\newcommand{\mcN}{\mathcal{N}}
\newcommand{\mcO}{\mathcal{O}}	\newcommand{\mcP}{\mathcal{P}}
\newcommand{\mcQ}{\mathcal{Q}}	\newcommand{\mcR}{\mathcal{R}}
\newcommand{\mcS}{\mathcal{S}}	\newcommand{\mcT}{\mathcal{T}}
\newcommand{\mcU}{\mathcal{U}}	\newcommand{\mcV}{\mathcal{V}}
\newcommand{\mcW}{\mathcal{W}}	\newcommand{\mcX}{\mathcal{X}}
\newcommand{\mcY}{\mathcal{Y}}	\newcommand{\mcZ}{\mathcal{Z}}


%---------------------------------------
% Bold Math Fonts :-
%---------------------------------------

%Captital Letters
\newcommand{\bmA}{\boldsymbol{A}}	\newcommand{\bmB}{\boldsymbol{B}}
\newcommand{\bmC}{\boldsymbol{C}}	\newcommand{\bmD}{\boldsymbol{D}}
\newcommand{\bmE}{\boldsymbol{E}}	\newcommand{\bmF}{\boldsymbol{F}}
\newcommand{\bmG}{\boldsymbol{G}}	\newcommand{\bmH}{\boldsymbol{H}}
\newcommand{\bmI}{\boldsymbol{I}}	\newcommand{\bmJ}{\boldsymbol{J}}
\newcommand{\bmK}{\boldsymbol{K}}	\newcommand{\bmL}{\boldsymbol{L}}
\newcommand{\bmM}{\boldsymbol{M}}	\newcommand{\bmN}{\boldsymbol{N}}
\newcommand{\bmO}{\boldsymbol{O}}	\newcommand{\bmP}{\boldsymbol{P}}
\newcommand{\bmQ}{\boldsymbol{Q}}	\newcommand{\bmR}{\boldsymbol{R}}
\newcommand{\bmS}{\boldsymbol{S}}	\newcommand{\bmT}{\boldsymbol{T}}
\newcommand{\bmU}{\boldsymbol{U}}	\newcommand{\bmV}{\boldsymbol{V}}
\newcommand{\bmW}{\boldsymbol{W}}	\newcommand{\bmX}{\boldsymbol{X}}
\newcommand{\bmY}{\boldsymbol{Y}}	\newcommand{\bmZ}{\boldsymbol{Z}}
%Small Letters
\newcommand{\bma}{\boldsymbol{a}}	\newcommand{\bmb}{\boldsymbol{b}}
\newcommand{\bmc}{\boldsymbol{c}}	\newcommand{\bmd}{\boldsymbol{d}}
\newcommand{\bme}{\boldsymbol{e}}	\newcommand{\bmf}{\boldsymbol{f}}
\newcommand{\bmg}{\boldsymbol{g}}	\newcommand{\bmh}{\boldsymbol{h}}
\newcommand{\bmi}{\boldsymbol{i}}	\newcommand{\bmj}{\boldsymbol{j}}
\newcommand{\bmk}{\boldsymbol{k}}	\newcommand{\bml}{\boldsymbol{l}}
\newcommand{\bmm}{\boldsymbol{m}}	\newcommand{\bmn}{\boldsymbol{n}}
\newcommand{\bmo}{\boldsymbol{o}}	\newcommand{\bmp}{\boldsymbol{p}}
\newcommand{\bmq}{\boldsymbol{q}}	\newcommand{\bmr}{\boldsymbol{r}}
\newcommand{\bms}{\boldsymbol{s}}	\newcommand{\bmt}{\boldsymbol{t}}
\newcommand{\bmu}{\boldsymbol{u}}	\newcommand{\bmv}{\boldsymbol{v}}
\newcommand{\bmw}{\boldsymbol{w}}	\newcommand{\bmx}{\boldsymbol{x}}
\newcommand{\bmy}{\boldsymbol{y}}	\newcommand{\bmz}{\boldsymbol{z}}

%---------------------------------------
% Scr Math Fonts :-
%---------------------------------------

\newcommand{\sA}{{\mathscr{A}}}   \newcommand{\sB}{{\mathscr{B}}}
\newcommand{\sC}{{\mathscr{C}}}   \newcommand{\sD}{{\mathscr{D}}}
\newcommand{\sE}{{\mathscr{E}}}   \newcommand{\sF}{{\mathscr{F}}}
\newcommand{\sG}{{\mathscr{G}}}   \newcommand{\sH}{{\mathscr{H}}}
\newcommand{\sI}{{\mathscr{I}}}   \newcommand{\sJ}{{\mathscr{J}}}
\newcommand{\sK}{{\mathscr{K}}}   \newcommand{\sL}{{\mathscr{L}}}
\newcommand{\sM}{{\mathscr{M}}}   \newcommand{\sN}{{\mathscr{N}}}
\newcommand{\sO}{{\mathscr{O}}}   \newcommand{\sP}{{\mathscr{P}}}
\newcommand{\sQ}{{\mathscr{Q}}}   \newcommand{\sR}{{\mathscr{R}}}
\newcommand{\sS}{{\mathscr{S}}}   \newcommand{\sT}{{\mathscr{T}}}
\newcommand{\sU}{{\mathscr{U}}}   \newcommand{\sV}{{\mathscr{V}}}
\newcommand{\sW}{{\mathscr{W}}}   \newcommand{\sX}{{\mathscr{X}}}
\newcommand{\sY}{{\mathscr{Y}}}   \newcommand{\sZ}{{\mathscr{Z}}}


%---------------------------------------
% Math Fraktur Font
%---------------------------------------

%Captital Letters
\newcommand{\mfA}{\mathfrak{A}}	\newcommand{\mfB}{\mathfrak{B}}
\newcommand{\mfC}{\mathfrak{C}}	\newcommand{\mfD}{\mathfrak{D}}
\newcommand{\mfE}{\mathfrak{E}}	\newcommand{\mfF}{\mathfrak{F}}
\newcommand{\mfG}{\mathfrak{G}}	\newcommand{\mfH}{\mathfrak{H}}
\newcommand{\mfI}{\mathfrak{I}}	\newcommand{\mfJ}{\mathfrak{J}}
\newcommand{\mfK}{\mathfrak{K}}	\newcommand{\mfL}{\mathfrak{L}}
\newcommand{\mfM}{\mathfrak{M}}	\newcommand{\mfN}{\mathfrak{N}}
\newcommand{\mfO}{\mathfrak{O}}	\newcommand{\mfP}{\mathfrak{P}}
\newcommand{\mfQ}{\mathfrak{Q}}	\newcommand{\mfR}{\mathfrak{R}}
\newcommand{\mfS}{\mathfrak{S}}	\newcommand{\mfT}{\mathfrak{T}}
\newcommand{\mfU}{\mathfrak{U}}	\newcommand{\mfV}{\mathfrak{V}}
\newcommand{\mfW}{\mathfrak{W}}	\newcommand{\mfX}{\mathfrak{X}}
\newcommand{\mfY}{\mathfrak{Y}}	\newcommand{\mfZ}{\mathfrak{Z}}
%Small Letters
\newcommand{\mfa}{\mathfrak{a}}	\newcommand{\mfb}{\mathfrak{b}}
\newcommand{\mfc}{\mathfrak{c}}	\newcommand{\mfd}{\mathfrak{d}}
\newcommand{\mfe}{\mathfrak{e}}	\newcommand{\mff}{\mathfrak{f}}
\newcommand{\mfg}{\mathfrak{g}}	\newcommand{\mfh}{\mathfrak{h}}
\newcommand{\mfi}{\mathfrak{i}}	\newcommand{\mfj}{\mathfrak{j}}
\newcommand{\mfk}{\mathfrak{k}}	\newcommand{\mfl}{\mathfrak{l}}
\newcommand{\mfm}{\mathfrak{m}}	\newcommand{\mfn}{\mathfrak{n}}
\newcommand{\mfo}{\mathfrak{o}}	\newcommand{\mfp}{\mathfrak{p}}
\newcommand{\mfq}{\mathfrak{q}}	\newcommand{\mfr}{\mathfrak{r}}
\newcommand{\mfs}{\mathfrak{s}}	\newcommand{\mft}{\mathfrak{t}}
\newcommand{\mfu}{\mathfrak{u}}	\newcommand{\mfv}{\mathfrak{v}}
\newcommand{\mfw}{\mathfrak{w}}	\newcommand{\mfx}{\mathfrak{x}}
\newcommand{\mfy}{\mathfrak{y}}	\newcommand{\mfz}{\mathfrak{z}}

\title{\Huge{21-235 Math Studies Analysis I}}
\author{\huge{Rohan Jain}}
\date{}

\begin{document}

\maketitle
\newpage% or \cleardoublepage
% \pdfbookmark[<level>]{<title>}{<dest>}
\pdfbookmark[section]{\contentsname}{toc}
\tableofcontents

\pagebreak
\chapter{269}
Insert 269 here.
\chapter{Dedekind Stuff}
\section{Ordered Fields (Review)}

\dfn{Order}{Let $E$ be a set. An \emph{order} on $E$ is a relation $<$ on $E$ such that for all $x, y, z \in E$:
\begin{enumerate}
    \item (Trichotomy) Exactly one of the following holds: $x < y$, $x = y$, or $x > y$.
    \item (Transitivity) If $x < y$ and $y < z$, then $x < z$.
\end{enumerate}}

\ex{Examples of Ordered Sets}{\begin{enumerate}
    \item This definition develops orders on basic number systems: e.g. $\ZZ$, $\QQ$, and $\RR$.
    \item Define $\lesssim$ on $\ZZ$ as follows: We say that $m \lesssim n$ for $m, n \in \ZZ$ if:
\begin{enumerate}
    \item $m$ is even and $n$ is odd
    \item $m,n$ are even and $m < n$
    \item $m,n$ are odd and $m < n$. 
\end{enumerate}
\end{enumerate}}
\noindent Key Concepts: 
\begin{itemize}
    \renewcommand\labelitemi{--}
    \item upper/lower bounds of sets
    \item bounded sets
    \item max/min
    \item supremum/infimum
    \item supremum/infimum property: An ordered set $E$ satisfies such a property if every nonempty set $A \subseteq E$ that's bounded above/below has a supremum/infimum in $E$.
    \item Fact: sup prop $\implies$ inf prop
\end{itemize}

\dfn{Ordered Field}{Let $\FF$ be a field with order $<$. We say that $\FF$ is an \emph{ordered field} provided that:
\begin{enumerate}
    \item For all $x, y, z \in \FF$, if $x < y$, then $x + z < y + z$.
    \item For all $x, y \in \FF$, if $0 < x$ and $0 < y$, then $0 < x \cdot y$.
\end{enumerate}}

\ex{}{$\QQ$ is a field.}
\noindent Facts of any ordered field:
\begin{enumerate}
    \item $0 < 1$
    \item $\nexists x \in \FF$ such that $x^2 = -1$.
\end{enumerate}

\dfn{Ordered Subfield, Homomorphism, Isomorphism}{Let $\FF$ be an ordered field. 
\begin{enumerate}
    \item A set $\mathbb K \subseteq \FF$ is called an \emph{ordered subfield} if $mathbb K$ is an algeraic subfield and $\mathbb K$ is an ordered field equipped with $<$ from $\FF$.
    \item Let $\mathbb G$ be an ordered field and let $f : \mathbb F \to \mathbb G$. We say that $f$ is an \emph{ordered field homomorphism} if it's a field homomorphism and $f(x) < f(y)$ whenever $x < y$.
    \item $f$ is an ordered field isomorphism if $f$ is an ordered field homomorphism and $f$ is bijective.
\end{enumerate}}
\nt{
\begin{enumerate}
    \item If $f : \mathbb F \to \mathbb G$ is an ordered field homomorphism, $f(\mathbb F)$ is an ordered subfield of $\mathbb G$.
    \item OF property $\implies$ $f$ is injective.
    \item $\therefore$ every ordered field homomorphism $f : \mathbb F \to \mathbb G$ is such that $f$ induces a bijection $f: \mathbb F \to f(\mathbb F) \subseteq \mathbb G$.
\end{enumerate}
}

\thm{$\mathbb Q$ is the smallest ordered field. More precisely, if $\FF$ is an ordered field, then there exists a canonical ordered field homomorphism $f : \QQ \to \FF$.}

\noindent Upshot/notation abuse: We identify $f(\QQ) = \QQ$ to view $\QQ \subseteq \FF$. 
In turn, $\NN \subset \ZZ \subset \QQ \subseteq \FF$. 


\section{Types of Ordered Fields}
\dfn{Archimedean, Dedekind complete}{Let $\FF$ be an ordered field.
\begin{enumerate}
    \item We say that $\FF$ is Archimedean if $\forall 0 < x \in \FF$, $\exists n \in \NN$ such that $n > x$.
    \item We say that $\FF$ is Dedekind complete if it satisfies the supremum property.
\end{enumerate}}

\noindent Facts:
\begin{enumerate}
    \item $\QQ$ is Archimedean.
    \item If $\FF$ is Dedekind complete, then $\forall 0 < x \in \FF$ and $\forall 0< n \in \NN$, $\exists !$  $0 < y \in \FF$ such that $y^n = x$.
    \item $\QQ$ is not Dedekind complete. ($\sqrt 2$ is a counterexample.)
\end{enumerate}

\thm{}{Suppose $\FF$ is a Dedekind complete ordered field. Then $\FF$ is Archimedean.}
\begin{proof}
    If not, then $\NN \subset \FF$ is bounded above, and so the supremum property provides $x \in \FF$ such that $x = \sup \NN$. But then $x - 1$ is an upper bound for $\NN$, so there exists $n \in \NN$ such that $x-1 < n$. Hence $x < n + 1$, which contradicts the definition of $x$ as an upper bound. Therefore, $\FF$ is Archimedean.
\end{proof}

\section{Dedekind Completion}
Throughout this section, let $\FF$ be an Archimedean ordered field.

\dfn{Dedekind cut}{We say a set $\mathcal C \subseteq \FF$ is \emph{Dedekind cut} if:
\begin{enumerate}
    \item $\mathcal C \neq \emptyset$ and $\mathcal C \neq \FF$.
    \item If $p \in \mathcal C$ and $q \in \FF$ such that $q < p$, then $q \in \mathcal C$.
    \item If $p \in \mathcal C$, then $\exists r \in \mathcal C$ such that $p < r$.
\end{enumerate}
We will write $\FF^*$ for the set of all Dedekind cuts in $\FF$. It is called the \emph{Dedekind completion} of $\FF$.}

\nt{Let $\mathcal C \subseteq \FF$ be a cut. Then:
\begin{enumerate}
    \item If $p \in \mathcal C$, then $q \notin \mathcal C$, then $p < q$.
    \item If $r \notin \mathcal C$, and $r < s \in \FF$, then $s \notin \mathcal C$.
\end{enumerate}}

\ex{Cut examples}{\begin{enumerate}
    \item Let $q \in \FF$ and define $\mathcal C_q = \{ p \in \FF \mid p < q\}$. Then $\mathcal C_q$ is a cut.
    \begin{proof}
        \begin{enumerate}
            \item $q - 1 < q \implies q - 1 \in \mathcal C_q$. $q \not < q \implies q \notin \mathcal C_q \implies \mathcal C_q \neq \FF$.
            \item Let $p \in \mathcal C_q$. Suppose $s \in \FF$ such that $s < p$. Then $s < q \implies s \in \mathcal C_q$.
            \item Let $p \in \mathcal C_q$. Then $p < \frac{p + q}{2} < q \implies \frac{p + q}{2} \in \mathcal C_q$.
        \end{enumerate}
    \end{proof}
    \item Suppose $\FF$ is such that $\nexists x \in \FF$ such that $x^2 = 2$. Let $\mathcal C = \{ p \in \FF \mid p \leq 0 \text{ or } 0 < p^2 < 2\}$. Then $\mathcal C$ is a cut.
    \begin{proof}
        \begin{enumerate}
            \item $1 \in \mathcal C$ and $1^2 = 1 < 2$. $2 \notin \mathcal C$ and $2^2 = 4 > 2$. 
            \item Let $p \in \mathcal C$ and $q \in \FF$ such that $q < p$. If $q \leq 0$, then $q \in \mathcal C$ trivially. Suppose $0 < q < p$. Then $0 < q^2 < p^2 < 2$, so $q \in \mathcal C$.
            \item Let $p \in \mathcal C$. If $p \leq 0$, then $1 \in \mathcal C$ and $p < 1$, so we're done. Suppose $0 < p^2 < 2$. Note, $0 < 2 - p^2$, so $\frac{2p + 1}{2 - p^2} > 0$. Then we can define $r = 1 + \frac{2p + 1}{2 - p^2} \geq \max(1, \frac{2p + 1}{2 - p^2})$. Then $(p + 1/r)^2 = p^2 + \frac{2p}{r} + \frac{1}{r^2}$. We have:
            \begin{align*}
                p^2 + \frac{2p}{r} + \frac{1}{r^2} &< p^2 + \frac{2p}{r} + \frac 1r \\
                &= p^2 + \frac{2p+1}{r} \\
                &\leq p^2 + 2-p^2 \\
                &= 2.
            \end{align*}
            So, $p < p + 1/r < 2$ and $p + 1/r \in \mathcal C$.
        \end{enumerate}
    \end{proof}
\end{enumerate}}

\subsection{Ordering $\FF^*$}

\mlenma{}{The following hold:
\begin{enumerate}
    \item If $\mathcal A, \mathcal B \in \FF^*$, then exactly one holds:
    \begin{itemize}
        \item $\mathcal A \subset \mathcal B$
        \item $\mathcal A = \mathcal B$
        \item $\mathcal B \subset \mathcal A$
    \end{itemize}
    \item If $\mathcal A, \mathcal B, \mathcal C \in \FF^*$ and $\mathcal A \subset \mathcal B$ and $\mathcal B \subset \mathcal C$, then $\mathcal A \subset \mathcal C$.
\end{enumerate}}
\begin{proof}
    Proof of $2$ is trivial, as well as the equality part for $1$. 
    \begin{itemize}
        \item If $\mathcal A = \mathcal B$, we're done.
        \item Suppose $\exists b \in \mathcal B \setminus\mathcal A$. If $a \in \mathcal A$, then $a < b$, but $\mathcal B$ is a cut so $a \in \mathcal B$, so $\mathcal A \subset \mathcal B$.
        \item Suppose $\exists a \in \mathcal A \setminus \mathcal B$. Then $a < b$ for all $b \in \mathcal B$, so $a \in \mathcal B$, so $\mathcal B \subset \mathcal A$.
    \end{itemize}
\end{proof}


\dfn{Order on cuts}{Given $\mathcal A, \mathcal B \in \FF^*$, we say that $\mathcal A < \mathcal B$ if $\mathcal A \subset \mathcal B$. The lemma above shows that this is infact an order.}


\mlenma{}{Let $E \subseteq \FF^*$ be nonempty and bounded above. Then $\mathcal B = \bigcup_{\mathcal A \in E} \mathcal A$ is a cut.}
\begin{proof}
    \begin{enumerate}
        \item Since $E \neq \emptyset$, there exists $\mathcal A \in E$. So $\mathcal A \neq \emptyset$, hence $\mathcal B \neq \emptyset$. 
        
        Since $E$ is bounded above, there exists $\mathcal C \in \FF^*$ such that $\mathcal A \subset \mathcal C$ for all $\mathcal A \in E$. Since $\mathcal C$ is a cut, there is $q \in \FF$ such that $q \notin \mathcal C$. Then $q \notin \mathcal A$ for all $\mathcal A \in E$, so $q \notin \mathcal B$.
        \item Let $p \in \mathcal B$ and $q \in \FF$ such that $q < p$. Since $\mathcal B$ is a union of cuts, it follows that $p \in \mathcal A$ for some $\mathcal A \in E$. Since $\mathcal A$ is a cut, $q \in \mathcal A \subseteq \mathcal B$.
        \item Let $p \in \mathcal B$. Then $p \in \mathcal A$ for some $\mathcal A \in E$. Since $\mathcal A$ is a cut, there exists $r \in \mathcal A$ such that $p < r$. Since $\mathcal A \subset \mathcal B$, we have $r \in \mathcal B$.
    \end{enumerate}
\end{proof}

\thm{}{$\FF^*$ equipped with the order $<$ satisfies the supremum property.}
\begin{proof}
    Let $E \subseteq \FF$ be a nonempty set that is bounded above.  From last time, we know that $\mathcal B = \cup_{\mathcal A \in E} \mathcal A$ is a cut. We claim that $\mathcal B = \sup E$. 

    If $\mathcal A \in E$, then $\mathcal A \subseteq \mathcal B$. And so $\mathcal A \leq \mathcal B$, so $\mathcal B$ is an upper bound for $E$.

    Next, suppose that $\mathcal C \in \FF^*$ is an upper bound of $E$. This means that $\mathcal A \leq \mathcal C$ for every $\mathcal A \in E$, meaning $\mathcal A \subseteq \mathcal C \forall \mathcal A \in E$. So $\mathcal B \subseteq \mathcal C$. As such, $\mathcal B \leq \mathcal C$, so $\mathcal B = \sup E$.
\end{proof}\

\noindent Remark: In none of the results leading up to this theorem did we use that $\FF$ is anything other than an ordered set. This shows that the cut construction of Dedekind works in general for ordered sets and yields $\FF^*$ that satisfies the supremum property. Also, $\{ \mathcal C_p \mid p \in \FF\} \subseteq \FF^*$. 

\subsection{Addition}
Idea: $\FF \cong \{ \mathcal C_p \mid p \in \FF\}$. 

\mlenma{}{Let $\mathcal A, \mathcal B \in \FF^*$. Then $\mathcal C = \{a  + b \mid a \in \mathcal A, b \in \mathcal B\}$ is a cut.}
\begin{proof}
    Claim: $\mathcal A, \mathcal B \neq \emptyset \implies \mathcal C \neq \emptyset$.

    $\mathcal A, \mathcal B$ are cuts, so $\exists M_1, M_2 \in \FF$ such that $a < M_1$ for all $a \in \mathcal A$ and $b < M_2$ for all $b \in \mathcal B$. Then $a + b < M_1 + M_2$ for all $a \in \mathcal A, b \in \mathcal B$, so $a + b < M_1 + M_2$, meaning $M_1 + M_2 \notin \mathcal C$.

    Also, let $c = a + b \in \mathcal C$ for $a \in \mathcal A, b \in \mathcal B$. Let $q < c \implies q-a < b \implies q - a \in \mathcal B$. Hence, $q = a + (q-a) \in \mathcal C$. 

    Thirdly, let $c = a + b \in\mathcal C$ for $a \in \mathcal A, b \in \mathcal B$. Since $\mathcal A, \mathcal B \in \FF^*$, $\exists r_a, r_b$ such that $a < r_a \in \mathcal A$, $b < r_b \in \mathcal B$. Then $c = a + b < r_a + r_b$, so $r_a + r_b \in \mathcal C$.

    As such, $\mathcal C$ is a cut.
\end{proof}
\noindent Before we define addition, we need to define the negative of a cut.


Heuristic: What we want is that $-\mathcal C_1 = \mathcal C_{-1}$. The way we do this is by defining $\mathcal C_{-p} = \{ q \in \FF \mid \exists p > q : p \in -\mathcal C_p^C\}$. This is the same as $\{ q\in \FF \mid \exists p > q : -p \notin \mathcal C_p\}$. 

Now we study $\{ q \in \FF \mid \exists p > q : -p \notin \mathcal C\}$.

\mlenma{}{Let $\mathcal C \in \FF^*$. Then $\{ q \in \FF \mid \exists p > q : -p \notin \mathcal C\}$ is a cut.}

\dfn{Addition}{For $\mathcal A, \mathcal B \in \FF^*$, we define $\mathcal A + \mathcal B = \{ a + b \mid a \in \mathcal A, b \in \mathcal B\}$ and $-\mathcal A = \{ q \in \FF \mid \exists p > q : -p \notin \mathcal A\}$.}

\thm{}{Define $0 = \mathcal C_0 \in \FF^*$. The following hold:
\begin{enumerate}
    \item $\mathcal A, \mathcal B \in \FF^* \implies \mathcal A + \mathcal B \in \FF^*$.
    \item $\mathcal A, \mathcal B \in \FF^* \implies \mathcal A + \mathcal B = \mathcal B + \mathcal A$.
    \item $\mathcal A, \mathcal B, \mathcal C \in \FF^* \implies (\mathcal A + \mathcal B) + \mathcal C = \mathcal A + (\mathcal B + \mathcal C)$.
    \item $\mathcal A \in \FF^* \implies \mathcal A + 0 = \mathcal A$.
    \item $\mathcal A \in \FF^* \implies \mathcal A + (-\mathcal A) = 0$.
\end{enumerate}
}
\begin{proof}
    Easy proof, too lazy to write out.
\end{proof}
Also: $\mathcal A, \mathcal B, \mathcal C \in \FF^*$ and $\mathcal A < \mathcal B \implies \mathcal A + \mathcal C < \mathcal B + \mathcal C$.

Important Remark: The Archimedean property is actually needed for the above theorem in orer to prove the 5th condition. 

\newpage
\subsection{Multiplication}
\mlenma{}{Let $\mathcal A, \mathcal B \in \FF^*$ such that $\mathcal A, \mathcal B > 0$. Then $\mathcal C = \{ p \in \FF \mid p \leq 0\} \cup \{ ab \mid a \in \mathcal A, b \in \mathcal B, a, b > 0\}$ is a cut.}

\mlenma{}{Let $\mathcal A \in \FF^*$ be such that $\mathcal A > 0$. Then $\mathcal C = \{ p \in \FF^* \mid p \leq 0\}\cup \{ 0 < q \in \FF \mid \exists p > q : p^{-1} \notin \mathcal A\}$ is a cut.}

\dfn{Multiplication}{Let $\mathcal A, \mathcal B \in \FF^*$. We define multiplication as:
\begin{enumerate}
    \item If $\mathcal A , \mathcal B > 0$, then $\mathcal A \cdot \mathcal B = \{ ab \mid 0 < a \in \mathcal A, 0 < b \in \mathcal B\}$.
    \item If $\mathcal A = 0$  or $\mathcal B = 0$, then $\mathcal A \cdot \mathcal B = 0$.
    \item If $\mathcal A > 0$ and $\mathcal B < 0$, then $\mathcal A \cdot \mathcal B = -(\mathcal A \cdot (-\mathcal B))$.
    \item If $\mathcal A < 0$ and $\mathcal B > 0$, then $\mathcal A \cdot \mathcal B = -((-\mathcal A) \cdot \mathcal B)$.
    \item If $\mathcal A, \mathcal B < 0$, then $\mathcal A \cdot \mathcal B = (-\mathcal A) \cdot (-\mathcal B)$.
\end{enumerate}
We define multiplication inversion via:
\begin{enumerate}
    \item If $\mathcal A > 0$, then $\mathcal A^{-1} = \{ q \in \FF \mid \exists p > q : p^{-1} \notin \mathcal A\}$.
    \item If $\mathcal A < 0$, then $\mathcal A^{-1} = -(-\mathcal A)^{-1}$.
\end{enumerate}}
\thm{}{Set $1 = \mathcal C_1$. The following hold:
\begin{enumerate}
    \item If $\mathcal A, \mathcal B \in \FF^*$, then $\mathcal A \cdot \mathcal B \in \FF^*$.
    \item If $\mathcal A, \mathcal B \in \FF^*$, then $\mathcal A \cdot \mathcal B = \mathcal B \cdot \mathcal A$.
    \item If $\mathcal A, \mathcal B, \mathcal C \in \FF^*$, then $(\mathcal A \cdot \mathcal B) \cdot \mathcal C = \mathcal A \cdot (\mathcal B \cdot \mathcal C)$.
    \item If $\mathcal A \in \FF^*$, then $\mathcal A \cdot 1 = \mathcal A$.
    \item If $\mathcal A \in \FF^*$, then $\mathcal A \cdot \mathcal A^{-1} = 1$.
\end{enumerate}}

Also if $\mathcal A, \mathcal B \in \FF^*$ and $\mathcal A, \mathcal B > 0$, then $\mathcal A \cdot \mathcal B > 0$. 

\thm{}{If $\mathcal A, \mathcal B, \mathcal C \in \FF^*$, then $\mathcal A \cdot(\mathcal B + \mathcal C) = \mathcal A \cdot \mathcal B + \mathcal A \cdot \mathcal C$.}

We now know that $\FF^*$ is an ordered field.


\section{Robert Reci}

\thm{}{$\QQ$ is the smallest ordered field.}
\begin{proof}
    Let $\FF$ be any ordered field. Let $1 \in \FF$. Let $\iota : \NN \to \FF$, $n \mapsto 1 + \cdots + 1$ $n$ times. Then $\iota(-n) = -\iota(n)$ for $n \in \NN_0$ and $-n \in \ZZ^-$. 

    Then we say $\iota(p/q) = \iota(p)\iota(q)^{-1}$ for $p/q \in \QQ$.
\end{proof}

\cor{Every ordered field is infinite}{$\iota[\QQ] \subseteq \FF$ is infinite.}

\subsubsection{Roots}
Let $\FF$ be a Dedekind complete ordered field, $0 < x \in \FF$, $n \in \NN$. Then $\exists ! y \in \FF$ such that $y > 0$ and $y^n = x$.

\begin{proof}
    $n = 1$ is silly. Assume $n \geq 2$. Let $E = \{ z \in \FF \mid z > 0 \text{ and } z^n < x\}$. Then $E$ is nonempty and bounded above by $x$. Let $y = \sup E$. We claim that $y^n = x$.

    We want to show that $y^n \ngtr > x$ and $y^n \nless < x$. 

    \mlenma{}{In any commutative ring R, $b^n - a^n = (b - a)(b^{n-1} + b^{n-2}a + \cdots + ba^{n-2} + a^{n-1})$.}

    And hence for $0 < a < b$ in $\FF$, we have $0 < b^n - a^n = (b-a)nb^{n-1}$. 

    Suppose $y^n < x$, so $x-y^n > 0$. We define $h = \frac 12 \min\left(1, \frac{x-y^n}{n(y+1)^{n-1}}\right)$. $0 < h < 1$, also $0 < h < \frac{x-y^n}{n(y+1)^{n-1}}$. 

    Then, by the inequality below the lemma, we have \begin{align*}
        0 &< (y+h)^n - y^n \\
        &< hn(y + h)^{n-1} \\
        &< hn(y + 1)^{n-1} \\
        &< {x-y^n},
    \end{align*}
    so $(y+h)^n < x$, which contradicts the definition of $y$ as the supremum.
\end{proof}

\dfn{Ring*}{A ring is a field where actually we don't care about inverses anymore. }

\dfn{Domain}{$R$ is a domain when $xy = 0 \implies x = 0 \wedge y = 0$.}

Let $R$ be a ring. For $(r, s) \in R \times R \setminus \{0\}$, we say $(r, s) \sim (r', s')$ if $rs' = r's$.

The field of fractions, Frac$(R)$ is the set of equivalence classes of $R \times R \setminus \{0\}$ under $\sim$ equipped with the operations $[(r, s)] + [(r', s')] = [(rs' + r's, ss')]$ and $[(r, s)] \cdot [(r', s')] = (rr', ss')$.

We check that $[(r , s)] \cdot [(s, r)] = [(rs, sr)] = [(1, 1)]$.

Let $\FF$ a field, $\FF[x]$ its polynomial ring. Let $\FF(x)$ be the field of fractions of $\FF[x]$. Then $\FF(x) := \text{Frac}(\FF[x])$ is the field of rational functions in $x$ with coefficients in $\FF$.

Given $p, q \in \FF[x]$, say $p/q > 0$ if $p$ and $q$ have the same sign. Say $f, g \in \FF(x)$, that $f > g$ when $f- g > 0$. 

\thm{}{$\FF(x)$ is never Archimedean.}
\begin{proof}
    $x$ is an upper bound for all $n \in \NN$. 
\end{proof}

\nt{If $\FF$ is Archimedean, $|\FF| \leq 2^{\aleph_0}$.}

\thm{}{Let $\lambda$ be an infinite cardinal. Then there is an ordered field of cardinality $\lambda$.}

\cor{}{The Archimedean property is not a first-order property.}

\section{Completeness}
\mlenma{}{Suppose $\FF$ is an ordered field that is not Dedekind complete. Then $\exists$ and infinite $E \subseteq \FF$ such that:
\begin{enumerate}
    \item $E$ bounded above, $\emptyset \neq U(E)$ is open, $\emptyset \neq U(E)^C$ is open.
    \item $a \in U(E)^C$, $b \in U(E) \implies a < b$.
    \item $f: \FF \to \FF$ with $f(x) = \begin{cases} 1 & x \in U(E) \\ 0 & x \in U(E)^C \end{cases}$ is differentiable with $f' = 0$. 
\end{enumerate}}

\thm{Characteristics of Dedekind Completeness}{
    Let $\FF$ be an ordered field. The following are equivalent:
    \begin{enumerate}
        \item $\FF$ is Dedekind complete.
        \item $\FF$ has the intermediate value property: If $f: [a, b] \to \FF$ is continuous and $\min(f(a), f(b)) < c < \max(f(a), f(b))$, then $\exists x \in [a, b]$ such that $f(x) = c$.
        \item $\FF$ satisfies the mean value property: If $f: [a, b] \to \FF$ is continuous and differentiable on $(a, b)$, then $\exists x \in (a, b)$ such that $f'(x) = \frac{f(b) - f(a)}{b-a}$.
        \item $\FF$ satisfies Cauchy mean value property: If $f,g: [a, b] \to \FF$ are both continuous and differentiable on $(a, b)$, then $\exists x \in (a, b)$ such that $\frac{f'(x)}{g'(x)} = \frac{f(b) - f(a)}{g(b) - g(a)}$.
        \item $\FF$ satisfies the extreme value property: If $f: [a, b] \to \FF$ is continuous, then $f$ attains a maximum and minimum on $[a, b]$.
    \end{enumerate}
}
\begin{proof}
    $1 \implies 2$: Let $f: [a, b] \to \FF$ and continuous. WLOG, assume $f(a) < c < f(b)$. Define $E = \{x \in [a,b] \mid f(x) < c\}$. $E$ is nonempty and bounded above by $b$. Let $x = \sup E$. We claim that $f(x) = c$. Since $f$ is continuous, $\exists \kappa > 0$ such that $f(t) < c \, \forall t \in [a, a + \kappa]$ and $f(t) > c \, \forall t \in [b - \kappa, b]$. So, $a + \frac \kappa 2 < x < b - \frac \kappa 2$. 

    Suppose BWOC $f(x) < c$. Again by continuity, $\exists \delta > 0$ such that $f(t) < c$ for all $t \in B(x, \delta) \subseteq [a, b]$. Then $x + \frac \delta 2 \in E$, contradiction. 

    Then suppose BWOC $f(x) > c$. Again, $\exists \delta > 0$ such that $f(t) > c$ for all $t \in B(x, \delta) \subseteq [a, b]$. Then $\exists z \in E$ such that $x - \frac \delta 2 < z  \leq x$ and $f(z) < c$. But then $c < f(z) < c$, contradiction. 

    So $f(x) = c$ by trichotomy.

    $2 \implies 1$: We'll show $\neg 1 \implies \neg 2$. Suppose $\FF$ is not Dedekind complete. Then we can let $f : \FF \to \FF$ be the strange function from the lemma, and we can pick $a < b$ with $a \in U(E)^C$ and $b \in U(E)$. Then $f$ is continuous on $[a, b]$, $f(a) - < 1 = f(b)$, but there is not $x \in [a,b]$ with $f(x) = \frac 12$, by construction.

    $1 \implies 5$: First we claim that if $\FF$ is Dedekind and $f: [a, b]\to \FF$ is continuous, then $f([a, b]) \subseteq \FF$ is a bounded set. We prove the claim.

    Consider $E = \{x \in [a, b] \mid f([a, x]) \text{ is bounded}\}$. $a \in E$ and $E$ is bounded, so we can let $s = \sup E$. Next note that by continuity, if $[c,d] \subseteq [a, b]$ such that $f([c, d])$ is bounded, then $\exists \delta  > 0$ such that $f([a,b] \cap [c - \delta, d + \delta])$ is bounded. Using this, deduce in turn that $a < s$, $s = \max E$, and $s = b$. 

    So now suppose $\FF$ is Dedekind complete and let $f: [a,b] \to \FF$ be continuous. The claim establishes that $f([a, b]) \subseteq \FF$ is a bounded set, so we can let$
    \begin{cases} \mu = \inf f([a, b]) \\ \lambda = \sup f([a, b]) \end{cases}$. Suppose BWOC that $f(x) < \lambda$ for all $x \in [a, b]$. Then teh function $g: [a,b] \to \FF$ defined by $g(x) = \frac{1}{\lambda - f(x)}$ is continuous and positive. So by the claim, there is $k > 0$ such that $g(x) \leq k$ for all $x \in [a, b]$. But then 
    \begin{align*}
        \frac{1}{\lambda - f(x)} \leq k \implies \frac 1k \leq \lambda - f(x) \implies f(x) \leq \lambda - \frac 1k,
    \end{align*}
    for all $x \in [a,b]$. But this contradicts the definition of $\lambda$, as we just found a better upper bound.

    Therefore, there does exists $M \in [a,b]$ such that $f(M) = \lambda$, which is $\max f([a,b])$. 

    The min follows from a similar argument. 

    $5 \implies 4$: Let $f, g: [a,b] \to \FF$ be continuous and differentiable on $(a, b)$. Let $h:[a,b]\to \FF$ via $h(x) = f(x)(g(b) - g(a)) - g(x)(f(b) - f(a))$. It suffices to show $\exists x \in (a, b)$ such that $h'(x) = 0$.

    By construction, $h(a) = h(b)$. If $h(x) = h(a)$ for all $x \in [a,b]$, then $h' = 0$ and we're done. Suppose then that $h$ is not constant. Then EVT shows that $f$ attains its maximal/minimum values, and at least one must occur at the point $x \in (a, b)$, therefore $h'(x) = 0$.

    $4 \implies 3$: Let $g(x) = x$. Done.

    $3 \implies 1$. We'll show $\neg 1 \implies \neg 3$. Suppose $\FF$ is not Dedekind complete. Then we can let $f: \FF \to \FF$ be the function from the lemma, and we can pick $a < b$ with $a \in U(E)^C$ and $b \in U(E)$. Then consider the restriction $f:[a,b] \to \FF$. Then $1 = 1 - 0 = f(b) - f(a)$. Then, $f'(x) (b-a) = 0 \cdot(b-a) = 0$ for all $x \in \FF$. $0 \neq 1$ so $\neg 3$ as desired.
\end{proof}

\chapter{$\RR, \CC, \bar \RR$}
\thm{}{$\RR$ is uncountable.}
\begin{proof}
    $\QQ \subseteq \RR$, so $\RR$ is definitely infinite. Suppose BWOC that there was a bijection $f: \NN \to \RR$. Set $I_0 = [f(0) + 1, f(0) + 2]$ and not that $f(0) \notin I_0$. Suppose we are given closed, nested, non-singleton intervals $I_n \subseteq I_{n-1} \subseteq \cdots \subseteq I_0$ such that $f(k) \notin I_k$ for $0 \leq k \leq n$. If $f(n+1) \notin I_n$, then set $I_{n+1} = I_n$. Otherwise, set $I_{n+1}$ to some non-singleton closed interval contained in $I_n$ such that $f(n+1) \notin I_{n+1}$.

    Since $\RR$ is Dedekind complete, we have that $\bigcap_{n=0}^\infty I_n \neq \emptyset$. So, there is an $x$ such that $x \in I_n$ for all $n \in \NN$. But then $x \neq f(n)$ for all $n \in \NN$, contradiction since $f$ is a bijection.
\end{proof}
\nt{Upshot: Most of $\RR$ is transcendental over $\QQ$.}

\section{Extended Reals: $\bar \RR$}
\dfn{Extended Reals}{$\bar \RR = \RR \cup \{-\infty, \infty\}$. We endow $\bar \RR$ with the following order: We write $x< y$ for $x, y\in \bar \RR$ if:
\begin{enumerate}
    \item $x, y \in \RR$ and $x < y$.
    \item $x = -\infty$ and $y \in \bar \RR \setminus \{-\infty\}$.
    \item $x \in \bar\RR \setminus \{\infty\}$ and $y = \infty$.
\end{enumerate}}

\noindent Facts:
\begin{itemize}
    \item $(\bar \RR, <)$ is an ordered set that satisfies the supremum property.
    \item All sets in $\bar \RR$ are bounded above.
    \item All sets in $\bar \RR$ admit a sup/inf, i.e.
    \begin{itemize}
        \item $\sup: \mathcal P(\bar \RR) \to \bar \RR$.
        \item $\inf: \mathcal P(\bar \RR) \to \bar \RR$.
    \end{itemize}
    Note: $\sup \emptyset = -\infty$ and $\inf \emptyset = \infty$. Also, $A \subseteq B \subseteq \bar \RR$ implies $\sup A \leq \sup B$ and $\inf A \geq \inf B$. And if $E \neq \emptyset$, then $\inf E \leq \sup E$. 
\end{itemize}
\nt{$\bar \RR$ isn't an OF because if it were, then it would be Dedekind complete and then there would exists an ordered field isomorphism $f: \RR \to \RR$ such that $f(x) = \infty$ for some $x \in \RR$. but then $f(x+1) = f(x) + f(1) = \infty + 1 = \infty$, which is not a true statement. }

\dfn{}{We endow $\bar \RR$ with the following ``algebra.''
\begin{enumerate}
    \item If $x \in \RR$, we set $x + \infty = \infty + x = \infty$.
    \item If $x \in \RR$, we set $x + (-\infty) = (-\infty) + x = -\infty$.
    \item $\infty + \infty = \infty$.
    \item $-\infty + (-\infty) = -\infty$.
    \item If $0 < x \in \bar \RR$, we set $x \cdot \infty = \infty \cdot x = \infty$.
    \item If $0 < x \in \bar \RR$, we set $x \cdot (-\infty) = (-\infty) \cdot x = -\infty$.
    \item If $0 > x \in \bar \RR$, we set $x \cdot \infty = \infty \cdot x = -\infty$.
    \item If $0 > x \in \bar \RR$, we set $x \cdot (-\infty) = (-\infty) \cdot x = \infty$.
    \item If $x \in \RR$, we set $\frac{x}{\infty} = \frac{x}{-\infty} = 0$.
    \item $\infty^{-1} = 0 = (-\infty)^{-1}$.
    \item If $0 < x \in \bar \RR$, we set $\frac x0 = \infty$.
    \item If $0 > x \in \bar \RR$, we set $\frac x0 = -\infty$.
\end{enumerate}}
\noindent Forbidden/undefined: $\infty + (-\infty)$, $\infty \cdot 0$, $\frac 00$, $\frac{\pm \infty}{\pm \infty}$, $\frac{\pm \infty}{\mp \infty}$.

\subsection{Sequences in $\bar \RR$}
\dfn{Sequence}{A sequence in $\bar \RR$ is $\{x_n \}^\infty_{n = \ell} \subseteq \bar \RR$ for $\ell \in \ZZ$.}

\noindent In turn, we define new sequences $\{a_N\}^\infty_{N = \ell}, \{b_N\}^\infty_{N = \ell}  \subseteq \bar \RR$:
\begin{itemize}
    \item $a_N = \inf \{x_n \mid n \geq N\}$.
    \item $b_N = \sup \{x_n \mid n \geq N\}$.
\end{itemize}
We then set $\liminf_{n \to \infty} x_n = \sup_{N \geq \ell} \inf_{n \geq N} x_n = \sup_{N \geq \ell} a_N$ and $\limsup_{n \to \infty} x_n = \inf_{N \geq \ell} \sup_{n \geq N} x_n = \inf_{N \geq \ell} b_N$.


\ex{}{Let $x_n = \begin{cases} (-1)^n & n \equiv 0 \mod 2 \\ n & n \equiv 1 \mod 2\end{cases}$. Then, $\limsup_{n \to \infty} x_n = \infty$ and $\liminf_{n \to \infty} x_n = 1$.}

\mprop{}{Let $\{x_n\}_{n = \ell}^\infty \subseteq \bar \RR$. Then $\liminf_{n \to \infty} x_n \leq \limsup_{n \to \infty} x_n$.}
\begin{proof}
    Let $M,N \geq \ell$ and $K = \max(M,N)$. Then, $\inf_{n > N} x_n \leq \inf_{n > K} x_n \leq \sup_{n \geq K} x_n \leq \sup_{n \geq M} x_n$.

    Thus, $\liminf_{n \to \infty} x_n  = \sup_{N \geq \ell} \inf_{n \geq N} x_n \leq \sup_{n \geq M} x_n$ for all $M \geq \ell$. So, $\liminf_{n \to \infty} x_n \leq \limsup_{n \to \infty} x_n$.
\end{proof}


\mprop{}{Let $a_n, b_n \in \bar \RR$ and suppose $\exists K \geq \ell$ such that $a_n \leq b_n$ for all $n \geq K$. Then, $\liminf_{n \to \infty} a_n \leq \liminf_{n\to \infty} b_n$ and $\limsup_{n \to \infty} a_n \leq \limsup_{n \to \infty} b_n$.}
\begin{proof}
    We can claim that if $k \geq K$, then 
    \begin{align*}
        \inf \{ a_n \mid n \geq k\} \leq \inf \{ b_n \mid n \geq k\} \\ 
        \sup \{ b_n \mid n \geq k\} \leq \sup \{ a_n \mid n \geq k\}.
    \end{align*}
    Indeed, if $\exists k \geq K$ such that $\inf \{ a_n \mid n \geq k\} > \inf \{ b_n \mid n \geq k\}$, then $\exists m \geq k$ such that $b_m < \inf\{a_n \mid n \geq k\} \leq a_m \leq b_m$, contradiction. Ditto for $\sup$. 

    Now define for $N \geq \ell$, $C_N = \inf_{n \geq N} a_n$, $D_N = \inf_{n \geq N} b_N$, $E_N = \sup_{n \geq N} a_n$, and $F_N = \sup_{n \geq N} b_n$. 

    The above claims show that $N \geq K$ then $C_N \leq D_N$ and $E_N \leq F_N$. Then we iterate to learn:
    \begin{align*}
        \liminf_{n \to \infty} a_n = \sup_{N \geq \ell} C_N \leq \sup_{N \geq \ell} D_N = \liminf_{n \to \infty} b_n \\
        \limsup_{n \to \infty} a_n = \inf_{N \geq \ell} E_N \leq \inf_{N \geq \ell} F_N = \limsup_{n \to \infty} b_n.
    \end{align*}
\end{proof}

\thm{}{Suppose $a_n, b_n \in \bar \RR$. The following hold:
\begin{enumerate}
    \item If $\limsup_{n \to \infty} a_n < x \in \bar \RR$, then $\exists N \geq \ell$ such that $a_n < x$ for all $n \geq N$.
    \item If $\liminf_{n \to \infty} a_n > x \in \bar \RR$, then $\exists N \geq \ell$ such that $a_n > x$ for all $n \geq N$.
    \item $\liminf_{n \to \infty} a_n = - \limsup_{n \to \infty} -a_n$.
    \item $\limsup_{n \to \infty} a_n = - \liminf_{n \to \infty} -a_n$.
    \item $\limsup_{n \to \infty} a_n + b_n \leq \limsup_{n \to \infty} a_n + \limsup_{n \to \infty} b_n$, provided that all arithmetic operations are well-defined. 
    \item $\liminf_{n \to \infty} a_n + \liminf_{n \to \infty} b_n \leq \liminf_{n \to \infty} a_n + b_n$, provided that all arithmetic operations are well-defined.
\end{enumerate}}

\begin{proof}
    \begin{enumerate}
        \item Suppose $\limsup_{n\to\infty}a_n = \inf_{N \geq \ell} \sup_{n\geq N} a_n < x$. This implies that $\exists N \geq \ell$ such that $\sup_{n \geq N} a_n < x$, meaning $a_n < x$ for all $n \geq N$.
        \item Similar as above.
        \item For any $\emptyset \neq X \subseteq \FF$, we have that $-\sup(-X) = \inf X$ and $-\inf(-X) = \sup X$. So the result follows.
        \item Same as above. 
        \item We break into cases:
        \begin{enumerate}
            \item $\limsup a_n = \infty$ or $\limsup b_n = \infty$. Then $\limsup a_n + b_n = \infty  \geq \limsup a_n + \limsup b_n$.
            \item Suppose either $\limsup a_n = -\infty$ or $\limsup b_n = -\infty$. WLOG consider the first option. Since $\limsup b_n < \infty$, then there eixsts $N_1 \geq \ell$ and $K \geq \RR$ such that $b_n < K$ for $n \geq N_1$. Now let $m \in \NN$ and note that $-\infty < -m - K$. We can use the first result of the theorem to pick $N_2 \geq \ell$ such that $n \geq N_2 \implies a_n < -m -K$. Then, if $n \geq \max(N_1, N_2)$, we have $a_n + b_n < -m$, so $\limsup a_n + b_n = -\infty \leq \limsup a_n + \limsup b_n$.
            \item $\limsup a_n, \limsup b_n \in \RR$. Let $\epsilon > 0$, then $\exists N_1, N_2 \geq \ell$ such that $n \geq N_1 \implies a_n < \limsup a_n + \frac{\epsilon}{2}$ and $n \geq N_2 \implies b_n < \limsup b_n + \frac{\epsilon}{2}$. Then, $n \geq \max(N_1, N_2) \implies a_n + b_n < \limsup a_n + \limsup b_n + \epsilon$, so $\limsup a_n + b_n \leq \limsup a_n + \limsup b_n + \epsilon$ for all $\epsilon$.
        \end{enumerate}
        \item Same as above.
    \end{enumerate}
\end{proof}


\mlenma{}{Let $x_n \subseteq\RR$. The following are equivalent for $x \in \RR$:
\begin{enumerate}
    \item $x_n \to x$ as $n \to \infty$.
    \item $\liminf_{n \to \infty} x_n = \limsup_{n \to \infty} x_n = x$.
\end{enumerate}}
\begin{proof}
    Let $\epsilon > 0$. Then $\exists N \geq \ell$ such that $n \geq N \implies x - \epsilon < x_n < x + \epsilon$. Thus, $x - \epsilon \leq \liminf_{n \to \infty} x_n \leq \limsup_{n \to \infty} x_n \leq x + \epsilon$ for all $\epsilon > 0$. This implies that $\liminf_{n \to \infty} x_n = \limsup_{n \to \infty} x_n = x$.

    Now let $\epsilon > 0$. Then by the previous theorem, there exists $N_1 , N_2 \geq \ell$ such that $\begin{cases}
        x - \epsilon < x_n & n \geq N_1 \\
        x_n < x + \epsilon & n \geq N_2
    \end{cases}$. Thus, $n \geq \max(N_1, N_2) \implies x - \epsilon < x_n < x + \epsilon$, so $x_n \to x$ as $n \to \infty$.
\end{proof}

\dfn{}{Let $x_n \in \bar \RR$ and $x \in \bar \RR$. We say that $x_n \to x$ as $n \to \infty$ if $\liminf_{n \to \infty} x_n = \limsup_{n \to \infty} x$.}
\noindent Remarks:
\begin{enumerate}
    \item The lemma shows this extends the notion of convergence in $\RR$.
    \item Limits are unique, when they exist.
\end{enumerate}

\ex{}{
    \begin{enumerate}
        \item $\lim_{n \to \infty} n = \infty$ ($n \to \infty$ as $n \to \infty$).
        \item Version of squeeze lemma
        \item TFAE:
        \begin{itemize}
            \item $x_n \to \infty$ as $n \to \infty$. 
            \item $\liminf_{n \to \infty} x_n = \infty$.
            \item $\forall M \in \NN$, there exists $N \geq \ell$ such that $n \geq N \implies M \leq x_N$.
        \end{itemize}
    \end{enumerate}
}

\chapter{Metric Spaces}
\dfn{Metric}{Let $X$ be a nonempty set. A metric on $X$ is a function $d: X \times X \to \RR$ such that:
\begin{enumerate}
    \item $d(x,y) \geq 0$ for all $x,y \in X$, and $d(x,y) = 0 \iff x = y$.
    \item $d(x,y) = d(y,x)$ for all $x,y \in X$.
    \item $d(x,y) \leq d(x,z) + d(z,y)$ for all $x,y,z \in X$.
\end{enumerate}}
\dfn{}{A metric space is $(X, d)$ for $X \neq \emptyset$ and $d$ a metric on $X$.}

\ex{}{\begin{enumerate}
    \item $\RR$ with $d(x, y) = |x - y|$.
    \item $\CC$ with $d(x, y) = |x - y|$.
    \item (Discrete Metric) Let $X \neq \emptyset$ be any set. Then $d: X \times X \to \{0, 1\}$ defined by $d(x, y) = \begin{cases} 0 & x = y \\ 1 & x \neq y \end{cases}$ is a metric on $X$.
    \item Let $V$ be a normed metric space with norm $\Vert \cdot \Vert$. Then $d(x, y) = \Vert x - y \Vert$ is a metric on $V$.
    \item Suppose $(Y, d)$ is a metric space and suppose $f: X \to Y$ is an injection where $X \neq \emptyset$ is a set. Then $\sigma: X \times X \to \RR$ defined by $\sigma(x, y) = d(f(x), f(y))$ is a metric on $X$.
    \begin{proof}
        We need to show that $\sigma$ satisfies the three properties of a metric. 
        \begin{enumerate}
            \item $\sigma(x, y) \geq 0$ because $d \geq 0$ and $\sigma(x, y) = 0 \iff d(f(x), f(y)) = 0 \iff f(x) = f(y) \iff x = y$.
            \item The other two are very trivial.
        \end{enumerate}
    \end{proof}
    \item Let $Y$ be a metric space and $\emptyset \neq X \subseteq Y$. Then $d: X \times X \to \RR$ defined by $d(x, y) = d_Y(x, y)$ is a metric on $X$.
    \item Consider $f:(0, \infty) \to \RR$ and $g: (0, \infty) \to \RR$ with $f(x) = \log x$ and $g(x) = \frac 1x$. Then $d_f(x, y) = \left| \log \frac xy \right|$ and $d_g(x, y) = \left| \frac 1x - \frac 1y \right| = \frac{|x - y|}{|x||y|}$ are metrics on $(0, \infty)$.
    \item Let $V,W$ be finite dimensional vector spaces over $\FF \in \{ \RR, \CC\}$. Let $L(V, W) = \{T: V \to W: T \text{ linear}\}$. Then define $\text{rk}(T) = \dim \ran T$ for $T \in L(V, W)$. Note that $\ran (T + S) = \{Tx + Sx \mid  x \in \FF\} \subseteq \{Tx + Sy \mid x, y \in \FF\} = \ran T + \ran S$. Then, $\text{rk}(T + S) \leq \text{rk}(T) + \text{rk}(S)$.
    
    Define $d(T, S)  = \text{rk}(T-S) \in \NN \subseteq [0, \infty]$. 
    \begin{itemize}
        \item $d(T, S) = 0 \iff \text{rk}(T-S) = 0 \iff T -S = 0$.
        \item Has symmetry.
        \item Triangle inequality: $d(T-S) = \text{rk}(T - R + R - S) \leq \text{rk}(T - R) + \text{rk}(R - S) = d(T, R) + d(R, S)$.
    \end{itemize}
    \item Let $f: \bar RR \to [-1, 1]$ via $f(x) = \begin{cases} 1 & x = \infty \\ -1 & x = -\infty \\ \frac{x}{\sqrt{1 + x^2}} & x \in \RR \end{cases}$. Then $d(x, y) = |f(x) - f(y)|$ is a metric on $\bar \RR$.
\end{enumerate}}

\dfn{}{Let $X$ be a metric space.
\begin{enumerate}
    \item For $x \in X$ and $r \geq 0$, we define $B(x, r) = \{y \in X \mid d(x, y) < r \}$. And $B[x, r] = \{y \in X \mid d(x, y) \leq r \}$.
    \item A set $E \subseteq XX$ is bounded if $\exists (R \geq 0)$ such that $E \subseteq B(x, R)$ for some $x \in X$.
    \item Let $Y$ be any set and $f:  Y \to X$. We say $f$ is a bounded function if $f(Y) \subseteq X$ is bounded. We write $\mathcal B(Y; X) = \{g : Y \to X \mid g \text{ is bounded}\}$.
\end{enumerate}

}

\ex{}{
    \begin{enumerate}
        \item $f : \RR \to \CC$ via $f(t) = e^{it} \implies f(t) = 1 \implies f(\RR) \subseteq B[0, 1]$ is bounded. So, $f \in \mathcal B(\RR; \CC)$.
        \item $f:(0, \infty) \to \RR$ via $f(t) = \frac{\log t}{\sqrt{1 + (\log t)^2}}$. So, $f \in \mathcal B((0, \infty); \RR)$.
        \item Let $X$ be a metric space and $Y$ a nonempty set. Consider $\mathcal B(X ; Y)$. If $f \in \mathcal B(X; Y)$, then $\exists y \in Y$ and $R \geq 0$ such that $d(f(x), y) \leq R$ for all $x$. Thus, $\sup_{x \in X} d(f(x), y) := \sup \{d(f(x), y) \mid x \in X\} \in [0, R]$.
        
        Similarly, if $f, g \in \mathcal B(X; Y)$, then exists $R \geq 0$ and $y_1, y_2 \in Y$ such that $d(f(x), y_1) \leq R$ and $d(g(x), y_2) \leq R$ for all $x \in X$. Then, $d(f(x), g(x)) \leq d(f(x), y_1) + d(y_1, y_2) + d(y_2, g(x)) \leq 2R + d(y_1, y_2) < \infty$ for all $x \in X$. So, $\sup_{x \in X} d(f(x), g(x)) < \infty$. We now define
        \begin{align*}
            d: \mathcal B(X; Y) \times \mathcal B(X; Y) &\to [0, \infty) \\
            (f, g) &\mapsto \sup_{x \in X} d(f(x), g(x)).
        \end{align*}
        \begin{proof}
            Consider the properties of a metric:
            \begin{itemize}
                \item $d(f, g) = 0 \iff \sup_{x \in X} d(f(x), g(x)) = 0 \iff d(f(x), g(x)) = 0 \iff f(x) = g(x)$ for all $x \in X \iff f = g$.
                \item Symmetry is trivial.
                \item Let $f, g, h\in \mathcal B(X; Y)$. Then, $d(f, h) = \sup_{x \in X} d(f(x), h(x)) \leq \sup_{x \in X} d(f(x), g(x)) + d(g(x), h(x)) \leq d(f, g) + d(g, h)$.
            \end{itemize}
        \end{proof}
    \end{enumerate}
}


\dfn{}{Let $X$ and $Y$ be metric spaces:
\begin{enumerate}
    \item A map $f : X \to Y$ is an isometric embedding if $d_Y(f(x), f(y)) = d_X(x, y)$ for all $x, y \in X$. Note, such an $f$ is injective.
    \item $f$ is an isometry if it's an isometric embedding and surjective.
    \item $X$ and $Y$ are isometric if there exists an isometry $f: X \to Y$.
\end{enumerate}}

\ex{}{
    \begin{enumerate}
        \item Consider $\RR^n$ with $|\cdot| = \Vert \cdot \Vert_2$, that is, 2-norm. 
        \item Recall $O(n) = \{\mathcal M \in \RR^{n \times n} \mid \mathcal M^T \mathcal M = I \}$ and $R \in O(n) \implies |Rx| = |x|$. 
        
        Let $a \in \RR^n$, $R \in O(n)$, and set $f: \RR^n \to \RR^n$ via $f(x) = a + Rx$. Then, 
        \begin{align*}
            |f(x) - f(y)| = |a + Rx - (a + Ry)| = |Rx - Ry| = |R(x - y)|.
        \end{align*}
        Also, $y = f(x) = a + Rx \iff y - a = Rx$.  So, $f$ is an isometry.
        \item Consider $x \mapsto ix \in \CC$ for $x \in \RR$. This is an isometric embedding but obviously not an isometry for it is not surjective.
    \end{enumerate}
}
The next example is so important that we call it a theorem. Recall $\mathcal B(X) = \mathcal B(X; \RR)$ for $X \neq \emptyset$ is a set. Note that if $V$ is a normed vector space, then $\mathcal B(X; V)$ is too: $\Vert f \Vert_{\mathcal B} = \sup_{x \in X}\Vert f(x) \Vert_V$ is a norm (exercise) and $d_{\mathcal B}(f, g) = \Vert f -g \Vert_{\mathcal B}$. 

\thm{}{
    Let $X$ be a metric space and fix an arbitrary element $a \in X$. For $x \in X$, we'll define $\varphi_x : X \to \RR$ via $\varphi_x(y) = d(x, y) - d(y, a)$. The following hold:
    \begin{enumerate}
        \item $\varphi_x \in \mathcal B(X)$ for all $x \in X$.
        \item Define $\Phi : X \to \mathcal B(X)$ via $\Phi(x) = \varphi_x$. Then, $\Phi$ is an isometric embedding.
    \end{enumerate}
}
\begin{proof}
    First note, $|\varphi_x(y)| = |d(x, y) - d(y, a)| \leq d(x, a)$ by the triangle inequality. So, $\Vert \varphi_X \Vert_{\mathcal B} = \sup_{y \in X} |\varphi_x(y)| \leq d(x, a) < \infty$. This shows the first result.

    Next, fix $x, z \in X$ and consider $\varphi_x(y) - \varphi_z(y) = d(x, y) - d(y, a) - d(z, y) + d(y, a)$. So,
    \begin{align*}
        |\varphi_x(y) - \varphi_z(y)| = |d(x, y) - d(y, z)| \leq d(x, z).
    \end{align*}
    Thus, $d_{\mathcal B}(\varphi_x, \varphi_y) = \Vert \varphi_x - \varphi_y \Vert_{\mathcal B} = \sup_{y \in X} |\varphi_x(y) - \varphi_z(y)| \leq d(x, z)$.

    On the other hand, $|\varphi_x(z) - \varphi_z(z)| = |d(x, z) - \cancelto{0}{d(z, z)}| = d(x, z)$. So, $d_{\mathcal B}(\varphi_x, \varphi_z) = d(x, z)$. 
\end{proof}

\chapter{Basic Metric Space Topology}


FILL IN LATER 


\mprop{}{
    Let $Y_1, \ldots, Y_n$ be metric spaces and consider $Y = \prod_{i=1}^n Y_i$, endowed with a $p$-metric from Homework 3. That is,
    \begin{align*}
        d_p(x, y) = \begin{cases}
        \left( \sum_{i=1}^n d^p_{Y_i}(x_i, y_i)\right)^{1/p} & 1 \leq p < \infty \\
        \max_{1 \leq i \leq n} d^p_{Y_i}(x_i, y_i) & p = \infty
        \end{cases}.
    \end{align*}
    Suppose $\{y_k\}_{k = \ell}^\infty \subseteq Y$ is given by $y_k = (y_{k, 1}, \ldots, y_{k, n})$. The following hold:
    \begin{enumerate}
        \item Let $y = (y_1, \ldots, y_n) \in Y$. Then $y_k \to y$ in $Y$ as $n \to \infty \iff y_{k, i} \to y_i$ in $Y_i$ as $k \to \infty$ for all $1 \leq i \leq n$. 
        \item $\{y_k\}_{k = \ell}^\infty$ is Cauchy in $Y$ if and only if $\{y_{k, i}\}_{k = \ell}^\infty$ is Cauchy in $Y_i$ for all $1 \leq i \leq n$.
    \end{enumerate}
}
\begin{proof}
    We'll only prove 1. as 2. is very similar. Suppose $y_k \to y$ as $k \to \infty$. Note that for $ 1 \leq i \leq n$, $d_i(y_{k, i}, y_i) \leq d_Y(y_k, y)$. Thus, for $\epsilon > 0$, we pick $K \geq \ell$ such that if $k \geq K$, then $d_Y(y_k, y) \leq \epsilon$. But then $k \geq K \implies d_i(y_{k ,i}) \leq d_Y(y_k, y) \leq \epsilon$ for all $1 \leq i \leq n$, meaning $y_{k ,i} \to y_i$ as $k \to \infty$ for $1 \leq i \leq n$. 

    Now suppose $y_{k ,i} \to y_i$ as $k \to \infty$ for all $1 \leq i \leq n$. Let $\epsilon > 0$ and pick $K_i \geq \ell$ such that $k \geq K_i \implies d_i(y_{k ,i}, y_i) < \frac{\epsilon}{n^{1/p}}$. Let $K = \max K_i \geq \ell$, and note $k \geq K \implies d_i(y_{k ,i}, y_i) < \frac{\epsilon}{n^{1/p}}$ for all $1 \leq i \leq n$. This means 
    \begin{align*}
        \begin{cases}
            \left(\sum_{i=1}^n d_i^p(y_{k, i}, y_i)\right)^{1/p}  \leq \left(\sum_{i=1}^n \frac{\epsilon^p}{n}\right)^{1/p} = \epsilon & 1 \leq p < \infty \\
            \max_i d_i(y_{k, i}, y_i)  < \epsilon & p = \infty
        \end{cases}
    \end{align*}
    So, $y_k \to y$ as $k \to \infty$.
\end{proof}

\dfn{}{Let $X \neq \emptyset$ be a set and $d_1, d_2$ be metrics on $X$. We say $d_1$ and $d_2$ are equivalent if $\exists c_1, c_2 >0$ such that $c_1d_1(x, y) \leq d_2(x, y) \leq c_2d_1(x, y)$ for all $x, y \in X$.}
The point is that equivalent metrics give the same notions of convergence, Cauchyness, and boundedness. 
\newpage
\ex{Equivalent Norms}{
    \begin{enumerate}
        \item All norms on $\FF^n$ are equivalent. 
        \item From recitation, $\Vert \cdot \Vert_p$ are all equivalent on $\FF^n$ for $1 \leq p \leq \infty$. 
        \item Let $Y_1, \ldots, Y_n$ be metric spaces and form $Y = \prod_{i=1}^n Y_i$. Then
        \begin{align*}
            d_p(x, y) = \Vert (d_1(x, y), \ldots, d_n(x, y)) \Vert_p \asymp \Vert (d_1(x, y), \ldots, d_n(x, y)) \Vert_q = d_q(x, y)
        \end{align*}
        Therefore, $d_p \asymp d_q$ in $Y$. 
        
        Note: This does not mean all metrics on $Y$ are equivalent. 
    \end{enumerate}
}
\ex{}{
    Let $V_1, \ldots, V_n, W$ be normed vector sapces over $\FF$. We define $\mathcal L(V_1, \ldots, V_n; W)$ is the set of $\{ T \in L(V_1, \ldots, V_n ; W) \mid \Vert T \Vert_{\mathcal L}\ < \infty\}$ where $\Vert T \Vert_{\mathcal L} := \sup \{\Vert T(v_1, \ldots, v_n) \Vert_W \mid v_i \in V_i : \Vert v_i \Vert_{V_i} < 1\} \in [0, \infty]$. Facts:
    \begin{enumerate}
        \item This is indeed a norm.
        \item $T \in \mathcal L \iff \Vert T(v_1, \ldots, v_n) \Vert_W \leq c\prod_{i=1}^n \Vert v_i \Vert_{V_i}$ for all $v_i \in V_i$ for some $0 \leq c < \infty$. $c = \Vert T \Vert_{\mathcal L}$ is the best constant. 
    \end{enumerate}
}

\thm{Algebra of Sequences}{
    Let $V_1, \ldots, V_n, W$ be normed vector spaces over a common field $\FF$. The following hold:
    \begin{enumerate}
        \item Let $\{v_{k ,i}\}^\infty_{k = \ell} \subseteq V_i$ for $1 \leq i \leq n$ be such that $v_{k, i} \to v_i$ in $V_i$ as $k \to \infty$. Let $\{T_k\}_{k = \ell}^\infty \subseteq \mathcal L(V_1, \ldots, V_n; W)$ be such that $T_k \to T$ as $k \to \infty$. Then $T_k(v_{k, 1}, \ldots, v_{k, n}) \to T(v_1, \ldots, v_n)$ in $W$ as $k \to \infty$.
        \item If $\{u_k\}, \{v_k\} \subseteq V_1$ are such that $u_k \to u$, $v_k \to v$ then $u_k + v_k \to u + v$ as $k \to \infty$. 
    \end{enumerate}
}
\begin{proof}
    We'll only do $1$ because $2$ is easy. We start with $n=2$ for simplicity. Suppose $\{x_k\} \subseteq V_1$, $\{y_k\} \subseteq V_w$ such taht $x_k \to x$ and $y_k \to y$ as $k \to \infty$. Then let $\sup_{k \geq \ell} \max \{\Vert x_k \Vert_{V_1}, \Vert y_k \Vert_{V_2}$, $\Vert T_k \Vert_{\mathcal L} \} = M < \infty$. Then,
    \begin{align*}
        T_k(x_k, y_k) - T(x, y)&= T_k(x_k, y_k - y) + T_k(x_k, y) - T(x, y) \\ 
        T_k(x_k, y_k + y) + T(x_k - x, y) + T_k(x, y) - T(x, y).
    \end{align*}
    This shows that 
    \begin{align*}
        \Vert T_k(x_k, y_k) - T(x, y) \Vert_W &\leq \Vert T_k \Vert_{\mathcal L} \Vert x_k \Vert_{V_1} \Vert y - y_k \Vert_{V_2}  + \Vert T_k \Vert_{\mathcal L} \Vert x - x_k \Vert_{V_1} \Vert y_k \Vert_{V_2}  + \Vert T - T_k \Vert_{\mathcal L} \Vert x_k \Vert_{V_1} \Vert y_k \Vert_{V_2}  \\
        \leq M^2 \Vert y - y_k \Vert_{V_2} + M^2 \Vert x - x_k \Vert_{V_1}  + M^2 \Vert T - T_k \Vert_{\mathcal L} \to 0
    \end{align*}
    as $k \to \infty$. 
\end{proof}
\dfn{}{\begin{enumerate}
    \item We say a metric space $X$ is complete if every Cauchy sequence in $X$ is convergent in $X$.
    \item We say a normed vector space is Banach if it's complete.
    \item We say an inner product space is a Hilbert space if it's Banach.
\end{enumerate}}
\ex{}{
    \begin{enumerate}
        \item $(\RR, | \cdot|)$ is complete.
        \item $X = \prod_{i=1}^n X_i$ with $p$-metric is complete if and only if each $X_i$ is complete. In particular, $(\RR^n, \Vert\cdot\Vert)$ is complete. 
        \item $\FF^n$ is complete with any more. 
        \item $\RR \setminus \{0\}$ is not complete with $|\cdot|$ as the metric.
        \item $\QQ^n$ with $|\cdot|$ is not complete. 
    \end{enumerate}
}

\ex{}{
    \begin{enumerate}
        \item $V$ is a finite dimensional normed vector spaces. $\varphi : \FF^n \to V$ isomorphism. Then $\FF^n \ni x \mapsto \Vert \varphi(x) \Vert_V \in [0, \infty)$ defines a norm on $\FF^N$, which we call $\Vert | x |\Vert$. Then $(\FF^n, \Vert | \cdot | \Vert)$ is isometric to $(V, \Vert \cdot \Vert_V)$, and hence $V$ is complete. 
        \item Let $\emptyset \neq X$ be a set endowed with the discrete metric. Suppose $\{x_n \}^\infty_{n = \ell} \subseteq X$ is Cauchy and pick $N \geq \ell$ such that $n,m \geq N \implies d(x_n, x_m) < 1$. Then $x_n = x_m = x_N$. So $x_n \to x_N$ as $n \to \infty$. Therefore $X$ is complete. 
    \end{enumerate}
}
Note that $Y = \prod Y_i$ is complete iff each individual $Y_i$ is complete. 
\thm{}{
    Let $V_1, \ldots, V_k, W$ be normed vector spaces over $\FF$. If $W$ is Banach, then so is $\mathcal L(V_1, \ldots, V_k)$.
}
\begin{proof}
    Suppose $\{T_n\}^\infty_{n = \ell} \subseteq \mathcal L(V_1, \ldots, V_k ; W$ is Cauchy. For fixed $v_1, \ldots, v_k) \in \prod_{i=1}^k V_i$, we bound
    \begin{align*}
        \Vert T_n(v_1, \ldots, v_k) - T_m(v_1, \ldots, v_k)\Vert_W \leq \Vert T_n - T_m \Vert_{\mathcal L}\prod_{i=1}^k \Vert v_i \Vert_{V_i}.
    \end{align*}
    Therefore, $\{T_n(v_1, \ldots, v_k)\}^\infty_{n = \ell} \subseteq  W$ is Cauchy and hence convergent. We may thus define $T: V_1 \times \cdots \times V_k \to W$ via $T(v_1, \ldots, v_k) = \lim_{n \to \infty} T_n(v_1, \ldots, v_k)$.
    \begin{enumerate}
        \item $T \in L(V_1, \ldots, V_k; W)$:
        
        \begin{align*}
            T_n(\alpha x  + \beta y, v_2,\ldots, v_k) = \alpha T_n(x, v_2, \ldots, v_k) + \beta T_n(y, v_2, \ldots, v_k)
        \end{align*}
        

        As $n \to \infty$, we get:

        \begin{align*}
            T(\alpha x + \beta y, v_2, \ldots, v_k) = \alpha T(x, v_2, \ldots, v_k) + \beta T(y, v_2, \ldots, v_k).
        \end{align*}
        Repeat in other slots if $k \geq 2$. As such, it is multilinear.

        \item $T \in \mathcal L(V_1, \ldots, V_k; W)$: Fix $v_i \in V_i$ with $\Vert v_i \Vert_{V_i} \leq 1$. Then 
        \begin{align*}
            \Vert T(v_1, \ldots, v_k) \Vert_W &= \lim_{n \to \infty} \Vert T_n(v_1, \ldots, v_k) \Vert_W \\
            &\leq \left(\limsup_{n \to \infty} \Vert T_n \Vert_{\mathcal L} \right) \prod_{n=1}^\infty \Vert v_i \Vert_{V_i} \leq \limsup_{n\to \infty} \Vert T_n \Vert_{\mathcal L} < \infty.
        \end{align*}
        \item $T_n \to T$ in $\mathcal L$ as $n \to \infty$: Let $\epsilon > 0$ and pick $N \geq \ell$ such that $n , m \geq N \implies \Vert T_n - T_m \Vert_{\mathcal L} < \frac \epsilon 2$. Then let $v_i \in V_i$ with $\Vert v_i \Vert_{V_i} \leq 1$. Then,
        \begin{align*}
            \Vert T(v_1, \ldots, v_k) - T_n (v_1, \ldots, v_k) \Vert_W = \lim_{m \to \infty} \Vert T_m(v_1, \ldots, v_k) - T_n (v_1, \ldots, v_k) \Vert_W \leq \lim_{m \to \infty} \Vert T_m - T_n \Vert_{\mathcal L} < \frac \epsilon 2.
        \end{align*}
        But this implies 
        \begin{align*}
            \Vert T(v_1, \ldots, v_k) - T_n(v_1, \ldots, v_k)\Vert_W \leq \frac \epsilon 2.
        \end{align*}
        By taking the supremum, we get that $\Vert T - T_n \Vert_{\mathcal L} \leq \frac \epsilon 2 < \epsilon$. 
    \end{enumerate}
\end{proof}

\cor{}{
    $V^* = \mathcal L (V; \FF)$ is always Banach.
}

\dfn{}{
    Let $X$ be a metric space, $E \subseteq X$. 
    \begin{enumerate}
        \item $x \in E$ is an interior point if $\exists \epsilon > 0$ such that $B(x, \epsilon) \subseteq E$. $E^\circ = \{ x \in E \mid x \text{ is an interior point}\}$. $E$ is open iff $E = E^\circ$. $E$ is closed iff $E^c$ is open.
        \item $x \in X$ is a boundary point of $E$ if $\forall \epsilon > 0$, $B(x, \epsilon) \cap E \neq \emptyset$ and $B(x, \epsilon) \cap E^c = \emptyset$. We write $\partial E = \{ x \in X \mid x \text{ is a boundary point of } E\}$. $\bar E = E^\circ \cup \partial E$.
        \item We say $x \in X$ is a limit point (accumulation point) of $E$ if $\forall \epsilon > 0$ $(B(x, \epsilon) \cap E) \setminus \{ x \} \neq \emptyset$. We write $E' = \{ x \in X \mid x \text{ is a limit point of } E\}$. If $x \in E \setminus E'$, then $x$ is an isolated point. 
    \end{enumerate}
}

\ex{}{
    Let $(X, disc)$ be given. Claim: all subsets of $X$ are both open and closed. 
    \begin{proof}
        $B(x, 1) = \{x\} \implies E \subseteq X$ can be written as 
        \begin{align*}
            E = \cup_{x \in E} B(x, 1),
        \end{align*}
        which is open. Therefore $E = (E^c)^c$ is also closed.
    \end{proof}
    Any metric space in which all sets are open and closed is called a discrete space.
}

\thm{}{
    Let $X$ be a metric space and $C \subseteq X$. The following are equivalent:
    \begin{enumerate}
        \item $C$ is closed.
        \item $C$ is sequentially closed; If $\{x_n \}_{n = \ell}^\infty \subseteq C$ is such that $x_n \to x$ in $X$ as $n \to \infty$, then $x \in C$. 
    \end{enumerate}
}
\begin{proof}
    $1 \to 2$. Let $\{x_n\} \subseteq C$ be such that $x_n \to x \in X$. Suppose BWOC that $x \in C^c$, which is open. Then $\exists N \geq \ell$ such that $n \geq N \implies x_n \in C^c \cup C$, which is a contradiction.

    $2 \to 1$. BWOC, suppose that $C$ is not closed, which emans $C^c$ is not open. Then $\exists x \in C^c$ such that we can pick $\{x_n \}_{n = 0}^\infty \subseteq C$ such that $x_n \in B(x, 2^{-n}) \cap C$. This means that $\{x_n \}_{n = 0}^\infty \subseteq C$ and $x_n \to x$ as $n \to \infty$. But $x \notin C$, so we have a contradiction. 
\end{proof}

\cor{}{
    Let $X$ be a complete metric space, and $\emptyset \neq C \subseteq X$. Then $C$ is closed in $X$ iff $C$ is a complete metric space with the metric from $X$. 
}
\begin{proof}
    $\implies$: Let $\{x_n\}_{n = \ell}^\infty \subseteq C$ be Cauchy. Then $x_n \to x \in X$ as $n \to \infty$ because $X$ is complete. By since $C$ is closed, $x \in C$.

    $\impliedby$: Let $\{x_n \} \subseteq C$ be such that $x_n \to x$ in $X$ as $n \to \infty$. Then $\{x _n \}$ is cauchy in $C$, meaning it's convergent in $C$, so $x \in C$, so $C$ is sequentially closed. 
\end{proof}

\dfn{}{Let $X$ be a metric space and $A \subseteq B \subseteq X$. We say $A$ is dense in $B$ if $\forall b \in B$, $\exists \{a_n\} \subseteq A$ such that $a_n \to b$ as $n \to \infty$.}

\ex{}{
    \begin{enumerate}
        \item $\QQ$ is dense in $\RR$. $\QQ^n$ is dense in $\RR^n$. $(\QQ^n + i \QQ^n) \subseteq \CC^n$ is dense. 
        \item $B(x, r) \subseteq \RR^n$ is dense in $B[x, r]$. 
        \item Let $X$ be given the discrete metric. $B(x, 1) = \{x\}$, but $B[x, 1] = X$, so as long as $X \neq \{x\}$, we do not have $B(x, 1) \subseteq B[x, 1]$ is dense.
    \end{enumerate}
}

\mprop{}{Let $X$ be a metrid space, $A \subseteq B \subseteq X$. The following are equivalent: 
\begin{enumerate}
    \item $A$ is dense in $B$.
    \item $ B \subseteq \bar A$.
    \item $\forall x \in B$ and $\epsilon > 0$, $\exists a \in A$ such that $d(x, a) < \epsilon$.
    \item $\forall x \in B$ and $\epsilon > 0$, $B(x, \epsilon) \cap A \neq \emptyset$.
\end{enumerate}
}
\begin{proof}
    Recall $\bar A = A \cup A'$. 

    $1 \implies 2$. Let $b \in B$. If $b \in A$, we're done. Otherwise $b \notin A$, but by density $\exists \{ a_n\}_{n = \ell}^\infty \subseteq A \setminus \{b\}$ such that $a_n \to b$ as $n \to \infty$. Thus, $b \in A'$. 

    $2 \implies 1$. Suppose $B \subseteq A \cup A' = \bar A$. Let $b \in B$. If $b \in A$, let $\{ a \}_{n = \ell}^\infty = b$ then we're done. 

    So suppose $b \in A' \setminus A$. By definition of limit point, we can pick a sequence $\{a_n\}$ such that $a_n \to b$ as $n \to \infty$. So $A$ is dense in $B$.

    $3 \iff 4$ is trivial. 

    $2 \iff 3$. Again, use $\bar A = A \cup A'$. 
\end{proof}


\cor{}{
    Let $X$ be a metric space and $A \subseteq B \subseteq X$. If $A$ is dense in $B$, then $A$ is also dense in $\bar B$. 
}
\begin{proof}
    $A \subseteq B$ is dense $\implies A \subseteq B \subseteq \bar A$. So $\bar B \subseteq \bar A$, meaning $A$ is dense is $\bar B$ as desired. 
\end{proof}


\dfn{}{
    Let $X$ be a metric space. We say $X$ is separable if $X$ has a countable dense subset.
}

\ex{Separable Vector Spaces}{
    \begin{enumerate}
        \item $\RR^n$ is separable, ditto for $\CC^n$. 
        \item Let $V$ be a finite dimensional normed vector space. Let $\varphi: \FF^n \to V$ be an isomorphism. Endow $\FF^n$ with norm $\Vert | x | \Vert = \Vert \varphi(x) \Vert_V$, which is equiavalent to $|\cdot|$ on $\FF^n$. Then $V$ is separable with $\varphi(\QQ^n)$ as a countable dense subset. 
        \item $\ell^\infty(\NN; \FF)$ is not separable, but $\ell^p(\NN; \FF)$ is for $1 \leq p < \infty$. 
    \end{enumerate}
}


\dfn{}{Let $X, X^*$ be metric spaces. We say that $X^*$ completes $X$ if:
\begin{enumerate}
    \item $X^*$ is complete.
    \item $\exists f : X \to X^*$ an isometric embedding. 
    \item $f(x) \subseteq X^*$ is dense.
\end{enumerate}}

\thm{Uniqueness of completions}{
    Let $X,Y,Z$ be metric spaces. Suppose $Y$ and $Z$ both complete $X$. Then $Y$ and $Z$ are isometric.
}
\begin{proof}
    Let $g: X \to Y$ and $h: X \to Z$ be isometric embeddings. We will construct an isometric $f: Y \to Z$. Let $y \in Y$. Since $g(X) \subseteq Y$ is dense, $\exists \{y_n\}_{n = \ell}^\infty \subseteq g(X)$ such that $y_n \to y$ as $n \to \infty$.

    Then $\exists! \{x_n \}_{n = \ell}^\infty \subseteq X$ such that $g(x_n) = y_n$ for all $n \geq \ell$. Then upon setting $z_n  = h(x_n) = h \circ g^{-1}(y_n)$, we have 
    \begin{align*}
        d_Z(z_n, z_m) = d_X(x_n, x_m) = d_Y(y_n, y_m).
    \end{align*}
    This means $\{z_n\}$ is Cauchy, and therefore convergent as $Z$ is complete. 

    Suppose $\{y_n'\}_{n = \ell}^\infty$ is another sequence such that $y_n' \to y$ as $n \to \infty$. Note 
    \begin{align*}
        d_Y(y_n, y_n') = d_X(g^{-1}(y_n), g^{-1}(y_n')) = d_Z(h(g^{-1}(y_n)), h(g^{-1}(y_n'))) = d_Z(z_n, z_n').
    \end{align*}
    Therefore, $\lim_{n \to \infty} z_n = \lim_{n \to \infty} z_n'$. So, we can define $f: Y \to Z$ as $f(y) = \lim_{n \to \infty} h(g^{-1}(y_n))$ for any sequence $\{y_n\} \subseteq g(X)$ such that $y_n \to y$ as $n \to \infty$.

    We claim that $f$ is an isometric embedding. Let $y , y' \in Y$ and pick $\{y_n\}_{n = \ell}^\infty$ and $\{y'_n\}_{n = \ell}^\infty$ such that $y_n \to y$ and $y'_n \to y'$ as $n \to \infty$. Then,
    \begin{align*}
        d_Y(y_n, y_n') = d_X(g^{-1}(y_n), g^{-1}(y_n')) = d_Z(h(g^{-1}(y_n)), h(g^{-1}(y_n'))) \to d_Z(f(y), f(y')) = d_Y(y, y'),
    \end{align*}
    so $f$ is an isometric embedding.

    We claim that $f$ is surjective. Let $z \in Z$ and pick $\{x_n\}_{n = \ell}^\infty$ such that $h(x_n) = z_n \to z$ as $n \to \infty$. Then let $y_n = g(x_n)$. Then $\{y_n\}_{n = \ell}^\infty \subseteq Y$ are Cauchy and hence convergent to $y \in Y$. Then $f(y) =\lim_{n \to \infty} h \circ g^{-1}(y_n) = \lim_{n \to \infty} z_n = z$. So $f: Y \to Z$ is an isometry!
\end{proof}
\nt{This is analogous to the uniqueness of Dedekind complete ordered fields. In principal, there can be different techniques for finding /constructing completions of a given metric space, but in the end they're isometric.}
\newpage


\thm{}{
    Let $X \neq \emptyset$ be a set and $Y$ be a metric space. Then $\mathcal B(X; Y)$ is complete if and only if $Y$ is complete. 
}
\begin{proof}
    HW5
\end{proof}

\cor{}{
    Let $X \neq \emptyset$ be a set. Then $\mathcal B(X) = \mathcal B(X; \RR)$ is a Banach space.
}
\begin{proof}
    $\RR$ is complete. 
\end{proof}

\thm{}{
    Let $X$ be a metric space. Then $X$ has a completion.
}
\begin{proof}
    Let $\Phi: X \to \mathcal B(X)$ be the isometric embedding we previously constructed. Let $X^* = \overline{\Phi(X)}$, which is closed in $\mathcal (B)$ and hence a complete metric space. By construction, $\Phi(X)$ is dense in $X^*$. So, $X^*$ is complete.
\end{proof}
\noindent Remarks:
\begin{enumerate}
    \item Why not just set $\RR = \bar \QQ$? It's cyclic!
    \item $\exists$ another construction of $X^*$ which is more ``direct'' and proceeds through $\text{Cauchy}(X)$ from HW4. This idea has room to play. It can be hacked to yield an alternate construction of $\bar \RR$ from $\QQ$ or any other Archimedean ordered field. 
\end{enumerate}

\section{Limits and Continuity}
\dfn{}{
    Let $X,Y$ be metric spaces, $E \subseteq X$, $z \in E', f: E \to Y$. We say that $f$ has limit $y \in Y$ as $x \to z$, written as $f(x) \to y$ as $x \to z$ or $\lim_{x \to z} f(x) = y$ if for all $\epsilon > 0$, there exists $\delta > 0$, such that $x \in E$ and $0 < d_X(x, z) < \delta \implies d_Y(f(x), y) < \epsilon$.
}
\noindent Remarks:
\begin{enumerate}
    \item limits are unique when they exist
    \item the definition only requires $z \in E'$, not $z \in E$. that is, $f(z)$ doesn't need to be defined and even if it is, the defintion doesn't care what it is. 
\end{enumerate}
\newpage
\thm{Sequential characterization of limits}{Let $X,Y$ be metric spaces, $E \subseteq X$, $f : E \to Y$, $z \in E'$, $y \in Y$. The following are equivalent:
\begin{enumerate}
    \item $f(x) \to y$ as $x \to z$.
    \item $\forall \epsilon > 0$, $\exists \delta > 0$ such that $f(B(z, \delta) \setminus \{z\}) \subseteq B_Y(y, \epsilon)$.
    \item If $\{x_n\}_{n = \ell}^\infty \subseteq E \setminus \{x\}$ is such that $x_n \to z$ as $n \to \infty$, then $f(x_n) \to y$ as $n \to \infty$. 
\end{enumerate}}
\begin{proof}
    $1 \iff 2$ is a triviality. Now we show $1 \implies 3$. Let $\{x_n\}_{n = \ell}^\infty \subseteq E \setminus \{z\}$ be such that $x_n \to z$ as $n \to \infty$. Let $\epsilon > 0$ and pick $\delta > 0$ such that $x \in E$ and $0 < d_X(x, z) < \delta \implies d_Y(f(x), y) < \epsilon$. Pick $N \geq \ell$ such that $n \geq N$ implies $0 < d_X(x_n, z) < \delta$. So, $d_Y(f(x_n), y) < \epsilon$. Therefore, $f(x_n) \to y$ as $n \to \infty$.

    Now for $3 \implies 1$. Suppose BWOC $\neg 1$. Then $\exists \epsilon > 0$ such that $\forall \delta > 0$, $\exists x \in E$ such that $0 < d(x, z) < \delta$, $d(f(x), y) \geq \epsilon$. 

    For $\delta = 2^{-n}$, $n \in \NN$, we then get $\{x_n \}_{n = 0}^\infty \subseteq E \setminus \{z\}$ such that $d(x_n, z) < 2^{-n}$, but $d(f(x_n), y) \geq \epsilon$. Now we use $3$: $x_n \to z$ as $n \to \infty$, so $f(x_n) \to y$ as $n \to \infty$. In particular, $\exists N \geq 0$ such that $n \geq N \implies d(f(x_n), y) < \epsilon$. This is a contradiction. 
\end{proof}
\thm{Limits and components}{Let $X, Y_1, \ldots, Y_n$ be metric spaces, and let $Y = \prod Y_i$ endowed with usual $p$-metric. Let $E \subseteq X$, $z \in E'$, $f: E \to Y$. Write $f = (f_1,\ldots, f_n)$ where $f_i : E \to Y_i$. The following are equivalent for $y = (y_1, \ldots, y_n) \in Y$:
\begin{enumerate}
    \item $f(x) \to y$ as $x \to z$.
    \item $f_i(x) \to y_i$ as $x \to z$ for $1\leq i \leq n$. 
\end{enumerate}}
\begin{proof}
    This follows from the sequential characterization of limits combined with the characterization of limits of sequences in the product space $Y$.
\end{proof}

\thm{
    Algebra of limits
}{
    Let $X$ be a metric space, $E\subseteq ZX$, $z \in E'$. The following hold:
    \begin{enumerate}
        \item Let $V$ be a normed vector space and suppose $f,g: E\to V$, $\alpha : E \to \FF$ are such that $f(x) \to v_1$, $g(x) \to v_2$, and $\alpha(x) \to \beta$ as $x \to z$. Then:
        \begin{enumerate}
            \item $f(x) + g(x) \to v_1 + v_2$ as $x \to z$.
            \item $\alpha(x)f(x) \to \beta v_1$ as $x \to z$.
        \end{enumerate}
        \item Let $V_1, \ldots, V_k, W$ be normed vector spaces over $\FF$. Suppose $f_i :E \to V_i$ and $T: E \to \mathcal L(V_1, \ldots, V_k; W)$ are such that $f_i(x) \to v_i$ as $x \to z$ and $T(x) \to M$ as $x \to z$. Then, 
        \begin{align*}
            E \ni x \mapsto T(x)(f_1(x), \ldots, f_k(x)) \in W
        \end{align*}
        satisfies $T(x)(f_1(x), \ldots, f_k(x)) \to M(v_1, \ldots, v_k)$ as $x \to z$.
    \end{enumerate}
}
\begin{proof}
    Use characterization of limits via sequences together with algebra of sequential limits.
\end{proof}
\dfn{}{
    Let $X,Y$ be metric spaces, $E\subseteq X$, $z \in E$, and $f: E \to Y$. We say $f$ is continuous at $z$ if for every $\epsilon > 0$ there exists $\delta > 0$ such that $x \in E$ and $d(x, z) < \delta \implies d(f(x), f(z)) < \epsilon$. We say $f$ is continuous on $E$ if $f$ is continuous at every point of $E$.
}


\newpage
\noindent Remarks:
\begin{enumerate}
    \item If $z$ is isolated, i.e. $z \in E \setminus E'$, then the definition of continuity is true vacuously and so $f$ is continuous as $z$. 
    \item Unlike when computing limits, we need $f(z)$ defined, and $x = z$ is allowed.
    \item We can think of $f: E \to Y$ with $E$ a metric space on its own with $d_E = d_X$.  
\end{enumerate}
\thm{Characterizations of continuity}{Let $X,Y$ be metric spaces and $z \in E \subseteq X$ and $f: E \to Y$. The following are equivalent:
\begin{enumerate}
    \item $f$ is continuous at $z$.
    \item $\forall \epsilon > 0$, $\exists \delta > 0$ such $f(E \cap B(z, \delta)) \subseteq B(f(z), \epsilon)$.
    \item If $z \in E'$, then $f(x) \to f(z)$ as $x \to z$.
    \item If $\{x_n\}_{n = \ell}^\infty  \subseteq E \setminus \{z\}$ is such that $x_n \to z$ as $n \to \infty$, then $f(x_n) \to f(z)$ as $n \to \infty$. 
    \item If $\{x_n\}_{n = \ell}^\infty \subseteq E$ is such that $x_n \to z$ as $n \to \infty$, then $f(x_n) \to f(z)$ as $n \to \infty$.
    \item If $\{x_n \}_{n = \ell}^\infty \subseteq E$ is such that $x_n \to z$ as $n \to \infty$, then $\{f(x_n)\}_{n = \ell}^\infty \subseteq Y$ is convergent. 
\end{enumerate}}
\begin{proof}
    $1 \iff 2$ is obvious as well as $3 \iff 4$ since we proved it in the sequential chracterization of limits.

    We'll prove $1 \implies 5 \implies 6 \implies 4$ and $3 \implies 1$. 

    $3 \implies 1$: If $z \in E \setminus E'$, we're done because of earlier remark. So let $z \in E \setminus E'$. Then $3$ is in play: $f(x) \to f(z)$ as $x \to z$. Let $\epsilon > 0$ and pick $\delta > 0$ such that $x \in E$ and $0 < d(x, z) < \delta \implies d(f(x), f(z)) < \epsilon$.

    Note, $x = z \iff d(x, z) = 0$, in which case $d(f(x), f(z)) = 0 < \epsilon$. So $f$ is continuous at $z$.

    $1 \implies 5$: Suppose $f$ is continuous at $z$ and let $\{x_n \} \subseteq E$ be such that $x_n \to z$ as $n \to \infty$. Let $\epsilon > 0$ and pick $\delta > 0$ such that $x \in E$ and $d(x, z) < \delta \implies d(f(x_n), f(z)) < \epsilon$. Pick $N \geq \ell$ such that $n \geq N$ implies $d(x_n , z) < \delta \implies d(f(x_n), f(z)) < \epsilon$. So $f(x_n) \to f(z)$ as $n \to \infty$. 

    $5 \implies 6$. Trivial

    $6 \implies 4$. Let $\{x_n\} \subseteq E \setminus \{z\}$ be such that $x_n \to z$ as $n \to \infty$. Define $\{y_n\} \subseteq E$ via 
    \begin{align*}
        y_n = \begin{cases}
        x_n & n = \ell + 2k \\
        z & n = \ell + 2k + 1
        \end{cases}.
    \end{align*}
    Then $y_n \to z$ as $n \to \infty$. 6 implies that $f(y_n)$ converges. So we can pick a subsequence to show that is converges to $f(z)$.
\end{proof}

\cor{Corallary 1}{Let $X, Y$ be metric spaces, $f : X \to Y$. $f$ is continuous if and only if if $ \{x_n \} \subseteq X$  is convergent, then $\{f(x_n)\} \subseteq Y$ is convergent.  }
\cor{Corallary 2}{Let $X, Y$ be metric spaces with $X$ separable. Let $f: X \to Y$ be continuous. Then $f(X) \subseteq Y$ is separable.}
\newpage
\thm{Continuity and products}{Let $X, Y_1,\ldots, Y_k$ be metric spaces and let $f: X \to Y := \prod Y_i$. Let $z \in X$. Write $f = (f_1,\ldots, f_k)$ where $f_i : X \to Y_i$. The following are equivalent:
\begin{enumerate}
    \item $f$ is continuous at $z$.
    \item Each $f_i$ is continuous at $z$.
\end{enumerate}}
\begin{proof}
    Proof direct from limit chracterization.
\end{proof}

\thm{Algebra of continuity}{Sum, product, and multilinear functions of continuous functions are continuous.}

\thm{Composition}{Let $X, Y, Z$ be metric spaces and $f: X \to Y$ and $g: Y \to Z$. Suppose $f$ is continuous at $z \in X$ and $g$ is continuous at $f(z)$. Then $g \circ f : X \to Z$ is continuous at $z$. }

\thm{}{Let $X, Y$ be metric spaces and $f: X \to Y$. The following are equivalent:
\begin{enumerate}
    \item $f$ is continuous.
    \item $f^{-1}(U)$ is open $\forall U \subseteq Y$ open.
    \item $f^{-1}(U)$ is closed $\forall U \subseteq Y$ closed.
\end{enumerate}}

\thm{Multilinearity and continuity}{
    Let $V_1, \ldots, V_k$ and $W$ be normed vector spaces over $\FF$, and let $T \in L(V_1, \ldots, V_k; W)$. The following are equivalent:
    \begin{enumerate}
        \item $T \in \mathcal L(V_1, \ldots, V_k; W)$ i.e. $T$ is a bounded multilinear map.
        \item $T$ is continuous.
        \item $T$ is continuous at $0 \in \prod V_i$. 
    \end{enumerate}
}
\begin{proof}
    $1 \implies 2$: Let $u = (u_1, \ldots, u_k)$, $v = (v_1, \ldots, v_k)$ be two vectors in $V_1 \times \ldots \times V_k$. We write
    \begin{align*}
        T(v_1, \ldots, v_k) - T(u_1, \ldots, u_k) &= T(v_1 - u_1, v_2, \ldots, v_k) + T(u_1, v_2 - u_2, \ldots, v_k) + \ldots + T(u_1, \ldots, u_{k-1} v_k - u_k).    \end{align*}
        This implies that $\Vert T(u) - T(v) \Vert \leq \Vert T \Vert_{\mathcal L} \left[ \Vert v_1 - u_1 \Vert_{V_1} \prod_{i = 2}^k \Vert v_i \Vert_{V_i}  + \Vert u_1 \Vert_{V_1} \Vert u_2 - v_2 \Vert_{V_2} \prod_{i = 3}^k \Vert u_i \Vert_{V_i} + \cdots + \prod_{i=1}^{k-1}\Vert u_1 \Vert_{V_i} \Vert u_k - v_k \Vert_{V_k} \right]$.

    From this est, it's easy to conclude that $T$ is continuous at $u$.

    $2 \implies 3$: trivial

    $3 \implies 1$: Suppose $T$ is continuous at 0. Let $\epsilon = 1$ and let $\delta > 0$ such that $\Vert u \Vert_p = \begin{cases}
    \left( \sum \Vert u_i \Vert_{V_i}^p \right)^{1/p} & p < \infty \\
    \max & p = \infty
    \end{cases} < \delta$. This implies that $\Vert T(u) \Vert_W = \Vert T(u) - T(0) \Vert_W < \epsilon = 1$.

    Let $u_i \in V_i$ be such that $\Vert u_i \Vert_{V_i} = 1$. Then 
    \begin{align*}
        \Vert \left( \frac{\delta}{2k^{1/p}} u_1, \ldots, \frac{\delta u_k}{2k^{1/p}}\right) = \frac \delta 2 < \delta.
    \end{align*}
    So, $T$ applied to that value is less than 1. But this means 
    \begin{align*}
        \left( \frac{\delta}{2k^{1/p}}\right) \Vert T(u) \Vert_W &< 1 \\
        \Vert T(u) \Vert_W &\leq \left( \frac{2k^{1/p}}{\delta} \right)^k.
    \end{align*}
    as well. By taking the supremum, we get that $T$ is bounded and in $\mathcal L$. 
\end{proof}

\dfn{}{Let $V,W$ be normed vector spaces over $\FF$. 
\begin{enumerate}
    \item Recall $L^k (V; W) = L(V_1, \ldots, V_k; W)$ and similarly for $\mathcal L$. Given $T \in L^k(V; W)$ and $v \in V$, we write $Tv^{\otimes k} = T(v^{\otimes k}) = T(v, \ldots, v)$.
    \item A polynomial is a map $p: V \to W$ given by $p(v) = \sum_{k=0}^d T_kv^{\otimes k}$ for $T_k \in L^k(V; W)$. We write $d=$ degree of $p$ given that $T_d \neq 0$. Note: by the continuity of $T_k \in \mathcal L^k(V; W)$ and algebra of continuity, all polynomials are continuous.
\end{enumerate}}
\dfn{}{Let $X,Y$ be metric spaces and $f: X \to Y$. 
\begin{enumerate}
    \item We say $f$ is uniformly continuous if $\forall \epsilon > 0$, $\exists \delta > 0$ such that $x, y \in X$ and $d_X(x, y) < \epsilon$ then $d_Y(f(x), f(y)) < \epsilon$.
    \item $f$ is Lipschitz if $\exists c \geq 0$ such that $d(f(x), f(y)) \leq c d(x, y)$ for all $x, y \in X$.
\end{enumerate}}
Facts:
\begin{enumerate}
    \item Lipschitz $\implies$ uniformly continuous $\implies$ continuous.
    \item Compositions of uniformly continuous functions are uniformly continuous.
    \item Compositions of Lipschitz functions are Lipschitz.
    \item Suppose $f,g: X \to V$ for $V$ a normed vector space. If $f, g$ are uniformly continuous or Lipschitz, then $\alpha f + \beta g$ are too $\forall \alpha, \beta \in \FF$.
\end{enumerate}
\mlenma{}{
    Suppose $X,Y$ are metric spaces, $f:  X \to Y$ is uniformly continuous if $\{x_n\}_{n = \ell}^\infty \subseteq X$ is Cauchy, then $\{f(x_n)\}_{n = \ell}^\infty \subseteq Y$ is Cauchy.
}
\begin{proof}
    Let $\epsilon > 0$, then there exists $\delta>0$ such that 
    \begin{align*}
        x,y \in X \land d_X(x, y) < \delta \implies d_Y(f(x), f(y)) < \epsilon.
    \end{align*}
    Pick $N \geq \ell$ such that $m,n \geq N \implies d(x_n, x_m) < \delta$. This means that $d(f(x_n), f(x_m)) < \epsilon$. Therefore $\{f(x_n)\}_{n = \ell}^\infty \subseteq Y$ is Cauchy.
\end{proof}
\newpage
\ex{}{
    \begin{enumerate}
        \item Let $f: \RR \to \RR$ via $f(x) = x^2$.
        \item Let $V, W$ be normed vector spaces, $a \in W$, $T \in \mathcal L(V, W)$. Then $f: V \to W$ via $f(x) = a + Tx$ is Lipschitz:
        \begin{align}
            \Vert f(x) - f(y) \Vert_W &= \Vert Tx - Ty \Vert_W \\
            &\leq \Vert T \Vert_{\mathcal L} \Vert x -y \Vert_V.
        \end{align}
        So, $f$ is Lipschitz. 

        But moving back to the first example, $f$ is not uniformly continuous. However, $f$ maps Cauchy sequences to Cauchy sequences. Indeed, suppose $\{x_n\}_n$ is Cauchy and bounded by $M$. Then,
        \begin{align*}
            |f(x_n) - f(x_m) | = |x^2_n - x^2_m| = |x_n + x_m||x_n - x_m| \leq 2M|x_n - x_m|.
        \end{align*}
        So $f$ is not uniformly continuous. Suppose not, then $\exists \delta > 0$ such that $| x- y| < \delta \implies |f(x) - f(y)| < 1$.

        Let $x = n \in \NN$ and $y = n + \frac \delta 2$. Then 
        \begin{align*}
            |x-y| &= \frac \delta 2 < \delta,
        \end{align*}
        so $1 > |f(y) - f(x)| = (n + \delta / 2)^2 - n^2 = \delta n + \frac {\delta^2} 4$. This is a contradiction.
        \item Let $X$ be a metric space. Let $a \in X$ and define $f:  X \to \RR$ via $f(x) = d(x, a)$. $f$ is Lipschitz as $|f(x) - f(y)| = |d(x, a) - d(y, a)| \leq d(x, y)$. 
        This can be generalized.
        \item Consider $\sin: \RR \to \RR$. 
        \begin{align*}
            |\sin(x) - \sin(y)| &= |\cos(w)(x-y)| \leq |x-y|
        \end{align*}
        for some $w$ below $x$ and $y$. Therefore $\sin$ is Lipschitz. Ditto for $\cos$.
        \item Let $X$ be  a metric space, $V_1, \ldots, V_k, W$ be normed vector spaces. And suppose $f_i: X \to V_i$ is uniformly continuous xor Lipschitz. Further suppose $T: X \to \mathcal L(V_1, \ldots, V_k; W)$ is uniformly continous xor Lipschitz. If $T, f_1, \ldots, f_k$ are all also bounded, then $X \ni x \mapsto T(x)(f_1 (x), \ldots, f_k(x)) \in W$ is uniformly continuous xor Lipschitz.
        
        \begin{proof}
            Recall 
            \begin{align*}
                T(u_1, \ldots, u_k) - T(v_1, \ldots, v_k) = \\
                T(u_1 - v_1, u_2, \ldots, u_k) + \cdots + T(v_1, \ldots, v_{k-1}, u_k - v_k).
            \end{align*}
            Now use this with $u_i = f_i(x)$ and $v_i = f_i(y)$. 

        \end{proof}
    \end{enumerate}
}

\dfn{}{Let $f: X\to Y$ for $X, Y$ metric spaces. We define $K(f) \in [0, \infty]$ to be 
\begin{align*}
    K(f) = \begin{cases}
    0 & |X| = 1 \\
    \sup_{x, y \in X , x \neq y} \frac{d_Y(f(x), f(y))}{d_X(x, y)} & \text{otherwise}
    \end{cases}.
\end{align*}
$K(f)$ is called the Lipschitz constant for $f$. }
\noindent Facts:
\begin{enumerate}
    \item $K(f) = 0 \iff f$ is constant. $K(f)$ is finite $\iff$ $f$ is Lipschitz. Also, $d(f(x), f(y)) \leq K(f) d(x, y)$. 
    \item Suppose $g: Y \to Z$, $Z$ is a metric space. Then $K(g \circ f) \leq K(g) K(f)$. 
    \begin{proof}
        $d_Z(g \circ f(x), g \circ f(y)) \leq K(g) d_Y(f(x), f(y)) \leq K(g)K(f)d(x, y)$. This yields the result. 
    \end{proof}
    \item If $Y = X$, i.e. $f: X \to X$, then $K(f^{(n)}) \leq K(f)^n$. 
    \begin{align*}
        f^{(0)} &= I_X \\
        f^{(n)} &= f \circ f^{(n-1)}.
    \end{align*}
\end{enumerate}

\dfn{}{Let $X, Y$ be metric spaces and $f: X \to Y$.
\begin{enumerate}
    \item We say $f$ is expansive if $\infty > K(f) > 1$.
    \item We say $f$ is non-expansion if $K(f) \leq 1$.
    \item We say $f$ is contractive if $K(f) < 1$.
    \item Suppose $Y=X$. We say $f$ is eventually contractive if $\exists 1 \leq n \in \N$ such that $f^{(n)}$ is contractive. 
\end{enumerate}}
\ex{}{
    Let $\alpha \in [0, 1]$, $\beta \in(0, 1)$, $\gamma \in [0, \infty)$ and $x \in (-\infty, 0]$. Set $f: \RR \to \RR$ via 
    \begin{align*}
        f(x) = \begin{cases}
        \beta & x \in (-\infty, 0] \\
        \beta + (1-\beta)\left(\frac x \beta\right)^{\alpha} & x \in [0, \beta] \\
        1 & x \in [\beta, 1] \\
            1 + \gamma(x-1) & x \in (1, \infty)
        \end{cases}.
    \end{align*}
    Exercise:
    \begin{align*}
        K(f) = \begin{cases}
        \infty, & \alpha \in (0, 1) \\
        \max\left(\gamma, \frac 1 \beta - 1 \right) & \alpha = 1
        \end{cases}.
    \end{align*}
    Also,
    \begin{align*}
        f^{(2)}(x) = \begin{cases}
            1 & x \leq 1 \\
            1 + \gamma^2 (x-1) & x > 1
        \end{cases}.
    \end{align*}
    So $K(f^{(2)}) = \gamma^2 \implies f^{(2)}$ is Lipschitz. If $\gamma < 1$ then $f$ is eventually contractive. 
}
\thm{Banach Fixed Point Theorem}{
    Let $X$ be a complete metric space and $f : X \to X$ be eventually contractive. Then there exists a unique fixed point $x_0 \in X$ such that $f(x_0) = x_0$.
}
\begin{proof}
    Suppose intitially that $f$ is contractive, i.e. $K(f) = \gamma \in [0, 1)$. Let $x_0 \in X$ arbitrarily. Inductively define $\{x_n\}_{n = 0}^\infty \subseteq X$ via $x_{n+1} = f(x_n)$, i.e. $x_n = f^{(n)}(x_0$). For $n > m \geq 0$, we bound
    \begin{align*}
        d(x_n, x_m) \leq d(x_n, x_{n-1}) + d(x_{n-1}, x_{m}) \leq \cdots \leq \sum_{i=m}^{n-1} d(x_i, x_{i+1}) = \sum_{i = m}^{n-1} d(f^{(i)}(x_0), f^{(i)}(x_1)) \leq \sum_{i=1}^{n-1} \gamma^i d(x_0, x_1) = d(x_0, x_1) \sum_{i=m}^{n-1} \gamma^i.
    \end{align*}
    Since $\gamma< 1$, the infinite sum of $\gamma^i$ converges. That is, $\left\{ \sum_{i=0}^k \gamma^i \right\}_{k = 0}^\infty$ is Cauchy. This and the bound implies that $\{x_n\}_{n = 0}^\infty$ is Cauchy.

    Now note that $x_{n+1} = f(x_n)$, and this converges to $x = f(x)$ because $f$ is continuous. Therefore $x$ is a fixed point. 

    Suppose $y \in X$ is such that $f(y) = y$. Then, $d(x, y) = d(f(x), f(y)) \leq \gamma d(x, y) \implies (1 - \gamma) d(x, y) \leq 0 \implies x = y$.

    Now consider the general case. $\exists z \leq n$ such that $f^{(n)}$ is contractive. By the previous analysis, there exists a unique $x \in x$ such that $f^{(n)} = x$. Thus, $f(x) = f^{(n+1)}(x) = f^{(n)}(f(x)) \implies f(x)$ is a fixed point of $f^{(n)}$. So this means $f(x) = x$. Suppose now $y = f(y)$, this means $y = f^{(n)}(y) \implies y = x$. 
\end{proof}
\nt{Say $f$ is contractive for simplicity. What we knew was that if $n > m \geq 0$, then 
\begin{align*}
    d(x_n, x_m) \leq d(x_0, x_1) \sum_{i = m}^{n-1} \gamma^i \leq d(x_0, x_1) \sum_{i=m}^\infty \gamma^i = d(x_0, x_1) \frac{\gamma^m}{1 - \gamma}.
\end{align*}
So,
\begin{align*}
    d(x, x_m) = \lim_{n \to \infty} d(x_n, x_m) \leq \frac{d(x_0, x_1)\gamma^m}{1 - \gamma}.
\end{align*}
}

\ex{putnam(?)}{
    Let $X \neq \emptyset$ be a set, $g: X \to \RR$ be bounded and $h : X \to \RR$ be arbitrary. Let $0 < \gamma < 1$. 
    
    Claim: $\exists ! f \in \mathcal B(X; \RR)$ such that $f(x) = g(x) + \gamma \cos(h(x) + f(x))$. 
    
    \begin{proof}
        Define $\Phi : \mathcal B(X) \to \mathcal B(X)$ as $\Phi(f) = g + \gamma \cos(h + f) \in \mathcal B(X)$. 

        Fact 1: $\mathcal B(X)$ is complete. 

        Fact 2: $\Phi(f_1) - \Phi(f_2) = \gamma [\cos(h + f_1) - \cos(h + f_2)]$. That is,
        \begin{align*}
            |\Phi(f_1)(x) - \Phi(f_2)(x)| &\leq \gamma |f_1(x) - f_2(x)| \\
            \Vert \Phi(f_1) - \Phi(f_2) \Vert_{\mathcal B(X)} &\leq \gamma \Vert f_1 - f_2 \Vert_{\mathcal B(X)}.
        \end{align*}
        Therefore, $\Phi$ is a contaction, meaning there is a unique $f \in \mathcal B(X)$ such that $f = \Phi(f) = g + \gamma \cos( h + f)$.
    \end{proof}
    Example 2: Solving the quadratic equation. Claim: suppose $V, W$ are Banach spaces over $\FF$, $A \in \mathcal L^2(V; W)$, $B \in \mathcal L(V; W), c \in W$. Suppose $A \neq 0$ and $B$ is invertible with $B^{-1} \in \mathcal L(W; V)$. We claim that there eixsts $x \in V$ such that $A(x, x) + Bx + c = 0$ provided that $4 \Vert B^{-1}\Vert^2_{\mathcal L(W; V)} \Vert A \Vert_{\mathcal L^2(V; W)} \Vert c \Vert_W < 1$. 
    \begin{proof}
        If $c = 0$, then $x=0$ does the job, so suppose $c \neq 0$. It suffices to prove that when $W =V$ and $B = I_V$. Indeed, suppose we proved this. Then, 
        \begin{align*}
            A(x, x) + Bx + c = 0 \text{ in } W \iff B^{-1} A(x, x) + x + B^{-1}c = 0 \text{ in } V.
        \end{align*}
        But $B^{-1} \circ A \in \mathcal L^2(V; V)$, $\Vert B^{-1} \circ A \Vert_{\mathcal L^2}\leq \Vert B^{-1} \Vert_{\mathcal L} \Vert A \Vert_{\mathcal L^2}$.

        $B^{-1} c \in V$, $\Vert B^{-1} c \Vert_V \leq \Vert B^{-1} \Vert_{\mathcal L} \Vert c \Vert_W$.

        Then, this means that $4 \Vert B^{-1} A \Vert_{\mathcal L^2} \Vert B^{-1}c \Vert_V < 1 \implies \exists x \in V$ such that we are done.

        \begin{proof}
            We prove the special case. We want to thsow that $A(x, x) + x + c = 0$ for $A \in \mathcal L^2(V; V), c \in  V \setminus \{0 \}$. Note,
            \begin{align*}
                A(x, x) + x + c = 0 \iff x = -c - A(x, x) \iff x \text{ is a fixed point of } f : V \to V, f(x) = -c - A(x, x).
            \end{align*}
            The idea is that if $A= 0$, then $x = -c$ is a solution. The strategy is to try to find $R \geq 0$ such that
            \begin{enumerate}
                \item $f: B[-c, R] \to B[-c, R]$,
                \item $f$ is eventually contractive on $B[-c, R]$.
            \end{enumerate}
            IF we can prove this, then $\exists ! x \in B[-c, R]$ such that $x = f(x) = -c - A(x, x) \implies A(x, x) + x + c = 0$.
        \end{proof}
    \end{proof}
    }

\ex{}{
    $\exists x \in V$ such that $A(x, x) + Bx + c = 0$. Reduce to case $ V + W$, $B = I$.

    Claim: $\exists x \in V$ such that $A(x, x) + x + c = 0$ where $A \in \mathcal L^2(V, V)$, $c \in V \setminus \{0\}$. 

    Let $f: V \to V$ such that $f(x)  = -c - A(x, x)$. Let $R \geq 0$ (TBD) and consider $x \in B[-c, R]$. 

    \begin{align*}
        f(x) = c = -A(x, x) = -[A(x + c -c, x + c - c)] &= -[A(x + c, x + c) - A(x+c, c) - A(c, x + c) + A(c, c)] \\
        &
        \implies \Vert f(x) + c \Vert \leq \Vert A \Vert [\Vert x + c \Vert^2 + 2 \Vert c \Vert \Vert x + c \Vert + \Vert c \Vert^2] \\
        &\leq \Vert A \Vert [R^2 + 2 \Vert c \Vert R + \Vert c \Vert ^2 ] \leq R.
    \end{align*}
    $\Vert A \Vert R^2 + (2 \Vert c \Vert \Vert R \Vert - 1)R + \Vert A \Vert \Vert c \Vert^2 \leq 0 \iff (2 \Vert A \Vert \Vert c \Vert - 1)^2 - 4 \Vert A \Vert^2  \Vert c \Vert^2 \geq 0$. This just gives 
    \begin{align*}
        1 - 4\Vert A \Vert \Vert c \Vert \geq 0 \implies 1 \geq 4\Vert A \Vert \Vert c \Vert.
    \end{align*}
    That is, if this holds, then $R \in [R_-, R_+]$. So,
    \begin{align*}
        R_{\pm} = \frac{1 - 2\Vert A \Vert \Vert C \Vert \pm \sqrt{1 - 4 \Vert A \Vert \Vert c \Vert}}{2 \Vert A \Vert}
    \end{align*}
    and $R_- > 0$. We now konw that $4 \Vert A \Vert \Vert c \Vert \leq 1 \implies$ for $R \in [R_-, R_+]$, $f: B[-c, R] \to B[-c, R]$.

    Next, for $x , y \in B[-c, R]$,
    \begin{align*}
        \Vert f(x) - f(y) \Vert = \Vert A(y, y) - A(x, x) \Vert &= \Vert A(y-x, x) + A(y, y-x)\Vert \\
        &= \Vert A(y-x, y+c) - A(y-x, c) + A(x+c, y-x) - A(c, y-x) \Vert \\
        &\leq \Vert A \Vert [\Vert x - y \Vert R + \Vert x-y \Vert \Vert c \Vert  + \Vert x -y \Vert R - \Vert x - y \Vert \Vert c \Vert] \\
        &= 2 \Vert A \Vert [R + \Vert c \Vert]\Vert x- y \Vert.
    \end{align*}
    We win if this quantity is strictly less than $1$. Note: 
    \begin{align*}
        2 \Vert A \Vert R_{\pm} + 2 \Vert A \Vert \Vert c \Vert = 1 \pm \sqrt{1 - 4\Vert A \Vert \Vert c \Vert}.
    \end{align*}
    This is why we choose $R_-$. So 
    \begin{align*}
        4 \Vert A \Vert \Vert c \Vert - 1 \implies f: B[-c, R_-] \to B[-c, R_-] \text{ is a contraction.}
    \end{align*}
    By Banach Fixed Point Theorem, $\exists ! x \in B[-c, R_-]$ such that $f(x) = x$. 
}

\nt{
 $f : B[-c, R_-] \rotatebox[origin=c]{270}{$\curvearrowleft$}$
}
\newpage
\section{Homeomorphisms}
\dfn{}{
    Let $X, Y$ be metric spaces and $f: X \to Y$ be a bijection. 
    \begin{enumerate}
        \item $f$ is a homeomorphism if $f, f^{-1}$ are continuous. We write $X \simeq_{hom} Y$ in this case.
        \item $f$ is a uniform homeomorphism if $f, f^{-1}$ are uniformly continuous. We write $X \simeq_{uni} Y$ in this case.
        \item $f$ is a bi-Lipschitz homeomorphism if $f, f^{-1}$ are Lipschitz continuous. We write $X \simeq_{bi-L} Y$ in this case.
    \end{enumerate}
}
\noindent Facts:
\begin{itemize}
    \item $\simeq_*$ are equivalence relations (also $X \simeq_{iso} Y \iff \exists \text{ an iso.})$. $[X]_{iso} \subseteq [X]_{bi-L} \subseteq [X]_{uni} \subseteq [X]_{hom}$.
    \item $f: X \to Y$ is a homeomorphism iff 
    \begin{align*}
        \begin{cases}
        f \text{ is bijective} & \\
        f^{-1}(U) \text{ is open } \text{ $\forall U \subseteq Y$ open.} \\
        f(V) \text{ is open } \text{ $\forall V \subseteq X$ open.}
        \end{cases}
    \end{align*}
    $f: X \to Y$ is bi-Lipschitz iff $f$ is surjective and $\exists c_0, c_1 > 0$ such that $c_0d(x, y) \leq d(f(x), f(y)) \leq c_1 d(x, y)$ for all $x, y\in X$. 
\end{itemize}

\ex{}{
    \begin{enumerate}
        \item Suppose $X$ is a finite metric space, $f: X \to Y$ a bijection. We claim that $f$ continuous implies $f$ Lipschitz. If the cardinality of $X$ is 1, then this is pointless (not really??). Suppose $ |X| \geq 2$. 
        
        Define $K(f) = \max \left\{ \frac{d(f(x), f(y))}{d(x, y)} \mid x \neq y \right\} < \infty$. 
        \item Consider $f: (0, 1) \to (1, \infty)$ via $f(x) = \frac{1}{x}$. This is a bijection and a homeomorphism. It's not a uniform homeomorphism. This shows that $[(0, 1)]_{uni} \subset [(0, 1)]_{hom}$.
        \item Let $V, W$ be normed vector spaces. We can add ``linear'' to any of our homeomorphism notions. In this case, linear bi-Lipschitz $\iff$ linear homeomorphism.
        
        Indeed, suppose that $T : V \to W$ is a linear map. This means that $T \in \mathcal L(V; W)$. We can use the same logic for $T^{-1}$. Thus, 
        \begin{align*}
            \Vert Tx - Ty \Vert_W \leq \Vert T \Vert_{\mathcal L} \Vert x- y\Vert_V.
        \end{align*}
        Also,
        \begin{align*}
            \Vert x - y \Vert_V &= \Vert T^{-1}T x - T^{-1}Ty \Vert_V \\
            &\leq \Vert T^{-1} \Vert_{\mathcal L} \Vert Tx - Tv\Vert_V \\
            &\implies \frac{1}{\Vert T \Vert_{\mathcal L}} \Vert x - y \Vert_V \leq \Vert Tx - Ty \Vert_W \leq \Vert T \Vert_{\mathcal L} \Vert x -y \Vert_V.
        \end{align*}
        Therefore, $T$ is bi-Lipschitz. 
    \end{enumerate}
}
\newpage 
\noindent \textbf{Question:} In general, is it the case that $[V]_{hom} = [V]_{bi-L}$ for $V \neq \{0\}$ a normed vector space?

\noindent \textbf{Answer:} No. In fact, $[V]_{bi-L} \subset [V]_{uni}$. Remember, we only care about metrics, not necessarily norms. 
\begin{proof}
    Fix $(V, \Vert \cdot \Vert)$ a normed vector space. Claim, $d: V \times V \to \RR$ given by $d(x, y) = \sqrt{\Vert x- y \Vert}$ is a metric on $V$. This is obviously symmetric and positive, so we just check the triangle inequality:
    \begin{align*}
        d(x, y) = \sqrt{\Vert x- y\Vert} \leq \sqrt{\Vert x - z \Vert + \Vert z - y\Vert} \leq \sqrt{\Vert x - z \Vert} + \sqrt{\Vert y - z \Vert} = d(x, z) + d(z, y).
    \end{align*}
    The last inequality is true by squaring. We'll now show that $(V, \Vert \cdot \Vert) \simeq_{uni} (V, d)$ with the identity map. That is $I: (V, \Vert \cdot \Vert) \leftrightarrow (V, d)$ is uniformly continuous in both directions.

    Let $\epsilon > 0$. Then $d(x, y) < \delta \iff \Vert x- y \Vert  < \delta^2$. So taking $\delta = \sqrt \epsilon$ shows that $I$ is uniformly continuous from $(V, d)$ to $(V, \Vert \cdot \Vert)$.

    Similarly, $\Vert x - y \Vert < \delta \iff d(x, y) < \sqrt{\delta}$. So take $\delta = \epsilon^2$ and we see that $I$ is uniformly continuous from $(V, \Vert \cdot \Vert)$ to $(V, d)$.

    Next, we claim that $(V, d)$ and $(V, \Vert \cdot \Vert)$ are not bi-Lipschitz homeomorphic. So suppose BWOC there exists an $f: (V, \Vert \cdot \Vert) \to (V, d)$ that is bi-Lipschitz homeomorphic. In particular, then there exists $c_0, c_1 > 0$ such that 
    \begin{align*}
       c_0 \Vert x- y \Vert \leq d(f(x), f(y)) = \sqrt{\Vert f(x) - f(y) \Vert} \leq c_1 \Vert x- y \Vert.
    \end{align*}
    for all $x,y \in V$. In particular, if $c:= c_1^2$, then $\Vert f(x) - f(y) \Vert \leq c \Vert x- y \Vert^2$ for $x,y \in V$. Let $x \neq y$ in $V$ and $1 \leq n \in \NN$. Let $x_i = x + \frac in (y - x)$ for $0 \leq i \leq n$. This yields 
    \begin{align*}
        \Vert x_{i+1} x_n \Vert = \frac{1}{n} \Vert x- y\Vert.
    \end{align*}
    Thus,
    \begin{align*}
        \Vert f(y) - f(x) \Vert \leq \sum_{i=0}^{n-1} \Vert f(x_{i+1}) - f(x_i) \Vert \leq \sum_{i=0}^{n-1} c \Vert x_{i+1} - x_i \Vert^2 = c\sum_{i=0}^{n-1} \frac{\Vert x - y\Vert^2}{n^2} = \frac{c \Vert x- y\Vert^2}{n}.
    \end{align*}
    Send $n \to \infty$ to get that $f(x) = f(y)$, contradiction! Therefore, $ [V]_{bi-L} \subset [V]_{uni}$.
\end{proof}
\noindent \textbf{Question:} Are there any non-linear bi-Lipschitz homeomorphisms on $V \neq \{ 0 \}$ a normed vector space. 

\noindent \textbf{Answer:} Yes, there are lots, at least if $V$ is complete. Suppose $V$ is a banach space and suppose $g: V \to V$ is a contraction. We claim that $f = I + g$ is a bi-Lipschitz homeomorphism.
\begin{proof} We follow the following steps:

    \begin{enumerate}
        \item $f$ is a bijection. We'll show $\forall y \in V$, $\exists  x \in V$ such that $x + g(x) = y$. Fix $y$, and define $h : V\to V$ via $h(x) = y - g(x)$. Then we're done if we can show that $\exists !  x \in X$ such that $h(x) = x$. But, $K(h) = K(g)$, so $h$ is a contraction and therefore the Banach fixed point theorem applies. 
        \item $f $ is bi-Lipschitz. We have 
        \begin{align*}
            \Vert f(x) - f(y) \Vert  = \Vert x- y  + g(x) - g(y) \Vert &\leq \Vert x-y \Vert  + \Vert g(x) - g(y) \Vert \\
            &\leq \Vert x - y \Vert + K(g) \Vert x- y\Vert = (1 + K(g)) \Vert x- y \Vert.
        \end{align*}
        This means that $K(f) \leq 1  + K(g)$. On the other hand, 
        \begin{align*}
            \Vert x- y \Vert &\leq \Vert x + g(x) - y - g(y) \Vert  + \Vert g(x) - g(y) \Vert\\
            &\leq \Vert f(x) - f(y) \Vert + K(g) \Vert x- y \Vert.
        \end{align*}
        This implies $(1 - K(g)) \Vert x- y\Vert \leq \Vert f(x) - f(y) \Vert \leq (1 + K(g)) \Vert x- y \Vert$. Therefore, $f$ is bi-Lipschitz. 
    \end{enumerate}
\end{proof}
\newpage
\noindent \textbf{Next:} Let $f:  V \to V$ be a bi-Lipschitz homeomorphism. Let $g: V \to V$ be a bi-Lipschitz homeomorphism with $K(g)K(f^{-1}) < 1$. We claim $h = f +g$ is a bi-Lipschitz homeomorphism. 
\begin{proof}
    Indeed, $h = f + g = f + g \circ f^{-1} \circ f = (I + g \circ f^{-1}) \circ f$. We just need to show that $g \circ f^{-1}$ is bi-Lipschitz because then $h$ will be. 

    But, $K(g \circ f^{-1}) \leq K(g)K(f^{-1}) < 1$ by assumption. Therefore, $g \circ f^{-1}$ is a bi-Lipschitz homeomorphism, and so is $h$. 
\end{proof}

\dfn{}{Let $X$ be a metric space 
\begin{enumerate}
    \item We say a property of $X$ is a topological property if it is common to $[X]_{hom}$. That is, it's true in $X$ if and only if it's true in $Y$ for all $Y \simeq_{hom} X$.
    \item We say a property of $X$ is a uniform property if it is common to $[X]_{uni}$.
    \item We say a property of $X$ is a strong property if it is common to $[X]_{bi-L}$.
\end{enumerate}}
\ex{}{
    \begin{enumerate}
        \item $(0, 1) \simeq_{hom} \RR$. Let $f:(0, 1) \to \RR$ be defined as 
        \begin{align*}
            f(x) &= \log\left( \frac 1x -  1\right) \\
            f^{-1}(y) &= \frac{1}{1 + e^{y}}.
        \end{align*}
        $(0, 1)$ is not complete and bounded, but $\RR$ is copmlete and unbounded. This is a strong example of how homeomorphisms do not maintain all properties. 
    \end{enumerate}
}

\mprop{}{
    Let $X \simeq_{bi-L} Y$. Then $X$ is bounded iff $Y$ is bounded. 
}
\begin{proof}
    $\exists c_0, c_1 > 0$ and a bijection $f: X \to Y$ such that $c_0d(x, y) \leq d(f(x), f(y)) \leq c_1d(x, y)$ for all $x, y \in X$. We'll show that $Y$ bounded implies $X$ bounded. The other direction is free. 

    So if $Y$ is bounded, then $Y \subseteq B(z, r)$ for some $z \in Y$. But, $z = f(x)$ for some $x \in X$. Therefore, $d(x, y) \leq \frac{1}{c_0}d(f(x), f(y))$ for all $y \in X$. But that distance is less than $r$, so we also have that $d(x, y) < \frac{r}{c_0}$. This means that $X \subseteq B(x, r/c_0)$. So $X$ is bounded. 
\end{proof}
\cor{}{Boundedness is a strong property. }
\newpage
\ex{}{
    Let $(X, d)$ be a metric space that is not bounded. We've seen that $\sigma = \frac{d}{1 + d}$ is also a metric on $X$. $(X, \sigma)$ is bounded because $X = B_{\sigma}[x, 1]$ for all $x \in X$. These are not bi-Lipschitz homeomorphic by the previous proposition, but they can be uniformly homeomorphic. We prove that the identity map does the job.

    To see this, let $\epsilon > 0$ and note 
    \begin{align*}
        d < \epsilon \implies \frac{d}{1+d} \leq d < \epsilon.
    \end{align*}
    This means $I: (X, d) \to (X, \sigma)$ is uniformly continuous. On the other hand, assume $\delta < 1$ and observe that
    \begin{align*}
        \frac{d}{d+1} < \delta \implies d(1 - \delta) < \delta \iff d < \frac{\delta}{1 - \delta}.
    \end{align*}
    We set $\epsilon := \frac{\delta}{1 - \delta}$. So then we hae $\delta = \frac{\epsilon}{1 + \epsilon}$, so $I: (X, \sigma) \to (X, d)$ is uniformly continuous.
}
\cor{}{Boundedness is a strong property (preserved by bi-Lipschitz, not by uniform).}
\mprop{}{
    Let $X,Y$ be metric spaces, $f: X \to Y$ be a uniform homeomorphism. The following hold:
    \begin{enumerate}
        \item $\{x_n \}_{n = \ell}^\infty \subseteq X $ is Cauchy if and onlyl if $\{f(x_n)\}_{n = \ell}^\infty \subseteq Y$ is Cauchy.
        \item $X$ complete if and only if $Y$ complete. 
    \end{enumerate}
}
\begin{proof}
We prove this in parts.
    \begin{enumerate}
        \item We know $g: X \to Y$ uniformly continuous implies that $\{g(x_n)\}_{n = \ell}^\infty \subseteq Y$ is Cauchy when $\{x_n\}_{n = \ell}^\infty \subseteq X$ is Cauchy. Now apply this to $g = f$ and $g = f^{-1}$ to get the result.
        \item Suppose $Y$ is complete. Let $\{x_n\}_{n = \ell}^\infty \subseteq X$ be Cauchy. Then we know $\{f(x_n)\}_{n = \ell}^\infty \subseteq Y$ is Cauchy and hence convergent to some $y \in Y$. Thus $x_n = f^{-1}(f(x_n)) \to f^{-1}(y)$ in $X$ as $n \to \infty$. Therefore $X$ is complete. The converse holds by symmetry. 
    \end{enumerate}
\end{proof}

\cor{}{Completeness is a uniform property. }
\begin{proof}
    The above proposition and $(0,1) \simeq_{hom} \RR$.
\end{proof}

\newpage
\section{More Metric Space Topology}
\dfn{}{Let $X$ be a metric space, $E \subseteq X$. We say $E$ is totally bounded if $\forall \epsilon > 0$, there exist $x_1, \ldots x_n \in X$ such that $E \subseteq \bigcup_{i=1}^n B(x_i, \epsilon)$.}
\noindent Facts:
\begin{enumerate}
    \item $E \subseteq X$ totally bounded $\implies$ $E$ is bounded. 
    \item If $A \subseteq E \subseteq X$ and $E$ is totally bounded, then $A$ is totally bounded.
    \item We don't have to use balls. $E \subseteq X$ is totall bounded if and only if $\forall \epsilon > 0$, $\exists A_1, \ldots, A_n \subseteq X$ such that $\mathrm{diam}(A_i) < \epsilon$ for $i = 1, \ldots, n$ and $E \subseteq \bigcup_{i = 1}^n A_i$.
\end{enumerate}
\begin{proof}
    We prove the third item. 

    First realize that $\mathrm{diam}(B(x, \epsilon)) = 2 \epsilon$. So if we know $E$ is totally bounded, we can pick $E \subseteq \bigcup_{i=1}^n B\left(x_i, \frac \epsilon 3\right)$. Then the diameter of each ball is $\frac{2\epsilon}{3} < \epsilon$, so let $A_i := B\left(x_i, \frac \epsilon 3\right)$ for $i = 1, \ldots, n$.

    Now suppose that $E \subseteq \bigcup_{i=1}^n A_i$ with $\mathrm{diam} < \frac \epsilon 3$. Then let $x_i \in A_i$ be arbitrary and note $A_i \subseteq B(x_i, \epsilon)$. This means $E\subseteq \bigcup_{i=1}^n B(x_i, \epsilon)$.
\end{proof}
\ex{}{
    \begin{enumerate}
        \item Suppose $E \subseteq X$ is finite. Then $E$ is totally bounded (just take the individual points).
        \item $a,b \in \RR$, $a<b$. Then $(a, b)$ and $[a,b]$ are totally bounded, but $\RR$ itself is not totally bounded.
        \item Let $X$ be an infinite set with the discrete metric. Then $B(x, 1)$ is a singleton set, so $X$ itself is not totally bounded. 
        \item Let $E \subseteq \{\chi_A \in \ell^\infty (\NN; \RR) \mid A \subseteq \NN\}$ where 
        \begin{align*}
            \chi_a(n) = \begin{cases}
            1 \quad n \in A \\
            0 \quad n \notin A
            \end{cases}.
        \end{align*}
        $E$ is bounded because $\Vert \chi_A \Vert_\infty \leq 1$. Also, if $A \neq B$, then $\Vert \chi_A - \chi_B \Vert_\infty = 1$. 

        Then $B(\chi_A, 1) \cap E = \{\chi_A\}$. Therefore $E$ is bounded but not totally bounded. 
        \item Let $V$ be a finite dimensional normed vector space. $B(x, r)$ and $B[x, r]$ are totally bounded. 
    \end{enumerate}
}
\mprop{}{
    Let $X$ be a metric space with $E \subseteq X$ totally bounded. Then $\overline E$ is totally bounded. 
}
\begin{proof}
    Recall $\overline E = E \cup E'$. Let $\epsilon > 0$ and pick $X_1, \ldots, X_n \in X$ such that $E \subseteq \bigcup_{i=1}^n B(x_i, \epsilon / 2)$. 
    
    Let $x \in E'$, which means that $\emptyset \neq (B(x, \epsilon / 2) \cap E) \setminus \{x\}$, so pick any $y$ in this set. In particular, $d(x, y) < \epsilon  /2 $. Since $y \in E$, there is an $x_i$ such that $d(x_i, y)  < \epsilon /2$. By the triangle inequality,
    \begin{align*}
        d(x, x_i) \leq d(x, y) + d(y, x_i) < \frac{2\epsilon}{2} = \epsilon.
    \end{align*}
    That is, $E' \subseteq \bigcup_{i=1}^n B(x_i, \epsilon)$. Therefore, $E$ itself will be contained in the same union of balls as well. 
\end{proof}
\newpage 
\thm{}{Let $X,Y$ be metric spaces.
\begin{enumerate}
    \item If $f : X \to Y$ is uniformly continuous and $E \subseteq X$ is totally bounded, then $f(E) \subseteq Y$ is totally bounded.
    \item If $X \simeq_{uni} Y$, then $X$ is totally bounded if and only if $Y$ is totally bounded. In particular, $TB$ is a uniform property. 
\end{enumerate}}
\begin{proof}
    If suffices to just prove the first item. 

    Let $\epsilon  >  0$ and pick $\delta > 0$ such that $f(B(x, \delta)) \subset B(f(x), \epsilon)$ for all $x \in X$. Since $E$ is totally bounded, there exist $x_1, \ldots, x_n$ in $X$ such that $E \subseteq \bigcup_{i=1}^n B(x_i, \delta)$. Thus, $f(E) \subseteq \bigcup_{i=1}^n f(B(x_i, \delta)) \subseteq \bigcup_{i=1}^n B(f(x_i), \epsilon)$. 
\end{proof}

\mprop{}{Let $X$ be a totally bounded metric space. Then $X$ is separable.}
\begin{proof}
    Since $X$ is totally bounded, for every $n \in \NN$ there is a finite set $\emptyset \neq E_n \subseteq X$ such that $X = \bigcup_{x \in E_n} B(x, 2^{-n})$. No we can define the countable set $E = \bigcup_{n \in \NN} E_n \subseteq X$. 

    Now given any $x \in X$, we know there exists $x_n \in E_n$ such that $d(x, x_n) < 2^{-n}$. 
\end{proof}

\ex{}{
    $\mathcal B(X; Y)$ is not separable when $X$ is an infinite set and $|Y| \geq 2$. 
}

\thm{}{Let $X$ be a metric space and $E \subseteq X$. The following are equivalent:
\begin{enumerate}
    \item $E$ is totally bounded.
    \item If $\{x_n\}_{n = \ell}^\infty \subseteq E$. Then there exists a Cauchy subsequence $\{x_{n_k}\}_{k = \ell}^\infty$.
\end{enumerate}}
\begin{proof}
    Suppose $E \subseteq X$ is totally bounded. We will use this notation for JUST this proof:

    \noindent Given two sequences $x = \{x_n\}_{n = \ell}^\infty, y = \{y_n\}_{n = \ell}^\infty \subseteq E$. We will write $x \sigma y$ to mean $x$ is a subsequence of $y$. Note, if $x \sigma y$ and $y \sigma z$, then $ x \sigma z$. 

    Set $x^\ell = x$, some given sequence $\{x_n\}_{n = \ell}^\infty \subseteq E$. $E$ is totally bounded, so $E \subseteq \bigcup_{y \in F_\ell} B(y, 2^{-\ell - 1})$ for $F_\ell$ finite. By the pigeonhole principle, $\exists y \in F_\ell$ such that $x^\ell_n \in B(y, 2^{-\ell - 1})$ for infinitely many $n$. We may thus select $x^{\ell + 1} \sigma x^{\ell}$ such that $x^{\ell + 1}_n \in B(y, 2^{-\ell - 1})$ for all $n$. 

    In particular, $d(x_{n}^{\ell + 1}, x_m^{\ell + 1}) < 2^{-\ell}$ for all $n,m \geq \ell$. Iterate this argument. This produces a sequence of subsequences. That is,
    \begin{align*}
        \cdots \sigma x^{m + 1} \sigma x^m \sigma \cdots \sigma = x
    \end{align*}
    such that $d(x^m_k, x^m_n) < 2^{-m +1}$ for all $m \geq \ell$, $n,k \geq \ell$. 

    Now let $\{y_n\}_{n = \ell}^\infty \subseteq E$ be given by $y_n = x^n_n$. Thus, for $m,n \geq N$, we know by construction that $d(y_n, y_m) = d(x_n^n , x^m_m) < 2^{-N + 1}$. This easily shows that $\{y_n\}_{n = \ell}^\infty \subseteq E$ is Cauchy.

    For the backward direction, we'll show that $\neg 1 \implies \neg 2$. Suppose $E$ is not totally bounded. Then there exists an $\epsilon > 0$ such that $E$ cannot be covered by finitely many $\epsilon$-balls. Let $x_0 \in E$ be arbitrary. $E \not\subseteq B(x_0, \epsilon)$. So there exists $x \in E$ such that $d(x_0, x) \geq \epsilon$. 

    Now suppose we have $x_0, \ldots, x_n \in E$ such that $d(x_i, x_j) \geq \epsilon$ for $i \neq j \leq n$. Note that $E \not\subseteq \bigcup_{i=0}^n B(x, \epsilon)$, so we can pick an $x_{n+1} \in E$ that has a minimal distance of $\epsilon$ to all the previous points. 

    Proceeding via induction, we find that there exists $\{x_n \}_{n= 0}^\infty$ such that $d(x_n, x_m) \geq \epsilon$ for all $m \neq n$ which cannot have a Cauchy subsequence. 
\end{proof}
\newpage 
\cor{}{If $X$ is totally bounded and complete, then it is sequentially compact, meaning all sequences in $X$ have convergent subsequences. }
\dfn{}{
    Let $X$ be a metric space. 
    \begin{enumerate}
        \item We say $\{U_\alpha\}_{\alpha \in A}$ ($A$ is any index set) is an open cover of $E$ if $E \subseteq \bigcup_{\alpha \in A} U_\alpha$ and each $U_\alpha$ is open.
        \item An open subcover of the open cover $\{U_\alpha\}_{\alpha \in A}$ is a collection $\{U_\beta\}_{\beta \in B}$ for any $B\subseteq A$ such that it remains an open cover of $E$. We say an open subcover $\{U_\beta\}_{\beta \in B}$ is finite if $B$ is finite. 
        \item We say $E$ is compact if each open cover of $E$ admits a finite open subcover. 
    \end{enumerate}
}
\ex{}{
    \begin{enumerate}
        \item Let $E \subseteq X$ be finite. Then $E$ is compact. 
        \item $(0, 1) \subset \RR$ is not compact. Take $\left\{\left(0, \frac{n}{n+1}\right)\right\}_{n=1}^{\infty}$. Suppose $B \subseteq \NN \setminus \{0\}$ is a finite set such that $\left\{\left(0, \frac{n}{n+1}\right)\right\}_{n= \in B}$ is an open subcover. Then let $N = \max B$. This implies that
        \begin{align*}
            (0, 1) \subseteq \left(0, \frac{N}{N+1}\right)
        \end{align*}
        which is a contradiction. 
        \item Let $E = \{0\} \cup \{2^{-n} \mid n \in \NN\} \subseteq \RR$. Then let $\{U_\alpha\}_{\alpha \in A}$ be an open cover.  Pick $\alpha_0 \in A$ such that $0 \in U_{\alpha_0}$. Since $2^{-n} \to 0$ as $n \to \infty$, in fact $U_{\alpha_0}$ contains all but finitely many of the $2^{-n}$ terms. That is, there exists $N$ such that $2^{-n \in U_{\alpha_0}}$ for all $n \geq N$. Now pick a finite subcover of $\{2^{-n} \mid 0 \leq n \leq N\}$ and we're done. $E$ ic ompact, though $E$ is infinite.
    \end{enumerate}    
}
\mprop{}{Let $X$ be a metric space and $K \subseteq X$ be compact. Then $K$ is closed and totally bounded, which in particular means that $K$ is bounded and that $K$ is a separable metric space, provided $K \neq \emptyset$. }
\begin{proof}
    Let $\epsilon > 0$ and note that $\{B(x, \epsilon)\}_{x \in X}$ is a cover of $K$. So by compactness, there exists $x_1, \ldots, x_n \in X$ such that $K \subseteq \bigcup_{i = 1}^n B(x_i, \epsilon)$. Therefore, $K$ is totally bounded which means it is bounded and separable.

    We now prove that $K$ is closed. If $K = \emptyset$ or $K^c = \emptyset$, then we are done. So, suppose otherwise. Let $x \in K^c$. For each $y \in K$, note that $B(y, d(x, y)/ 2) \cap B(x, d(x, y)  2) = \emptyset$. Also, $\{B(y, d(x, y)/2)\}_{y \in K}$ is an open cover of $K$.

    $K$ being compact implies that $K \subseteq \bigcup_{i=1}^n B(y_i, d(x, y_i)/2)$ for some $y_1, \ldots, y_n \in K$. 

    Let $\delta = \min\{d(x, y_i)/2 \mid 1 \leq i \leq n\}$. Then, $K \cap B(x, \delta) \subseteq \bigcup_{i=1}^n B(y_i, d(x, y_i)/2) \cap B(x, \delta) = \emptyset$. This means that $B(x, \delta) \subseteq K^c$. As such, $K^c$ is open and $K$ is closed.
\end{proof}


\mprop{}{Let $X$ be a metric space and $\emptyset \neq Y \subseteq X$ and $K \subseteq Y \subseteq X$. Then $K$ is compact in $Y$ if and only if $K$ is compact in $X$.}
\begin{proof}
    Suppose $K$ is compact in $Y$. Then let $\{U_\alpha\}_{\alpha \in A}$ be an open cover of $K$ in $X$. Then, $\{Y \cap U_\alpha\}_{\alpha \in A}$ is an open cover of $K$ in $Y$. By compactness, $K \subseteq \bigcup_{i=1}^n (Y \cap U_{\alpha_i}) \subseteq \bigcup_{i=1}^n U_{\alpha_i}$. So, $K$ is compact in $X$.

    Now suppose $K$ is compact in $X$. Let $\{V_{\alpha}\}_{\alpha \in A}$ be an open cover of $K$ in $Y$. Then, there exists an open (in $X$) $U_\alpha$ such that $V_\alpha = Y \cap U_\alpha$ for all $\alpha \in A$. THus, $\{U_\alpha\}_{\alpha \in A}$ is an open cover of $K$ in $X$. So,
    \begin{align*}
        K \subseteq \bigcup_{i=1}^n U_{\alpha_i} \implies K = K \cap Y \subseteq \bigcup_{i=1}^n Y_n \cap U_{\alpha_i} = \bigcup_{i=1}^n V_{\alpha_i}.
    \end{align*}
    So, $K$ is compact in $Y$.
\end{proof}
\nt{In light of this, we will mostly study compact metric spaces rather than compact subsets.}

\thm{}{Let $X$ and $Y$ be metric spaces with $X$ compact and $f: X \to Y$ be continuous. Then $f(X) \subseteq Y$ is compact.}
\begin{proof}
    Let $\{U_\alpha\}_{\alpha \in A}$ be an open cover of $f(X)$. Note that $f$ being continuous implies that $f^{-1}(U_\alpha) \subseteq X$ is open for all $\alpha \in A$. Thus, $\{f^{-1}(U_\alpha)\}_{\alpha in A}$ is an open cover of $X$. So since $X$ is compact, we have that $X = \bigcup_{i=1}^n f^{-1}(U_{\alpha_i})$. Therefore, $f(X) \subseteq \bigcup_{i=1}^n U_{\alpha_i}$. So, $f(X)$ is compact.
\end{proof}
\cor{}{
    Compactness is a topological property.
}
\mprop{}{Let $X$ be a compact metric space. Then $C \subseteq X$ si compact if and only if $C$ is closed.}
\begin{proof}
    Forward direction has already been proved. So now assume $C$ is closed and let $\{U_\alpha\}_{\alpha \in A}$ be an open cover of $C$. Then, $C^c \cup \{U_{\alpha}\}_{\alpha in A}$ is an open cover of $X$. So, $X = C^c \cup \bigcup_{i=1}^n U_{\alpha_i} \implies C \subseteq \bigcup_{i=1}^n U_{\alpha_i}$, so $C$ is compact.
\end{proof}
\thm{}{
    Let $X,Y$ be metric spaces with $X$ compact and suppose $f: X \to Y$ is a continuous bijection. Then $f^{-1} : Y \to X$ is continuous and $f$ is a homeomorphism.
}
\begin{proof}
    First note that if $C \subseteq X$ is closed, then $C$ is compact by the above proposition. So, $f(C)$ is compact and hence closed in $Y$. Define $g = f^{-1}: Y \to X$ and note that for $C \subseteq X$ closed, $g^{-1}(C) = f(C)$ which is closed for all such $C$. As such, $g$ is continuous by the closet set characterization of continuity.
\end{proof}

\ex{}{The theorem fails if $X$ is not compact:
\begin{enumerate}
    \item $[0, 2\pi) \ni t \mapsto (\cos(t), \sin(t)) \in \mathbb S^1$. This is continuous, but its inverse is not.
    \item $(0, 2\pi) \ni t \mapsto (\sin(2t), \sin(t)) \in \RR^2$. This sort of creates a figure 8 as the image. 
\end{enumerate}}
\newpage
\thm{}{Let $X$ be a metric space, $f : X \to Y$ continuous. The following hold:
\begin{enumerate}
    \item Suppose $\emptyset \neq K \subseteq X$ is compact. Then $\forall \epsilon > 0$, $\exists \delta > 0$ such that $x \in K$ and $y \in X$ and $d_X(x, y) < \delta$ implies that $d_Y(f(x), f(y)) < \epsilon$.
    \item If $X$ is compact, then $f$ is uniformly continuous. 
\end{enumerate}}
\begin{proof}
    Clearly $1  \implies 2$, so weonly need to prove 1. Let $\emptyset \neq K \subseteq X$ be compact and $\epsilon > 0$. Since $f$ is continuous, at each $x \in K$, $\exists \delta_x > 0$ such that 
    \begin{align*}
        y \in X \land d_X(x, y) < \delta_x \implies d_Y(f(x), f(y)) < \frac \epsilon 2. 
    \end{align*}
    So then, $\{B(x, \delta_x /2)\}_{x \in K}$ is an open cover in $K$. But since $K$ is compact, we have that 
    \begin{align*}
        K \subseteq \bigcup_{i=1}^n B(x_i, \delta_i /2).
    \end{align*}
    Now let $\delta = \min\{\delta_1/2, \ldots, \delta_n / 2\} > 0$. Now let $x \in K$ and $y \in X$ be such that $d_X(x, y) < \delta$. Since $x \in K$, there eixsts $1 \leq m \leq n$ such that $d_X(x, x_n) < \delta_m /2 $. Thus,
    \begin{align*}
        d_X(x_m, y) &\leq d_X(x_m, x) + d_X(x, y) \\
        &\leq  \frac{\delta_m}{2} + \delta \\
        &\leq \frac{2\delta_m}{2} = \delta_m.
    \end{align*}
    As such,
    \begin{align*}
        d_Y(f(x), f(y)) &\leq d_Y(f(x), f(x_m)) + d_Y(f(x_m), f(y)) \\
        &\leq \frac{\epsilon}{2} + \frac{\epsilon}{2} = \epsilon
    \end{align*}
    as desired.
\end{proof}
\cor{}{Suppose $X$ is a compact metric space. Then 
\begin{align*}
    [X]_{hom} = [X]_{uni}.
\end{align*}}
\newpage
\thm{Borel-Lebesgue}{Let $X$ be a metric space. The following are equivalent:
\begin{enumerate}
    \item $X$ satisfies the Bolzano-Weierstrass limit point property: If $E \subseteq X$ is infinite, then it has a limit point. 
    \item $X$ is sequentially compact: all sequences in $X$ have convergent subsequences.
    \item $X$ is complete and totally bounded.
    \item $X$ is compact.
\end{enumerate}}
\begin{proof}
    (BW $\implies$ sequentially compact): Let $\{x_n\}_{n = \ell}^\infty \subseteq X$ and let $F = \{x_n \mid n \geq \ell\} \subseteq X$. If $F$ is finite then we can extract a constant subsequence, which trivially converges. Assume then that $F$ is an infinite set, which means that $\exists x \in F'$. Pick $n_\ell \geq \ell$ such that $x_{n_\ell} \in B(x, 2^{-\ell}) \cap F \setminus \{x\}$. Now suppose we have $n_\ell < n_{\ell+1} < \dots < n_k$ and $x_{n_\ell}, \ldots, x_{n_k} \in X$ such that
    \begin{align*}
        x_{n_j} \in B(x, 2^{-j}) \cap F \setminus \{x\}.
    \end{align*}
    Note that $B(x, 2^{-(k+1)}) \cap F \setminus \{x\} \neq \emptyset$. Note that this also must be infinite as otherwise we could just shrink the interval as much as we want to not include any of the points. Thus, we can choose $n_{k+1} > n_k$ and $x_{n_{k+1}} \in B(x, 2^{-k - 1}) \cap F \setminus \{x\}$. By induction we know have $\{x_{n_k}\}_{k = \ell}^\infty$ such that $d(x, x_{n_k}) < 2^{-k}$ for all $k \geq \ell$. Therefore $x_{n_k} \to x$ as $k \to \infty$. Therefore, $X$ is sequentially compact.

    (sequentially compact $\implies$ complete and totally bounded): Let $\{x_n\}_{n=\ell}^\infty \subseteq X$ be Cauchy. Sequential compactness give us a convergent subsequence $\{x_{n_k}\}_{k=\ell}^\infty$. We know that Cauchy and convergent subsequence implies convergence, so $\{x_n\}_{n=\ell}^\infty$ is convergent. Also, all convergence subsequences are Cauchy, so all sequences in $X$ have Cauchy subsequences, so $X$ is totally bounded. 

    (complete + totally bounded $\implies$ compact): Let $\{U_\alpha\}_{\alpha \in A}$ be an open cover of $X$. Claim: $\exists \epsilon > 0$ such that $\forall x \in X$, $\exists \alpha \in A$ such that $B(x, \epsilon) \subseteq U_\alpha$. We now prove this claim.

    Suppose the contrapositive. Then for each $n \in \NN$, there is an $x_n \in X$ such that $B(x_n, 2^{-n}) \cap U_\alpha^c \neq \emptyset$ for all $\alpha \in A$. $X$ is totally bounded, so there exists a Cauchy subsequence $\{x_{n_k}\}_{k=0}^\infty$, so this sequence has to be convergent by completeness of $X$. Pick an $\alpha \in A$ such that $x \in U_\alpha$. Since $U_\alpha$ is open, we can pick an $r > 0$ such that $B(x, r) \subseteq U_\alpha$. Now pick $M \geq 0$ such that if $k \geq M$, then $d(x_{n_k}, x) < r/2$ and $2^{-n_k} < r/2$. Now let $y \in B(x_{n_M}, r/2)$. Then $d(x, y) \leq d(x, x_{n_M}) + d(x_{n_M}, y) < r/2 + r/2 = r$. Therefore, $y \in B(x, r)$ and $B(x_{n_M}, r/2) \subseteq B(x, r) \subseteq U_\alpha$. In turn, $B(x_{n_M}, 2^{-n_k}) \subseteq B(x_{n_M}, r/2) \subseteq U_\alpha$ which is a contradiction. 

    So back to the main proof: Let $\{U_\alpha\}_{\alpha \in A}$ be an open cover and let $\epsilon > 0$ be as given by the claim. Then, $X = \bigcup_{i=1}^n B(x_i, \epsilon)$ thanks to totally boundedness. But the claim guarantees that $\exists \alpha_i \in A$ such that $B(x_i, \eps) \subseteq U_{\alpha_i}$, so $X = \bigcup_{i=1}^n U_{\alpha_i}$. Therefore $X$ is compact.

    (compact $\implies$ BW): Let $E \subseteq X$ and suppose that $E' = \emptyset$. Then $\forall x  \in X$, $\exists \eps_x > 0$ such that $E \cap B(x, \eps_x) \setminus \{x\} = \emptyset$. Then, $\{B(x, \eps_x)\}_{x\in X}$ is an open cover of $X$. So, by compactness, $X = \bigcup_{i=1}^n B(x_i, \eps_{x_i})$. Thus, $E = E \cap X = \bigcup_{i=1}^n B(x_i, \eps_{x_i}) \subseteq \bigcup_{i+1}^n \{x_i\}$. Therefore $E$ is finite. But now we're done as if $E$ is infinite, $E'$ cannot be empty. 
\end{proof}
\cor{Heine-Borel}{Let $V$ be a finite dimensional normed vector space and $E \subseteq V$. Then $E$ is compact if and only if $E$ is closed and bounded. }
\begin{proof}
    We saw that sequential compactness is true if and only if closed and bounded when $V = \RR^n$. Pick your favorite linear homeomorphism $T : V \to \RR^n$. We can do this because we can always consider this as $V$ as a vector space over $\RR$. Now $E \subseteq V$ is compact if and only if $E\subseteq V$ is sequentially compact if and only if $T(E) \subseteq \RR^n$ is sequentially compact if and only if $T(E) \subseteq \RR^n$ is closed and bounded.

    Heavily note that since $T$ is linear, it is bi-Lipschitz, which means $T(E)$ is closed and bounded if and only if $E$ is closed and bounded.
\end{proof}
\newpage

\ex{}{
    $B[x, r] \subseteq V$ are totally bounded.
}
\mlenma{}{
    Suppose $\emptyset \neq K \subseteq \RR$ be compact. Then there exists $x_0 = \min k$ and $x_1 = \max K$. 
}
\begin{proof}
    $K$ compact implies $K$ is closed and bounded. Let $M = \sup K$ and $m = \inf K$. Now pick $\{m_n\}_{n}$ and $\{M_n\}_{n}$ such that $m_n \to m$ and $M_n \to M$ as $n \to \infty$. By closedness, $m, M \in K$ so we have a minimum and maximum as desired.
\end{proof}
\thm{Extreme Value Theorem}{
    Let $X$ be a compact metric space and $f : X \to \RR$ be continuous. Then there exists $x_0, x_1 \in X$ such that $f(x_0) \leq f(x) \leq f(x_1)$ for all $x \in X$. 
}
\begin{proof}
    $f(X) \subseteq \RR$ is compact because $f$ is continuous and $X$ is compact. Therefore, the lemma above shows that there is $y_0, y_1 \in f(X)$ such that $y_0 < f(x) < y_1$ for all $x \in X$. So we can write this as 
    \begin{align*}
        f(x_0) \leq f(x) \leq f(x_1)
    \end{align*}
    where $x_0, x_1 \in X$ such that $f(x_0) = y_0$ and $f(x_1) = y_1$.
\end{proof}

\dfn{}{
    Let $X$ be a metric space. 
    \begin{enumerate}
        \item We say $X$ is connected if the only sets in $X$ that are clopen are $\emptyset$ and $X$.
        \item $E \subseteq X$ is a connected subset if 
        \begin{align*}
            \begin{cases}
                E = \emptyset \\
                E \text{ is a connected metric space when endowed with the metric from $X$}
            \end{cases}
        \end{align*}
    \end{enumerate}
}
\nt{Connectness is automatically intrinsic. That is, if $E \subseteq X \subseteq Y$ for $Y$ a metric space, $X \neq \emptyset$, then $E$ is a connected subset of $X$.}
\ex{}{
    \begin{enumerate}
        \item $(0, 1) \cup (1, 2)$ with the metric from $\RR$ is disconnected.
        \item Let $X$ be a metric space and $E = \{x\} \subseteq X$. Then $E$ is trivially connected.
        \item Let $\FF \subset \RR$ be an archimedean field. Pick $z \in \RR \setminus \FF$ and let $L_z = \FF \cap (-\infty, z)$ and $R_z = \FF \cap (z , \infty)$. Then $R_z = L^c_z$ so they are both clopen. 
        \item Let $X$ be a discrete metric space with cardinality greater than 2. $X$ is disconnected. In fact, the only connected subsets are singletons.
    \end{enumerate}
}
\newpage
\thm{}{
    Let $V$ be a normed vector space and $C \subseteq V$ be convex. That is $x, y \in C$ and $t \in [0, 1] \implies (1-t)x + ty \in C$. Then $C$ is connected. In particular, $V$ itself is connected.
}
\begin{proof}
    Suppose BWOC $C$ is disconnected. Then $ \exists \emptyset \neq E \subset C$ that's open and closed. That is, $\emptyset \neq E^c \subset C$ is also open and closed. Let $x \in E$ and $y \in E^c$. Define 
    \begin{align*}
        S = \{s \in [0, 1] \mid (1-t)x + ty \in E \text{ for all $0 \leq t \leq s$}\}.
    \end{align*}
    Note $x \in E$ implies that $0 \in S$. So we can define $s = \sup S \in [0, 1]$. Now define $z := (1-s) + sy \in C$. It must be the case that $z \in E$ or $z \in E^c$.
    \begin{itemize}
        \item Let $z \in E$. Then $E$ is open, so $B(z, \epsilon) \subseteq E$. Then for $0 < \delta < \eps$, the point $w_\delta = x + \left(s + \frac{d}{\Vert x- y \Vert}\right) (y-x)$ is such that 
        \begin{align*}
            \Vert w_\delta - z \Vert = \delta  <\eps \implies w_\delta \subseteq B(z, eps) \subseteq E.
        \end{align*}
        Then, $s + \frac{\eps}{2\Vert x- y \Vert} \in S$, so we have a contradiction to $s$ being an upper bound.
        \item Let $z \in E^c$, but $E^c$ is also open so $B(z, \eps) \subseteq E^c$. Arguing as above shoes that the point $x + \left(s - \frac{\delta}{\Vert x- y\Vert}\right)(y-x) \in B(z, \eps) \subseteq E^c$ for all $0 < \delta< \eps$. Thus, $s - \frac{\eps}{2 \Vert x-y\Vert}$ is an upper bound of $s$, a contradiction.
    \end{itemize}
    Therefore, $C$ is connected.
\end{proof}

\thm{}{Let $X$ and $Y$ be metric spaces with $X$ connected. Let $f: X \to Y$ be continuous. Then $f(X) \subseteq Y$ is connected.}
\begin{proof}
    Suppose not, i.e. $f(X)$ is a disconnected metric space. Pick $\emptyset \neq V \subset f(X)$ such that $V$ is clopen. Note $f: X \to f(X)$ is still continuous. Thus, $f^{-1}(V) \subseteq X$ is clopen. Since $\emptyset \neq V \subset f(X)$, we must have that $\emptyset \neq f^{-1}(V) \subset X$. Therefore, $X$ is disconnected, which is a contradiction.
\end{proof}

\cor{}{Connectedness is a topological property.}

\ex{}{
    Let $f: [0, 2\pi] \to \RR^2$ via $f(t) = (\cos t, \sin t)$. Note that the image is connected, but not convex. Specifically, $\mathbb S^1$ is not convex.
}
\newpage 
\thm{Characterizations of disconnectedness}{
    Let $X$ be a metric space. The following are equivalent:
    \begin{enumerate}
        \item $X$ is disconnected.
        \item $\exists A, B \subseteq X$ nonempty such that $A \cap \overline B = \overline A \cap B = \emptyset$ and $A \cup B = X$.
        \item $\exists$ nonempty closed sets $C, D \subseteq X$ such that $C \cup D = X$ and $C \cap D = \emptyset$. 
        \item $\exists$ nonempety open sets $U, V \subseteq X$ such that $U \cap V = X$ and $U \cap V = \emptyset$.
        \item $\exists$ continuous surjection $f: X \to \{0, 1\}$. 
        \item $\exists$ a discrete metric space $Y$ with cardinality greater than 1 and a continuous surjection $f: X \to Y$.
    \end{enumerate}
}
\begin{proof}
    We go in order.

    (1) $\implies$ (2): $X$ is disconnected, so there exists a clopen set $\emptyset \neq A \subset X$. Let $B := A^c$, which is also clopen and such that $\emptyset \neq B \subset X$. Obviously $A \cup B = X$. Now we check that $A \cap \overline B = A \cap B = \emptyset = \overline A \cap B$ as desired.

    (2) $\implies$ (3): Let $A$ and $B$ as in (2). Note that $\overline A = \overline A \cap (A \cup B) = [\overline A \cap A] \cup [\overline A \cap B] = \overline A \cap A = A$. As such, $A$ is closed. Similarly, $B$ is also closed. Let $C = A, D = B$.

    (3) $\implies$ (4): Let $C, D$ as in (3). Let $U = C^c, V = D^c$.

    (4) $\implies$ (5): Let $U,V$ be the open sets from (4). Define $f: X \to \{0, 1\}$ via
    \begin{align*}
        f(x) = \begin{cases}
            1 & \text{if } x \in U \\
            0 & \text{if } x \in V
        \end{cases}
    \end{align*}
    $f$ is obviously surjective. Note,
    \begin{itemize}
        \item $f^{-1}(\emptyset) = \emptyset$.
        \item $f^{-1}(\{0\}) = U$.
        \item $f^{-1}(\{1\}) = V$.
        \item $f^{-1}(\{0, 1\}) = U \cup V = X$.
    \end{itemize}
    These are all open sets, so $f$ is continuous.

    (5) $\implies$ (6): Let $f: X \to Y$ as in (5). Let $Y = \{0, 1\}$. 

    (6) $\implies$ (1): Let $f: X \to Y$ as in (6) with $Y$ a discrete space (all sets are clopen) and cardinality of $Y$ at least 2. Pick $y , z \in Y$ such that $y \neq z$. Let $E = f^{-1}(\{y\})$, which is clopen because $f$ is continuous. As $f$ is surjective, $E$ is nonempty. Also, $z \neq y \implies E \neq X$. Therefore, $E$ is a nonempty clopen set. As such, $X$ is disconnected.
\end{proof}
\cor{}{
    $C \subseteq \RR$ is connected if and only if $C$ is convex.
}
\begin{proof}
    The backwards direction was done already, so now we prove the forward. Let $x, y \in C$ and suppose $x < z < y$ but $z \notin C$. Let 
    \begin{align*}
        \begin{cases} L_z = C \cap (-\infty, z) \\ R_z = C \cap (z, \infty) \end{cases}.
    \end{align*}
    Both $L_z$ and $R_z$ are open and nonempty, and $L_z \cap R_z = C$, therefore $C$ is disconnected. This is a contradiction, so $x,y \in C \implies (1-t)x + ty \in C$ for all $t \in [0, 1]$, so $C$ is convex.
\end{proof}
\newpage
\thm{Intermediate Value Theorem}{
    Let $X$ be a metric space. Then the following are equivalent:
    \begin{enumerate}
        \item $X$ is connected.
        \item $X$ satisfies the intermediate value theorem: if $f: X \to \RR$ is continuous and $\alpha , \beta \in f(X)$ with $\alpha < \beta$ and $\alpha < \gamma < \beta$, then there exists $x \in X$ such that $f(x) = \gamma$. 
    \end{enumerate}
}
\begin{proof}
    (1) $\implies$ (2): Let $X$ be connected. Since $f$ is continuous, $f(X) \subseteq \RR$ is connected and therefore convex. 

    (2) $\implies$ (1): We'll show $\neg (1) \implies \neg (2)$. Suppose $X$ is disconnected. Then, there exists a continuous surjection $f: X \to \{0, 1\}$. Therefore (2) is false as $f$ does not take on values in $(0, 1)$. 
\end{proof}
\mprop{}{
    Let $X$ be a connected metric space with cardinality of $X$ greater than or equal to 2. Then $X$ is uncountable. 
}
\begin{proof}
    Let $y, z\in X$ be distinct and define $f: X \to \RR$ via $f(x) = d(x, z)$. Note, $f$ is continuous, $f(z) = 0$, and $f(y) = d(y, z) > 0$. Thus, by the intermediate value theorem,  we have the inclusion $[0, d(y, z)] \subseteq f(X)$. Therefore, $f(X)$ is uncountable. Now if $X$ were countable, then there would exist $g: \NN \to X$ that's surjective, meaning $f \circ g : \NN \to f(X)$ is a surjection, implying $f(X)$ is countable which is a contradiction. As such, $X$ is uncountable.
\end{proof}

\chapter{Spaces of Functions}
\dfn{}{
    Let $X \neq \emptyset$ be a set, $Y$ a metric space, and $f, f_n: X \to Y$ for $(n \geq \ell)$. 
    \begin{enumerate}
        \item We say $f_n \to f$ uniformly if $\forall \eps > 0$ there exists $N \geq \ell$ such that 
        \begin{align*}
            n \geq N \implies d(f(x), f_n(x)) < \eps \text{ for all $x \in X$.}
        \end{align*}
        \item We say $f_n \to f$ pointwise if for all $x$, $\forall \eps > 0$, there exists $N \geq \ell$ such that
        \begin{align*}
            n \geq N \implies d(f(x), f_n(x)) < \eps \text{ for all $x \in X$.}
        \end{align*}
        \item We say $\{f_n\}_{n=\ell}^\infty$ is uniformly Cauchy if for every $\eps > 0$, there is $N \geq \ell$ such that 
        \begin{align*}
            n,m \geq N \implies d(f_n(x), f_m(x)) < \eps \text{ for all $x \in X$.}
        \end{align*}
    \end{enumerate}
}
\noindent Clearly, uniformly convergent implies uniformly Cauchy and pointwise convergent.
\nt{Uniform convergence is the same as metric convergence in $\mathcal B(X;Y)$ if we know the sequence is in $\mathcal B(X: Y)$. Otherwise it's not really metric convergence. Can we hack this?}
\noindent \textbf{Idea}: An extended metric space is a set $X$ equipped with a map $d: X \times X \to [0, \infty]$ such that 
\begin{enumerate}
    \item $d(x, y) = 0 \iff x = y$
    \item $d(x, y) \leq d(x, z) + d(z, y)$
    \item $d(x, y) = d(y, x)$
\end{enumerate}

\ex{}{
    Let $X \neq \emptyset$ be a set and $Y$ a metric space. 
    \begin{align*}
        \mathcal F (X; Y)= \{f : X \to Y\}
    \end{align*}
    equipped with $d(f, g) = \sup_{x \in X} d(f(x), g(x))$ is an extended metric.
}
\noindent \textbf{Idea}: In an extended metric space, let $x \sim y \iff d(x,y) < \infty$. This is an equivalence relation. So $d$ restricted to each equivalence class $[x]$ is a metric space. Therefore
\begin{align*}
    X = \bigsqcup [x].
\end{align*}
The moral is that uniform convergence is convergence in $\mathcal F(X; Y)$ with the extended metric. Ditto for Cauchy.

\noindent Things preserved by uniform limits. Suppose $f_n : X \to Y$, $f: X \to Y$ such that $f_n \to f$ uniformly as $n \to \infty$.
\begin{enumerate}
    \item If each $f_n$ is bounded, then $f$ is bounded.
    \item If each $f_n$ is continuous at $x\in X$, then $f$ is continuous at $x$.
    \item If each $f_n$ is uniformly continuous, then $f$ is uniformly continuous.
\end{enumerate}
\noindent This breaks with just pointwise:
\begin{enumerate}
    \item Suppose $f: X \to Y$ is an unbounded function. Fix a point $z \in X$. Let $f_n : X\to Y$ via
    \begin{align*}
        f_n(x) = \begin{cases}
        f(x) & \text{ if $f(x) \in B(f(z), 2^n)$} \\
        f(z) & \text{ otherwise}
        \end{cases}.
    \end{align*}
    Each $f_n$ is bounded and converges to $f$. So $f$ is not necessarily bounded.
    \item Let $f_n : [0, 1]\to \RR$ where $f_n(x) > x^n$ for $n \in \NN$. Then $f_n \to f$ pointwise as $n \to \infty$ where 
    \begin{align*}
        f(x) = \begin{cases} 0 & 0 \leq x < 1 \\
        1 & x = 1
        \end{cases}.
    \end{align*}
    However, each $f_n$ is uniformly continuous but $f$ is not.
    \item Suppose $X, Y$ are metric spaces, $M \geq 0$. Let $L_M = \{f: X \to Y \mid K(f) \leq M\}$. Then $L_M$ is closed under pointwise limits. Indeed, suppose $\{f_n\}_{n=\ell}^\infty \subseteq L_M$ and $f_n \to f$ pointwise as $n \to \infty$. By assumption, $d_Y(f_n(x), f_n(y)) \leq Md_X(x, y)$ for all $x, y \in X$. Fix a pair $x, y\in X$. Then,
    \begin{align*}
        \begin{cases}
            f_n(x) \to f(x) \\
            f_n(y) \to f(y)
        \end{cases}
    \end{align*}
    as $n \to \infty$. Thus, $d_Y(f(x), f(y)) = \lim_{n\to \infty} d_Y(f_n(x), f_n(y)) \leq Md_X(x, y)$. As such, $f$ is Lipschitz with $K(f) \leq M$.
\end{enumerate}

\dfn{}{
    Let $X \neq \emptyset$ and $f_n : X \to \overline \RR$ for $n \geq \ell$. We say $\{f_n\}_{n=\ell}^\infty$ is
    \begin{itemize}
        \item non-decreasing if $m < n \implies f_m(x) \leq f_n(x)$ for all $x \in X$.
        \item non-increasing if $m < n \implies f_n(x) \leq f_m(x)$ for all $x \in X$.
        \item monotone if either are true.
    \end{itemize} 
}
\thm{Dini}{
    Let $X$ be a compact metric space and suppose $\{f_n\}_{n=\ell}^\infty \subseteq C^0 (X; \RR)$ is monotone. Further suppose that there exists a continuous $f : X \to \RR$ such that $f_n \to f$ pointwise as $n \to \infty$. Then $f_n \to f$ as $n \to \infty$. 
}
\begin{proof}
    Suppose WLOG that $\{f_n\}_{n=\ell}^\infty$ is nondecreasing. Let $\eps > 0$. First note that for all $x \in X$, the pointwise converge guarantees $\exists N_x \geq \ell$ such that 
    \begin{align*}
        n \geq N_x \implies \Vert f(x) - f_n(x) \Vert < \frac \eps 3.
    \end{align*}
    In turn, since both $f$ and $f_{N_x}$ are continuous at $x$, $\exists \delta_x > 0$ such that 
    \begin{align*}
        y \in B(x, \delta_x) \implies \begin{cases} \Vert f(x) - f(y)\Vert < \frac \eps 3 \\
        \Vert f_{N_x}(x) - f_{N_x}(y) \Vert < \frac \eps 3
        \end{cases}.
    \end{align*}
    $X$ is compact and $\{B(x, \delta_x)\}_{x\in X}$ is an open cover of $X$. So, $\exists x_1, \ldots, x_n$ such that 
    \begin{align*}
        X = \bigcup_{i=1}^n B(x_i, \delta_i).
    \end{align*}
    Note that if $y \in B(x, \delta_x)$ for some $x\in X$, then 
    \begin{align*}
        \Vert f(y) - f_{N_x}(y) \Vert \leq \Vert f(y) - f(x) \Vert + \Vert f(x) - f_{N_x}(x) \Vert + \Vert f_{N_x}(x) - f_{N_x}(y) \Vert < \eps.
    \end{align*}
    Given any $y \in X$, we know $y \in B(x_i, \delta_i)$ for some $1 \leq i \leq n$. So, 
    \begin{align*}
        \Vert f(y) - f_{N_{x_i}}(y)| < \eps.
    \end{align*} 
    Thus for $n \geq N$, we have that 
    \begin{align*}
        \Vert f(y) - f_n(y) \Vert = f(y) - f_{n}(y) \leq f(y) - f_{N_{x_i}}(y) \leq \Vert f(y) - f_{N_{x_i}}(y) < \eps. 
    \end{align*}
    As such, we have uniform convergence of $f_n \to f$. 
\end{proof}
\dfn{}{
    Let $X,Y$ be metric spaces. Recall that 
    \begin{align*}
        UC^0(X; Y) \subseteq C^0(X; Y) \subseteq \mathcal F(X; Y)
    \end{align*}
    and that 
    \begin{align*}
        UC_b^0(X; Y) \subseteq C_b^0(X; Y) \subseteq \mathcal B(X; Y) \subseteq \mathcal F(X; Y).
    \end{align*}
    We also have 
    \begin{align*}
        C^{0, 1}(X; Y) &\subseteq C^0(X; Y)
        C^{0, 1}_b(X; Y) &\subseteq C^{0}_b(X; Y)
    \end{align*}
    where 
    \begin{align*}
        C^{0, 1}(X; Y) = \{f: X \to Y \mid \text{ $f$ is Lipschitz}\}
    \end{align*}
    and 
    \begin{align*}
        C^{0, 1}_b(X; Y) = \{f: X \to Y \mid \text{ $f$ is Lipschitz and bounded}\}.
    \end{align*}
}
\noindent Fact:
\begin{enumerate}
    \item $Y$ is complete $\iff$ any of $C^0_b, UC^0_b, \mathcal B$ are complete.
\end{enumerate}
But what about $C^{0, 1}_b(X; Y)$? Unfortunately this is not true :(
\newpage
\thm{Weierstrass' Monster}{
    There exists $\{f_n\}_{n=0}^\infty \subseteq C_b^{0, 1}(\RR; [0, 1])$ such that
    \begin{align*}
        f_n \to f \in UC_b^0(\RR; [0, 1])
    \end{align*}
    but $f \notin C_b^{0, 1}(\RR; [0, 1])$. More precisely, for every $x\in \RR$ and $0 < m  \in \NN$, there will exist $\delta_m^{\pm} \in \left\{\frac{8^{-m}}{2}, \frac{3\cdot 8^{-m}}{2}\right\}$ such that
    \begin{align}
        \frac{|f(x \pm \delta^{\pm}_m) - f(x)|}{|\delta_m^\pm|} \geq \frac{7^m + 1}{48}
    \end{align}
    and
    \begin{align}
        \left(\frac{f(x + \delta_m^+) - f(x)}{\delta_m^+} \right) \left(\frac{f(x - \delta^-_m) - f(x)}{\delta^-_m}\right) < 0.
    \end{align}
}
\cor{}{$\exists f \in UC^0_b (\RR ; [0, 1])$ such that $f$ is nowhere differentiable. Moreover, for $x \in \RR$, if $\lim_{y\to 0^-} \frac{f(x+y) - f(x)}{y} = L \in \overline \RR$ and $\lim_{y \to 0^+} \frac{f(x+y) - f(x)}{y} = R \in \overline \RR$, then $L,R \in \{+\infty, -\infty\}$ and $LR < 0$.}
\begin{proof}
    The fact that $f$ is not differentiable at any given $x \in \RR$ follows from $(5.1)$. For the second point, $(5.1)$ also implies that $|L| = |R| = \infty$. And $(5.2)$ implies that $LR < 0$.
\end{proof}

\noindent\textit{Proof of Theorem 5.0.2.} We begin by defining $\varphi: \RR \to [0, 1]$ via 
\begin{align*}
    \varphi(x) = \mathrm{dist}(x, 2\ZZ).
\end{align*}
That is, the distance between $x$ and the closest even integer. Also,
\begin{align}
    |\varphi(x) - \varphi(y)| < 1 \quad \forall x,y\in \RR
\end{align}
and 
\begin{align}
    \varphi(x+2) = \varphi(x) \quad \forall x \in \RR.
\end{align}
Now for $N \in \NN$, we let $f_N \in C_b^{0, 1}(\RR ; \RR)$ be given by 
\begin{align*}
    f_N(x) = \frac 18 \sum_{n=0}^N \left( \frac 78 \right)^n \varphi(8^n x).
\end{align*}
This is Lipschitz because it is the finite sum of Lipschitz functions. Notice that 
\begin{align*}
    0 \leq f_N(x) \leq \frac 18 \sum_{n=0}^\infty \left( \frac 78 \right)^n = \frac 18 \cdot \frac{1}{1 - \frac 78} = 1.
\end{align*}
Also if $M \geq N \geq 0$, then 
\begin{align*}
    |f_M(x) - f_N(x)| &\leq \frac 18 \sum_{n=N + 1}^M \left( \frac 78\right)^n \\
    \sup_{x \in \RR} |f_M(x) - f_N(x) &\leq \frac 18 \sum_{n=N + 1}^M \left( \frac 78\right)^n.
\end{align*}
Therefore, $\{f_N\}_{N=0}^\infty$ is Cauchy in $UC_b^0(\RR; [0, 1])$. This is complete set, so 
\begin{align*}
    \exists f \in UC_b^0(\RR; [0, 1]) \quad \text{such that} \quad f_N \to f.
\end{align*}
So,
\begin{align*}
    f(x) = \frac 18 \sum_{n=0}^\infty \left(\frac 78 \right)^n \varphi(8^n x).
\end{align*}
Now fix $x \in \RR$ and $0 < m \in \NN$. Pick $!k \in \ZZ$ such that $8^mx \in [2k, 2k + 2)$, and set 
\begin{align*}
    \delta^+_m&= 8^{-m}\begin{cases}
    \frac 12 & 8^m x \in [2k, 2k + \frac 12) \\
    \frac 32 & 8^m x \in [2k + \frac 12, 2k +1) \\
    \frac 12 & 8^m x \in [2k + 1, 2k + \frac 32) \\
    \frac 32 & 8^m x \in [2k + \frac 32, 2k + 2)
    \end{cases} \\
    \delta^-_m&= 8^{-m}\begin{cases}
        \frac 32 & 8^m x \in [2k, 2k + \frac 12) \\
        \frac 12 & 8^m x \in [2k + \frac 12, 2k +1) \\
        \frac 32 & 8^m x \in [2k + 1, 2k + \frac 32) \\
        \frac 12 & 8^m x \in [2k + \frac 32, 2k + 2)
    \end{cases}.
\end{align*}
The point is that 
\begin{align*}
    \begin{cases}
        |\varphi(y + 1/2) - \varphi(y)| = \frac 12 \\
        |\varphi(y + 3/2) - \varphi(y)| = \frac 12
    \end{cases}.
\end{align*}
Next, set 
\begin{align*}
    \gamma_n^\pm = \frac{\varphi(8^n(x \pm \delta^{\pm}_m)) - \varphi(8^n x)}{\delta^{\pm}_m}
\end{align*}
for $n \in \NN$. Note that $0 \leq n < m \implies |\gamma_n^\pm| \leq \frac{|8^n \delta_m^\pm|}{|\delta^{\pm}_m|} = 8^n$. Also, $n = m \implies |\gamma_m^\pm| = \frac 12 / |\delta^{\pm}_m \in \{8^m, 8^m/3\}$. 

\noindent Now we consider $m < n$. In this situation, $8^n \delta_m^\pm \in 2\ZZ$, which means that $\gamma_m^\pm = 0$ due to the 2-periodicity of $\varphi$. Thus, for $N > m$:
\begin{align*}
    \frac{8|f_N(x \pm \delta_m^\pm) - f_N(x)|}{|\delta_m^\pm|} = \left| \sum_{n=0}^N \left( \frac 78 \right)^n \gamma_n^\pm\right| = \left| \sum_{n=0}^m \left(\frac 78\right)^n \gamma_n^\pm \right| \geq \left(\frac 78 \right)^m |\gamma_m^\pm| - \sum_{n=0}^{m-1} \left( \frac 78\right)^n |\gamma_n^\pm|.
\end{align*}
But this is also greater than
\begin{align*}
    \left( \frac 78 \right)^m \cdot \frac{8^m}{3} - \sum_{n=0}^{m-1} 7^n = \frac{7^m}{3} - \frac{7^m - 1}{6} = \frac{7^m +1}{6}.
\end{align*}
Sending $N$ to infinity, we get 
\begin{align*}
    \frac{|f(x \pm \delta_m^\pm) - f(x)|}{|\delta_m^\pm|} \geq \frac 18 \cdot \frac{7^m + 1}{6}.
\end{align*}
In particular, $K(f) \geq \liminf_{m \to\infty} '' \geq \liminf_{m \to \infty} \frac{7m + 1}{48} = \infty$. So, $f \notin C^{0,1}_b$, but $f \in UC_b^0$. A similar argument shows the other inequality.
$\hfill \qed$
\newpage
\nt{Remarks:
\begin{enumerate}
    \item Weierstrass' original proof used trigonometric functions in place of $\varphi$. The upshot is that there exists sequences of smooth functions that converge uniformly to a uniformly continuous and bounded function that is nowhere differentiable.
    \item There's room to play.
\end{enumerate}}
\dfn{}{
    Let $V$ be a normed vector space, $X$ a metric space. We define a seminorm on $C^{0,1}(X  ;V)$ by
    \begin{align*}
        K(f) = [f]_1 = \begin{cases}
            \sup_{x \neq y} \frac{\Vert f(x) - f(y) \Vert_V}{d(x,y)} & |X| \geq 2 \\
            0 & \text{otherwise}
        \end{cases}.
    \end{align*}
    This is a seminorm: $[\text{constant}] = 0$. $C^{0,1}_b(X; V)$ is normed:
    \begin{align*}
        \Vert f\Vert_{C^{0,1}_b} = \Vert f \Vert_{C^0_b} + [f]_1.
    \end{align*}
}
\thm{}{
    Let $X$ be a metric space and $V$ a normed vector space. Then $C^{0,1}_b(X; V)$ is complete if and only if $V$ is complete.
}
\begin{proof}
    We only do the backwards direction. Suppose $V$ is Banach and $\{f_n\}_{n=\ell}^\infty C_b^{0,1}$ is Cauchy. Then it's Cauchy in $C^0_b$ as well, and hence convergent in $C^0_b$ to some $f \in C^0_b(X; V)$. 

    Let $\eps > 0$ and pick $N \geq \ell$ such that $m,n \geq N \implies \Vert f_n - f_m \Vert_{C_b^{0,1}} < \frac{\eps}{2} \implies [f_n - f_m] < \frac{\eps}{2}$. As such,
    \begin{align*}
        \Vert (f_n(x) - f_m(x)) - (f_n(y) - f_m(y))\Vert_V \leq \frac{\eps}{2} d(x, y)
    \end{align*}
    for all $x,y \in X$. Send $m \to \infty$ for $n \geq N$, we have $[f_n -f] \leq \frac{\eps}{2}$.
\end{proof}
\newpage
\section{Stone-Weierstrass}
\dfn{}{
    A vector space $V$ is an algebra if it is equipped with an associative multiplication. That is,
    \begin{align*}
        v\cdot w \in V \text{ for $v,w \in V$} \\
        v\cdot(w \cdot z) = (v\cdot w)\cdot z \\
        \alpha(v\cdot w) = (\alpha v) \cdot w = v \cdot (\alpha w) \\
        v(w+z) = vw + vz \\
        (w+z)v = wv + zv.
    \end{align*}
    We say $V$ is unital if there exists an $e \in V$ such that $ev = ve = v$ for all $v \in V$. 

    A normed algebra is an algebra $V$ that's a normed vector space such that 
    \begin{align*}
        \Vert vw \Vert \leq \Vert v \Vert \Vert w \Vert.
    \end{align*}
    A Banach algebra is a complete normed algebra.
}
\ex{Normed Algebras}{
    \begin{enumerate}
        \item $\FF \in \{ \RR, \CC\}$ a field.
        \item $\FF^{n \times m}$ with the operator norm.
        \item Let $V$ be a normed vector space. Then $\mathcal L(V)$ is a normed algebra. $T,S \in \mathcal L(V)$, $TS = T \circ S$.
        \item Let $A$ be a normed algebra. $\mathcal B(X; A)$ is a normed algebra. $f,g \in \mathcal (X; A)$, $fg \in \mathcal(B; X)$, $(fg)(x) = f(x)g(x)$. We  can check that $A$ unital $\iff$ $\mathcal B(X; A)$ unital.
        \item $\ell^p(\NN; A)$ is a normed algebra. $x,y \in \ell^p \implies (xy)_n = x_n y_n \in A$. Also, 
        \begin{align*}
            \Vert xy \Vert_p = \left( \sum_{n=0}^\infty \Vert x_n y_n \Vert_A^p \right)^{\frac{1}{p}} \leq \left( \sum \Vert x_n\Vert^p \Vert y_n\Vert^p \right)^{\frac{1}{p}} \leq \Vert x \Vert_\infty \Vert y \Vert_\infty \leq \Vert x \Vert^p \Vert y \Vert^p.
        \end{align*}
        \item Let $X$ be a metric space. Then $C^0(b(X; A))$, $UC_b^0(X; A)$ and $C^{0,1}_b(X; A)$ are normed algebras.
        \item Now let $Y \subset X$ as above. $S = \{f \in C^0_b(X; A) \mid f(y) = 0 \,  \forall y \in Y\}$ is a normed algebra.
    \end{enumerate}
}
\noindent Remarks:
\begin{enumerate}
    \item Suppose $A$ is a normed algebra. If we complete $A$ to get $A^*$, then $A^*$ is a Banach algebra. 
    \item Suppose $A$ is a normed algebra and $B \subseteq A$ is a subalgebra. Then $\overline B$ is also a normed algebra. 
\end{enumerate}

\thm{}{
    Let $X$ be a unital Banach algebra. Then for every $x \in X$, the sequence $\{\sum_{n=0}^N \frac{x^n}{n!}\}_{N=0}^\infty \subseteq X$ converges. Thus, we may define the map $\exp : X \to X$ via $\exp(x) = \sum_{n=0}^\infty \frac{x^n}{n!}$.
}
\begin{proof}
    Recall that 
    \begin{align*}
        \sum_{n=0}^\infty \frac{r^n}{n!} \in \RR
    \end{align*}
    is well defined (convergent) because it is equal to 
    \begin{align*}
        \lim_{N\to \infty} \sum_{n=0}^N \frac{r^n}{n!}
    \end{align*}
    for all $r \in \RR$. Now note for $N  > M \geq 0$ and some $x \in X$,
    \begin{align*}
        \left\Vert \sum_{n=M+1}^N  \frac{x^n}{n!} \right\Vert \leq \sum_{n=M+1}^N \frac{\Vert x^n \Vert}{n!} \leq \sum_{n=M+1}^N \frac{\Vert x \Vert^n}{n!}
    \end{align*}
    Therefore, $\left\{ \sum_{n=0}^N \frac{x^n}{n!} \right\}_{N=0}^\infty$ is Cauchy and therefore converges.
\end{proof}
\noindent Question: Suppose $A \subseteq C^0_b(X; \FF)$ is a subalgebra. Are there algebraic conditions on $A$ that guarantee that $A$ is dense in $C_b^0(X; \FF)$? 

\noindent Answer: Yes.

\mlenma{}{There eixst $\{f_n\}_{n=0}^\infty \subseteq UC_b^0([0, 1]; \RR)$ such that the following hold:
\begin{enumerate}
    \item Each $f_n$ is a polynomial satisfying $0 \leq f_n(x) \leq \sqrt x$ for all $x \in [0, 1]$. Also, $f_n \leq f_{n+1}$. 
    \item $f_n \to f$ uniformly as $n \to \infty$ where $f(x) = \sqrt x$. 
\end{enumerate}
}
\begin{proof}
    We define $\{f_n\}_{n=0}^\infty$ inductively as follows:
    \begin{itemize}
        \item $f_0(x) = 0$.
        \item Given $f_n$, define $f_{n+1}(x) = f_n(x) + \frac{x - (f_n(x))^2}{2}$.
    \end{itemize}
    We claim $0 \leq f_n(x) \leq \sqrt x$ for all $n \in \NN$ and $x \in [0,1]$. We proceed by induction on $n$:

    Suppose true for $n$, we show truth for $n+1$. Then $0 \leq f_n(x) \leq \sqrt x \implies 0 \leq \frac{f_n(x) + \sqrt x}{2} \leq \sqrt x \leq 1$. So,
    \begin{align*}
        \sqrt x  - f_{n+1}(x) = \sqrt x  - f_n(x) - \frac{x - f_n^2(x)}{2}= (\sqrt x - f_n(x))\left( 1 - \frac{\sqrt x  + f_n(x)}{2} \right)^2 \geq 0,
    \end{align*}
    so $\sqrt x \geq f_{n+1}(x) \geq f_n(x) \geq 0$. Next, note that 
    \begin{align*}
        f_{n+1}(x) = f_n(x) + \frac{x - f_n^2(x)}{2} \geq f_{n}(x)
    \end{align*}
    so the sequence of $f_n$'s is non-decreasing. Thus, for all $x \in [0,1]$, 
    \begin{align*}
        0 \leq f_n(x) &\leq \sqrt x \\
        f_n(x) &\leq f_{n+1}(x).
    \end{align*}
    Now since the sequence is bounded and monotone, it converges pointwise to its supremum. Now,
    \begin{align*}
        f_{n+1}(x) = f_n(x) + \frac{x - f_n^2(x)}{2} \implies f(x) = f(x) + \frac{x + f^2(x)}{2} \implies f(x) = \sqrt x
    \end{align*}
    as desired. Therefore, Dini tells us the convergence of $f_n \to f$ is uniform, since $[0,1]$ is compact.
\end{proof}
\newpage
\thm{}{
    Let $X$ be a set and $A \subseteq \mathcal B(X; \RR)$ a subalgebra. The following hold:
    \begin{enumerate}
        \item If $f \in A$, then $|f| \in \overline A$. 
        \item If $f \in \overline A$, then $|f| \in \overline A$. 
        \item If $f,g \in \overline A$, then $f \vee g, f \wedge g \in \overline A$ where 
        \begin{align*}
            f \vee g(x) &= \max\{f(x), g(x)\} \\
            f \wedge g(x) &= \min\{f(x), g(x)\}.
        \end{align*}
    \end{enumerate}
}
\begin{proof} We go in order:
    \begin{enumerate}
        \item Let $\{p_n\}_{n=0}^\infty \subseteq UC_b^0( [0, 1], [0,1])$ be as in the previous lemma. That is, $p_n$ are polynomials such that $p_n \to \sqrt \cdot$ uniformly. If $f = 0$, then $|f| = 0 = |f| \in A \subseteq \overline A$, so suppose $f \neq 0$. Then let 
        \begin{align*}
            g = \left(\frac{f}{\Vert f \Vert_{\mathcal B}}\right)^2 \in A.
        \end{align*}
        Note that $0 \leq |g(x)| \leq 1$ for all $x \in X$. Also, $p_n(0) = 0$ for all $n \in \NN$. So, $p_n \circ g \in A$ as well. Now note that $d_{\mathcal B} (p_n \circ g,\sqrt g) = \sup_{x \in X} |p_n(g(x)) - \sqrt{g(x)}| \leq \sup_{y \in [0,1]} |p_n(y) - \sqrt y|$, which tends to 0 as $n \to \infty$ by construction. 
        
        Therefore, $p_n \circ g \to \sqrt g \in \mathcal B(X; \RR)$, so $\sqrt{g} \in \overline A$. But, $\sqrt g = \frac{|f|}{\Vert f \Vert_{\mathcal B}}$. So,
        \begin{align*}
            |f| = \Vert f \Vert_{\mathcal B} \sqrt{g}.
        \end{align*}
        So since $\sqrt g \in \overline A$, so is $|f|$. 
        \item Let $f \in \overline A$. Pick $\{f_n\}_{n=0}^\infty \subseteq A$ such that $f_n \to f$ in $\mathcal B$. By 1, $|f_n| \in \overline A$ for all $n$. Now notice 
        \begin{align*}
            | |f_n| - |f|| \leq |f_n - f|,
        \end{align*}
        so $|f_n| \to |f|$ as $n \to \infty$, but $\overline A$ is closed, so $|f| in \overline A$. 
        \item Note that $f \vee g = \frac{f + g + |f-g|}{2}$, $f \wedge g = \frac{f + g - |f-g|}{2}$. Because of algebra, $f,g \in A \implies f \vee g, f \wedge g \in \overline A$.
    \end{enumerate}
\end{proof}
\noindent Remark: There's nor eason to stick with a single min/max. 3. + induction shows that if $f_1, \ldots f_n \in A$, then $\min_{i} f_i, \max_i f_i \in \overline A$. 

\dfn{}{
    Let $X \neq \emptyset$ and $A \subseteq \mathcal B(X; \FF)$ a subalgebra. We say 
    \begin{enumerate}
        \item $A$ contains the constants if $\forall c \in \FF$, the constant function $c \in A$. 
        \item $A$ is non-vanishing if $\forall x \in X$, there exists $f \in A$ such that $f(x) \neq 0$. 
        \item $A$ separates points if for every $x,y \in X$ such that $x \neq y$, there exists $f \in A$ such that $f(x) \neq f(y)$. 
        \item $A$ satisfies the two-point property if for every distinct $x,y \in X$ and $\forall \alpha, \beta \in \FF$ , there exists $f \in A$ such that $f(x) = \alpha$ and $f(y) =\beta$. 
    \end{enumerate}
}
\mprop{}{
    Let $X$ be a set and $A \subseteq \mathcal B(X; \FF)$ a subalgebra. Then the following hold:
    \begin{enumerate}
        \item If $A$ contains constants, then $A$ is non-vanishing.
        \item $A$ separates points and is non-vanishing implies $A$ satisfies the two-point property.
        \item If $|X| \geq 2$, then the two-point property implies $A$ separates points and is non-vanishing.
    \end{enumerate}
}
\begin{proof}
    Lots of definition chasing.
    \begin{enumerate}
        \item Duh.
        \item Fix $x,y \in X$ such that $x \neq y$. $A$ separates, so there is $g \in A$ such that $g(x) \neq g(y)$. As $A$ is non-vanshing, there exists $h,k \in A$ such that $h(x) \neq 0$ and $k(x) \neq 0$. Let $u,v \in A$ be given by 
        \begin{align*}
            \begin{cases}
                u(z) = (g(z) - g(x))k(z) \\
                v(z) = (g(z) - g(y))h(z)
            \end{cases}.
        \end{align*}
        Note that $u(x) = 0$ and $u(y) \neq 0$. Also $v(x) \neq 0$ and $v(y) =0$. For $\alpha, \beta \in \FF$, let $f \in A$ be given by 
        \begin{align*}
            f(z) = \frac{\alpha}{v(x)} v(z) + \beta\frac{u(z)}{u(y)}.
        \end{align*}
        Note that $f(x) = \alpha$ and $f(y) =\beta$. 
        \item Given any $x,y \in X$ such that $x \neq y$, there is $f \in A$ such that $f(x) = 1$ and $f(y) = 2$. Therefore, $A$ separates and is non-vanshing.
    \end{enumerate}
\end{proof}
\thm{Stone-Weierstrass}{
    Let $X$ be a compact metric space and $\{0\} \neq A \subseteq C^0(X; \RR)$ be a subalgebra. Then the following are equivalent:
    \begin{enumerate}
        \item $A$ satisfies the two-point property.
        \item $A$ is dense in $C^0(X; \RR)$.
    \end{enumerate}
}
\begin{proof}
    If $|X| = 1$, then infact $A = C^0(X; \RR)$ and we're done. So assume $|X| \geq 2$. 
    \begin{itemize}
        \item $2 \implies 1$. We'll prove $\neg 1 \implies \neg 2$. If $A$ doesn't satisfy the two-point property, then either $A$ is not non-vanshing or $A$ does not separate.
        \begin{itemize}
            \item If $A$ is not non-vanishing, there is $x \in X$ such that $f(x) = 0$ for all $f \in A$. Thus, if $f \in \overline A$ and $f_n \to f$ in $C^0(X; \RR)$, then $f_n(x) = 0$ for all $n$, which means $f(x) = 0$. Therefore, $A$ cannot be dense because there are functions where $f(x) \neq 0$. 
            \item If $A$ does not separate, then there exists $x \neq y$ such that $f(x) = f(y)$ for all $f \in A$. Therefore, $f(x) = f(y)$ for all $f \in \overline A$. But $f \in C^0(X; \RR)$ given by $f(z) = d(z, y)$ is such that $f(x) = d(x,y) > 0$, but $f(y) = 0$. As such, we have the same conclusion.
        \end{itemize}
        \item $1 \implies 2$: Let $f \in C^0(X; \RR)$ and $\eps > 0$. We will find $h \in A$ such that $d_{\mathcal B}(f, h) = \sup_{x\in X} | f(x)  - h(x)| < \eps$. Once we do this, we then know $\overline A = C^0(X; \RR)$. 
        
        Claim: for every $z \in X$, there is $g \in A$ such that $g(z) = f(z)$ and $g(y) > f(y) - \frac{\eps}{2}$ for all $y$. To prove the claim, we first use the two point property. For each $x \in X \setminus \{z\}$, we can choose $\psi_x \in A$ such that 
        \begin{align*}
            \psi_x(x) = f(x) \quad \quad \psi_x(z) = f(z).
        \end{align*}
        Also define $\psi_z \in A$ via \begin{align*}
            \psi_z(z) = f(z) \quad \quad \psi_z(y) = 0 \text{ for all $y \neq z$}.
        \end{align*}
        Consider $x \in X$. Both $\psi_x$ and $f$ are continuous, so $\exists \delta_x > 0$ such that
        \begin{align*}
            y \in B(x, \delta_x) \implies |\psi_x(y) - f(y)| = |\psi_x(y) - f(y) - (\psi_x(x) - f(x))| < \frac{\eps}{2}.
        \end{align*}
        Note that $\{B(x, \delta_x)\}_{x\in X}$ is an open cover, so by compactness there exist $x_1 , \ldots, x_n \in X$ such that $X = \bigcup_{i=1}^n B(x_i, \delta_i)$. Define $g = \max (\psi_{x_1}, \ldots, \psi_{x_n}) \in \overline A$. Note that 
        \begin{align*}
            g(z) = \max\{f(z), \ldots, f(z)\} = f(z).
        \end{align*}
        And if $y \in X$, then $y \in B(x_i, \delta_i)$, meaning 
        \begin{align*}
            | \psi_{x_i}(y) - f(y) < \frac \eps 2 \implies \psi_{x_i}(y) > f(y) - \frac{\eps}{2}.
        \end{align*}
        This means that $g(y) > f(y) - \frac \eps 2$. 

        So now for each $z \in X$, the first step yields $g_z \in \overline A$ such that $g_z(z) = f(z)$ and $g_z(y) > f(y) - \frac \eps 2$ for all $y \in X$. Since $g_z$ and $f$ are continuous, there will exist $\eta_z > 0$ such that 
        \begin{align*}
            y \in B(z, \eta_z) \implies |g_z(y) - f(y)| = |g_z(y) - g_z(z) - (f(y) -f(z))| < \frac \eps 2 \implies g_z(y) < f(y) + \frac \eps 2.
        \end{align*}
        Now pick $z_1, \ldots, z_n \in X$ such that $X = \bigcup_{i=1}^n B(z_i, \eta_i)$. Now set $h = \min\{g_{z_1}, \ldots, g_{z_n}\} \in \overline A$. Then $h(y) = \min \{g_{z_1}(y), \ldots g_{z_n}(y)\} > f(y) - \frac \eps 2$ for all $ \in X$. And, $y \in X$ implies that $y \in B(z_i, \eta_i)$, meaning $g_{z_i}(y) < f(y) + \frac \eps 2$. So $h(y) , f(y) + \frac \eps 2$. This finishes the proof.
    \end{itemize}
\end{proof}
\ex{Stone Weierstrass fails on $\CC$}{
    Let $X = \{z \in \CC \mid |z| = 1\}$. Then let 
    \begin{align*}
        A = \left\{ f: X \to \CC \mid f(z) = \sum_{n=0}^N a_nz^n \text{ for $a_n \in \CC$}\right\}.
    \end{align*}
    Exercise: $A$ is a subalgebra of $C^0(X; \CC)$ that satisfies the two-point property. Note that $f \in A$ means that $[0, 2\pi] \ni t \mapsto f(e^{it}) \in \CC$.

    Then for $f \in A$, analyze
    \begin{align*}
        \int_0^{2\pi} f(e^{it}) \text dt &= \int_0^{2\pi} \sum_{n=0}^N  e^{it} \sum_{n=0}^N a_n e^{int} \text dt \\
        &= \sum_{n=1}^{N+1} a_n \int_0^{2\pi} e^{i(n+1)t} \text dt \\
        &= 0.
    \end{align*}
    Therefore, $f \in \overline A$ means $\int_0^{2\pi} e^{it}f(e^{it}) \text dt = 0$. Consider $f \in C^0 (X; \CC)$ given by $f(z) = \overline z$. Then, 
    \begin{align*}
        \int_0^{2\pi} e^{it}f(e^{it}) \text dt &= \int_0^{2\pi} e^{it} e^{-it} \text dt = 2 \pi \neq 0.
    \end{align*}
    As such, $\overline A \subset C^0(X; \CC)$.

    Observe $f \in A \iff f(z) = \sum a_nz^n$. So 
    \begin{align*}
        \overline f(z) = \sum_{n=0}^N \overline a_n \overline z^n = \sum_{n=0}^N \frac{\overline a_n}{z^n},
    \end{align*}
    which is not a polynomial, meaning $\overline f \notin A$. 
}

\thm{Stone-Weierstrass in $\CC$}{
    Let $X$ be a compact metric space and $\{0\} \neq A \subseteq C^0(X; \RR)$ be a subalgebra that satisifies the two-point property and is self-adjoint $(f \in A \implies \overline f \in A)$. Then $A$ is dense in $C^0(X; \CC)$.
}
\begin{proof}
    Note: given $f \in C^0(X; \RR)$, we have 
    \begin{align*}
        \mathrm {Re} f = \frac{f + \overline f}{2} \quad \quad \mathrm{Im} f = \frac{f - \overline f}{2}.
    \end{align*}
    This means that $f \in A \implies \mathrm{Re} f, \mathrm{Im} f \in A$. Then $R = \{\mathrm{Re} f \mid f \in A\} = \{ \mathrm{Im} f \mid f\ \in A\} \subseteq A$. It is left as an exercise to show that $R$ is a real subalgebra of $C^0(X; \RR)$ that satisfies the two-point property. As such, $R$ is dense in $C^0(X; \RR)$ by the real version of Stone-Weierstrass. Now let $f \in C^0(X; \CC)$ and $\eps > 0$. 

    Pick $a,b \in \RR$ such that $\Vert \mathrm{Re} f - a \Vert_{\mathcal B} < \frac{\eps}{2}$ and $\Vert \mathrm{Im} f - b \Vert_{\mathcal B} < \frac{\eps}{2}$. Then $a + ib \in A$ and 
    \begin{align*}
        \Vert f - (a + ib) \Vert_{\mathcal B} = \Vert (\mathrm{Re}f - a) + i(\mathrm{Im} - b) \Vert_{\mathcal B} < \frac{\eps}{2} + \frac{\eps}{2} = \eps.
    \end{align*}
    As such, $\overline A = C^0(X; \CC)$.
\end{proof}
\ex{}{
    \begin{enumerate}
        \item Let $\emptyset \neq K \subset \RR^n$ be a compact set and consider 
        \begin{align*}
            \mathcal P_{\FF}(K) = \{ p\restriction K \mid p:  \RR^n \to \FF \text{ is a polynomial}\} \subseteq C^0(K; \FF).
        \end{align*}
        It's easy to check that $\mathcal P_{\FF}(K)$ is a subalgebra and it's not $\{0\}$. If $K$ is a singleton, we're done so suppose otherwise.

        It suffices to show that $\mathcal P_{\FF}(K)$ contains constants and separates points. Constants are trivial. 

        Now let $x,y \in K$ with $x_i \neq y_i$. Let $p: \RR^n \to \FF$ via 
        \begin{align*}
            p(z) = 2(z_i - x_i) + (z_i - y_i).
        \end{align*}
        Then $p(x) = x_i + y_i$ and $p(y) = 2(y_i - x_i) \neq y_i - x_i$.

        If $\FF = \RR$, then Stone Weierstrass says that $\mathcal P_{\FF}(K)$ is dense in $C^0(K; \RR)$. On the other hand, $f \in \mathcal P_{\CC}(K)$ if and only if $f(x) = \sum_{|\alpha| \leq d} a_\alpha x^{\alpha}$ for $a_\alpha \in \CC$. This means that $\overline f(x) = \sum_{|\alpha| \leq d} \overline{a_\alpha}x^\alpha \implies \overline f \in \mathcal P_{\CC}(K)$, so the algebra is self-adjoint and therefore dense in $C^0(K; \CC)$.
        \item Let $0 < R < \infty$ and set $X = \{z \in \CC \mid |z| = R\}$. 
        \begin{align*}
            A \coloneq \{f : X \to \CC \mid \exists n \text{ such that }f(z) = \sum_{k = -n}^n a_k z^k , a_k \in \CC\} \subseteq C^0(X; \CC).
        \end{align*}
        $A$ is a subalgebra and it's an exercise to show that it satisfies the two-point property. Note that for $z \in X$, $\overline   = \frac{R^2}{2}$. So, 
        \begin{align*}
            f(z) = \sum_{|k| \leq n} a_k z^k \implies \overline f(z) = \sum_{|k|\leq n} \overline a_k \overline z^k = \sum_{|k| \leq n} \overline a_k \left(\frac{R^2}{z}\right)^k = \sum_{|k| \leq n}R^{2k} \overline a_k z^{-k} = \sum_{|k| \leq n} R^{-2k} \overline a_{-k} z^k.
        \end{align*}
        Now by Stone-Weierstrass, $\overline A = C^0(X; \CC)$. 
        \item Let $X$ be a compact metric space. Note that $L = C^{0,1}(X; \FF) \subseteq C^0(X; \FF)$. We claim that $\overline L = C^0(X; \FF)$. $L$ is a subalgebra:
        \begin{align*}
            |f(x)g(x) - f(y)g(y)| \leq |f(x) - f(y)||g(x)| + |f(y)||g(x) - g(y)| \leq [f]\cdot \Vert g\Vert_{C^0}|x-y| + \Vert f\Vert_{C^0} [g]|x-y|.
        \end{align*}
        Let $x_0 \neq x_1$, $\alpha, \beta \in \FF$ and define $f: X \to \FF$ via 
        \begin{align*}
            f(x) \ frac{\alpha d(x, x_1) + \beta d(x, x_0)}{d(x_0, x_1)}.
        \end{align*}
        Then $f \in L$ and $f(x_0) = \alpha$, $f(x_1) = \beta$. Therefore $L$ satisfies the two point property. On the other hand, if $\FF = \CC$ and $f \in L$, then $\overline f \in L$, so $L$ is self-adjoint. Therefore by Stone Weierstrass, $\overline L = C^0(X; \FF)$.
    \end{enumerate}
}
\chapter{Advanced Topics in Metric Space Theory}
we will come back here in the last week of the semester i guess
\chapter{Series}
\dfn{}{
    Let $X$ be a normed vectors pace and $\{x_n\}_{n=\ell}^\infty \subseteq X$. For $N \geq \ell$, let $S_N = \sum_{n=\ell}^N x_n \in X$ ($N$-th partial sum). If the sequence $\{S_N\}_{N=\ell}^\infty$ converges to $s\in X$, we say that the infinite series $\sum_{n=\ell}^\infty x_n$ converges and set $\sum_{n=\ell}^\infty x_n = \lim_{N \to \infty} S_n =s$. Otherwise, we say $\sum_{n=\ell}^\infty x_n$ diverges. 
}

\mlenma{}{
    Suppose $\{x_n\}_{n=\ell}^\infty \subseteq X$ for $X$ a normed vector space and $\sum_{n=\ell}^\infty x_n$ converges. Then $x_n \to 0$ as $n \to \infty$. 
}
\begin{proof}
    We can write $x_n = s_n -s_{n-1}$ for $n \geq \ell + 1$. As the RHS converges to 0, $x_n$ must as well. 
\end{proof}
\mlenma{}{
    Suppose $\{x_n\}_{n=\ell}^\infty \subseteq X$, $X$ a normed vector space, is such that $x_n \to x$ as $n \to \infty$. Let $\{y_n\}_{n=\ell}^\infty \subseteq X$ be such that $y_n = x_{n+1} -x_n$. Then $\sum_{n=\ell}^\infty y_n$ converges and equals $x - x_\ell$.
}
\begin{proof}
    We have 
    \begin{align*}
        \sum_{n=\ell}^N y_n = \sum_{n=\ell}^N x_{n+1} - x_n = x_{N+1} - x_\ell.
    \end{align*}
    So this goes to $x - x_{\ell}$ as $N \to \infty$.
\end{proof}

\thm{Cauchy Criterion}{
    Let $X$ be a Banach space and $\{x_n\}_{n=\ell}^\infty \subseteq X$. The following are equivalent:
    \begin{enumerate}
        \item $\sum_{n=\ell}^\infty x_n$ converges.
        \item For every $\eps > 0$, there is $N \geq \ell$ such that $M > N \implies \left\Vert\sum_{n=N}^M x_n\right\Vert < \eps$.
    \end{enumerate}
    That is, $\sum_{n=\ell}^\infty x_n$ converges if and only if $\left\{S_N = \sum_{n=\ell}^N x_n\right\}_{N}$ converges if and only if $\{S_N\}_{N}$ is Cauchy if and only if 2.
}
\newpage 
\thm{}{
    Let $X$ be a normed vector space. The following are equivalent:
    \begin{enumerate}
        \item $X$ is Banach.
        \item If $\{x_n\}_{n=\ell}^\infty$ is such that $\sum_{n=\ell}^\infty \Vert x_n\Vert < \infty$, then $\sum_{n=\ell}^\infty x_n$ converges.
    \end{enumerate}
}
\begin{proof}
    We go in order.
    \begin{itemize}
        \item $(1) \implies (2)$: Note: $\left\Vert \sum_{n=N}^M\right\Vert \leq \sum_{n=N}^M \Vert x_n \Vert$. Apply Cauchy Criterion in $\RR$, then $X$.
        \item $(2) \implies (1)$: Let $\{x_n\}_{n=\ell}^\infty$ be Cauchy. Extract a subsequence $\{x_{n_k}\}_{k=\ell}^\infty$ such that $\Vert x_{n_{k+1}} - x_{n_k}\Vert \leq 2^{-k}$. Now define $y_k = x_{n_{k+1}} - x_{n_k}$ for $k \geq \ell$. Then 
        \begin{align*}
            \Vert y_k \Vert  \leq 2^{-k} \implies \sum_{k=\ell}^\infty \Vert y_k\Vert \leq \sum_{k=\ell}^\infty 2^{-k} < \infty,
        \end{align*}
        so (2) implies $\sum_{k=\ell}^\infty y_k = y$ converges in $X$. 

        Then $\sum_{k=\ell}^M y_k = x_{n_{M+1}} - x_{n_{\ell}}$. So, 
        \begin{align*}
            \lim_{M \to \infty} x_{n_{M+1}} = y + x_{n_\ell}.
        \end{align*}
        So $\{x_{n_k}\}_{k=\ell}^\infty$ is convergent, so $\{x_n\}_{n=\ell}^\infty$ also converges and therefore $X$ is Banach.
    \end{itemize}
\end{proof}
\dfn{}{
    Let $X$ be a normed vector space, $\{x_n\}_{n=\ell}^\infty \subseteq X$. We say $\sum_{n=\ell}^\infty x_n$ converges absolutely if $\sum_{n=\ell}^\infty \Vert x_n\Vert < \infty$.
}
\nt{If $X$ is Banach, then aboslute convergence implies convergence.}
\thm{Root Test}{
    Let $X$ be a Banach space and $\{x_n\}_{n=\ell}^\infty \subseteq X$. Define $\alpha = \limsup_{n \to \infty}\Vert x_n\Vert^{1/n} \in [0, \infty]$. The following hold:
    \begin{enumerate}
        \item If $0 \leq \alpha < 1$, then $\sum_{n=\ell}^\infty x_n$ converges absolutely.
        \item If $\alpha > 1$, then $\sum_{n=\ell}^\infty x_n$ diverges.
    \end{enumerate}
}
\begin{proof}
    We go in order.
    \begin{enumerate}
        \item Suppose $0 \leq \alpha < 1$. Then we can pick $\alpha < \beta < 1$. More precisely, there is $N \geq \max(1, \ell)$ such that $n \geq N$ means $\Vert x_n\Vert^{1/n} < \beta$. In particular, $\Vert x_n\Vert < \beta^n$. Thus,
        \begin{align}
            \sum_{n =\ell}^\infty \Vert x_n\Vert = \sum_{n=\ell}^N \Vert x_n\Vert + \sum_{n=N+1}^\infty \Vert x_n\Vert \leq \sum_{n=\ell}^N \Vert x_n\Vert + \frac{1}{\beta^{N+1}}\frac{1}{1-\beta} < \infty.
        \end{align}
        \item Suppose $\alpha > 1$. Then $\exists$ a subsequence $\{x_{n_k}\}_{k=\ell}^\infty$ such that $\Vert x_{n_{k}}\Vert^{1/n_k} \to \alpha > 1$. In particular, $\Vert x_{n_k}\Vert \geq 1$ for $k$ large enough. So we see that $\sum_{n=\ell}^\infty \Vert x_{n}\Vert$ diverges as it has infinitely many elements greater than $1$.
    \end{enumerate}
\end{proof}

\mlenma{}{
    Let $\{a_n\}_{n=\ell}^\infty \subseteq[0, \infty]$ be such that $\frac{a_{n+1}}{a_n}$ is well-defined in $\overline \RR$ for all $n \geq \ell$. Then 
    \begin{align*}
        \limsup_{n\to\infty} a_n^{1/n} \leq \limsup_{n\to\infty}\frac{a_{n+1}}{a_n}.
    \end{align*}
}
\begin{proof}
    If the right hand side is infinity, then we are done, so suppose it is equal to some $\gamma < \infty$. Then we can pick $\gamma < \beta < \infty$ and $N \geq \max(\ell, 1)$ such that $\frac{a_{n+1}}{a_n} < \beta$ for all $n \geq N$. Then $a_{N+1} \leq \beta a_N$, $a_{N+2} \leq \beta a_{N+1} \leq \beta^2 a_{N}$. Continuing this, we get 
    \begin{align*}
        a_{N+n} \leq \beta^n a_N
    \end{align*}
    for all $n \geq 0$. Thus 
    \begin{align*}
        (a_{N+n})^{\frac{1}{N+n}} \leq \beta^{\frac{n}{N+n}} a_N^{\frac{1}{N+n}}.
    \end{align*}
    So now,
    \begin{align*}
        \limsup_{n\to\infty} a_n^{1/n} \leq \limsup_{n\to\infty} \beta^{\frac{n}{N+n}} a_N^{\frac{1}{N+n}} \leq \beta.
    \end{align*}
    So we have that 
    \begin{align*}
        \limsup_{n\to\infty} a_n^{1/n} \leq \gamma
    \end{align*}
    as desired.
\end{proof}

\thm{Ratio Test}{
    Let $\{x_n\}_{n=\ell}^\infty \subseteq X$, $X$ a Banach space, and suppose 
    \begin{align*}
        \frac{\Vert x_{n+1}\Vert}{\Vert x_n\Vert} \in [0, \infty]
    \end{align*}
    for all $n \geq \ell$. Let $\alpha = \limsup_{n\to\infty} \frac{\Vert x_{n+1}\Vert}{\Vert x_n\Vert} \in [0, \infty]$. The following hold:
    \begin{enumerate}
        \item If $0 \leq \alpha < 1$, then $\sum_{n=\ell}^\infty x_n$ converges absolutely.
        \item If $\frac{\Vert x_{n+1}\Vert}{\Vert x_n\Vert}$ for all $n \geq N \in \NN$, then $\sum_{n=\ell}^\infty x_n$ diverges.
    \end{enumerate}
}
\begin{proof}
    We go in order.
    \begin{enumerate}
        \item Use the lemma and the root test.
        \item Then, $\Vert x_{n+1}\Vert \geq \Vert x_n\Vert$ for all $n \geq N$. Without loss of generality, assume $x_N \neq 0$. So,
        \begin{align*}
            \Vert x_{N+n} \Vert \geq \Vert x_N\Vert > 0
        \end{align*}
        for all $n$, meaning the sum of $x_n$'s cannot converge as it is strictly increasing.
    \end{enumerate}
\end{proof}
\newpage
\section{Series of functions}
\dfn{}{
    Let $X \neq \emptyset$ and $Y$ a normed vector space. For $n \geq \ell \in \ZZ$, suppose $f_n : X\to Y$.
    \begin{enumerate}
        \item We define $\Gamma_c = \{x \in X \mid \sum_{n=\ell}^\infty f_n(x) \text{ converges in }Y\}$. We let $\sum_{n=\ell}^\infty : \Gamma_c \to Y$ via 
        \begin{align*}
            \left( \sum_{n=\ell}^\infty f_n\right)(x) = \sum_{n=\ell}^\infty f_n(x).
        \end{align*}
        We also write $\Gamma_d = X\setminus \Gamma_c$ (divergent set). 
        \item Let $A \subseteq \Gamma_c$. We say $\sum_{n=\ell}^\infty f_n$ converges uniformly on $A$ if $\left\{S_n = \sum_{n=\ell}^\infty f_n\right\}_{n=\ell}^\infty$ converges uniformly on $A$. 
    \end{enumerate}
}
\noindent \textbf{Facts:}
Suppose $f = \sum_{n=\ell}^\infty f_n$ converges uniformly on $X$, a metric space.
\begin{enumerate}
    \item If each $f_n$ is bounded, then $f$ is bounded.
    \item If each $f_n$ is continuous at $z \in X$, then $f$ is continuous at $z$. 
    \item If each $f_n$ is uniformly continuous at $z \in X$, then $f$ is uniformly continuous at $z$.
\end{enumerate}
\dfn{}{
    Let $X \neq \emptyset$ be a set and $Y$ a Banach space. Suppose $f_n: X \to Y$ for all $n \geq \ell$. We say $\sum_{n=\ell}^\infty f_n$ converges pointwise absolutely on $A \subseteq X$ if $\sum_{n=\ell}^\infty \Vert f_n(x)\Vert < \infty$ for all $x \in A$. 

    We say $\sum_{n=\ell}^\infty f_n$ converges uniformly absolutely on $A$ if 
    \begin{align*}
        A \ni x \mapsto \sum_{n=\ell}^\infty \Vert f_n(x)\Vert \in \RR
    \end{align*}
    converges uniformly where the sum is a limit of partial sums.
}

\thm{Weierstrass M-test}{
    Let $X \neq \emptyset$ and $Y$ be a Banach space and usppose $f_n: X \to Y$ for all $n \geq \ell$. Suppose $\{M_n\}_{n=\ell}^\infty \subseteq [0, \infty)$ such that $\sup_{x \in X} \Vert f_n(x) \Vert \leq M_n$ for all $n \geq \ell$ and $\sum_{n=\ell}^\infty M_n < \infty$. Then
    \begin{enumerate}
        \item The series $f = \sum_{n=\ell}^\infty f_n$ converges uniformly absolutely. In particular, pointwise absolutely and pointwise. That is, $\Gamma_c(f) = X$.
        \item $f: X \to Y$ is bounded and satisfies $\Vert f \Vert_{\mathcal B} = \sup_{x \in X}\Vert f(x) \Vert \leq \sum_{n=\ell}^\infty M_n < \infty$.
        \item If $X$ is a metric space, then if $z \in X$ and each $f_n$ is continuous at $z$, then $f$ is continuous at $z$.
        \item If $X$ is a metric space, then each $f_n$ uniformly continuous implies that $f$ itself is uniformly continuous.
    \end{enumerate}
}
\begin{proof}
    For $N > K \geq \ell$, we can bound 
    \begin{align*}
        \sup_{x \in X} \sum_{n=K}^N \Vert f_n(x) \Vert \leq \sum_{n=K}^N M_n.
    \end{align*}
    Thus, the partial sums $S_n(x) \coloneq \sum_{n=\ell}^N \Vert f_n(x) \Vert$ are uniformly Cauchy, and hence uniformly convergent.Thus $f = \sum_{n=\ell}^\infty f_n$ is convergent uniformly absolutely. Then
    \begin{align*}
        \sum_{n=\ell}^N \Vert f_n(x) \Vert \leq \sum_{n =\ell}^N M_n \leq \sum_{n=\ell}^\infty M_n
    \end{align*}
    for all $N$. So, 
    \begin{align*}
        \Vert f(x) \Vert \leq \sum_{n=\ell}^\infty \Vert f_n(x)\Vert \leq \sum_{n=\ell}^\infty M_n
    \end{align*}
    for all $x$, which shows 2.

    We know that $f = \sum_{n=\ell}^\infty f_n$ is convergent uniformly. So 3 and 4 follow from properties preserved by uniform convergence.
\end{proof}
\ex{}{Recall if $X$ is a unital Banach algebra, we define $\exp: X \to X$ via 
\begin{align*}
    \exp(x) = \sum_{n=0}^\infty \frac{x^n}{n!}.
\end{align*}
Consider $f_n(x) = \frac{x^n}{n!}$ for $n \geq 0$. Each $f_n$ is continuous on $X$ and uniformly continuous when restricted to $B[0, R]$ for $0 \leq R < \infty$. 
}
\noindent If we restrict $f_n \restriction B[0, R]$, then 
\begin{align*}
    \sup_{x \in B[0, R]}\Vert f_n(x)\Vert \leq \frac{R^n}{n!}.
\end{align*}
And also 
\begin{align*}
    \sum_{n=0}^\infty \frac{R^n}{n!} = \exp(R) < \infty.
\end{align*}
So applying the M-test then shows that $\exp$ is bounded and uniformly continuous on $B[0, R]$ for all $0 \leq R < \infty$. If particular, we can send $R$ to infinity and show that $\exp$ is continuous on all of $X$.
\section{Power Series}
\noindent Notation: 
\begin{itemize}
    \item $\mathcal L^n(X; Y)$ bounded multilinear maps from $X^n \to Y$. Also $\mathcal L^0(X; Y) \coloneq Y$. 
    \item $p: X \to Y$ is a polynomial if 
    \begin{align*}
        p(x) = \sum_{n=0}^N T_n(x^{\otimes n})
    \end{align*}
    where $T_n \in \mathcal L^n(X; Y)$ and given $T \in \mathcal L^n(X; Y)$. Also,
    \begin{align*}
        T(x^{\otimes n}) = T(x, x, x, x, x, x, x, x, x ,x, x, x, x, x, x, x, x, x, x, x, x, x, , x, x, x, x, x)
    \end{align*}
\end{itemize}


\dfn{}{
    Let $X,Y$ be normed vector spaces and for $n \in \NN$, let $T_n \in \mathcal L^n(X; Y)$. Let $x_0 \in X$.
    \begin{enumerate}
        \item A power series centered at $x_0$ is a series of functions 
        \begin{align*}
            f = \sum_{n=0}^\infty T_n((\cdot  - x_0)^{\otimes n}).
        \end{align*}
        That is, $f_n(x) = T_n((x - x_0)^{\otimes n})$.
        \item We define the analytic radius of convergence to be 
        \begin{align*}
            R(f) \coloneq \left( \limsup_{n \to \infty} \Vert T_n\Vert_{\mathcal L^n}^{1.n} \right)^{-1} \in [0, \infty].
        \end{align*}
        \item We define the uniform radius of convergence to be 
        \begin{align*}
            R_{uni}(f) \coloneq \begin{cases}
            \infty & X = \{0\} \\
            \left(\limsup_{n \to \infty} \sup_{\Vert u \Vert_X = 1} \Vert T_n u^{\otimes n} \Vert_{Y}^{1/n} \right)^{-1} & X \neq \{0\}
            \end{cases}.
        \end{align*}
        \item We define the pointwise radius of convergence to be 
        \begin{align*}
            R_{pw}(f) = \begin{cases}
                \infty & X = \{0\} \\
                \inf_{\Vert u\Vert_X = 1} \left(  \limsup_{n \to \infty} \Vert T_n u^{\otimes n}\Vert_Y^{1/n}\right)^{-1} & X \neq \{0\}
            \end{cases}.
        \end{align*}
        Note that 
        \begin{align*}
            \inf_{\Vert u\Vert_X = 1} \left(  \limsup_{n \to \infty} \Vert T_n u^{\otimes n}\Vert_Y^{1/n}\right)^{-1} =  \left( \sup_{\Vert u\Vert_X = 1} \limsup_{n \to \infty} \Vert T_n u^{\otimes n}\Vert_Y^{1/n}\right)^{-1}.
        \end{align*}
    \end{enumerate}
}
\noindent Remark: each of the radii are independent of $x_0$.
\thm{}{
    Let $X$ be a metric space, $Y$ a Banach space, $T_n \in \mathcal L^n(X; Y)$ for all $n \in \NN$. Consider the power series 
    \begin{align*}
        f = \sum_{n=0}^\infty T_n((\cdot - x_0)^{\otimes n})
    \end{align*}
    for some $x_0 \in X$. The following hold:
    \begin{enumerate}
        \item Define $\rho : \partial B_X(0, 1)\to [0, \infty]$ via 
        \begin{align*}
            \rho(u) = \left( \limsup_{n \to \infty} \Vert T_n u^{\otimes n}\Vert_Y^{1/n}\right)^{-1}.
        \end{align*}
        For every $u \in \partial B_X(0, 1)$, $\rho(u) = \rho(-u)$ and $f$ converges pointwise absolutely on $x_0 + [0, \rho(u))u$ and diverges on the ray $x_0 + (\rho(u), \infty) u$. In particular, 
        \begin{align*}
            \{x + tu \mid |t| < \rho(u) \}& \subseteq \Gamma_c(f) \\
            \{x + tu \mid |t| > \rho(u) \}& \subseteq \Gamma_d(f).
        \end{align*}
        \item $R(f) \leq R_{uni}(f) \leq R_{pw}(f)$.
        \item $f$ converges pointwise absolutely on $B_X(x_0, R_{pw}(f))$ and if $R_{pw}(f) < t$, then $f$ does not converge pointwise on $B_X(x_0, t)$.
        \item If $0 < S < R_{uni}(f)$, then $f$ converges uniformly absolutely on $B_X[x_0, S]$ and $f: B_X[x_0, S] \to Y$ is bounded and uniformly continuous. If $R_{uni}(f) < t$< then $f$ is not uniformly convergent on $B_X(x_0, t)$.
        \item $f: B_X(x_0, R_{uni}(f)) \to Y$ is continuous.
    \end{enumerate}
}
\noindent Remark: It is know that 
\begin{align*}
    \begin{cases}
        \dfrac{R_{uni}}{e} \leq R(f) \\
        R_{uni}(f) = R(f) \text{ if $X,Y$ are Hilbert.}
    \end{cases}
\end{align*}
\begin{proof}
    I wasn't in class for the first 10 minutes oops.
    \begin{enumerate}
        \item missing
        \item missing
        \item Since $R_{pw}(f) \leq \rho(u)$ for all $\Vert u \Vert_X = 1$, the inclusion $B_X(x_0, R_{pw}(f)) \subseteq \Gamma_c(f)$ follows from 1. On the other hand, if $R_{pw}(f) < t$, we pick $R_{pw}(f) < s < t$ there exists $\Vert u \Vert = 1$ such that $\rho(u)<s \implies x_0 + su \in \Gamma_d(f)$. 
        \item Let $0 < s < R_{uni}(f)$ and pick $\frac{1}{R_{uni}(f)} < \gamma < \frac 1s$. 
        \begin{align*}
            \frac{1}{R_{uni}(f)} = \limsup_{n \to \infty} \sup_{\Vert u\Vert = 1} \Vert T_n u^{\otimes n}\Vert_Y^{1/n}.
        \end{align*}
        So there is $N$ such that $n \geq N \implies \sup_{\Vert u \Vert = 1} \Vert T_n u^{\otimes n} \Vert_Y^{1/n} < \gamma$. Then for $n \geq N$ and $x \in B_X(x_0, s)$, we have 
        \begin{align*}
            \Vert T_n(x - x_0)^{\otimes n}\Vert_Y^{1/n} \leq \Vert x - x_0\Vert_X \sup_{\Vert u \Vert = 1} \Vert T_n x^{\otimes n} \Vert_Y^{1/n} < s\gamma.
        \end{align*}
        Then $\sum_{n=0}^\infty (\gamma s)^n < \infty$, so the M-test implies that $f$ converges uniformly absolutely on $B_X[x_0, s]$ to a bounded and uniform continuous $f_n$.

        Now let $R_{uni}(f) < s < t$. Pick a subsequence such that 
        \begin{align*}
            \sup_{\Vert u \Vert = 1} \Vert T_{n_k} u^{\otimes n_k}\Vert^{1/n_k} \to \frac{1}{R_{uni}(f)}
        \end{align*}
        as $k \to \infty$. Now pick $K \in \NN$ such that $k \geq K$ implies 
        \begin{align*}
            \frac 1s < \sup_{\Vert u \Vert = 1} \Vert T_{n_k} u^{\otimes n_k}\Vert^{1/n_k}_Y \implies \exists \Vert u_k\Vert = 1
        \end{align*}
        such that 
        \begin{align*}
            \frac 1s < \Vert T_{n_k} u_k^{\otimes n_k}\Vert^{1/n_k}_Y \implies 1 < \Vert T_{n_k} (su_k)^{\otimes n_k}\Vert^{1/n_k}_Y.
        \end{align*}
        Suppose BWOC that $f$ is convergent uniformly in $B(x_0, t)$. Then the partial sums 
        \begin{align*}
            \left\{S_n = \sum_{n=0}^N T_n((\cdot - x_0)^{\otimes n})\right\}_{N=0}^\infty
        \end{align*}
        converges uniformly on $B(x_0, t)$. Then $su_k \in B(x_0, t)$ for all $k \geq K$. So,
        \begin{align*}
            1 < \Vert T_{n_k}(su_k)^{\otimes n_k} \Vert_Y = \Vert S_{n_k}(x_0 + su_k) - S_{n_k - 1}(x_0 + su_k)\Vert_Y \leq \sup_{x \in B(x_0, t)} \Vert S_{n_k} - S_{n_k - 1}(x) \Vert_Y \to 0.
        \end{align*}
        This is a contradiction.
        \item Follows from 4 because $f$ is continuous in $B[x_0, S]$ for all $s < R_{uni}(f)$.
    \end{enumerate}
\end{proof}
\newpage
As an example, let $X = \ell^2(\NN)$. Let 
\begin{align*}
    f(x) = \sum_{n=0}^\infty x_n^n.
\end{align*}
That is,
\begin{align*}
    f = \sum_{n=0}^\infty T_n(\cdot^{\otimes n})
\end{align*}
for $T \in \mathcal L^n(X; \RR)$ given by 
\begin{align*}
    T_n(x^{(1)}, \ldots, x^{(n)}) = \prod_{j=1}^n x^{(j)}_n.
\end{align*}
Then 
\begin{align*}
    |T_n(\cdot)| \leq \prod_{j=1}^n |x^{(j)}_n \implies \Vert T_n \Vert_{\mathcal L} = 1 \implies R(f) = 1.
\end{align*}
Similarly $R_{uni}(f) = 1$. Note that $x \in X = \ell^2 \implies x_n \to 0$ as $n \to \infty$. So
\begin{align*}
    \limsup_{n \to \infty} |x_n^n|^{1/n} = \limsup_{n \to \infty} |x_n| = 0 \implies R_{pw}(f) = \infty.
\end{align*}
Rewinding, when we build PS we consider $T_n \in \mathcal L^n(X; Y)$ and study $x \mapsto T_n(x^{\otimes n})$. In particular, this means 
\begin{align*}
    T_n(x^{\otimes n}) = \mathcal S T_n(x^{\otimes n})
\end{align*}
where... Fix a map $T \in L^n(X; Y)$ for $n \geq 2$. 
\begin{enumerate}
    \item $T$ is symmetric if $T(x_1, \ldots x_n) = T(x_{\pi(1)}, \ldots, x_{\pi(n)})$ where $\pi$ is a permutation of $n$ elements.
    \item Given $T$, we can ``symmetrize it'': 
    \begin{align*}
        \mathcal ST \in L^n(X; Y), \mathcal ST(x_1, \ldots, x_n) = \frac{1}{n!} \sum_{n \in S_n} T(x_{\pi(1)}, \ldots, x_{\pi(n)}).
    \end{align*}
    Clearly $\mathcal ST$ is symmetric.
    \item Given $T \in \mathcal L_{sym}^n(X; Y)$ and $0 \leq k \leq n$, we can define 
    \begin{align*}
        T^{(k)} \in \mathcal L_{sym}^{n-k}(X; \mathcal L^k_{sym}(X; Y))
    \end{align*}
    as 
    \begin{align*}
        T^{(k)}(x_1, \ldots, x_{n_k})(v_1, \ldots, v_k) = T(x_1, \ldots, x_{n-k}, v_1, \ldots v_k).
    \end{align*}
\end{enumerate}
The point is that if we consider $f = \sum_{n=0}^\infty T_n((\cdot - x_0)^{\otimes n}) = \sum_{n=0}^\infty \mathcal ST_n((\cdot - x_0)^{\otimes n})$, we can reduce to studying PS built from $T_n \in \mathcal L^n_{sym}(X; Y)$.
\newpage
\noindent Facts about symmetric multilinear maps:

Let $X,Y$ be a normed vector space over $\FF$:
\begin{enumerate}
    \item $\mathcal S: \mathcal L^n(X; Y) \to \mathcal L^n_{sym}(X; Y)$ is bounded and linear and 
    \begin{align*}
        \Vert \mathcal ST\Vert_{\mathcal L^n} \leq \Vert T\Vert_{\mathcal L^n}.
    \end{align*}
    \item $\mathcal L^n_{sym}(X; Y) \subseteq \mathcal L^n(X; Y)$ is a closed subspace. 
    \item Let $0 \leq k \leq n$. The map 
    \begin{align*}
        \mathcal L^n_{sym}(X; Y) \ni T \mapsto T^{(k)} \in \mathcal L^{n-k}_{sym}(X; \mathcal L^k_{sym}(X; Y))
    \end{align*}
    is also bounded and linear, and $\Vert T^{(k)}\Vert_{\mathcal L^{n-k} \mathcal L^k} \leq \Vert T \Vert_{\mathcal L^n}$.
    \item On $\mathcal L^n_{sym}( X; Y)$ we can define for $T \in \mathcal L^n_{sym}$,
    \begin{align*}
        \Vert T \Vert_{\mathcal L^n_{sym}} = \sup_{\Vert u \Vert_X \leq 1} \Vert Tu^{\otimes n} \Vert_Y.
    \end{align*}
    If $X \neq \{0\}$, then we just need $\Vert u \Vert = 1$. This is a norm, and 
    \begin{align*}
        \Vert T \Vert_{\mathcal L^n_{sym}} \leq \Vert T \Vert_{\mathcal L^n} \leq \frac{n^n}{n!}\Vert T\Vert_{\mathcal L^n_{sym}}.
    \end{align*}
    That is, it's an equivalent norm. In particular, if $T \in \mathcal L^n_{sym}(X; Y)$ for all $n \in \NN$, then the power series 
    \begin{align*}
        f =\sum_{n=0}^\infty T_n((\cdot - x_0)^{\otimes n}),
    \end{align*}
    we have:
    \begin{align*}
        \frac 1e R_{uni}(f) \leq R(f) \leq R_{uni}(f).
    \end{align*}
    \item If $X$ is an inner product space, then $\Vert T\Vert_{\mathcal L^n_{sym}} = \Vert T\Vert_{\mathcal L^n}$. So $R_{uni}(f) = R(f)$ for power series.
    
    That is, we might as well assume $T_n \in \mathcal L^n_{sym}$ when working with power series. This actually helps a bit.
\end{enumerate}

\thm{Analyticity Theorem}{
    Let $X,Y$ be normed vector spaces over $\FF$ with $Y$ complete. Assume that $T_n \in \mathcal L^n_{sym}(X; Y)$ for all $n \in \NN$ and consider the power series $f = \sum_{n =0}^\infty T_n((\cdot - x_0)^{\otimes n})$. Suppose $R(f) > 0$, and let $0 < r <R(f)$ and $x_1 \in B(x_0, r)$. The following hold:
    \begin{enumerate}
        \item For each $k \in \NN$, the series 
        \begin{align*}
            S_k = \sum_{n=k}^\infty \binom nkT_n^{(k)}((x_1 - x_0)^{\otimes (n-k)}) \in \mathcal L^k_{sym}(X; Y)
        \end{align*}
        converges absolutely in $\mathcal L^k_{sym}$.
        \item The power series $F = \sum_{k=0}^\infty S_k((\cdot - x_1)^{\otimes k})$ satisfies \begin{align*}
            R_{uni}(F) \geq r - \Vert x_1 - x_0\Vert_X
        \end{align*}
        and $F=f$ in $B(x_1, r - \Vert x_1 - x_0\Vert)$.
    \end{enumerate}
}
\newpage
\begin{proof}
    First note that $\binom nk^{1/n} = \left[\frac{n(n-1)\cdots(n-k+1)}{k!}\right]^{1/n} \leq \left(\frac{n^k}{k!}\right)^{1/n} \to 1$ as $n \to \infty$. This implies for $r < R(f)$,
    \begin{align*}
        \limsup_{n\to\infty} \left[ \binom nk \Vert T_n\Vert_{\mathcal L^n} r^{n-k}\right]^{1/n} \leq \frac{r}{R(f)} < 1.
    \end{align*}
    Hence, if $x_1 \in B(x_0, r)$, then 
    \begin{align*}
        \limsup_{n\to\infty}\left\Vert \binom nk T_n^{(k)} ((x_1 - x_0)^{\otimes (n-k)}) \right\Vert_{\mathcal L^k_{sym}}^{1/n} \leq \limsup_{n\to\infty}\left[ \binom nk \Vert T_n\Vert_{\mathcal L^n} r^{n-k}\right]^{1/n} \leq \frac{r}{R(f)} < 1.
    \end{align*}
    Therefore, by the root test, the series $S_k$ converges absolutely.
    Similarly, $\limsup_{n\to\infty} \left[ \Vert T_n\Vert_{\mathcal L^n} r^n\right]^{1/n} = \frac{r}{R(f)} < 1$. Then again by the root test, we have 
    \begin{align*}
        \sum_{n=0}^\infty \Vert T_n\Vert_{\mathcal L^n} r^n < \infty.
    \end{align*}
    Now fix $x \in B(x_1, r - \Vert x_1 - x_0\Vert)$. Then 
    \begin{align*}
        \Vert x - x_1 \Vert < r - \Vert x_1 - x_0\Vert \implies \sum_{k=0}^n \binom nk \Vert T_n\Vert_{\mathcal L^n} \Vert x_0 - x_1\Vert^{n-k} \Vert x - x_1 \Vert^k \leq \sum_{k=0}^n \binom nk \Vert T_n\Vert_{\mathcal L^n} \Vert x_0 - x_1\Vert^{n-k} (r - \Vert x_1 - x_0\Vert)^k.
    \end{align*}
    Applying the binomial theorem, we get that this equal to 
    \begin{align*}
        \Vert T_n\Vert_{\mathcal L^n} (\Vert x_0 - x_1\Vert + r - \Vert x_0 - x_1\Vert)^n = \Vert T_n\Vert_{\mathcal L^n}r^n.
    \end{align*}
    By the Weierstrass M-test, we get that 
    \begin{align*}
        \sum_{n=0}^\infty \sum_{k=0}^n \binom nk T_n^{(k)} ((x_1 - x_0)^{\otimes (n-k)}) (\cdot - x_1)^{\otimes k}
    \end{align*}
    converges uniformly absolutely in $B(x_1, r - \Vert x_1 - x_0\Vert)$. Now note that 
    \begin{align*}
        T_n((x-x_0)^{\otimes n}) = \sum_{k=0}^n \binom nk T_n^{(k)} ((x_1 - x_0)^{\otimes (n-k)}) (x - x_1)^{\otimes k}.
    \end{align*}
    Thus, 
    \begin{align*}
        f(x) &= \sum_{n=0}^\infty T_n((x-x_0)^{\otimes n}) \\
        &= \sum_{n=0}^\infty \sum_{k=0}^n \binom nk T_n^{(k)} ((x_1 - x_0)^{\otimes (n-k)}) (x - x_1)^{\otimes k} \\
        &= \sum_{k=0}^\infty \sum{n=k}^\infty  \binom nk T_n^{(k)} ((x_1 - x_0)^{\otimes (n-k)}) (x - x_1)^{\otimes k}\\
        &= \sum_{k=0}^\infty S_k((x - x_1)^{\otimes k}) = F(x).
    \end{align*}
    Therefore, $f = F$ in $B(x_1, r - \Vert x_1 - x_0\Vert)$ and $R_{uni}(F) \geq r - \Vert x_1 - x_0\Vert$.
\end{proof}
\newpage
\dfn{}{
    Let $X$ and $Y$ be normed vector spaces with $Y$ complete. Suppose $\emptyset \neq U \subseteq X$ is open and $f: U \to Y$. We say $f$ is analytic if $\forall x_0 \in U$, there is $R > >0$ with $B(x_0, R) \subseteq U$ such that 
    \begin{align*}
        f(x) = \sum_{n=0}^\infty T_n((x - x_0)^{\otimes n})
    \end{align*}
    for all $x \in B(x_0, R)$ where $T_n \in \mathcal L^n_{sym}(X; Y)$ for all $n \in \NN$ and $R \leq R_{uni}(f)$.
}
\nt{Analytic implies continuous. Also, if we're given $f = \sum T_n(\cdot - x_0)^{\otimes n}$ for $T_n \in \mathcal L^n_{sym}$ for all $n$, then $f$ is analytic in $B(x_0, R(f))$, provided $R(f) > 0$.}

\chapter{Differential Calculus}
High level motivation: hierarchy of complexity of functions $f: X \to Y$:
\begin{itemize}
    \item Simplest functions are constant.
    \item Continuous functions are locally well-approximated by constants.
\end{itemize}
Now say $X$ and $Y$ are normed vector spaces. 
\begin{itemize}
    \item Then we include affine functions.
    \item Quadratic polynomials.
    \item Higher order polynomials.
    \item Then differentiable functions are locally well-approximated by affine functions.
\end{itemize}
\dfn{}{
    Let $X,Y$ be normed vector spaces over $\FF$ and $\emptyset \neq U \subseteq X$ be open and $f: U \to Y$. $f$ is differentiable at $x \in U$ if there exists $\lambda \in \mathcal L(X; Y)$ such that
    \begin{align*}
        \lim_{h \to 0} \frac{f(x+h) - f(x) - \lambda h}{\Vert h\Vert} = 0.
    \end{align*}
}
\mlenma{}{
    Same hypothesis but assume $\lambda_1, \lambda_2 \in \mathcal L(X; Y)$ are both sufficient for the condition. Then $\lambda_1 = \lambda_2$. 
}
\begin{proof}
    Let $g_i(h) = \frac{f(x + h) - f(x) - \lambda_i h}{\Vert h \Vert}$ for $\Vert h \Vert$ small enough. Then fix $u \in X$ with $\Vert u \Vert =1 $. THen for $t$ small, 
    \begin{align*}
        \lambda_1(tu) - \lambda_2(tu) = \Vert tu\Vert [g_2(tu) - g_1(tu)].
    \end{align*}
    But this implies 
    \begin{align*}
        \lambda_1(u) - \lambda(u) = \Vert u\Vert [g_2(tu) - g_1(tu)] \to 0
    \end{align*}
    as $t \to 0$. Therefore, $\lambda_1 u = \lambda_2 u$ for all $\Vert u \Vert = 1$, meaning $\lambda_1 = \lambda_2$.
\end{proof}
\dfn{}{
    Following the definition before, we define $Df(x) = \lambda \in \mathcal L(X; Y)$. We also set 
    \begin{align*}
        D^1(U; Y) = \{g: U \to Y \mid g \text{ is differentiable at all points in $U$}\}.
    \end{align*}
}
\mlenma{}{
    Same hypothesis. Suppose $f$ is differentiable at $x \in U$, then $x$ is continuous at $x$.
}
\begin{proof}
    We have 
    \begin{align*}
        f(x + h) - f(x) &= \frac{f(x + h) - f(x) - Df(x) h}{\Vert h \Vert} \Vert h \Vert + Df(x) h.
    \end{align*}
    But this right hand side goes to 0 as $h \to 0$ because $Df(x) \in \mathcal L(X; Y)$.
\end{proof}
\cor{}{
    $D^1(U; Y) \subseteq C^0(U ;Y)$.
}
\noindent Remarks:
\begin{enumerate}
    \item We always need $X$ and $Y$ to be over the same field. If one is over $\CC$ and the other is over $\RR$, then we demote $\CC$ to $\RR$.
    \item Differentiability is local; $f$ differentiable at $x \in U$ iff $f \restriction V$ is differentiable at $x \in V \subseteq U$.
    \item It's easy to check that if we change to equivalent norms on $X$ and $Y$, then the notion of differentiability does not change.
    \item If $X = \FF$ then $\mathcal L(\FF; Y) \simeq Y$.
\end{enumerate}
\ex{}{
    \begin{enumerate}
        \item All affine functions are differentiable.
        \item Consider $f: \CC \to \CC$ as $f(z) = \overline z$.
        \begin{align*}
            f(z+h) - f(z) &= \overline{z + h} - \overline z = \overline h.
        \end{align*}
        But $\overline h$ is not $\CC$-linear, so $f$ is not differentiable.
        \item $F: \RR^2 \to \RR^2$ by $F(x) = (x_1 - x_2)$ is differentiable.
    \end{enumerate}
}
In $\RR$, MVT says $f(b) - f(a) = f'(c)(b-a)$ for some $c \in [a,b]$. But what if we take it to multiple dimensional output. Then consider 
\begin{align*}
    f(t) = (\cos(t), \sin(t)) \implies f'(t) = (-\sin(t), \cos(t)).
\end{align*}
This means that $|f'(t)| = 1$. If MVT holds, then there exists $c \in [0, 2\pi]$ such that 
\begin{align*}
    f(2\pi) - f(0) = (2\pi - 0) f'(c) \implies 2\pi f'(c) = 0.
\end{align*}
\thm{MVT, $X = \RR$ case}{
    Let $Y$ be a normed vector space over $\RR$, $a,b \in \RR$ such that $a < b$. Suppose $f:[a,b] \to Y$ and $\varphi: [a, b] \to \RR$ are continuous on $[a,b]$ and are differentiable on $[a,b] \setminus E$ for $E \subseteq (a,b)$ a countable set. If $\Vert f'(x) \Vert \leq \varphi'(x)$ for all $x \in (a,b) \setminus E$, then 
    \begin{align*}
        \Vert f(x) - f(a) \Vert \leq \varphi(x) - \varphi(a)
    \end{align*}
    for all $x \in [a,b]$. 
}
\thm{MVT, general case}{
    Let $X$ and $Y$ be normed vector spaces, $\emptyset \neq U \subseteq X$ open, $f: U \to Y$ continuous. Fix $x,y \in U$ and suppose $L(x,y) = \{(1-\alpha) x + \alpha y \mid \alpha \in [0,1]\} \subseteq U$. Further suppose that $f$ is differentiable on $L(x,y) \setminus E$ for some countable set $E$. Then the following hold:
    \begin{enumerate}
        \item If $M \coloneq \sup_{z \in L(x,y) \setminus E} \Vert Df(z) \Vert_{\mathcal L} < \infty$, then
        \begin{align*}
            \Vert f(y) - f(x) \Vert \leq M \Vert x -y\Vert.
        \end{align*}
        \item Let $T \in \mathcal L(X; Y)$ and suppose $K = \sup_{z \in L(x,y) \setminus E} \Vert Df(z) - T\Vert_{\mathcal L} < \infty$. Then
        \begin{align*}
            \Vert f(y) - f(x) - T(x - y) \Vert \leq K \Vert x - y \Vert.
        \end{align*}
    \end{enumerate}
}
\begin{proof}
    2 follows from 1 applied to $g(z) = f(z) - Tz$, so we just prove 1. 

    Define $h : [0,1] \to U$ via 
    \begin{align*}
        h(t) = (1-t)x + ty.
    \end{align*}
    Note that $h([0,1]) = L(x,y)$. $h$ is continuous on $[0,1]$ and differentiable on $(0,1)$ with $h'(t) = y-x$. Consider $g:[0,1] \to Y$ via $g = f\circ h$. By chain rule, $g$ is differentaible on $(0,1) \setminus F$ for $F = h^{-1}(E)$. Then 
    \begin{align*}
        Dg(t) &= Df(h(t))h'(t) \\
        &= Df(h(t))(y-x).
    \end{align*}
    Let $\varphi: [0,1] \to \RR$ via $\varphi(t) = tM\Vert y-x\Vert$. Then $\varphi' = M\Vert x-y\Vert$. Then
    \begin{align*}
        \Vert g'(t) \Vert \leq \Vert Df(h(t)) \Vert_{\mathcal L} \Vert y-x\Vert \leq M \Vert x-y\Vert.
    \end{align*}
    By MVT, we have 
    \begin{align*}
        \Vert g(1) - g(0)\Vert \leq \varphi(1) - \varphi(0) = M\Vert x-y\Vert
    \end{align*}
    for all $x \in [0,1] \setminus E$.
\end{proof}

\mlenma{}{
    Let $k \geq 1$, $X_1, \ldots X_k, Y$ normed vector spaces, $T \in \mathcal L(X_1, \ldots, X_k; Y)$. Then $T$ is differentiable on $X_1 \times \cdots \times X_k$ and 
    \begin{align*}
        DT(x_1, \ldots, x_k)(y_1, \ldots y_k) &= T(y_1, x_2, \ldots, x_k) + T(x_1, y_2, x_3, \ldots x_k) + \cdots + T(x_1, \ldots, x_{k-1}, y_k).
    \end{align*}
}
\begin{proof}
    Let $k=2$ then induct after. $T: X_1 \times X_2 \to Y$ bilinear. Then for $x = (x_1, x_2) \in X_1 \times X_2$ and $(h_1, h_2) \in X_1 \times X_2$, we write 
    \begin{align*}
        T(x_1 + h_1, x_2 + h_2) = T(x_1, x_2) + T(h_1, h_2) + T(x_1, h_2) + T(h_1, x_2).
    \end{align*}
    Let $L: X_1 \times X_2 \to Y$ via 
    \begin{align*}
        L(y_1, y_2) = T(x_1, y_2) + T(y_1, x_2).
    \end{align*}
    Check that $L$ is continuous and linear. Observe that 
    \begin{align*}
        T(x_1 + h_1, x_2 + h_2) - T(x_1, x_2) - Lh = T(h_1, h_2),
    \end{align*}
    so we can write 
    \begin{align*}
        \left|\frac{T(x_1 + h_1, x_2 + h_2) - T(x_1, x_2) - Lh}{\Vert h\Vert_{X_1 \times X_2}}\right| &= \left| \frac{T(h_1, h_2)}{\Vert h \Vert_{X_1 \times X_2}}\right| \\
        &\leq \frac{\Vert T \Vert_{\mathcal L} \Vert h_1 \Vert_{X_1} \Vert h_2 \Vert_{X_2}}{\Vert h \Vert_{X_1 \times X_2}} \\
        &\leq \Vert T\Vert_{\mathcal L} \Vert h \Vert_{X_1 \times X_2} \to 0.
    \end{align*}
\end{proof}
\thm{Chain Rule}{
    Let $X,Y,Z$ be normed vector spaces, $\emptyset \neq U \subseteq X$ and $\emptyset \neq V \subseteq Y$ open sets. Then let $f: U \to Y$ and $g: V \to Z$ such that $f[U] \subseteq V$. If $f$ is differentiable at $x \in U$ and $g$ is differentiable at $f(x) \in V$, then $g\circ f$ is differentiabl at $x$ and 
    \begin{align*}
        D(g\circ f)(x) = Dg(f(x)) \circ Df(x).
    \end{align*}
}
\mlenma{}{
    Let $X, Y_1, \ldots Y_m$ be normed vector spaces, $Y = Y_1 \times \cdots \times Y_m$, $\emptyset \neq U \subseteq X$ open, $f: U \to Y$. Product equipped with  $\pi_i Y \to Y_i$ which maps $y \mapsto y_i$. Let $f_i = \pi_i \circ f$. Then $f$ is differentiable at $x \in U$ if and only if each $f_i$ is differentiable at $x$.
}
\begin{proof}
    We start with the easy direction.
    \begin{itemize}
        \item $(\implies)$: Immediate as $\pi_i \in \mathcal L(Y, Y_i)$ and $f$ is differentiable.
        \item $(\impliedby)$: Check that $(Df_1(x) , Df_2(x), \ldots Df_m(x))$ makes this work.
    \end{itemize}
\end{proof}

\thm{Product Rule}{
    Fix $2 \leq k \in \NN$, $X, Y_1, \ldots Y_k$, $Z$ normed vector spaces, $\emptyset  \neq U \subseteq X$ open. Suppose $f_i : U \to Y_i$ are all differentiable at some point $x \in U$. Also suppose that $T: U \to \mathcal L(Y_1, \ldots Y_k ; Z)$ is differentiable at $x$. Then $S: U \to Z$ via $S(x) = T(x)(f_1(x), \ldots, f_k(x))$ is differentiable at $x$, and 
    \begin{align*}
        DS(x)h = (DT(x)h)(f_1(x), \ldots, f_k(x)) + T(x)(Df_1(x)h, f_2(x), \ldots, f_k(x)) + \\
        T(x)(f_1(x), Df_2(x)h, \ldots, f_k(x)) + \cdots + T(x)(f_1(x), \ldots, f_{k-1}(x), Df_k(x)h).
    \end{align*}
}
\begin{proof}
    We start by defining
    \begin{align*}
        \Phi: U \to \mathcal L(Y_1, \ldots Y_k; Z) \times Y_1 \times \cdots \times Y_k
    \end{align*}
    via
    \begin{align*}
        \Phi(x) = (T(x), f_1(x), \ldots, f_k(x)).
    \end{align*}
    By the second lemma, $\Phi$ is differentiable at $x$. Then let $M \in \mathcal L( \mathcal L(Y_1, \ldots Y_k; Z), Y_1, \ldots, Y_k; Z)$ via 
    \begin{align*}
        M(\Lambda, y_1, \ldots, y_k) = \Lambda(y_1, \ldots, y_k).
    \end{align*}
    By the first lemma, $M$ is differentiable. Observe $S = M \circ \Phi$. By the chain rule, $S$ is differentiable at $x$, where we have 
    \begin{align*}
        DS(x) = DM(\Phi(x)) \circ D\Phi(x).
    \end{align*}
    We have that 
    \begin{align*}
        DM(\Lambda, \lambda_1, \ldots, \lambda_k)(H, h_1, \ldots, h_k) &= M(H, \lambda_1, \ldots, \lambda_k) + M(\Lambda, h_1, \lambda_2, \ldots, \lambda_k) + \cdots + M(\Lambda, \lambda_1, \ldots, \lambda_{k-1}, h_k).
    \end{align*}
    And,
    \begin{align*}
        D\Phi(x)\eta = (DT(x)\eta, Df_1(x)\eta, \ldots, Df_k(x)\eta).
    \end{align*}
    Now we win because we have
    \begin{align*}
        DS(x)\eta &= [DM(\Phi(x)) \circ D\Phi(x)]\eta \\
        &= (DT(x)\eta)(f_1(x), \ldots, f_k(x)) + T(x)(Df_1(x)\eta, f_2(x), \ldots, f_k(x)) + \\
        &T(x)(f_1(x), Df_2(x)\eta, \ldots, f_k(x)) + \cdots + T(x)(f_1(x), \ldots, f_{k-1}(x), Df_k(x)\eta).
    \end{align*}
\end{proof}
\newpage

\thm{}{
    Let $a,b \in \RR$ with $a<b$, $Y$ a normed vector space over $\RR$. Suppose $f:[a,b] \to Y$, $\varphi: [a,b] \to \RR$ are continuous on at$[a,b]$ and differentiable on $(a,b) \setminus E$ for $E$ countable. Suppose $\Vert f'(x) \Vert \leq \varphi'(x)$ for all $x \in (a,b) \setminus E$. Then $\Vert f(x) - f(a) \Vert \leq \varphi(x) - \varphi(a)$ for all $x \in [a,b]$.
}
\begin{proof}
    We first do a crazy construction. Let $F = E \cup \{a\}$, which is certainly countable. So let $F = \{r_n\}_{n=0}^\infty$ with the last term repeated infinitely many times if $F$ is finite.

    For $n \in \NN$, let $\zeta_n \in\mathrm{Step}([a,b]; \RR)$ given by $\zeta_n(x) = \begin{cases}
        0 & x \in [a, r_n] \\
        1 & x \in (r_n, b]
    \end{cases}$. So let $\zeta = \sum_{n=0}^\infty 2^{-n} \zeta_n \in \mathrm{Reg}([a,b])$.  Note that this converges uniformly. We now cover some properties of $\zeta$:
    \begin{enumerate}
        \item $\Vert \zeta \Vert_{\mathcal B} \leq \sum 2^{-n} \Vert \zeta_n\Vert_{\mathcal B} \leq 2$.
        \item $a \leq x < y \leq b \implies \zeta_n(x) \leq \zeta_n(y)$ for all $n$, but this implies the same property for $\zeta$, i.e. that it is nondecreasing.
        \item For $s \in (a, b]$, we have that 
        \begin{align*}
            \lim_{x \to s^-} \zeta(x) \leq \zeta(s).
        \end{align*}
        \item Suppose $r_n < x \leq b$ for $n \in \NN$. Then 
        \begin{align*}
            2^{-n}\zeta_n(r_n) + 2^{-n} = 2^{-n} = 2^{-n} \zeta_n(x).
        \end{align*}
        This mean that 
        \begin{align*}
            \zeta(r_n) + 2^{-n} \leq \zeta(x).
        \end{align*}
    \end{enumerate}
    Now let $\eps > 0$ and let
    \begin{align*}
        S_\eps = \{x \in [a,b] \mid \Vert f(y) - f (a) \Vert \leq \varphi(y) - \varphi(a) + \eps(y-a) + \eps \zeta(y) \,\, \forall a \leq y \leq x \}.
    \end{align*}
    Note that $a \in S_\eps \subseteq[a,b]$, so there exists $s = \sup S_\eps \in [a,b]$. We claim that $s = b$. 
    \begin{itemize}
        \item We want to show $s = \max S_\eps$. If $s = a$, we're done. So suppose $s \neq a$. By construction, we have that 
        \begin{align*}
            \Vert f(y) - f(a) \Vert \leq \varphi(y) - \varphi(a) + \eps(y-a) + \eps \zeta(y) \,\, \forall y \in [a,s).
        \end{align*}
        Thus, from the properties outlined above,
        \begin{align*}
            f(s) - f(a) = \lim_{y \to s^-}\Vert f(y) - f(a) \Vert \leq \varphi(s) - \varphi(a) + \eps(s-a) + \eps\zeta(s).
        \end{align*}
        So this shows that $s \in S_\eps$.
        \item We want to show $s\notin (a,b) \setminus E$. Suppose BWOC $s \in (a,b) \setminus E$. Since $\varphi,f$ are differentiable at $s$, we can pick $\eta > 0$ such that 
        \begin{align*}
            \begin{cases}
                \Vert f(x) - f(s) - f'(s)(x-s) \Vert < |x-s| \frac \eps 2 \\
                \Vert \varphi(x) - \varphi(s) - \varphi'(s)(x-s) \Vert < |x-s| \frac \eps 2
            \end{cases}
        \end{align*}
        for all $x \in B(s, \eta) \cap [a,b]$. Then, for $x \in [s, s+\eta) \cap [a,b]$, we have 
        \begin{align*}
            \Vert f(x) - f(s) \Vert \leq |x-s| \frac \eps 2 + \Vert f'(s)(x-s)\Vert &= \frac \eps 2 (x-s) + |x-s|\Vert f'(s)\Vert \\
            &\leq \frac \eps 2 (x-s) + (x-s) \varphi'(s) \\
            &\leq \eps(x-s) + \varphi(x) - \varphi(s).
        \end{align*}
        On the other hand, $s \in S_\eps \implies \Vert f(s) - f(a) \Vert \leq \varphi(s) - \varphi(a) + \eps(s-a) + \eps\zeta(s)$. Combined, this yields that
        \begin{align*}
            \Vert f(x) - f(a) \Vert \leq \Vert f(x) - f(s)\Vert + \Vert f(s) - f(a)\Vert \leq \varphi(x) - \varphi(a) + \eps(x-a) + \eps\zeta(x) + \eps\cancelto{\zeta(x)}{\zeta(s)}.
        \end{align*}
        But this a contradiction to the maximality of $s$, so we are done.
        \item Now we show $x \notin F = E \cup \{a\}$. Suppose by way of contradiction that $s \in F$. This means that $s = r_n$ for some $n \in \NN$. Since $f ,\varphi$ are continuous, there exists $\delta > 0$ such that $x \in B(r_n, \delta) \cap [a,b]$ implies that 
        \begin{align*}
            \Vert f(x) - f(r_n) \Vert < \eps \cdot 2^{-n-1}
        \end{align*}
        and 
        \begin{align*}
            | \varphi(x) - \varphi(r_n) | < \eps \cdot 2^{-n-1}.
        \end{align*}
        Thus, for $x \in (s, s + \delta) \cap [a,b]$, we have that 
        \begin{align*}
            \Vert f(x) - f(a) \Vert &\leq \Vert f(x) - f(s) \Vert + \Vert f(s) -f(a) \Vert \\
            &\leq \eps \cdot 2^{-n-1} + \varphi(s) - \varphi(a) + \eps(s-a) + \eps\zeta(s) \\
            &\leq \varphi(x) - \varphi(a) + \eps(s-a) + \eps(\zeta(r_n) + 2^{-n}) \\
            &\leq \varphi(x) - \varphi(a) + \eps(x-a) + \eps \zeta(x).
        \end{align*}
        This shows that $x \in S_\eps$, which is a contradiction.
    \end{itemize}
    By combining the above results, we see that $s = \max S_\eps = b$. So, we now know that 
    \begin{align*}
        \Vert f(x) - f(a) \Vert \leq \varphi(x) - \varphi(a) + \eps[(x-a) + \zeta(x)]
    \end{align*}
    for all $x \in [a,b]$. Now send $\eps \to 0$ for the desired result.
\end{proof}
\cor{}{
    Let $X,Y$ be normed vector spaces, $\emptyset \neq U \subseteq X$ be open and convex, and $f: U \to Y$ be differentiable except on countable $E \subseteq U$. Suppose 
    \begin{align*}
        M = \sup_{U \setminus E} \Vert Df(x) \Vert_{\mathcal L} < \infty.
    \end{align*}
    Then 
    \begin{align*}
        \Vert f(x) - f(y) \Vert \leq M \Vert x - y \Vert.
    \end{align*}
    That is, $f$ is Lipschitz.
}
\begin{proof}
    $U$ is convex, so we have that $L(x,y) \subseteq U$ for all $x ,y \in U$. So, the mean value theorem implies that 
    \begin{align*}
        \Vert f(x) - f(y) \Vert \leq \Vert x - y \Vert \sup_{z \in L(x,y) \setminus E} \Vert Df(z) \Vert_{\mathcal L} \leq M \Vert x - y \Vert.w
    \end{align*}
\end{proof}
\thm{}{
    Let $X,Y$ be normed vector spaces, $\emptyset \neq U \subseteq X$ be open and $Y \neq \{0\}$. Then the following are equivalent:
    \begin{enumerate}
        \item $U$ is connected.
        \item If $f : U \to Y$ is continuous everywhere and is differentiable on $U \setminus E$ for $E$ countable, adn $Df(x) = 0$ for all $x \in U \setminus E$, then $f$ is constant.
    \end{enumerate}
}
\begin{proof}
    We go in order:
    \begin{itemize}
        \item $\implies$: Suppose $U$ is connected and $f$ as stated. Pick $z \in U$ and write $y = f(z) \in Y$. Consider $S = f^{-1}(\{y\})$, which is relatively closed in $U$. On the other hand, for $x \in S$, pick $B(x, r) \subseteq U$ and note that $B(x, r)$ is convex. Then for $w \in B(x, r)$, we have 
        \begin{align*}
            \Vert f(w) - y\Vert = \Vert f(w) - f(x) \Vert \leq \Vert w -x \Vert \sup_{L(w, x) \setminus E} \Vert Df\Vert = 0.
        \end{align*}
        This means $f = y$ in this ball, meaning $B(x, r) \subseteq S$.
        \item $\impliedby$: AFSOC. If $U$ is not connected, then there exists open sets $V,W \subseteq U$ suchj that $U = V \cup W$ and $V \cap W = \emptyset$. 
        
        Let $y \in Y \setminus \{0\}$ and define $f : U \to Y$ via
        \begin{align*}
            f(x) = \begin{cases}
                0 & \text{if } x \in V \\
                y & \text{if } x \in W
            \end{cases}.
        \end{align*}
        Since $V,W$ are open and $V \cap W = \emptyset$, we have that $f$ is differentiable on all of $U$ and $Df = 0$. Then $\neg (2)$ because $Y \neq \emptyset$. 
    \end{itemize}
\end{proof}
missing stuff about partial derivatives n shit. also higher order derivatives but we do not care
\mlenma{}{
    Let $a,b \in \RR$ with $a < b$ and $Y$ a Banach space with $\FF$ as its underlying field.. Suppose $f \in C^k((a, b); Y)$ for $k \geq 1$. If $a < \alpha \leq \beta < b$, then $f( \beta) = \sum_{j=0}^{k-1} D^jf(\alpha) \frac{(\beta - \alpha)^j}{j!} + \int_\alpha^\beta \frac{(\beta - t)^{k-1}}{(k-1)!} D^kf(t) \, \text dt$.
}
\begin{proof}
    First note that $D^if \in C^0([\alpha, \beta]; Y)$ for $0 \leq j \leq k$, so the integral is well defined.

    Suppose initially that $g \in C^k((a, b); \FF)$, $h \in C^k((a, b); Y)$. Then,
    \begin{align*}
        D \sum_{j=0}^{k-1} (-1)^j D^{j} g D^{k-1 - j} h &= \sum_{j=0}^{k-1} [(-1)^j D^{j+1} g D^{k-1-j} h + (-1)^j D^jg D^{k-j} h] \\
        &= \sum_{j=1}^k -(-1)^j D^jg D^{k-j}h + \sum_{j=0}^{k-1}(-1)^j D^jg D^{k-j}h \\
        &= -(-1)^k D^k g h + gD^k h.
    \end{align*}
    Let $g: (a,b) \to \RR \subseteq \FF$ via $g(t) = \frac{(\beta - t)^{k-1}}{(k-1)!}$ and $h =f$. Then,
    \begin{align*}
        \frac{(\beta - t)^{k-1}}{(k-1)!} D^kf(t) &= \left(\sum_{j=0}^{k-1} (-1)^j D^j g(t) D^{k-1-j} h(t) \right)' \\
        &\implies \int_\alpha^\beta \frac{(\beta - t)^{k-1}}{(k-1)!} D^kf(t) \, \text dt \\
        &= \sum \bigg |_{t =\beta} - \sum \bigg|_{t = \alpha} \\
        &= f(\beta) - \sum_{j=0}^{k-1} D^jf(\alpha) \frac{(\beta - \alpha)^j}{j!}.
    \end{align*}
\end{proof}
\newpage
\thm{Taylor}{
    Let $X$ be a normed vector space and $Y$ a Banach space. Let $\emptyset \neq U \subseteq X$ be open. Suppose $f \in C^k(U; Y)$ for $k \geq 1$> Further suppose $x ,y \in U$ and $L(x,y) = \{(1-t)x + ty \mid t \in [0,1]\} \subseteq U$. Then, 
    \begin{align*}
        f(y) = \sum_{j=0}^{k-1} \frac{1}{j!} D^jf(x) (y-x)^{\otimes j} + \left( \int_0^1 \frac{(1-t)^{k-1}}{(k-1)!} D^k f(ty + (1-t)x) \, \text dt \right)(y-x)^{\otimes j}.
    \end{align*}
    In particular, for all $\eps > 0$, $\exists \delta > 0$ such that $x \in B_X[y, \delta]$ implies that 
    \begin{align*}
        \Vert f(y) - \sum_{j=0}^{k} \frac{1}{j!} D^jf(x) (y-x)^{\otimes j} \Vert_Y \leq \eps \Vert y-x \Vert^k_X.
    \end{align*}
}
\begin{proof}
    Since $U$ is open, $\exists r > 0$ such that 
    \begin{align*}
        \{(1-t)x + ty \mid t \in [-r, 1+r] \} \subseteq U.
    \end{align*}
    Define $F: (-r, 1+r) \to Y$ via $F(t) = f((1-t)x + ty)$. Then $F \in C^k((-r, 1+r) ; Y)$ and $DF^j(t) = D^j((1-t)x + ty) (y-x)^{\otimes j}$. Then,the desired value for $f(y)$ follows by applying the previous lemma with $a = -r$, $b = 1+r$, $\alpha = 0$, and $\beta = 1$. To complete the proof, 
    \begin{align*}
        \int_0^1 \frac{(1-t)^{k-1}}{(k-1)!} \, \text dt = \frac{1}{k!}.
    \end{align*}
    Thus, 
    \begin{align*}
        f(y) - \sum_{j=0}^{k} \frac{1}{j!} D^jf(x) (y-x)^{\otimes j} &= \int_0^1 \frac{(1-t)^{k-1}}{(k-1)!} D^{k} f((1-t)x + ty) (y-x)^{k} \, \text dt  - \frac{D^kf(x)}{k!}(y-x)^k\\
        &= \left(\int_0^1 \frac{(1-t)^{k-1}}{(k-1)!} [D^{k} f((1-t)x + ty) - D^kf(x)]  \, \text dt \right)(y-x)^{k}.
    \end{align*}
    So the desired result follows from the continuity of $D^kf$.
\end{proof}
\noindent Remarks:
\begin{enumerate}
    \item $Y$ does not have to be a Banach space for the second estimate. To prove this, let $f: U \to Y \subseteq Y^*$ where $Y^*$ is the completion of $Y$.
    \item In fact, the converse of Taylor's theorem is true if we add convexity to $U$. This completes the claim that $f \in C^k \iff f$ is locally well approximated by a polynomial of degree $k$.
\end{enumerate}
\newpage
\thm{Power series are smooth}{
    Let $X$ be a normed vector space and $Y$ a Banach space. For $n \in \NN$, let $T_n \in \mathcal L^n_{sym}(X; Y)$. Fix $x_0 \in X$ and define the power series 
    \begin{align*}
        f = \sum_{n=0}^\infty T_n((\cdot - x_0)^{\otimes n}).
    \end{align*}
    Finally suppose $R_{uni}(f) = \left(\limsup_{n\to\infty} \Vert T_n\Vert^{1/n}_{\mathcal L^n_{sym}}\right) \in (0, \infty]$. Then the following hold:
    \begin{enumerate}
        \item $f \in C^\infty(B_X(x_0, R_{uni}(f)); Y)$.
        \item For $k \in \NN$, we have that $D^kf(x) = \sum_{n=k}^\infty D^k T_n(x-x_0)^{\otimes (n-k)}$, which is equal to 
        \begin{align*}
            \sum_{n=k}^\infty \frac{n!}{(n-k)!} T_n^{(k)} ((x-x_0)^{\otimes (n-k)}) \in \mathcal L^k(X; Y).
        \end{align*}
        with series converging pointwise in $B_X(x_0, R_{uni}(f))$ and uniformly on any ball smaller. In particular, $T_k = \frac{1}{k!} D^kf(x_0)$. That is,
        \begin{align*}
            f(x) = \sum_{n=0}^\infty \frac{n!}{(n-k)!} D^nf(x_0) ((x-x_0)^{\otimes (n-k)})
        \end{align*}
        \item If $0 \leq s \leq R_{uni}(f)$, then $f \in UC_b^k (B_X[x_0, s]; Y)$.
    \end{enumerate}
}
\begin{proof}
    For $k \in \NN$, define the power series with values in $\mathcal L^k(X; Y)$ by 
    \begin{align*}
        g_k = \sum_{n=k}^\infty \frac{n!}{(n-k)!} T_n^{(k)}(\cdot - x_0)^{\otimes (n-k)}.
    \end{align*}
    Recall that 
    \begin{align*}
        T^{(k)}(h_1, \ldots h_{n-k})(v_1, \ldots, v_k) = T(h_1, \ldots, h_{n-k}, v_1, \ldots, v_k).
    \end{align*}
    So we have that 
    \begin{align*}
        g_k = \sum_{n=0}^\infty \frac{(n+k)!}{n!} T_{n+k}^{(k)}(\cdot - x_0)^{\otimes n}.
    \end{align*}
    Note that $g_0 = f$ and $g_k$ is the formal guess for the $k$-th derivative of $f$. For $k \geq 1$, we bound (use Thm 8.2.23),
    \begin{align*}
        \Vert T_{n_k}(h_1^{\otimes n}, h_2^{\otimes k}) \Vert_Y \leq \sqrt{\frac{(n+k)^{n+k}}{n^nk^k}} \Vert T_{n+k}\Vert_{\mathcal L^{n+k}_{sym}}
    \end{align*}
    for $\Vert h_1 \Vert = \Vert h_2 \Vert = 1$. Thus, 
    \begin{align*}
        \Vert T_{n+k}^{(k)}\Vert_{\mathcal L^{n-k}_{sym}(X; \mathcal L_{sym}^k(X; Y))} &= \sup_{\Vert h_1\Vert = 1} \Vert T_{n+k}^{(k)} (h_1^{\otimes n})\Vert_{\mathcal L^k_{sym}} \\
        &=\sup_{\Vert h_1\Vert = 1} \sup_{\Vert h_2\Vert = 1} \Vert T_{n+k}(h_1^{\otimes n}, h_2^{\otimes k})\Vert_{Y} \\
        &\leq \sqrt{\frac{(n+k)^{n+k}}{n^nk^k}} \Vert T_{n+k}\Vert_{\mathcal L_{sym}}.
    \end{align*}
    Note that if $z > 0 \in \RR$, then $1 + z \leq \sum_{n=0}^\infty \frac{z^n}{n!} = e^z$. Therefore, $1 + \frac kn \leq e^{k/n} \implies (1 + k/n)^n \leq e^k$. Then,
    \begin{align*}
        \frac{(n+k)^{n+k}}{n^nk^k} = \frac{(n+k)^n}{n^n} \frac{(n+k)^k}{k^k} = (1 + k/n)^n \frac{(n+k)^k}{k^K }\leq e^k \frac{(n+k)^k}{k^k}.
    \end{align*}
    So if $n \geq k \geq 1$, then 
    \begin{align*}
        \left\Vert \frac{(n+k)!}{n!} T_{n+k}^{(k)}\right\Vert^{1/n}_{\mathcal L^n_{sym} \mathcal L_{sym}^k} &\leq [(n+k)\cdots(n+1)]^{1/n} \left( \frac{(n+k)^{n+k}}{n^nk^k}\right)^{1/2n} \cdot \Vert T_{n+k} \Vert_{\mathcal L_{sym}}^{1/n} \\
        &\leq (2n)^{k/n}e^{k/2n} (2n)^{k/2n} \left(\Vert T_{n+k} \Vert^{\frac{1}{n+k}}\right)^{\frac{n+k}{n}}.
    \end{align*}
    So,
    \begin{align*}
        \limsup_{n\to\infty} \left\Vert \frac{(n+k)!}{n!} T_{n+k}^{(k)}\right\Vert^{1/n}_{\mathcal L^n_{sym} \mathcal L_{sym}^k} \leq \limsup_{n\to\infty} (2n)^{k/n}e^{k/2n} (2n)^{k/2n} \left(\Vert T_{n+k} \Vert^{\frac{1}{n+k}}\right)^{\frac{n+k}{n}} = \frac{1}{R_{uni}(f)}.
    \end{align*}
    But note that the left hand side of this is $\frac{1}{R_{uni}(g_k)}$. So we get that $R_{uni}(f) \leq R_{uni}(g_k)$. Thus, if we set $U \coloneq B_X(x_0, R_{uni}(f))$, we get that each $g_k$ converges pointwise in $U$ and defines a continuous $g_K U \to \mathcal L^k_{sym}(X; Y)$. Moreover, each $g_k$ converges uniformly absolutely on $B_X[x_0, S]$ for $S < R_{uni}(f)$. So actually, $g_k$ is bounded and uniformly continuous on $B_X[x_0, S]$. To conclude, we claim that each $g_k$ is differentiable with $Dg_k = g_{k+1}$. 

    Note 
    \begin{align*}
        D\sum_{n=N}^\infty \frac{n!}{(n-k)!} T_n^{(k)}(\cdot - x_0)^{\otimes(n-k)} = \sum_{n=k+1}^N \frac{n!}{(n-k-1)!} T_n^{(k+1)} (\cdot - x_0)^{\otimes n-k-1}.
    \end{align*}
    If we send $N \to \infty$, we get that result as a result of homework 12 problem 5. 
    
\end{proof}
\thm{}{
    Let $X$ be a normed vector space and $Y$ a Banach space, $\emptyset \neq U \subseteq X$ open and connected. Suppose $f,g : U \to Y$ are analytic. The following hold:
    \begin{enumerate}
        \item If there is $z \in U$ such that $D_n f(z) = D^n g(z)$ for all $n$, then $f=g$ in $U$.
        \item If $f =g$ in an open set $V \subseteq U$, then $f=g$ in $U$.
    \end{enumerate}
}
\begin{proof}
    1 implies 2 trivially, so we just prove 1. So let 
    \begin{align*}
        E = \{x \in U \mid D^nf(x) = D^ng(x) \,\, \forall n\in \NN\}.
    \end{align*}
    If this is clopen then we show that $E$ is just $U$. Once this is established, we then have that for all $x_0 \in U$,
    \begin{align*}
        f(x) &= \sum \frac{D^n f(x_0)(x-x_0)^{\otimes n}}{n!} \\
        g(x) &= \sum \frac{D^n g(x_0)(x-x_0)^{\otimes n}}{n!}
     \end{align*}
     which would show $f=g$. So now now that 
     \begin{enumerate}
        \item If $x_0 \in E$, then $B(x_0, r) \subseteq E$ via the previous argument. Therefore $E$ is open.
        \item Suppose $\{x_m\}_{m=0}^\infty \subseteq E$ such that $x_m \to x \in U$ as $m \to \infty$. Then we know that $D^n f(x_m) = D^n g(x_m)$ for all $n$ and $m$. So if we fix $n$ and send $m \to \infty$, we get that $D^n f(x) = D^n g(x)$ by continuity. Therefore, $x \in E$ and therefore $E$ is closed.
     \end{enumerate}
\end{proof}

\chapter{ODEs?}
\noindent Idea: we want to solve equations involving derivatives. An ordinary differential equaiton is an equation with one independent variable. For example, find $x: [0, T] \ to X$ a banach space, such that 
\begin{align*}
    \begin{cases}
        x'(t) = f(t, x(t)) & \forall t \in (0, T)\\
        x(0) = x_0 \in X & \text{ for given data $x_0 \in X$}
    \end{cases}.
\end{align*}
And given $f: [0, T] \times X \to X$. 

\end{document}