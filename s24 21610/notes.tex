\documentclass{report}

%%%%%%%%%%%%%%%%%%%%%%%%%%%%%%%%%
% PACKAGE IMPORTS
%%%%%%%%%%%%%%%%%%%%%%%%%%%%%%%%%


\usepackage[tmargin=2cm,rmargin=1in,lmargin=1in,margin=0.85in,bmargin=2cm,footskip=.2in]{geometry}
\usepackage{amsmath,amsfonts,amsthm,amssymb,mathtools}
\usepackage[varbb]{newpxmath}
\usepackage{xfrac}
\usepackage[makeroom]{cancel}
\usepackage{mathtools}
\usepackage{bookmark}
\usepackage{enumitem}
\usepackage{hyperref,theoremref}
\hypersetup{
	pdftitle={Assignment},
	colorlinks=true, linkcolor=doc!90,
	bookmarksnumbered=true,
	bookmarksopen=true
}
\usepackage[most,many,breakable]{tcolorbox}
\usepackage{xcolor}
\usepackage{varwidth}
\usepackage{varwidth}
\usepackage{etoolbox}
%\usepackage{authblk}
\usepackage{nameref}
\usepackage{multicol,array}
\usepackage{tikz-cd}
\usepackage[ruled,vlined,linesnumbered]{algorithm2e}
\usepackage{comment} % enables the use of multi-line comments (\ifx \fi) 
\usepackage{import}
\usepackage{xifthen}
\usepackage{pdfpages}
\usepackage{transparent}

% \usepackage{extsizes}

\newcommand\mycommfont[1]{\footnotesize\ttfamily\textcolor{blue}{#1}}
\SetCommentSty{mycommfont}
\newcommand{\incfig}[1]{%
    \def\svgwidth{\columnwidth}
    \import{./figures/}{#1.pdf_tex}
}

\usepackage{tikzsymbols}
\renewcommand\qedsymbol{$\Laughey$}


%\usepackage{import}
%\usepackage{xifthen}
%\usepackage{pdfpages}
%\usepackage{transparent}


%%%%%%%%%%%%%%%%%%%%%%%%%%%%%%
% SELF MADE COLORS
%%%%%%%%%%%%%%%%%%%%%%%%%%%%%%



\definecolor{myg}{RGB}{56, 140, 70}
\definecolor{myb}{RGB}{45, 111, 177}
\definecolor{myr}{RGB}{199, 68, 64}
\definecolor{mytheorembg}{HTML}{F2F2F9}
\definecolor{mytheoremfr}{HTML}{00007B}
\definecolor{mylenmabg}{HTML}{FFFAF8}
\definecolor{mylenmafr}{HTML}{983b0f}
\definecolor{mypropbg}{HTML}{f2fbfc}
\definecolor{mypropfr}{HTML}{191971}
\definecolor{myexamplebg}{HTML}{F2FBF8}
\definecolor{myexamplefr}{HTML}{88D6D1}
\definecolor{myexampleti}{HTML}{2A7F7F}
\definecolor{mydefinitbg}{HTML}{E5E5FF}
\definecolor{mydefinitfr}{HTML}{3F3FA3}
\definecolor{notesgreen}{RGB}{0,162,0}
\definecolor{myp}{RGB}{197, 92, 212}
\definecolor{mygr}{HTML}{2C3338}
\definecolor{myred}{RGB}{127,0,0}
\definecolor{myyellow}{RGB}{169,121,69}
\definecolor{myexercisebg}{HTML}{F2FBF8}
\definecolor{myexercisefg}{HTML}{88D6D1}


%%%%%%%%%%%%%%%%%%%%%%%%%%%%
% TCOLORBOX SETUPS
%%%%%%%%%%%%%%%%%%%%%%%%%%%%

\setlength{\parindent}{1cm}
%================================
% THEOREM BOX
%================================

\tcbuselibrary{theorems,skins,hooks}
\newtcbtheorem[number within=section]{Theorem}{Theorem}
{%
	enhanced,
	breakable,
	colback = mytheorembg,
	frame hidden,
	boxrule = 0sp,
	borderline west = {2pt}{0pt}{mytheoremfr},
	sharp corners,
	detach title,
	before upper = \tcbtitle\par\smallskip,
	coltitle = mytheoremfr,
	fonttitle = \bfseries\sffamily,
	description font = \mdseries,
	separator sign none,
	segmentation style={solid, mytheoremfr},
}
{th}

\tcbuselibrary{theorems,skins,hooks}
\newtcbtheorem[number within=chapter]{theorem}{Theorem}
{%
	enhanced,
	breakable,
	colback = mytheorembg,
	frame hidden,
	boxrule = 0sp,
	borderline west = {2pt}{0pt}{mytheoremfr},
	sharp corners,
	detach title,
	before upper = \tcbtitle\par\smallskip,
	coltitle = mytheoremfr,
	fonttitle = \bfseries\sffamily,
	description font = \mdseries,
	separator sign none,
	segmentation style={solid, mytheoremfr},
}
{th}


\tcbuselibrary{theorems,skins,hooks}
\newtcolorbox{Theoremcon}
{%
	enhanced
	,breakable
	,colback = mytheorembg
	,frame hidden
	,boxrule = 0sp
	,borderline west = {2pt}{0pt}{mytheoremfr}
	,sharp corners
	,description font = \mdseries
	,separator sign none
}

%================================
% Corollery
%================================
\tcbuselibrary{theorems,skins,hooks}
\newtcbtheorem[number within=section]{Corollary}{Corollary}
{%
	enhanced
	,breakable
	,colback = myp!10
	,frame hidden
	,boxrule = 0sp
	,borderline west = {2pt}{0pt}{myp!85!black}
	,sharp corners
	,detach title
	,before upper = \tcbtitle\par\smallskip
	,coltitle = myp!85!black
	,fonttitle = \bfseries\sffamily
	,description font = \mdseries
	,separator sign none
	,segmentation style={solid, myp!85!black}
}
{th}
\tcbuselibrary{theorems,skins,hooks}
\newtcbtheorem[number within=chapter]{corollary}{Corollary}
{%
	enhanced
	,breakable
	,colback = myp!10
	,frame hidden
	,boxrule = 0sp
	,borderline west = {2pt}{0pt}{myp!85!black}
	,sharp corners
	,detach title
	,before upper = \tcbtitle\par\smallskip
	,coltitle = myp!85!black
	,fonttitle = \bfseries\sffamily
	,description font = \mdseries
	,separator sign none
	,segmentation style={solid, myp!85!black}
}
{th}


%================================
% LENMA
%================================

\tcbuselibrary{theorems,skins,hooks}
\newtcbtheorem[number within=section]{Lenma}{Lenma}
{%
	enhanced,
	breakable,
	colback = mylenmabg,
	frame hidden,
	boxrule = 0sp,
	borderline west = {2pt}{0pt}{mylenmafr},
	sharp corners,
	detach title,
	before upper = \tcbtitle\par\smallskip,
	coltitle = mylenmafr,
	fonttitle = \bfseries\sffamily,
	description font = \mdseries,
	separator sign none,
	segmentation style={solid, mylenmafr},
}
{th}

\tcbuselibrary{theorems,skins,hooks}
\newtcbtheorem[number within=chapter]{lenma}{Lenma}
{%
	enhanced,
	breakable,
	colback = mylenmabg,
	frame hidden,
	boxrule = 0sp,
	borderline west = {2pt}{0pt}{mylenmafr},
	sharp corners,
	detach title,
	before upper = \tcbtitle\par\smallskip,
	coltitle = mylenmafr,
	fonttitle = \bfseries\sffamily,
	description font = \mdseries,
	separator sign none,
	segmentation style={solid, mylenmafr},
}
{th}


%================================
% PROPOSITION
%================================

\tcbuselibrary{theorems,skins,hooks}
\newtcbtheorem[number within=section]{Prop}{Proposition}
{%
	enhanced,
	breakable,
	colback = mypropbg,
	frame hidden,
	boxrule = 0sp,
	borderline west = {2pt}{0pt}{mypropfr},
	sharp corners,
	detach title,
	before upper = \tcbtitle\par\smallskip,
	coltitle = mypropfr,
	fonttitle = \bfseries\sffamily,
	description font = \mdseries,
	separator sign none,
	segmentation style={solid, mypropfr},
}
{th}

\tcbuselibrary{theorems,skins,hooks}
\newtcbtheorem[number within=chapter]{prop}{Proposition}
{%
	enhanced,
	breakable,
	colback = mypropbg,
	frame hidden,
	boxrule = 0sp,
	borderline west = {2pt}{0pt}{mypropfr},
	sharp corners,
	detach title,
	before upper = \tcbtitle\par\smallskip,
	coltitle = mypropfr,
	fonttitle = \bfseries\sffamily,
	description font = \mdseries,
	separator sign none,
	segmentation style={solid, mypropfr},
}
{th}


%================================
% CLAIM
%================================

\tcbuselibrary{theorems,skins,hooks}
\newtcbtheorem[number within=section]{claim}{Claim}
{%
	enhanced
	,breakable
	,colback = myg!10
	,frame hidden
	,boxrule = 0sp
	,borderline west = {2pt}{0pt}{myg}
	,sharp corners
	,detach title
	,before upper = \tcbtitle\par\smallskip
	,coltitle = myg!85!black
	,fonttitle = \bfseries\sffamily
	,description font = \mdseries
	,separator sign none
	,segmentation style={solid, myg!85!black}
}
{th}



%================================
% Exercise
%================================

\tcbuselibrary{theorems,skins,hooks}
\newtcbtheorem[number within=section]{Exercise}{Exercise}
{%
	enhanced,
	breakable,
	colback = myexercisebg,
	frame hidden,
	boxrule = 0sp,
	borderline west = {2pt}{0pt}{myexercisefg},
	sharp corners,
	detach title,
	before upper = \tcbtitle\par\smallskip,
	coltitle = myexercisefg,
	fonttitle = \bfseries\sffamily,
	description font = \mdseries,
	separator sign none,
	segmentation style={solid, myexercisefg},
}
{th}

\tcbuselibrary{theorems,skins,hooks}
\newtcbtheorem[number within=chapter]{exercise}{Exercise}
{%
	enhanced,
	breakable,
	colback = myexercisebg,
	frame hidden,
	boxrule = 0sp,
	borderline west = {2pt}{0pt}{myexercisefg},
	sharp corners,
	detach title,
	before upper = \tcbtitle\par\smallskip,
	coltitle = myexercisefg,
	fonttitle = \bfseries\sffamily,
	description font = \mdseries,
	separator sign none,
	segmentation style={solid, myexercisefg},
}
{th}

%================================
% EXAMPLE BOX
%================================

\newtcbtheorem[number within=section]{Example}{Example}
{%
	colback = myexamplebg
	,breakable
	,colframe = myexamplefr
	,coltitle = myexampleti
	,boxrule = 1pt
	,sharp corners
	,detach title
	,before upper=\tcbtitle\par\smallskip
	,fonttitle = \bfseries
	,description font = \mdseries
	,separator sign none
	,description delimiters parenthesis
}
{ex}

\newtcbtheorem[number within=chapter]{example}{Example}
{%
	colback = myexamplebg
	,breakable
	,colframe = myexamplefr
	,coltitle = myexampleti
	,boxrule = 1pt
	,sharp corners
	,detach title
	,before upper=\tcbtitle\par\smallskip
	,fonttitle = \bfseries
	,description font = \mdseries
	,separator sign none
	,description delimiters parenthesis
}
{ex}

%================================
% DEFINITION BOX
%================================

\newtcbtheorem[number within=section]{Definition}{Definition}{enhanced,
	before skip=2mm,after skip=2mm, colback=red!5,colframe=red!80!black,boxrule=0.5mm,
	attach boxed title to top left={xshift=1cm,yshift*=1mm-\tcboxedtitleheight}, varwidth boxed title*=-3cm,
	boxed title style={frame code={
					\path[fill=tcbcolback]
					([yshift=-1mm,xshift=-1mm]frame.north west)
					arc[start angle=0,end angle=180,radius=1mm]
					([yshift=-1mm,xshift=1mm]frame.north east)
					arc[start angle=180,end angle=0,radius=1mm];
					\path[left color=tcbcolback!60!black,right color=tcbcolback!60!black,
						middle color=tcbcolback!80!black]
					([xshift=-2mm]frame.north west) -- ([xshift=2mm]frame.north east)
					[rounded corners=1mm]-- ([xshift=1mm,yshift=-1mm]frame.north east)
					-- (frame.south east) -- (frame.south west)
					-- ([xshift=-1mm,yshift=-1mm]frame.north west)
					[sharp corners]-- cycle;
				},interior engine=empty,
		},
	fonttitle=\bfseries,
	title={#2},#1}{def}
\newtcbtheorem[number within=chapter]{definition}{Definition}{enhanced,
	before skip=2mm,after skip=2mm, colback=red!5,colframe=red!80!black,boxrule=0.5mm,
	attach boxed title to top left={xshift=1cm,yshift*=1mm-\tcboxedtitleheight}, varwidth boxed title*=-3cm,
	boxed title style={frame code={
					\path[fill=tcbcolback]
					([yshift=-1mm,xshift=-1mm]frame.north west)
					arc[start angle=0,end angle=180,radius=1mm]
					([yshift=-1mm,xshift=1mm]frame.north east)
					arc[start angle=180,end angle=0,radius=1mm];
					\path[left color=tcbcolback!60!black,right color=tcbcolback!60!black,
						middle color=tcbcolback!80!black]
					([xshift=-2mm]frame.north west) -- ([xshift=2mm]frame.north east)
					[rounded corners=1mm]-- ([xshift=1mm,yshift=-1mm]frame.north east)
					-- (frame.south east) -- (frame.south west)
					-- ([xshift=-1mm,yshift=-1mm]frame.north west)
					[sharp corners]-- cycle;
				},interior engine=empty,
		},
	fonttitle=\bfseries,
	title={#2},#1}{def}



%================================
% Solution BOX
%================================

\makeatletter
\newtcbtheorem{question}{Question}{enhanced,
	breakable,
	colback=white,
	colframe=myb!80!black,
	attach boxed title to top left={yshift*=-\tcboxedtitleheight},
	fonttitle=\bfseries,
	title={#2},
	boxed title size=title,
	boxed title style={%
			sharp corners,
			rounded corners=northwest,
			colback=tcbcolframe,
			boxrule=0pt,
		},
	underlay boxed title={%
			\path[fill=tcbcolframe] (title.south west)--(title.south east)
			to[out=0, in=180] ([xshift=5mm]title.east)--
			(title.center-|frame.east)
			[rounded corners=\kvtcb@arc] |-
			(frame.north) -| cycle;
		},
	#1
}{def}
\makeatother

%================================
% SOLUTION BOX
%================================

\makeatletter
\newtcolorbox{solution}{enhanced,
	breakable,
	colback=white,
	colframe=myg!80!black,
	attach boxed title to top left={yshift*=-\tcboxedtitleheight},
	title=Solution,
	boxed title size=title,
	boxed title style={%
			sharp corners,
			rounded corners=northwest,
			colback=tcbcolframe,
			boxrule=0pt,
		},
	underlay boxed title={%
			\path[fill=tcbcolframe] (title.south west)--(title.south east)
			to[out=0, in=180] ([xshift=5mm]title.east)--
			(title.center-|frame.east)
			[rounded corners=\kvtcb@arc] |-
			(frame.north) -| cycle;
		},
}
\makeatother

%================================
% Question BOX
%================================

\makeatletter
\newtcbtheorem{qstion}{Question}{enhanced,
	breakable,
	colback=white,
	colframe=mygr,
	attach boxed title to top left={yshift*=-\tcboxedtitleheight},
	fonttitle=\bfseries,
	title={#2},
	boxed title size=title,
	boxed title style={%
			sharp corners,
			rounded corners=northwest,
			colback=tcbcolframe,
			boxrule=0pt,
		},
	underlay boxed title={%
			\path[fill=tcbcolframe] (title.south west)--(title.south east)
			to[out=0, in=180] ([xshift=5mm]title.east)--
			(title.center-|frame.east)
			[rounded corners=\kvtcb@arc] |-
			(frame.north) -| cycle;
		},
	#1
}{def}
\makeatother

\newtcbtheorem[number within=chapter]{wconc}{Wrong Concept}{
	breakable,
	enhanced,
	colback=white,
	colframe=myr,
	arc=0pt,
	outer arc=0pt,
	fonttitle=\bfseries\sffamily\large,
	colbacktitle=myr,
	attach boxed title to top left={},
	boxed title style={
			enhanced,
			skin=enhancedfirst jigsaw,
			arc=3pt,
			bottom=0pt,
			interior style={fill=myr}
		},
	#1
}{def}



%================================
% NOTE BOX
%================================

\usetikzlibrary{arrows,calc,shadows.blur}
\tcbuselibrary{skins}
\newtcolorbox{note}[1][]{%
	enhanced jigsaw,
	colback=gray!20!white,%
	colframe=gray!80!black,
	size=small,
	boxrule=1pt,
	title=\textbf{Note:},
	halign title=flush center,
	coltitle=black,
	breakable,
	drop shadow=black!50!white,
	attach boxed title to top left={xshift=1cm,yshift=-\tcboxedtitleheight/2,yshifttext=-\tcboxedtitleheight/2},
	minipage boxed title=1.5cm,
	boxed title style={%
			colback=white,
			size=fbox,
			boxrule=1pt,
			boxsep=2pt,
			underlay={%
					\coordinate (dotA) at ($(interior.west) + (-0.5pt,0)$);
					\coordinate (dotB) at ($(interior.east) + (0.5pt,0)$);
					\begin{scope}
						\clip (interior.north west) rectangle ([xshift=3ex]interior.east);
						\filldraw [white, blur shadow={shadow opacity=60, shadow yshift=-.75ex}, rounded corners=2pt] (interior.north west) rectangle (interior.south east);
					\end{scope}
					\begin{scope}[gray!80!black]
						\fill (dotA) circle (2pt);
						\fill (dotB) circle (2pt);
					\end{scope}
				},
		},
	#1,
}

%%%%%%%%%%%%%%%%%%%%%%%%%%%%%%
% SELF MADE COMMANDS
%%%%%%%%%%%%%%%%%%%%%%%%%%%%%%


\newcommand{\thm}[2]{\begin{Theorem}{#1}{}#2\end{Theorem}}
\newcommand{\cor}[2]{\begin{Corollary}{#1}{}#2\end{Corollary}}
\newcommand{\mlenma}[2]{\begin{Lenma}{#1}{}#2\end{Lenma}}
\newcommand{\mprop}[2]{\begin{Prop}{#1}{}#2\end{Prop}}
\newcommand{\clm}[3]{\begin{claim}{#1}{#2}#3\end{claim}}
\newcommand{\wc}[2]{\begin{wconc}{#1}{}\setlength{\parindent}{1cm}#2\end{wconc}}
\newcommand{\thmcon}[1]{\begin{Theoremcon}{#1}\end{Theoremcon}}
\newcommand{\ex}[2]{\begin{Example}{#1}{}#2\end{Example}}
\newcommand{\dfn}[2]{\begin{Definition}[colbacktitle=red!75!black]{#1}{}#2\end{Definition}}
\newcommand{\dfnc}[2]{\begin{definition}[colbacktitle=red!75!black]{#1}{}#2\end{definition}}
\newcommand{\qs}[2]{\begin{question*}{#1}{}#2\end{question*}}
\newcommand{\mpf}[2]{\begin{myproof}[#1]#2\end{myproof}}
\newcommand{\nt}[1]{\begin{note}#1\end{note}}

\newcommand*\circled[1]{\tikz[baseline=(char.base)]{
		\node[shape=circle,draw,inner sep=1pt] (char) {#1};}}
\newcommand\getcurrentref[1]{%
	\ifnumequal{\value{#1}}{0}
	{??}
	{\the\value{#1}}%
}
\newcommand{\getCurrentSectionNumber}{\getcurrentref{section}}
\newenvironment{myproof}[1][\proofname]{%
	\proof[\bfseries #1: ]%
}{\endproof}

\newcommand{\mclm}[2]{\begin{myclaim}[#1]#2\end{myclaim}}
\newenvironment{myclaim}[1][\claimname]{\proof[\bfseries #1: ]}{}

\newcounter{mylabelcounter}

\makeatletter
\newcommand{\setword}[2]{%
	\phantomsection
	#1\def\@currentlabel{\unexpanded{#1}}\label{#2}%
}
\makeatother




\tikzset{
	symbol/.style={
			draw=none,
			every to/.append style={
					edge node={node [sloped, allow upside down, auto=false]{$#1$}}}
		}
}


% deliminators
\DeclarePairedDelimiter{\abs}{\lvert}{\rvert}
\DeclarePairedDelimiter{\norm}{\lVert}{\rVert}

\DeclarePairedDelimiter{\ceil}{\lceil}{\rceil}
\DeclarePairedDelimiter{\floor}{\lfloor}{\rfloor}
\DeclarePairedDelimiter{\round}{\lfloor}{\rceil}

\newsavebox\diffdbox
\newcommand{\slantedromand}{{\mathpalette\makesl{d}}}
\newcommand{\makesl}[2]{%
\begingroup
\sbox{\diffdbox}{$\mathsurround=0pt#1\mathrm{#2}$}%
\pdfsave
\pdfsetmatrix{1 0 0.2 1}%
\rlap{\usebox{\diffdbox}}%
\pdfrestore
\hskip\wd\diffdbox
\endgroup
}
\newcommand{\dd}[1][]{\ensuremath{\mathop{}\!\ifstrempty{#1}{%
\slantedromand\@ifnextchar^{\hspace{0.2ex}}{\hspace{0.1ex}}}%
{\slantedromand\hspace{0.2ex}^{#1}}}}
\ProvideDocumentCommand\dv{o m g}{%
  \ensuremath{%
    \IfValueTF{#3}{%
      \IfNoValueTF{#1}{%
        \frac{\dd #2}{\dd #3}%
      }{%
        \frac{\dd^{#1} #2}{\dd #3^{#1}}%
      }%
    }{%
      \IfNoValueTF{#1}{%
        \frac{\dd}{\dd #2}%
      }{%
        \frac{\dd^{#1}}{\dd #2^{#1}}%
      }%
    }%
  }%
}
\providecommand*{\pdv}[3][]{\frac{\partial^{#1}#2}{\partial#3^{#1}}}
%  - others
\DeclareMathOperator{\Lap}{\mathcal{L}}
\DeclareMathOperator{\Var}{Var} % varience
\DeclareMathOperator{\Cov}{Cov} % covarience
\DeclareMathOperator{\E}{E} % expected

% Since the amsthm package isn't loaded

% I prefer the slanted \leq
\let\oldleq\leq % save them in case they're every wanted
\let\oldgeq\geq
\renewcommand{\leq}{\leqslant}
\renewcommand{\geq}{\geqslant}

% % redefine matrix env to allow for alignment, use r as default
% \renewcommand*\env@matrix[1][r]{\hskip -\arraycolsep
%     \let\@ifnextchar\new@ifnextchar
%     \array{*\c@MaxMatrixCols #1}}


%\usepackage{framed}
%\usepackage{titletoc}
%\usepackage{etoolbox}
%\usepackage{lmodern}


%\patchcmd{\tableofcontents}{\contentsname}{\sffamily\contentsname}{}{}

%\renewenvironment{leftbar}
%{\def\FrameCommand{\hspace{6em}%
%		{\color{myyellow}\vrule width 2pt depth 6pt}\hspace{1em}}%
%	\MakeFramed{\parshape 1 0cm \dimexpr\textwidth-6em\relax\FrameRestore}\vskip2pt%
%}
%{\endMakeFramed}

%\titlecontents{chapter}
%[0em]{\vspace*{2\baselineskip}}
%{\parbox{4.5em}{%
%		\hfill\Huge\sffamily\bfseries\color{myred}\thecontentspage}%
%	\vspace*{-2.3\baselineskip}\leftbar\textsc{\small\chaptername~\thecontentslabel}\\\sffamily}
%{}{\endleftbar}
%\titlecontents{section}
%[8.4em]
%{\sffamily\contentslabel{3em}}{}{}
%{\hspace{0.5em}\nobreak\itshape\color{myred}\contentspage}
%\titlecontents{subsection}
%[8.4em]
%{\sffamily\contentslabel{3em}}{}{}  
%{\hspace{0.5em}\nobreak\itshape\color{myred}\contentspage}



%%%%%%%%%%%%%%%%%%%%%%%%%%%%%%%%%%%%%%%%%%%
% TABLE OF CONTENTS
%%%%%%%%%%%%%%%%%%%%%%%%%%%%%%%%%%%%%%%%%%%

\usepackage{tikz}
\definecolor{doc}{RGB}{0,60,110}
\usepackage{titletoc}
\contentsmargin{0cm}
\titlecontents{chapter}[3.7pc]
{\addvspace{30pt}%
	\begin{tikzpicture}[remember picture, overlay]%
		\draw[fill=doc!60,draw=doc!60] (-7,-.1) rectangle (-0.9,.5);%
		\pgftext[left,x=-3.5cm,y=0.2cm]{\color{white}\Large\sc\bfseries Chapter\ \thecontentslabel};%
	\end{tikzpicture}\color{doc!60}\large\sc\bfseries}%
{}
{}
{\;\titlerule\;\large\sc\bfseries Page \thecontentspage
	\begin{tikzpicture}[remember picture, overlay]
		\draw[fill=doc!60,draw=doc!60] (2pt,0) rectangle (4,0.1pt);
	\end{tikzpicture}}%
\titlecontents{section}[3.7pc]
{\addvspace{2pt}}
{\contentslabel[\thecontentslabel]{2pc}}
{}
{\hfill\small \thecontentspage}
[]
\titlecontents*{subsection}[3.7pc]
{\addvspace{-1pt}\small}
{}
{}
{\ --- \small\thecontentspage}
[ \textbullet\ ][]

\makeatletter
\renewcommand{\tableofcontents}{%
	\chapter*{%
	  \vspace*{-20\p@}%
	  \begin{tikzpicture}[remember picture, overlay]%
		  \pgftext[right,x=15cm,y=0.2cm]{\color{doc!60}\Huge\sc\bfseries \contentsname};%
		  \draw[fill=doc!60,draw=doc!60] (13,-.75) rectangle (20,1);%
		  \clip (13,-.75) rectangle (20,1);
		  \pgftext[right,x=15cm,y=0.2cm]{\color{white}\Huge\sc\bfseries \contentsname};%
	  \end{tikzpicture}}%
	\@starttoc{toc}}
\makeatother
\newcommand{\id}{\mathrm{id}}
\newcommand{\taking}[1]{\xrightarrow{#1}}
\newcommand{\inv}{^{-1}}

%From M170 "Introduction to Graph Theory" at SJSU
\DeclareMathOperator{\diam}{diam}
\DeclareMathOperator{\ord}{ord}
\newcommand{\defeq}{\overset{\mathrm{def}}{=}}

%From the USAMO .tex files
\newcommand{\ts}{\textsuperscript}
\newcommand{\dg}{^\circ}
\newcommand{\ii}{\item}

% % From Math 55 and Math 145 at Harvard
% \newenvironment{subproof}[1][Proof]{%
% \begin{proof}[#1] \renewcommand{\qedsymbol}{$\blacksquare$}}%
% {\end{proof}}

\newcommand{\liff}{\leftrightarrow}
\newcommand{\lthen}{\rightarrow}
\newcommand{\opname}{\operatorname}
\newcommand{\surjto}{\twoheadrightarrow}
\newcommand{\injto}{\hookrightarrow}
\newcommand{\On}{\mathrm{On}} % ordinals
\DeclareMathOperator{\img}{im} % Image
\DeclareMathOperator{\Img}{Im} % Image
\DeclareMathOperator{\coker}{coker} % Cokernel
\DeclareMathOperator{\Coker}{Coker} % Cokernel
\DeclareMathOperator{\Ker}{Ker} % Kernel
\DeclareMathOperator{\rank}{rank}
\DeclareMathOperator{\Spec}{Spec} % spectrum
\DeclareMathOperator{\Tr}{Tr} % trace
\DeclareMathOperator{\pr}{pr} % projection
\DeclareMathOperator{\ext}{ext} % extension
\DeclareMathOperator{\pred}{pred} % predecessor
\DeclareMathOperator{\dom}{dom} % domain
\DeclareMathOperator{\ran}{ran} % range
\DeclareMathOperator{\Hom}{Hom} % homomorphism
\DeclareMathOperator{\Mor}{Mor} % morphisms
\DeclareMathOperator{\End}{End} % endomorphism

\newcommand{\eps}{\epsilon}
\newcommand{\veps}{\varepsilon}
\newcommand{\ol}{\overline}
\newcommand{\ul}{\underline}
\newcommand{\wt}{\widetilde}
\newcommand{\wh}{\widehat}
\newcommand{\vocab}[1]{\textbf{\color{blue} #1}}
\providecommand{\half}{\frac{1}{2}}
\newcommand{\dang}{\measuredangle} %% Directed angle
\newcommand{\ray}[1]{\overrightarrow{#1}}
\newcommand{\seg}[1]{\overline{#1}}
\newcommand{\arc}[1]{\wideparen{#1}}
\DeclareMathOperator{\cis}{cis}
\DeclareMathOperator*{\lcm}{lcm}
\DeclareMathOperator*{\argmin}{arg min}
\DeclareMathOperator*{\argmax}{arg max}
\newcommand{\cycsum}{\sum_{\mathrm{cyc}}}
\newcommand{\symsum}{\sum_{\mathrm{sym}}}
\newcommand{\cycprod}{\prod_{\mathrm{cyc}}}
\newcommand{\symprod}{\prod_{\mathrm{sym}}}
\newcommand{\Qed}{\begin{flushright}\qed\end{flushright}}
\newcommand{\parinn}{\setlength{\parindent}{1cm}}
\newcommand{\parinf}{\setlength{\parindent}{0cm}}
% \newcommand{\norm}{\|\cdot\|}
\newcommand{\inorm}{\norm_{\infty}}
\newcommand{\opensets}{\{V_{\alpha}\}_{\alpha\in I}}
\newcommand{\oset}{V_{\alpha}}
\newcommand{\opset}[1]{V_{\alpha_{#1}}}
\newcommand{\lub}{\text{lub}}
\newcommand{\del}[2]{\frac{\partial #1}{\partial #2}}
\newcommand{\Del}[3]{\frac{\partial^{#1} #2}{\partial^{#1} #3}}
\newcommand{\deld}[2]{\dfrac{\partial #1}{\partial #2}}
\newcommand{\Deld}[3]{\dfrac{\partial^{#1} #2}{\partial^{#1} #3}}
\newcommand{\lm}{\lambda}
\newcommand{\uin}{\mathbin{\rotatebox[origin=c]{90}{$\in$}}}
\newcommand{\usubset}{\mathbin{\rotatebox[origin=c]{90}{$\subset$}}}
\newcommand{\lt}{\left}
\newcommand{\rt}{\right}
\newcommand{\bs}[1]{\boldsymbol{#1}}
\newcommand{\exs}{\exists}
\newcommand{\st}{\strut}
\newcommand{\dps}[1]{\displaystyle{#1}}

\newcommand{\sol}{\setlength{\parindent}{0cm}\textbf{\textit{Solution:}}\setlength{\parindent}{1cm} }
\newcommand{\solve}[1]{\setlength{\parindent}{0cm}\textbf{\textit{Solution: }}\setlength{\parindent}{1cm}#1 \Qed}
% Things Lie
\newcommand{\kb}{\mathfrak b}
\newcommand{\kg}{\mathfrak g}
\newcommand{\kh}{\mathfrak h}
\newcommand{\kn}{\mathfrak n}
\newcommand{\ku}{\mathfrak u}
\newcommand{\kz}{\mathfrak z}
\DeclareMathOperator{\Ext}{Ext} % Ext functor
\DeclareMathOperator{\Tor}{Tor} % Tor functor
\newcommand{\gl}{\opname{\mathfrak{gl}}} % frak gl group
\renewcommand{\sl}{\opname{\mathfrak{sl}}} % frak sl group chktex 6

% More script letters etc.
\newcommand{\SA}{\mathcal A}
\newcommand{\SB}{\mathcal B}
\newcommand{\SC}{\mathcal C}
\newcommand{\SF}{\mathcal F}
\newcommand{\SG}{\mathcal G}
\newcommand{\SH}{\mathcal H}
\newcommand{\OO}{\mathcal O}

\newcommand{\SCA}{\mathscr A}
\newcommand{\SCB}{\mathscr B}
\newcommand{\SCC}{\mathscr C}
\newcommand{\SCD}{\mathscr D}
\newcommand{\SCE}{\mathscr E}
\newcommand{\SCF}{\mathscr F}
\newcommand{\SCG}{\mathscr G}
\newcommand{\SCH}{\mathscr H}

% Mathfrak primes
\newcommand{\km}{\mathfrak m}
\newcommand{\kp}{\mathfrak p}
\newcommand{\kq}{\mathfrak q}

% number sets
\newcommand{\RR}[1][]{\ensuremath{\ifstrempty{#1}{\mathbb{R}}{\mathbb{R}^{#1}}}}
\newcommand{\NN}[1][]{\ensuremath{\ifstrempty{#1}{\mathbb{N}}{\mathbb{N}^{#1}}}}
\newcommand{\ZZ}[1][]{\ensuremath{\ifstrempty{#1}{\mathbb{Z}}{\mathbb{Z}^{#1}}}}
\newcommand{\QQ}[1][]{\ensuremath{\ifstrempty{#1}{\mathbb{Q}}{\mathbb{Q}^{#1}}}}
\newcommand{\CC}[1][]{\ensuremath{\ifstrempty{#1}{\mathbb{C}}{\mathbb{C}^{#1}}}}
\newcommand{\PP}[1][]{\ensuremath{\ifstrempty{#1}{\mathbb{P}}{\mathbb{P}^{#1}}}}
\newcommand{\HH}[1][]{\ensuremath{\ifstrempty{#1}{\mathbb{H}}{\mathbb{H}^{#1}}}}
\newcommand{\FF}[1][]{\ensuremath{\ifstrempty{#1}{\mathbb{F}}{\mathbb{F}^{#1}}}}

% number sets without arguments
\newcommand{\R}{\ensuremath{\mathbb{R}}}
\newcommand{\N}{\ensuremath{\mathbb{N}}}
\newcommand{\Z}{\ensuremath{\mathbb{Z}}}
\newcommand{\Q}{\ensuremath{\mathbb{Q}}}
\newcommand{\C}{\ensuremath{\mathbb{C}}}
\newcommand{\F}{\ensuremath{\mathbb{F}}}

% expected value
\newcommand{\EE}{\ensuremath{\mathbb{E}}}
\newcommand{\charin}{\text{ char }}
\DeclareMathOperator{\sign}{sign}
\DeclareMathOperator{\Aut}{Aut}
\DeclareMathOperator{\Inn}{Inn}
\DeclareMathOperator{\Syl}{Syl}
\DeclareMathOperator{\Gal}{Gal}
\DeclareMathOperator{\GL}{GL} % General linear group
\DeclareMathOperator{\SL}{SL} % Special linear group

%---------------------------------------
% BlackBoard Math Fonts :-
%---------------------------------------

%Captital Letters
\newcommand{\bbA}{\mathbb{A}}	\newcommand{\bbB}{\mathbb{B}}
\newcommand{\bbC}{\mathbb{C}}	\newcommand{\bbD}{\mathbb{D}}
\newcommand{\bbE}{\mathbb{E}}	\newcommand{\bbF}{\mathbb{F}}
\newcommand{\bbG}{\mathbb{G}}	\newcommand{\bbH}{\mathbb{H}}
\newcommand{\bbI}{\mathbb{I}}	\newcommand{\bbJ}{\mathbb{J}}
\newcommand{\bbK}{\mathbb{K}}	\newcommand{\bbL}{\mathbb{L}}
\newcommand{\bbM}{\mathbb{M}}	\newcommand{\bbN}{\mathbb{N}}
\newcommand{\bbO}{\mathbb{O}}	\newcommand{\bbP}{\mathbb{P}}
\newcommand{\bbQ}{\mathbb{Q}}	\newcommand{\bbR}{\mathbb{R}}
\newcommand{\bbS}{\mathbb{S}}	\newcommand{\bbT}{\mathbb{T}}
\newcommand{\bbU}{\mathbb{U}}	\newcommand{\bbV}{\mathbb{V}}
\newcommand{\bbW}{\mathbb{W}}	\newcommand{\bbX}{\mathbb{X}}
\newcommand{\bbY}{\mathbb{Y}}	\newcommand{\bbZ}{\mathbb{Z}}

%---------------------------------------
% MathCal Fonts :-
%---------------------------------------

%Captital Letters
\newcommand{\mcA}{\mathcal{A}}	\newcommand{\mcB}{\mathcal{B}}
\newcommand{\mcC}{\mathcal{C}}	\newcommand{\mcD}{\mathcal{D}}
\newcommand{\mcE}{\mathcal{E}}	\newcommand{\mcF}{\mathcal{F}}
\newcommand{\mcG}{\mathcal{G}}	\newcommand{\mcH}{\mathcal{H}}
\newcommand{\mcI}{\mathcal{I}}	\newcommand{\mcJ}{\mathcal{J}}
\newcommand{\mcK}{\mathcal{K}}	\newcommand{\mcL}{\mathcal{L}}
\newcommand{\mcM}{\mathcal{M}}	\newcommand{\mcN}{\mathcal{N}}
\newcommand{\mcO}{\mathcal{O}}	\newcommand{\mcP}{\mathcal{P}}
\newcommand{\mcQ}{\mathcal{Q}}	\newcommand{\mcR}{\mathcal{R}}
\newcommand{\mcS}{\mathcal{S}}	\newcommand{\mcT}{\mathcal{T}}
\newcommand{\mcU}{\mathcal{U}}	\newcommand{\mcV}{\mathcal{V}}
\newcommand{\mcW}{\mathcal{W}}	\newcommand{\mcX}{\mathcal{X}}
\newcommand{\mcY}{\mathcal{Y}}	\newcommand{\mcZ}{\mathcal{Z}}


%---------------------------------------
% Bold Math Fonts :-
%---------------------------------------

%Captital Letters
\newcommand{\bmA}{\boldsymbol{A}}	\newcommand{\bmB}{\boldsymbol{B}}
\newcommand{\bmC}{\boldsymbol{C}}	\newcommand{\bmD}{\boldsymbol{D}}
\newcommand{\bmE}{\boldsymbol{E}}	\newcommand{\bmF}{\boldsymbol{F}}
\newcommand{\bmG}{\boldsymbol{G}}	\newcommand{\bmH}{\boldsymbol{H}}
\newcommand{\bmI}{\boldsymbol{I}}	\newcommand{\bmJ}{\boldsymbol{J}}
\newcommand{\bmK}{\boldsymbol{K}}	\newcommand{\bmL}{\boldsymbol{L}}
\newcommand{\bmM}{\boldsymbol{M}}	\newcommand{\bmN}{\boldsymbol{N}}
\newcommand{\bmO}{\boldsymbol{O}}	\newcommand{\bmP}{\boldsymbol{P}}
\newcommand{\bmQ}{\boldsymbol{Q}}	\newcommand{\bmR}{\boldsymbol{R}}
\newcommand{\bmS}{\boldsymbol{S}}	\newcommand{\bmT}{\boldsymbol{T}}
\newcommand{\bmU}{\boldsymbol{U}}	\newcommand{\bmV}{\boldsymbol{V}}
\newcommand{\bmW}{\boldsymbol{W}}	\newcommand{\bmX}{\boldsymbol{X}}
\newcommand{\bmY}{\boldsymbol{Y}}	\newcommand{\bmZ}{\boldsymbol{Z}}
%Small Letters
\newcommand{\bma}{\boldsymbol{a}}	\newcommand{\bmb}{\boldsymbol{b}}
\newcommand{\bmc}{\boldsymbol{c}}	\newcommand{\bmd}{\boldsymbol{d}}
\newcommand{\bme}{\boldsymbol{e}}	\newcommand{\bmf}{\boldsymbol{f}}
\newcommand{\bmg}{\boldsymbol{g}}	\newcommand{\bmh}{\boldsymbol{h}}
\newcommand{\bmi}{\boldsymbol{i}}	\newcommand{\bmj}{\boldsymbol{j}}
\newcommand{\bmk}{\boldsymbol{k}}	\newcommand{\bml}{\boldsymbol{l}}
\newcommand{\bmm}{\boldsymbol{m}}	\newcommand{\bmn}{\boldsymbol{n}}
\newcommand{\bmo}{\boldsymbol{o}}	\newcommand{\bmp}{\boldsymbol{p}}
\newcommand{\bmq}{\boldsymbol{q}}	\newcommand{\bmr}{\boldsymbol{r}}
\newcommand{\bms}{\boldsymbol{s}}	\newcommand{\bmt}{\boldsymbol{t}}
\newcommand{\bmu}{\boldsymbol{u}}	\newcommand{\bmv}{\boldsymbol{v}}
\newcommand{\bmw}{\boldsymbol{w}}	\newcommand{\bmx}{\boldsymbol{x}}
\newcommand{\bmy}{\boldsymbol{y}}	\newcommand{\bmz}{\boldsymbol{z}}

%---------------------------------------
% Scr Math Fonts :-
%---------------------------------------

\newcommand{\sA}{{\mathscr{A}}}   \newcommand{\sB}{{\mathscr{B}}}
\newcommand{\sC}{{\mathscr{C}}}   \newcommand{\sD}{{\mathscr{D}}}
\newcommand{\sE}{{\mathscr{E}}}   \newcommand{\sF}{{\mathscr{F}}}
\newcommand{\sG}{{\mathscr{G}}}   \newcommand{\sH}{{\mathscr{H}}}
\newcommand{\sI}{{\mathscr{I}}}   \newcommand{\sJ}{{\mathscr{J}}}
\newcommand{\sK}{{\mathscr{K}}}   \newcommand{\sL}{{\mathscr{L}}}
\newcommand{\sM}{{\mathscr{M}}}   \newcommand{\sN}{{\mathscr{N}}}
\newcommand{\sO}{{\mathscr{O}}}   \newcommand{\sP}{{\mathscr{P}}}
\newcommand{\sQ}{{\mathscr{Q}}}   \newcommand{\sR}{{\mathscr{R}}}
\newcommand{\sS}{{\mathscr{S}}}   \newcommand{\sT}{{\mathscr{T}}}
\newcommand{\sU}{{\mathscr{U}}}   \newcommand{\sV}{{\mathscr{V}}}
\newcommand{\sW}{{\mathscr{W}}}   \newcommand{\sX}{{\mathscr{X}}}
\newcommand{\sY}{{\mathscr{Y}}}   \newcommand{\sZ}{{\mathscr{Z}}}


%---------------------------------------
% Math Fraktur Font
%---------------------------------------

%Captital Letters
\newcommand{\mfA}{\mathfrak{A}}	\newcommand{\mfB}{\mathfrak{B}}
\newcommand{\mfC}{\mathfrak{C}}	\newcommand{\mfD}{\mathfrak{D}}
\newcommand{\mfE}{\mathfrak{E}}	\newcommand{\mfF}{\mathfrak{F}}
\newcommand{\mfG}{\mathfrak{G}}	\newcommand{\mfH}{\mathfrak{H}}
\newcommand{\mfI}{\mathfrak{I}}	\newcommand{\mfJ}{\mathfrak{J}}
\newcommand{\mfK}{\mathfrak{K}}	\newcommand{\mfL}{\mathfrak{L}}
\newcommand{\mfM}{\mathfrak{M}}	\newcommand{\mfN}{\mathfrak{N}}
\newcommand{\mfO}{\mathfrak{O}}	\newcommand{\mfP}{\mathfrak{P}}
\newcommand{\mfQ}{\mathfrak{Q}}	\newcommand{\mfR}{\mathfrak{R}}
\newcommand{\mfS}{\mathfrak{S}}	\newcommand{\mfT}{\mathfrak{T}}
\newcommand{\mfU}{\mathfrak{U}}	\newcommand{\mfV}{\mathfrak{V}}
\newcommand{\mfW}{\mathfrak{W}}	\newcommand{\mfX}{\mathfrak{X}}
\newcommand{\mfY}{\mathfrak{Y}}	\newcommand{\mfZ}{\mathfrak{Z}}
%Small Letters
\newcommand{\mfa}{\mathfrak{a}}	\newcommand{\mfb}{\mathfrak{b}}
\newcommand{\mfc}{\mathfrak{c}}	\newcommand{\mfd}{\mathfrak{d}}
\newcommand{\mfe}{\mathfrak{e}}	\newcommand{\mff}{\mathfrak{f}}
\newcommand{\mfg}{\mathfrak{g}}	\newcommand{\mfh}{\mathfrak{h}}
\newcommand{\mfi}{\mathfrak{i}}	\newcommand{\mfj}{\mathfrak{j}}
\newcommand{\mfk}{\mathfrak{k}}	\newcommand{\mfl}{\mathfrak{l}}
\newcommand{\mfm}{\mathfrak{m}}	\newcommand{\mfn}{\mathfrak{n}}
\newcommand{\mfo}{\mathfrak{o}}	\newcommand{\mfp}{\mathfrak{p}}
\newcommand{\mfq}{\mathfrak{q}}	\newcommand{\mfr}{\mathfrak{r}}
\newcommand{\mfs}{\mathfrak{s}}	\newcommand{\mft}{\mathfrak{t}}
\newcommand{\mfu}{\mathfrak{u}}	\newcommand{\mfv}{\mathfrak{v}}
\newcommand{\mfw}{\mathfrak{w}}	\newcommand{\mfx}{\mathfrak{x}}
\newcommand{\mfy}{\mathfrak{y}}	\newcommand{\mfz}{\mathfrak{z}}
\usepackage{fix-cm}
\usepackage[T1]{fontenc}

\title{\Huge{21-610}\\Algebra I}
\author{\huge{Rohan Jain}}
\date{}

\begin{document}

\maketitle
\newpage% or \cleardoublepage
% \pdfbookmark[<level>]{<title>}{<dest>}
\pdfbookmark[section]{\contentsname}{toc}
\tableofcontents

\pagebreak

\chapter{}
\section{1/17 - Group Actions}

\dfn{Action}{With a group $G$ and set $X$, an \emph{action} of $G$ on $X$ is a HM from $G$ to $\Sigma_X$ (the group of permutations of $X$).}

\dfn{$g \cdot x$}{If $\phi : G \to \Sigma_X$ is an action, then for $g \in G$ and $x \in X$, we write $g \cdot x$ for $\phi(g)(x)$.}
\nt{People will eventually lose the $\cdot$. So, $g \cdot x$ will be written as $gx$.}
\ex{actions}{\begin{itemize}
    \item $1 \cdot x = x$ $(*)$
    \item $g \cdot (h \cdot x) = \phi(g)(\phi(h)(x)) = (\phi(g) \circ \phi(h))(x) = \phi(gh)(x) = (gh) \cdot x$ $(**)$
\end{itemize}}
If $\cdot : G \times X \to X$ satisfies $(*)\cdot \& (**)$, then there's unique action $\phi : G \to \Sigma_X$ such that $g \cdot x = \phi(g)(x)$.
\begin{proof}
    Define $\phi: G \to ^X X$ by $\phi(g)(x) = g \cdot x$. 

    $\phi(g^{-1})$ is 2-sided inverse of $\phi(g)$, $\phi(g) \in \Sigma_X$. So $\phi$ is an HM by $(**)$.
\end{proof}

\dfn{Orbit Equivalence Relation}{Let $G$ act on $X$. The \emph{orbit equivalence relation} on $X$ is induced by action: $x \sim y$ if $\exists g \in G$ such that $g \cdot x = y$.}
\dfn{Orbits}{The equivalence classes of this relation are called \emph{orbits}. They are defined as \[O_x = \{y : x \sim y\} = \{y : \exists g, g\cdot x = y\}\]}

\dfn{Stabilizer}{Let $G$ act on $X$. The \emph{stabilizer} of $x \in X$ is the subgroup of $G$ defined as \[G_x = \{g \in G : g \cdot x = x\}\]}
\newpage
Note that $G_x \leq G$.

\begin{proof}
    We need to show that $G_x$ is a subgroup of $G$.
    \begin{itemize}
        \item $1 \cdot x = x$, so $1 \in G_x$.
        \item $g \in G_x \implies g \cdot x = x \implies g^{-1} \cdot x = x \implies g^{-1} \in G_x$.
        \item $g,h \in G_x$. $(gh) \cdot x = g \cdot (h \cdot x) = g \cdot x = x$, so $gh \in G_x$.
    \end{itemize}
\end{proof}
A calculation:
\begin{align*}
    g_1 \cdot x = g_2 \cdot x &\iff g_2^{-1} \cdot (g_1 \cdot x) = x \\
    &\iff (g_2^{-1}g_1) \cdot x = x \\
    &\iff g_2^{-1}g_1 \in G_x \\
    &\iff g_1 \in g_2G_x \\
    &\iff g_1G_x = g_2G_x
\end{align*}
This gives a bijection between $O_x$ and set of left cosets of $G_x$. So, we have the orbit-stabilizer theorem:
\thm{Orbit-Stabilizer Theorem}{Let $G$ act on $X$. Then for all $x \in X$, $|O_x| = [G : G_x]$.}
\dfn{Fixed Point}{Let $G$ act on $X$. A \emph{fixed point} of the action is an $x \in X$ such that $g \cdot x = x$ for all $g \in G$. That is, $G_x = G$.}
\dfn{Fixed-Point Set}{Let $G$ act on $X$. Choose a $g \in G$. The \emph{fixed-point set} of $g$ is the set of all $x \in X$ such that $g \cdot x = x$ and is denoted $X_g$.}
\section{1/19 - Group Actions}
\ex{Automorphism Groups}{$$\Aut(G) = \{f : G \to G : f \text{ is an isomorphism}\}$$
$\phi \in \Sigma_G$, $\phi(ab) = \phi(a)\phi(b)$. Recall conjugate of $h$ by $g$ is $h^g = ghg^{-1}$.

Fact 1: For any $g \in G$, $h \mapsto h^g$ is an automohism of $G$.

Fact 2: If $\phi : G \to \Aut(G)$, $\phi : g \mapsto (h \mapsto h^g)$, then $\phi$ is an HM for $G$ to $\Aut(G)$.

$G$ acts on $G$ by automorphisms. $g \cdot h = h^g = ghg^{-1}$.

In this setting:
\begin{enumerate}
    \item Orbit equivalence relation is conjugacy.
    \item Orbits are conjugacy classes.
    \item For $h \in G$, the stabilizer of $h$ for conjugation action $= \{g : h^g = g\}$. \begin{align*}
        h^g = h &\iff ghg^{-1} = h \\
        &\iff gh = hg \\
        &\iff g \in C_G(h)
    \end{align*}
\end{enumerate}
}

\dfn{Centralizer}{Let $G$ act on $X$. The \emph{centralizer} of $x \in X$ is the subgroup of $G$ defined as \[C_G(x) = \{g \in G : g \cdot x = x \cdot g\}\]}

\thm{Orbit-Stablizer Equation for conjugation action}{$$|\text{conj class of }h| = [G : C_G(h)]$$}
$$|G| = \sum_{C \text{ conj. class} }|C|$$
So, if $C = $ class of $h$, $|C| = [G : C_G(h)]$.

Recall the definition of a fixed-point. So, for $G$ acting on $G$ by conjugation, $X_g = C_G(g)$. That is,
\begin{align*}
    h \text{ fixed point} \iff h^g = h \iff hg = gh \tag{for all $g$}
\end{align*}
\dfn{Center}{The \emph{center} of $G$ is $Z(G) = \{g \in G : gh = hg \text{ for all } h \in G\}$. In fact, $Z(G)$ is normal in $G$. That is, $Z(G) \lhd G$.}
\thm{}{Let $p$ be prime. Let $G$ be a group of order $p^n$. Then $Z(G) \neq 1$.}
\begin{proof}
    Let $G$ act on $G$ by conjugation. $G$ is partitioned into orbits (i.e. conjugacy classes). 

    For any $h$, we know that the size of the class of $h$ is $[G : C_G(h)]= \dfrac{p^n}{|C_G(h)|}$. Each orbit has size $1$ or a power of $p$. So, $|C_G(h)|$ is a power of $p$.

    Note in any action of $G$ onto $X$, $x$ being a fixed point implies $O_x = \{x\}$. So, $|O_x| = 1$.

    So, $|G| = A + B$ where $A$ is the number of orbits of size 1 and $B = \sum |C|$ where $C$ is a conjugacy class of size $p^n$ for $n > 0$. 

    So, $A = p^n - B$. So $p | A$. As $Z(G) \neq \emptyset$, $|Z(G)| > 0, p | |Z(G)|$. So, $|Z(G)| \geq p$, which is at least 2, so $Z(G) \neq 1$.
\end{proof}
\thm{Cauchy's Theorem}{Let $G$ be a finite group. If $p$ is a prime dividing $|G|$, then $G$ has an element or subgroup of order $p$.}

Facts to remember from undergraduate group theory:
\begin{itemize}
    \item Let $N \lhd G$. Then subgroups of $G/N$ are in bijection with $\{H : N \leq H \leq G\}$. In fact $H \mapsto H/N$ is a bijection.
    \item Normal subgroups of $G/N$ are uniquely of the form $H/N$ where $H \lhd G$ and $N \leq H$.
    \item $H/N \lhd G/N$ , $\frac{G/N}{H/N} \cong G/H$.
\end{itemize}
\newpage
\section{1/22 - Using Group Actions to Prove Theorems}
Now we prove Cauchy's Theorem:
\begin{proof}
    Let $X = \{ (g_1, \ldots, g_p) \in G^p : g_1 \cdots g_p = 1\}$. Some remarks:
    \begin{itemize}
        \item $(g_1, \ldots, g_p) \in X \iff (g_1\ldots g_{p-1})g_p = 1$. So, $g_p = (g_1\ldots g_{p-1})^{-1}$ and $(g_p, g_1, \ldots, g_{p-1}) \in X$. So, $|X| = |G|^{p-1}$.
        \item $X \neq \emptyset$ as $(1, \ldots, 1) \in X$.
    \end{itemize}
    So now it's easy to define an action of $C_p$(cyclic group of order $p$) on $X$. Explicitly, if $C_p = \langle a \rangle>$, then $a \cdot (g_1, \ldots, g_p) = (g_2, \ldots, g_p, g_1)$. 

    Now we analyze the fixed-points. $(g_1, \ldots, g_p)$ if and only if all the $g_i$ are equal. So, fixed points in the action of $C_p$ on $X$ are $(g, \ldots, g) \in X$ where $g^p = 1$. 

    As $p \mid |G|$, $p \mid |X| = |G|^{p-1}$. As $p$ is prime, $|C_p| = p$. So all orbits have size 1 or $p$. $X$ is partitioned into orbits, say $$|X| = C + D_p$$
    So $p | C$ where $C$ is the number of fixed-points for this action. As $(1, \ldots, 1)$ is a fixed-point, $C>0$, so $C \geq p > 1$. So, there is a fixed-point $(g, \ldots, g) \in X$ where $g^p = 1$. So, $g$ has order $p$.
\end{proof}
\dfn{$\Syl_p(G)$}{$\Syl_p(G) = \{H : H \leq G, |H| = p^k \text{ for some } k \geq 1 \text{ for largest } k\}$}
\nt{If $p \nmid |G|$, then $\Syl_p(G) = \{1\}$.}
\thm{Sylow's Theorem}{Let $G$ be a finite group. Let $p$ be a prime dividing $|G|$. \begin{enumerate}
    \item If $H \leq G$ and $|H|$ is a power of $p$, there is $K \in \Syl_p(G)$ such that $H \leq K$.
    \item If $K_1, K_2 \in \Syl_p(G)$, then $K_1$ and $K_2$ are conjugate.
    \item $|\Syl_p(G)| \equiv 1 \pmod{p}$ and divides $|G|$.
\end{enumerate}}
\noindent Notes before proof:
\begin{itemize}
    \item Let $G$ be a group. $\alpha \in \Aut(G), H\leq G$. Then $\alpha [H] = \{\alpha(h) : h \in H\}$ is a subgroup of $G$  and $\alpha[H] \cong H$. $\alpha$ is a bijection from $H$ to $\alpha[H]$. 
    \item In particular, for $g \in G$, if $\alpha$ is ``conjugation by $g$'', then $\alpha[H] = gHg^{-1}$ or $H$ goes to $H^g$. We can check: $G$ acts on $\{H : H \leq G\}$. $g\cdot H = H^g = gHg^{-1}$.
    \item $H$ is a fixed point of this action if and only if $H^g = H$ for all $g \in G$. That is, $H \lhd G$.
    \item For any $H$, stabilizer of $H$ for this action is $N_G(H) = \{g \in G : gHg^{-1} = H\}$.
\end{itemize}
\dfn{Normalizer}{Let $H \leq G$. The \emph{normalizer} of $H$ in $G$ is $N_G(H) = \{g \in G : gHg^{-1} = H\}$.}
\begin{itemize}
    \item Let $G$ act on $X$. Then we know that if $Y \subseteq X$, and $g \cdot y \in Y$ for all $g \in G$ and $y \in Y$, then $Y$ is a union of orbits and we then get an action for $G$ onto $Y$.
    \item Let $G$ act on $X$. Let $H \leq G$, now easily $H$ acts on $X$. Each $G$-orbit breaks up as a union of $H$-orbits.
\end{itemize}
\newpage
\section{1/26 - Using Group Actions to Prove Theorems}
Now we finally prove Sylow's.
\begin{proof}
    
\end{proof}

\section{1/29 - Series in Groups}
\dfn{Subnormal Series}{A \emph{subnormal series} for a group $G$ is a sequence $(G_i)$ of subgroups \[1  = G_0 \lhd G_1 \lhd \cdots \lhd G_n = G\]}
\dfn{Normal Series}{A subnormal series is a \emph{normal series} if $G_i \lhd G$ for all $i$.}
\dfn{Characteristic Series}{A \emph{characteristic series} is a normal series $(G_i)$ such that $G_i \text{char} G$ for all $i$.}
\dfn{Commutator}{Let $G$ be a group. The \emph{commutator} of $g,h \in G$ is $[g,h] = ghg^{-1}h^{-1}$.}
\nt{$[g,h] = 1$ if and only if $gh = hg$.}
\dfn{Commutator Subgroups}{The \emph{commutator subgroup} of $G$ is $[G,G] = \langle [g,h] : g,h \in G\rangle$.}
\nt{If $\alpha \in \Aut(G)$, then $\alpha([g,h]) = [\alpha(g), \alpha(h)]$. So $\alpha([G,G]) = [G,G]$. So $[G,G] \lhd G$.}
\noindent What does it mean for $G/N$ to be abelian? We get:
\begin{align*}
    &[aN, bN] = 1 \forall a ,b \\
    &\iff [a,b]N = 1 \\
    &\iff [a,b] \in N \\
    &\iff [G,G] \leq N
\end{align*}
So $[G, G]$ is the least normal subgroup of $G$ such that $G/N$ is abelian.
\dfn{Solvable}{A group $G$ is \emph{solvable} iff there is a subnormal series $(G_i)$ in $G$ such that $G_{i+1}/G_i$ is abelian for all $i < n$.}
Trivially, abelian groups are solvable. But there are non-abelian solvable groups. For example, $S_3$ is solvable. This is because $1 \lhd A_3 \lhd S_3$. $S_3/A_3 \cong C_2$ is abelian.
\newpage
\thm{}{If $G$ is solvable and $H \leq G$, then $H$ is solvable.}
\begin{proof}
    Let $(G_i)$ be a subnormal series for $G$ such that $[G_{i+1}, G_{i+1}] \leq G_i$ for all $i < n$. Then $(H \cap G_i)$ is a subnormal series for $H$ such that $[H \cap G_{i+1}, H \cap G_{i+1}] \leq H \cap G_i$ for all $i < n$.
\end{proof}

\thm{}{If $G$ is solvable and $N \lhd G$, then $G/N$ is solvable.}
\begin{proof}
    Note: If we have $N \lhd G$ and $H \leq G$ and $\phi_N : G \to G/N$ is the natural homomorphism, then $\phi_N(H) \leq  = HN/N \leq G/N$.

    Let $(G_i)$ be a subnormal series for $G$ such that $[G_{i+1}, G_{i+1}] \leq G_i$ for all $i < n$. Then $(G_iN/N)$ is a subnormal series for $G/N$ such that $[G_{i+1}N/N, G_{i+1}N/N] \leq G_iN/N$ for all $i < n$.
\end{proof}

\section{1/31 - }
\thm{}{If $N \lhd G$ with $N$ and $G/N$ both solvable, then $G$ is solvable.}
\begin{proof}
    Let $(N_i)$ be a subnormal series for $N$ such that $[N_{i+1}, N_{i+1}] \leq N_i$ for all $i < n$. Let $(V_i)$ be a subnormal series for $G/N$ such that $[V_{i+1}, V_{i+1}] \leq V_i$ for all $i < n$.

    For each $j$, let $G_j$ be the unique subgroup of $G$ such that $N \leq G_j$ and $G_j/N = V_j$. 

    What we want is that $N_0, N_1, \ldots, N_m = N = G_0, G_1, \ldots, G_m = G$ is a subnormal series for $G$ such that $[G_{i+1}, G_{i+1}] \leq G_i$ for all $i < n$.

    Since $V_j \lhd V_{j+1}$, run time to check that $G_j \lhd G_{j+1}$. We know that $N_{j+1} / N_j$ and $V_{j+1} / V_j$ are abelian. So we have that $V_{j+1} / V_j = G_{j+1} / N / G_j / N \cong G_{j+1} / G_j$ is abelian. So $G_{j+1} / G_j$ is abelian. So $G$ is solvable.
\end{proof}

\dfn{Derived Series}{Let $G$ be a group. Then the \emph{derived series} of $G$ is a sequence of subgroups of $G$ defined as \[G^{(0)} = G, G^{(1)} = [G,G], G^{(2)} = [G^{(1)}, G^{(1)}], \ldots, G^{(n+1)} = [G^{(n)}, G^{(n)}]\]}

\nt{$G^{(n)} / G^{(n+1)}$ is abelian. In fact, $G^{(n+1)}$ is the best normal subgroup of $G^{(n)}$ such that $G^{(n)} / G^{(n+1)}$ is abelian.}
\nt{$G^{(n)} \text{char} G$ for all $n$.}
\mprop{}{If there is $n$ with $G^{(n)} = 1$, then $G$ is solvable.}
\begin{proof}
    $G^{(n)} = 1 \implies G^{(n-1)}$ is abelian. So $G^{(n-1)} / G^{(n)} = G^{(n-1)}$ is abelian. So $G$ is solvable.
\end{proof}
\newpage
\mprop{}{If $G$ is solvable, then there is $n$ with $G^{(n)} = 1$.}
\begin{proof}
    Let $(G_i)$ be a subnormal series for $G$ such that $[G_{i+1}, G_{i+1}] \leq G_i$ for all $i < n$. We show by induction on $s$ that $G^{(s)} \leq G_{n-s}$ for $0 \leq s \leq n$. 

    If this works, then $G^{(n)} \leq G_0 = 1$. So $G^{(n)} = 1$.

    \textbf{Base Case:} $s = 0$. $G^{(0)} = G \leq G_n = G$.

    Now suppose $s < n$ and $G^{(s)} \leq G_{n-s}$. Then $G^{(s+1)} = [G^{(s)}, G^{(s)}] \leq [G_{n-s}, G_{n-s}] \leq G_{n-s-1}$.

    So $G^{(n)} = 1$.
\end{proof}

\dfn{Simple}{$G$ is \emph{simple} if $G\neq 1$ and $N \lhd G$ implies $N = 1$ or $N = G$.}

\cor{}{If $G$ is abelian and simple, then $G$ is cyclic of prime order.}

\thm{}{For $n \geq 5$, $A_n$ is simple.}
\thm{}{If $G$ has a nonabelian simple subgroup, then $G$ is not solvable.}
\begin{proof}
    \begin{enumerate}[label=(\alph*)]
        \item If $K$ is simple and nonabelian, $K$ is definitely not solvable.
        \item Subgroups of solvable groups are solvable.
    \end{enumerate}
\end{proof}

\section{2/2 - (absent)}
\dfn{Nilpotence}{A group $G$ is \emph{nilpotent} if there is a subnormal series $(G_i)$ for $G$ such that $G_{i+1} \lhd G_i / G_{i+1}$ for all $i$.}

\section{2/5 - Nilpotency}
\dfn{Central Series}{A \emph{central series} for a group $G$ is a sequence $(G_i)$ of subgroups \[1  = G_0 \lhd G_1 \lhd \cdots \lhd G_n = G\] such that $G_i / G_{i-1} \leq Z(G/G_{i-1})$ for all $i$.}
Note that $G \text{ nilpotent} \iff G$ has a central series.
\dfn{Descending Central Series}{A \emph{descending central series} for a group $G$ is a sequence $(G_i)$ of subgroups \[G = G_0 \triangleright G_1 \triangleright \cdots \triangleright G_n \triangleright \cdots\] where for each $n$, $G_{n+1} = [G_n, G]$. In other words, $L_1(G) = G$, $L_{n+1}(G) = [G, L_n(G)]$.}
Note that $G \text{ nilpotent} \iff G$ has a descending central series which has $n$ such that $L_n(G) = 1$.
\dfn{Ascending Central Series}{An \emph{ascending central series} for a group $G$ is a sequence $(G_i)$ of subgroups \[1 = Z_0 \lhd Z_1 \lhd \cdots \lhd Z_n \lhd \cdots\] where each successive group is defined by $Z_{n+1} = \{g \in G : \forall y \in G, [x, y] \leq Z_i\}$. In other words, $U_0(G) = 1$, $U_{n+1}(G) / U_n(G) = Z(G/U_n(G))$.}
Note that $G \text{ nilpotent} \iff G$ has an ascending central series which has $n$ such that $U_n(G) = G$.

For $G$ nilpotent:

The number of steps it takes the acs to reach $G$ is the same number of steps the dcs takes to reach $1$, which are both the least length of a central series.

\dfn{Nilpotency Degree}{If $G$ nilpotent, $G$ has \emph{nilpotency degree} $t$ if either series takes $t$ steps.}
\ex{Nilpotency Degrees}{
    \begin{itemize}
        \item Degree 0: $G = 1$.
        \item Degree 1: $G$ abelian, $G \neq 1$. 
    \end{itemize}
}
Fact (``normalizer property''):
\begin{itemize}
    \item $G$ nilpotent and $H < G \implies N_G(H) > H$.
\end{itemize}
\begin{proof}
    We induct on nilpotence degree:

    \textbf{Base Case:} $t = 0$ or $t=1$. $G$ abelian. $H < G$. $N_G(H) = G > H$.

    Now suppose the nilpotenecy degree of $G = t +1$ and we have the normalizer property for all groups of nilpotency degree $\leq t$. Let $H < G$. We proceed with case analysis involving the center. We know that $Z(G) \leq N_G(H)$, so:
    \begin{itemize}
        \item Case 1: If $Z(G) \not\leq H$, then $H \neq N_G(H)$, done. 
        \item Case 2: If $Z(G) \leq H$, consider the acs. We have that $U_0 = 1, U_1 = Z(G)$. We take for granted that the nilpotency degree of $G/Z(G) = t$. So consider the acs of $G/Z(G)$, where $U_0 = Z(G)/Z(G) = 1$, $U_1 = G/Z(G)$. So, $Z(G) \leq H < G \implies H/Z(G) < G/Z(G)$ and $G/Z(G)$ has nilpotency degree $t$. So $N_{G/Z(G)}(H/Z(G)) > H/Z(G)$. So $N_G(H) > H$.
    \end{itemize}
    
\end{proof}

\thm{}{If $G_1, G_2$ are niltpotent, then $G_1 \times G_2$ is nilpotent.}
Recall that if $M,N \lhd G$ and $M \cap N = 1$, then $mn = nm$ for all $m \in M$ and $n \in N$, then $MN \cong M \times N$ and $MN \lhd G$. This generalizes for more than 2 subgroups.

Also recall if $G$ finite and $H \in \Syl_p(G)$, then $N_G(H)  = N_G(N_G(H))$.
\begin{proof}
    $H \lhd N_G(H)$, so $H$ is the unique Sylow $p$-subgroup of $N_G(H)$. So now if $g \in N_G(N_G(H))$, then $N_G(H)^g = N_G(H)$, so $H^g$ is the unique Sylow subgroup of $N_G(H)^g = H$, so $g \in N_G(H)$.
\end{proof}

Now let $G$ be finite and nilpotent. Let $|G| = p_1^{t_1}p_2^{t_2}\ldots p_k^{t_k}$. We claim that for each $i$, $G$ has a unique normal Sylow $p_i$-subgroup, $P_i$.
\begin{proof}
    Let $P_i \in \Syl_{p_i}(G)$. Then $N_G(P_i) = N_G(N_G(P_i))$. By the normalizer property, $N_G(P_i) = G$. So $P_i \lhd G$ and $P_i$ is the unique Sylow $p_i$-subgroup of $G$.
\end{proof}
Then by Lagrange, $P_i \cap P_j = 1$ for $i \neq j$. So $P_1P_2\ldots P_k \lhd G$ and $P_1 \times P_2 \times \ldots \times P_k \cong P_1P_2\ldots P_k$. So $P_1P_2 \ldots P_k = G \cong P_1 \times P_2 \times \ldots \times P_k$.

\section{2/7 - Jordan Holder}
\thm{Jordan-Holder}{Let $G$ be a group. Let $(H_i)_{0 \leq i \leq n}$ and $(G_j)_{0 \leq j \leq m}$ be two composition series of $G$. Then $n = m$ and there is a permutation $\sigma$ of $\{0, 1, \ldots, n\}$ such that $H_i / H_{i-1} \cong G_{\sigma(i)} / G_{\sigma(i-1)}$ for all $i$.}
\begin{proof}
    If $G = 1$ or if $G$ is simple, then the result is trivial. So assume $G$ is not simple or trivial so that $m, n > 1$.

    Let $H = H_{n-1}$, so $1 < H \lhd G$. $G/H$ is simple and $(H_i)_{0 \leq i \leq n-1}$ is a composition series for $H$.

    Let $k$ be least $k$ such taht $G_k \not\leq H$. Note that $0 < k < m$. ($G_0 = 1 \leq H, G_m = G \not\leq H$.) Form the subnormal series $(G_iH/H)_{0 \leq j \leq m} \in G/H$. As $G/H$ is simple, this series is a series of $H/H$'s followed by series of $G/H$'s. 

    For $j<k$, $G_j \leq H, G_jH = H \Rightarrow G_jH/H = H/H = 1$.

    So for $j \geq k$, $G_j H/H = G/H$, so $G = G_jH$. 

    For $j \geq k$, $G/H = G_jH/H \cong G_j/(G_j \cap H)$.

    So $(G/H$ simple), $G_j / (G_j \cap H)$ simple for $j \geq k$. 

    Recall $(H_i)_{0 \leq i \leq n-1}$ is for $h$ with $n-1$ steps, so our IH applies to $H$. 

    Define subnormal series in $H$ that is $(G_j \cap H)_{0 \leq j \leq m}$.

    We want to argue two things:
    \begin{enumerate}[label=(\alph*)]
        \item Deleting one repetition in $(G_j \cap H)_{0 \leq j \leq m}$, we obtain a composition series for $H$.
        \item By IH applied to $H$, $m-1 = n-1$ so $m=n$. Argue that we can find IM's between quotients of $G_j$ and $H_j$.
    \end{enumerate}
    Note that for $j < k, G_j \leq H$, so $G_j = G_j \cap H$. 

    Let $j > k$. $G_{j-1} \lhd G_j, G_j \cap H \lhd G_j$. Then we have that $G_{j_1}(G_j \cap H) = G_{j}$. This yields:
    \begin{align*}
        \frac{G_{j-1}(G_j \cap H)}{(G_j \cap H)} &\lhd \frac{G_j}{G_j \cap H} \tag{this RHS is simple} \\
    \end{align*}
    So either $G_{j-1}(G_j \cap H) = G_j$ or $G_{j-1}(G_j \cap H) = G_j \cap H$. However, since $j >k$, we have that $G_{j-1} \not\leq H$, so $G_{j-1} (G_j \cap H) = G_j$. So:
    \begin{align*}
        \frac{G_j}{G_{j-1}} = \frac{G_{j-1}(G_j \cap H)}{G_{j-1}} \\
        \cong \frac{G_j \cap H}{(G_{j} \cap H)\cap G_{j-1}} = \frac{G_j \cap H}{G_{j-1} \cap H}
    \end{align*}
    To finish, we have that $G_{k-1} = G_{k-1} \cap H = G_k \cap H$. We also want to show that $G/H \cong G_k / G_{k-1}$. From $G_{k-1} = G_k \cap H$, $G_k / G_{k-1} \cong G_k / (G_k \cap H) \cong G/H$, by fact already proved. 

    If $G$ has a composition series with $n$ steps, then a $U$ composition series have $n$ steps and ``same quotients''.

    $G \neq 1$, G not simple. So $n \geq 2$. Fix a compostion series $(H_i)_{0 \leq i \leq n}$ so that $H = H_{n-1} \leq G$. 

    Get another composition series $(G_j)_{0 \leq j \leq m}$. Choose $k$ such that $G_k \not\leq H$. Recall that $G_j \cap H = G_j$ for $j < k$ and that for $j>k$, $\frac{G_j \cap H}{G_{j-1} \cap H} \cong \frac{G_j}{G_{j-1}}$.

    We are left to prove that $G_{k-1} = G_k \cap H$ and $\frac{G_k}{G_{k-1}} \cong \frac{G}{H}$. 

    Assessing this, $G_{k-1} \cap H = G_k \cap H$ and seq$(G_j \cap H : j \neq k, 0 \leq j \leq m)$ is a composition series for $H$ with $m-1$ steps with quotients $\frac{G_j}{G_{j-1}}$ for $0 < j  \leq m, j \neq k$. 

    We have 2 composition series for $h$ with 
    \begin{align*}
        (H_i)_{0 \leq i \leq n-1} &\text{ and } (G_j \cap H)_{0 \leq j \leq m-1} \\
    \end{align*}
    By the inductive hypothesis, $m-1 = n-1$, so $m = n$ and up to isomorphism, we can match the quotients of the two series. 
\end{proof}

\section{2/9 - Free Groups}
Recall that $G$ is cyclic if $G = \langle g \rangle$ for some $g \in G$. In this case, there is a unique HM $\phi : (\ZZ, +) \to G$ such that $\phi(1) = g$. Clearly $\phi$ is surjective, so $G \cong \ZZ / \ker \phi$. So $G$ is abelian.

If $\ker(\phi) = 0 = 0 \ZZ$, then $|G| = \infty$, $\phi$ is an isomorphism from $\ZZ$ to $G$. So there is a unique $n > 0$ such that $\ker(\phi) = n \ZZ$ and $|g| = n$. $\phi$ induces an IM from $\ZZ / n \ZZ$ to $G$.

Let's talk about groups that are generated by two elements $G = \langle a, b \rangle$. 
\begin{enumerate}
    \item Any symmetric group $S_n$ for $n \geq 2$ is generated by two elements.
    \begin{align*}
        S_n = \langle (12), (1 \ldots n)\rangle
    \end{align*}
    \item Any dihedral group $D_k$ for finite $k$. 
\end{enumerate}
Our goal is to find a group $F = \langle a, b \rangle$ such that for all $G$ and all $g,h \in G$, there is a unique HM $\phi : F \to G$ such that $\phi(a) = g$ and $\phi(b) = h$. 

We start by trying to construct $F$ using words. Let $\Sigma = \{ a, b\}$. A \emph{word} in $\Sigma$ is a finite sequence of elements of $\Sigma \times \ZZ$. 
\ex{Words}{\begin{itemize}
    \item Write $(a, 3)(b, 5)(a, 2)(b, 1)$ as $a^3b^5a^2b$.
\end{itemize}}
\noindent A word is reduced if
\begin{itemize}
    \item No pairs of form $(s, 0)$ for $s \in \Sigma$
    \item No successive pairs of form $(s, i)(s, j)$ for $s \in \Sigma$.
\end{itemize}
\dfn{Reduction}{Remove terms of the form $s^0$ and replace successive terms of the form $s^is^j$ with $s^{i+j}$.}

\section{2/12 - More Alphabet Stuff}
Let $\sigma$ be a set of any symbols. Words are finite sequences of elements of $\sigma \times \ZZ$. A word is reduced if it has no terms of the form $(s, 0)$ and no successive terms of the form $(s, i)(s, j)$.

Let $W$ be the set of all reduced words in $\sigma^*$. We want to associated to each $s \in \Sigma$, a permutation $\pi_s$ of $W$. The idea is that $\pi_s(w)$ is a unique reduction of $sw$. 

$\pi_s(w)=sw$ makes $w$ start with a term $s^i$ for some $i \in \ZZ$.

If $w = s^i \ldots$ and $i \neq -1$, then $\pi_s(w) = s^{i+1} \ldots$.

If $w = s^{-1}\ldots$, then $\pi_s(w) = \ldots$.

Define $\rho_s$, whose intuition is to reduce $s^{-1}w$. 

$\rho_s(w) = s^{-1}w$ unless $w$ starts with a power of $s$. 

If $w = s^i \ldots$ and $i \neq 1$, then $\rho_s(w) = s^{i-1} \ldots$.

If $w = s \ldots$, then $\rho_s(w) = \ldots$.

One can check that $\pi_s \circ \rho_s = \rho_s \circ \pi_s = \id_W$. As such, $\pi_s$ is a permutation and $\rho_s = \pi_s^{-1}$.

Let $G$ be the subgroup of $\Sigma_w$ (the group of permutations of $w$) generated by $\{\pi_s : s \in \Sigma\}$. If $w$ is a word (not necessarily reduced), say $w = s_1^{i_1}s_2^{i_2}\ldots s_n^{i_n}$, then $\pi_{w} = \pi_{s_1}^{i_1}\pi_{s_2}^{i_2}\ldots \pi_{s_n}^{i_n} \in G$. 

Key fact 1: If $w'$ is obtained from $w$ by a single reduction step, then $\pi_{w'} = \pi_{w}$.

Key fact 2: $G = \{\pi_w : w \in W\}$.
\begin{proof}
    from key fact 1. 
\end{proof}
Key fact 3: If $w \in W$, then $\pi_w(\epsilon) = w$. 

So every $g \in G$ is of form $\pi_w$ for unique reduced word $w$, where $w = g(\epsilon)$. 

Key fact 4: For any words $v, w$, $\pi_v \circ \pi_w = \pi_{vw}$.

Key fact 5: if $w$ is a word and reduced $v$ is obtained from $w$ by some reduction sequence, then $\pi_v(\epsilon) = \pi_w(\epsilon) = v \Rightarrow$ $v$ is determined by $w$. 

Key fact 6: If $v,w \in W$ and $u$ is the unique redcuced word obatined by reducing $vw$, then $\pi_u = \pi_v \circ \pi_w$.

Conclusion: $G$ is isomorphic to the group of whose underlying set is $W$ and whose group operation is ``concatenate and reduce''.

Key point: $G_\Sigma$ has the following ``universal property'': For any group $H$ and any function $\phi : \Sigma \to H$, there is a unique HM $\Phi : G \to H$ such that $\Phi \circ \pi_s = \phi(s)$ for all $s \in \Sigma$.

\dfn{Free Group}{The \emph{free group} on $\Sigma$ is the group $G_\Sigma$ with the universal property.}

\section{2/14 - Free Groups}
Key property of $Free(\Sigma)$ is called the ``universal property.'' Basically, for all groups $H$ and functions $f : \sigma \to H$, there is unique HM $\alpha : Free(\Sigma) \to H$ such that $\alpha(s') = f(s)'$ for all $s \in \Sigma$.

Let's proceed with a proof of the universal property.
\begin{proof}
    \begin{enumerate}
        \item If such an $\alpha$ exists, then it must be given by the formula 
        $$\alpha(s_1^{i_1}s_2^{i_2}\ldots s_n^{i_n}) = f(s_1)^{i_1}f(s_2)^{i_2}\ldots f(s_n)^{i_n}$$
        for all reduced words $s_1^{i_1}s_2^{i_2}\ldots s_n^{i_n}$.
        \item Need to verify that function $\alpha$ specified by $(x)$ is a group HM.
        
        Let $v= s_1^{i_1}s_2^{i_2}\ldots s_n^{i_n}$ and $w = t_1^{j_1}t_2^{j_2}\ldots t_m^{j_m}$ be reduced words. Then $vw$ is a reduced word and $\alpha(vw) = \alpha(v)\alpha(w)$.
    \end{enumerate}
\end{proof}

\section{2/19 - Rings}
What do we mean when we say $R$ is a ring? For now, they will be commutative rings with unity. 

\dfn{Ring Homomorphism}{Let $R$ and $S$ be rings. Then $\phi: R \to S$ is a \emph{ring homomorphism} iff:
\begin{align*}
    \phi(a+b) &= \phi(a) + \phi(b) \\
    \phi(ab) &= \phi(a)\phi(b) \\
    \phi(1_R) &= 1_S
\end{align*}
for all $a,b \in R$.}
\dfn{Subring}{Let $R \subseteq S$ be a subring. This means that the inclusion map $\iota : R \to S$ is a ring homomorphism.}
\dfn{Ideal}{$I \subseteq R$ is an \emph{ideal} iff $I \leq (R, +)$ and $tr \in I$ for all $t \in R$ and $r \in I$. }
\nt{If $\phi: R \to S$ is a ring homomorphism, then $\ker(\phi) = \{r \in R : \phi(r) = 0\}$ is an ideal of $R$.

Also, if $I$ is an ideal, then $I = R$ iff $1 \in I$. 

$\{0\}$ and $R$ are always ideals. `zero ideal'}

\dfn{Quotient Ring}{Let $I \lhd R$. Then $R/I = \{r + I : r \in R\}$ is the \emph{quotient ring} of $R$ by $I$.}
\nt{$(s + I)(t + I) = st + I$. $\phi_I : R \to R/I$ is a quotient homormophism. $\phi_I(r) = r + I$. $\ker(\phi_I) = I$.}

\dfn{$R$-module}{Let $R$ be a ring. $M$ is an \emph{$R$-module} if $M$ is equipped with operations $+: M \times M \to M$ and scalar multiplication$: R \times M \to M$ with the following properties:
\begin{itemize}
    \item $(M, +)$ is an abelian group.
    \item $r(m+n) = rm + rn$ for all $r \in R$ and $m,n \in M$.
    \item $(r+s)m = rm + sm$ for all $r,s \in R$ and $m \in M$.
    \item $(rs)m = r(sm)$ for all $r,s \in R$ and $m \in M$.
    \item $1m = m$ for all $m \in M$.
    \item $0m = 0$ for all $m \in M$.
\end{itemize}}

\ex{}{\begin{enumerate}
    \item If $R$ is a field, $R$-modules are just vector spaces over $R$. 
    \item Let $R = \ZZ$. We claim that $\ZZ$-modules are just abelian groups. Let $G$ be a $\ZZ$-module. Just forget scalar multiplication to get an abelian group.
    
    Converse: Let $(G, +)$ be an abelian group. We have that $ 0g = 0$, $1g = g, (n+1)g = ng + g$, and $(-n)g = 0 - (ng)$. 
    \item Let $K$ be a field, $R = K[x]$. Let $M$ be a $K[x]$-module. 
    \begin{enumerate}
        \item $K$ is a subring of $K[x]$. So $M$ is a $K$-module that is a vector space over $K$.
        \item Let $T: M \to M$ be given by $T(m) = xm$. Then $T$ is a $K$-linear map. Note that we can recover $K[x]$-module structure from $K$-vector space structure and $T$. Namely, $(a_0 + a_1x + \ldots + a_nx^n)m = a_0m + a_1T(m) + \ldots + a_nT^n(m)$.
    \end{enumerate}
    \item For any ring $R$, $R$ is an $R$-module.
    \item Let $M, N$ be $R$-modules. $M \dotplus N = \{(m,n) : m \in M, n \in N\}$ is an $R$-module with coordinatewise operations.
    \item Let $M, N$ be $R$-modules. $Hom_R(M,N) = \{f : M \to N : f \text{ is an $R$-module HM}\}$ is an $R$-module that is $R$ linear. 
\end{enumerate}}
\dfn{Submodule}{If $N$ is an $R$-module then $M \subseteq N$ is a \emph{submodule} iff $M$ is a subgroup of $(N, +)$ and $rm \in M$ for all $r \in R$ and $m \in M$.}
\ex{}{
    \begin{enumerate}
        \item Submodules of vector spaces over $R$ are just subspaces.
        \item In abelian groups, submodules are just subgroups.
        \item If $R$ is a ringe viewed as an $R$-module, then submodules are just ideals.
        \item If $V$ is a vector space over $K$ with $T: V \to V$ linear, view $V$ as a $K[x]$-module. Then submodules of $V$ are just $T$-invariant subspaces.
    \end{enumerate}
}
\dfn{Quotient Module}{If $M \leq N$, $N/M$ is defined in the obvious way. $N/M = \{n + M : n \in N\}$.}
$\phi_M : N \to N/M$ is a quotient homomorphism. 

Also, if $T: M \to N$ is a linear homomorphism, then $\ker(T) = \{m \in M : T(m) = 0\}$. also, $im(T) \cong M / \ker(T)$.
\dfn{Cokernel}{The \emph{cokernel} of $T$ is $N / im(T)$.}
\newpage
\section{Some Random Category Stuff}
% define \span as \operataorname{span}
\def\span{\operatorname{span}}
\dfn{Finitely Generated}{An $R$-module $M$ is \emph{finitely generated} if there is $X \subseteq M$ finite such that $M = \span(X)$.}
\dfn{Noetherian}{An $R$-module $M$ is \emph{Noetherian} if every submodule of $M$ is finitely generated.}
\dfn{Noetherian Ring}{A ring $R$ is \emph{Noetherian} iff $R$ is Noetherian as an $R$-module.}
\ex{}{PID's are Noetherian rings.}
\nt{For any ring $R$, $R$ is the $R$-span of $\{1\}$. So $R$ is finitely generated.}
\ex{Non-Noetherian Ring}{Let $R = \mathbb{Z}[x_1, x_2, \ldots]$ be the ring of polynomials in infinitely many variables. Then $R$ is not Noetherian.

We have that $I = (x_0, x_1, \cdots)$, which is $p \in R$ such that the constant term of $p$ is 0. Supposed by contradiction that $I$ is finitely generated. Say that $I = R$-span of $(p_0, \cdots, p_{n-1})$. That is, $I = (\sum_{i=0}^{n-1} a_i p_i : a_i \in R)$. Then $x_n \in I$, so $x_n = \sum_{i=0}^{n-1} a_i p_i$ for some $a_i \in R$. Let $R$ be so large that $x_k$ does not appear in any $p_j$. $x_k \in I$, so $x_k = \sum_{i=0}^{n-1} a_i p_i$. 

Consider assigning values to $x$ like:
\begin{itemize}
    \item $x_k \to 1$
    \item $x_i \to 0$ for $i \neq k$
\end{itemize}
This gives $1 = \sum_{i=1}^{n-1} a_i  0 = 0$, which is a contradiction.
}
\thm{}{Let $M$ be an $R$-module. Then the following are equivalent:
\begin{enumerate}
    \item $M$ is Noetherian.
    \item Whenever $(N_i)$ such that $N_i \leq N_{i+1}$ and $N_i \leq M$, $N_i$ is eventually constant.
    \item Any nonempty famiy of submodules of $M$ has a maximal element.
\end{enumerate}}
\begin{proof}
    \begin{itemize}
        \item $1 \implies 2$. Let $(N_i)$ be a chain of submodules of $M$. Then $N = \bigcup_{i=1}^\infty N_i$ is a submodule of $M$. $N$ is finitely generated, so fix $X \subseteq N$ where $X$ is finite. $N = \span(X)$, find $i$ such that $X \subseteq N_i$. Now $N_i \subseteq N$, so $N = \span(X) \subseteq N_i$. So $N_i = N_j = N$ for $j \geq i$.
        \item $\neg 1 \implies \neg 2$. 
    \end{itemize}
\end{proof}
\end{document}
