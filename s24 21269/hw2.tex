%% This document created by Scientific Word (R) Version 3.5

\documentclass{article}%
\usepackage{amsmath}
\usepackage{graphicx}
\usepackage{amsfonts}
\usepackage{amssymb}%
\setcounter{MaxMatrixCols}{30}
%TCIDATA{OutputFilter=latex2.dll}
%TCIDATA{Version=5.50.0.2960}
%TCIDATA{CSTFile=LaTeX article (bright).cst}
%TCIDATA{Created=Friday, August 29, 2008 15:08:24}
%TCIDATA{LastRevised=Sunday, January 28, 2024 07:18:29}
%TCIDATA{<META NAME="GraphicsSave" CONTENT="32">}
%TCIDATA{<META NAME="SaveForMode" CONTENT="1">}
%TCIDATA{BibliographyScheme=Manual}
%TCIDATA{<META NAME="DocumentShell" CONTENT="General\SW\Blank - Standard LaTeX Article">}
%BeginMSIPreambleData
\providecommand{\U}[1]{\protect\rule{.1in}{.1in}}
%EndMSIPreambleData
\providecommand{\U}[1]{\protect\rule{.1in}{.1in}}
\providecommand{\U}[1]{\protect\rule{.1in}{.1in}}
\newtheorem{theorem}{Theorem}
\newtheorem{acknowledgement}[theorem]{Acknowledgement}
\newtheorem{algorithm}[theorem]{Algorithm}
\newtheorem{axiom}[theorem]{Axiom}
\newtheorem{case}[theorem]{Case}
\newtheorem{claim}[theorem]{Claim}
\newtheorem{conclusion}[theorem]{Conclusion}
\newtheorem{condition}[theorem]{Condition}
\newtheorem{conjecture}[theorem]{Conjecture}
\newtheorem{corollary}[theorem]{Corollary}
\newtheorem{criterion}[theorem]{Criterion}
\newtheorem{definition}[theorem]{Definition}
\newtheorem{example}[theorem]{Example}
\newtheorem{exercise}[theorem]{Exercise}
\newtheorem{lemma}[theorem]{Lemma}
\newtheorem{notation}[theorem]{Notation}
\newtheorem{problem}[theorem]{Problem}
\newtheorem{proposition}[theorem]{Proposition}
\newtheorem{remark}[theorem]{Remark}
\newtheorem{solution}[theorem]{Solution}
\newtheorem{summary}[theorem]{Summary}
\newenvironment{proof}[1][Proof]{\textbf{#1.} }{\ \rule{0.5em}{0.5em}}
\begin{document}

\begin{center}
\textbf{Math 21-269, Vector Analysis I, Spring 2024}

\textbf{Assignment 2}
\end{center}

\textbf{The due date for this assignment is Friday, February 9.}

Given $N$ bounded intervals $I_{1},\ldots,I_{N}\subset\mathbb{R}$, a
\emph{rectangle} in $\mathbb{R}^{N}$ is a set of the form
\[
R:=I_{1}\times\cdots\times I_{N}.
\]
The elementary measure of a rectangle is given by%
\[
\operatorname*{meas}R:=(b_{1}-a_{1})\cdot(b_{2}-a_{2})\cdot\cdots\cdot
(b_{n}-a_{n})
\]
where $a_{n}\leq b_{n}$ are the endpoints of $I_{n}$. 

Given a rectangle $R$, by a \emph{partition} $\mathcal{P}$ of $R$ we mean a
finite set of rectangles $R_{1}$, \ldots, $R_{n}$ such that $R_{i}\cap
R_{j}=\emptyset$ if $i\neq j$ and%
\[
R=\bigcup_{i=1}^{n}R_{i}.
\]
Let's take for granted that if $R\subset\mathbb{R}^{N}$ is a rectangle and
$\mathcal{P}=\left\{  R_{1},\ldots,R_{n}\right\}  $ is a partition of $R$,
then\footnote{This is a good exercise, in case you are bored.}%
\[
\operatorname*{meas}R=\sum_{i=1}^{n}\operatorname*{meas}R_{i}.
\]
Given a rectangle $R\subset\mathbb{R}^{N}$ and two partitions $\mathcal{P}$
and $\mathcal{Q}$ of $R$, we say that $\mathcal{Q}$ is a \emph{refinement} of
$\mathcal{P}$, if each rectangle of $\mathcal{Q}$ is contained in some
rectangle of $\mathcal{P}$. 

Let $R\subset\mathbb{R}^{N}$ be a rectangle and let $f:R\rightarrow\mathbb{R}$
be a bounded function. Given a partition $\mathcal{P}=\{R_{1},\ldots,R_{n}\}$
of $R$, we define the \emph{lower} and \emph{upper sums} of $f$ for the
partition $\mathcal{P}$ respectively by%
\begin{align*}
L\left(  f,\mathcal{P}\right)   &  :=\sum_{i=1}^{n}\operatorname*{meas}%
R_{i}\inf_{\boldsymbol{x}\in R_{i}}f\left(  \boldsymbol{x}\right)  ,\\
U\left(  f,\mathcal{P}\right)   &  :=\sum_{i=1}^{n}\operatorname*{meas}%
R_{i}\sup_{\boldsymbol{x}\in R_{i}}f\left(  \boldsymbol{x}\right)  .
\end{align*}
Given a rectangle $R$ and a bounded function $f:R\rightarrow\mathbb{R}$, the
\emph{upper Riemann integral} $\overline{\int_{R}}f\left(  \boldsymbol{x}%
\right)  \,d\boldsymbol{x}$ of $f$ and the \emph{lower Riemann integral}
$\underline{\int_{R}}f\left(  \boldsymbol{x}\right)  \,d\boldsymbol{x}$ of $f$
are defined as%
\begin{align*}
\overline{\int_{R}}f\left(  \boldsymbol{x}\right)  \,d\boldsymbol{x} &
=\inf\left\{  U\left(  f,\mathcal{P}\right)  :\,\mathcal{P}\text{ partition of
}R\right\}  ,\\
\underline{\int_{R}}f\left(  \boldsymbol{x}\right)  \,d\boldsymbol{x} &
=\sup\left\{  L\left(  f,\mathcal{P}\right)  :\,\mathcal{P}\text{ partition of
}R\right\}  .
\end{align*}


If a nonempty set $E\subseteq\mathbb{R}$ is not bounded from above, we define
$\sup E:=\infty$. Similarly, if $E$ is not bounded from below, we set $\inf
E:=-\infty$.

\begin{enumerate}
\item Let $R\subset\mathbb{R}^{N}$ be a rectangle and let $f:R\rightarrow
\mathbb{R}$ and $g:R\rightarrow\mathbb{R}$  be bounded functions.

\begin{enumerate}
\item Prove that if $\mathcal{P}$ and $\mathcal{Q}$ are two partitions of $R$
and $\mathcal{Q}$ is a refinement of $\mathcal{P}$, then
\[
U(f,\mathcal{Q})\leq U(f,\mathcal{P}),\quad L(f,\mathcal{P})\leq
L(f,\mathcal{Q}).
\]


\item Prove that%
\[
\underline{\int_{R}}f\left(  \boldsymbol{x}\right)  \,d\boldsymbol{x}%
\leq\overline{\int_{R}}f\left(  \boldsymbol{x}\right)  \,d\boldsymbol{x}.
\]


\item Prove that if $\mathcal{P}$ and $\mathcal{Q}$ be two partitions of $R$
and $\mathcal{S}$ is a refinement $\mathcal{S}$ of both $\mathcal{P}$ and
$\mathcal{Q}$, then
\begin{align*}
L\left(  f,\mathcal{P}\right)  +L(g,\mathcal{Q})  & \leq L\left(
f+g,\mathcal{S}\right)  ,\\
U\left(  f+g,\mathcal{S}\right)    & \leq U\left(  f,\mathcal{P}\right)
+U(g,\mathcal{Q}).
\end{align*}


\item Prove that%
\begin{align*}
\underline{\int_{R}}f\left(  \boldsymbol{x}\right)  \,d\boldsymbol{x}%
+\underline{\int_{R}}g\left(  \boldsymbol{x}\right)  \,d\boldsymbol{x}  &
\leq\underline{\int_{R}}\left(  f\left(  \boldsymbol{x}\right)  +g\left(
\boldsymbol{x}\right)  \right)  \,d\boldsymbol{x},\\
\overline{\int_{R}}\left(  f\left(  \boldsymbol{x}\right)  +g\left(
\boldsymbol{x}\right)  \right)  \,d\boldsymbol{x}  & \leq\overline{\int_{R}%
}f\left(  \boldsymbol{x}\right)  \,d\boldsymbol{x}+\overline{\int_{R}}g\left(
\boldsymbol{x}\right)  \,d\boldsymbol{x}.
\end{align*}

\end{enumerate}

\item For every $\boldsymbol{x}\in\mathbb{R}^{N}$, let%
\[
\Vert\boldsymbol{x}\Vert_{1}:=|x_{1}|+\cdots+|x_{N}|,\quad\Vert\boldsymbol{x}%
\Vert_{\infty}:=\max\{|x_{1}|,\ldots,|x_{N}|\}.
\]


\begin{enumerate}
\item Prove that $\Vert\cdot\Vert_{1}$ and $\Vert\cdot\Vert_{\infty}$ are norms.

\item For every $\boldsymbol{x}\in\mathbb{R}^{N}$, let%
\[
\Vert\boldsymbol{x}\Vert:=\inf\{\Vert\boldsymbol{y}\Vert_{1}+\Vert
\boldsymbol{z}\Vert_{\infty}:\,\boldsymbol{y}\in\mathbb{R}^{N}%
,\,\boldsymbol{z}\in\mathbb{R}^{N},\,\boldsymbol{x}=\boldsymbol{y}%
+\boldsymbol{z}\}.
\]
Prove that $\Vert\cdot\Vert$ is a norm.
\end{enumerate}

\item Compute the liminf and the limsup of the following sequences

\begin{enumerate}
\item $x_{n}=\frac{1+(-1)^{n}}{n}$,

\item $x_{n}=\sin\frac{n\pi}{2}$,

\item $x_{n}=(-1)^{n}n\sin\frac{1}{n}$,

\item $x_{n}=x^{n}$, where $x\in\mathbb{R}$.
\end{enumerate}
\end{enumerate}

\begin{center}
\textbf{JUSTIFY\ ALL\ YOUR\ ANSWERS}
\end{center}
\begin{enumerate}
    \item \begin{enumerate}
        \item $\mathcal{Q}$ being a refinement of $\mathcal{P}$ means that for every $P \in \mathcal{P}$, there exists $\{Q_i\}_{1 \leq i \leq r}$ such that $\bigcup_{1 \leq i \leq p} Q_i = P$. So, we have:
        \begin{align*}
            U(f, \mathcal{Q}) = \sum_{Q \in \mathcal{Q}} \operatorname{meas} Q \sup_{x \in Q}f(x)
        \end{align*}
        So, let $\{Q_i\}_{1 \leq i \leq r} = S$. We argue that:
        \begin{align*}
            \sum_{Q \in S}\operatorname{meas}Q\sup_{x \in Q}f(x) \leq \sum_{Q \in S}\operatorname{meas}Q\sup_{x \in P}f(x) = \operatorname{meas} P \sup_{x \in P}f(x)
        \end{align*}
        This is true because the supremum of the union of some sets is very obviously greater than or equal to the supremum of each individual set. We then proceed by realizing
        \begin{align*}
            \sum_{Q \in \mathcal{Q}} \operatorname{meas} Q \sup_{x \in Q}f(x) = \sum_{P \in \mathcal{P}} \sum_{Q \in S}\operatorname{meas}Q\sup_{x \in Q}f(x)
        \end{align*}
        since the partitions are still included exactly one time. Combining everything, we get :
        \begin{align*}
            U(f, \mathcal{Q}) &= \sum_{Q \in \mathcal{Q}} \operatorname{meas} Q \sup_{x \in Q}f(x)\\
            &= \sum_{P \in \mathcal{P}} \sum_{Q \in S}\operatorname{meas}Q\sup_{x \in Q}f(x) \\
            &\leq \sum_{P \in \mathcal{P}} \operatorname{meas} P \sup_{x \in P}f(x) \\
            &= U(f, \mathcal{P})
        \end{align*}
        For the $L$ case, we can use the same argument to show that $L(f, \mathcal{P}) \leq L(f, \mathcal{Q})$.
        \begin{align*}
            L(f, \mathcal{Q}) &= \sum_{Q \in \mathcal{Q}} \operatorname{meas} Q \inf_{x \in Q}f(x)\\
            &= \sum_{P \in \mathcal{P}} \sum_{Q \in S}\operatorname{meas}Q\inf_{x \in Q}f(x) \\
            &\geq \sum_{P \in \mathcal{P}} \operatorname{meas} P \inf_{x \in P}f(x)\\
            &= L(f, \mathcal{P})
        \end{align*}
        \item We know $L(f, \mathcal{P}) \leq U(f, \mathcal{P})$ for all partitions $\mathcal{P}$ of $R$. This can be derived from the definitions of $L$ and $U$ and the fact that $\inf_{x \in R}f(x) \leq \sup_{x \in R}f(x)$. Because of this, we essentially want to show that $\sup\{L(f, \mathcal{P})\} \leq \inf\{U(f, \mathcal{P})\}$. We basically have that for all $x \in \sup\{L(f, \mathcal{P})\}$ and $y \in \inf\{U(f, \mathcal{P})\}$, $x \leq y$. We also have that $\sup\{L(f, \mathcal{P})\}$ is bounded from above and $\inf\{U(f, \mathcal{P})\}$ is bounded from below. As such, we know that all $y$ is an upper bound for all $x$. This means that $\sup\{L(f, \mathcal{P})\}$ is a lower bound for $\inf\{U(f, \mathcal{P})\}$. As such, we have that $\sup\{L(f, \mathcal{P})\} \leq \inf\{U(f, \mathcal{P})\}$.
        \item (a) gave us the following results:
        \begin{itemize}
            \item $L(f, \mathcal{P}) + L(g, \mathcal{Q}) \leq L(f, \mathcal{S}) + L(g, \mathcal{S})$
            \item $U(f, \mathcal{S}) + U(g, \mathcal{S}) \leq U(f, \mathcal{P}) + U(g, \mathcal{Q})$
        \end{itemize}
        We now show that $\inf\{f(x)\} + \inf\{g(x)\} \leq \inf\{f(x) + g(x)\}$. We have:
        \begin{align*}
            f(x) + g(x) &\geq \inf\{f(x)\} + \inf\{g(x)\},
        \end{align*}
        which means that we have a lower bound for $\{f(x) + g(x)\}$. As such:
        \begin{align*}
            \inf\{f(x)\} + \inf\{g(x)\} &\leq \inf\{f(x) + g(x)\}.
        \end{align*}
        So if we apply this given inequality to each element of $\mathcal{S}$, we get our desired result:
        \begin{align*}
            L(f, \mathcal{P}) + L(g, \mathcal{Q}) &\leq L(f, \mathcal{S}) + L(g, \mathcal{S})\\
            &\leq L(f + g, \mathcal{S})
        \end{align*}
        We can do the same thing for the upper sums to get the other inequality. We proceed by showing that $\sup\{f(x) + g(x)\} \leq \sup\{f(x)\} + \sup\{g(x)\}$. We have:
        \begin{align*}
            f(x) + g(x) &\leq \sup\{f(x)\} + \sup\{g(x)\},
        \end{align*}
        which means that we have an upper bound for $\{f(x) + g(x)\}$. As such:
        \begin{align*}
            \sup\{f(x) + g(x)\} &\leq \sup\{f(x)\} + \sup\{g(x)\}.
        \end{align*}
        So if we apply this given inequality to each element of $\mathcal{S}$, we get our desired result:
        \begin{align*}
            U(f + g, \mathcal{S}) &\leq U(f, \mathcal{P}) + U(g, \mathcal{Q})\\
            &\leq U(f, \mathcal{S}) + U(g, \mathcal{S})
        \end{align*}
        \newpage
        \item 
        We have that for all partitions $\mathcal{P}_\circ$ of $R$, we have:
        \begin{align*}
            \inf\{U(f, \mathcal{P}) | \mathcal{P} \text{ partitions } R\} + \inf\{U(g, \mathcal{P}) | \mathcal{P} \text{ partitions } R\} \leq U(f, \mathcal{P}_\circ) + U(g, \mathcal{P}_\circ)
        \end{align*}
        This means that the LHS is a lower bound for the following set:
        \begin{align*}
            \{U(f + g, \mathcal{P}) | \mathcal{P} \text{ partitions } R\}
        \end{align*}
        So the infimum of this set is greater than or equal to the aforementioned LHS, leading to:
        \begin{align*}
            \underline{\int_{R}}f\left(  \boldsymbol{x}\right)  \,d\boldsymbol{x}%
+\underline{\int_{R}}g\left(  \boldsymbol{x}\right)  \,d\boldsymbol{x}  &
\leq\underline{\int_{R}}\left(  f\left(  \boldsymbol{x}\right)  +g\left(
\boldsymbol{x}\right)  \right)  \,d\boldsymbol{x}
        \end{align*}
        We repeat this process for upper integrals with $L$ instead of $U$ and sups instead of infs. So, for any partition $\mathcal{P}_\circ$ of $R$, we have:
        \begin{align*}
            \sup\{L(f, \mathcal{P}) | \mathcal{P} \text{ partitions } R\} + \sup\{L(g, \mathcal{P}) | \mathcal{P} \text{ partitions } R\} \geq L(f, \mathcal{P}_\circ) + L(g, \mathcal{P}_\circ)
        \end{align*}
        This means that the LHS is an upper bound for the following set:
        \begin{align*}
            \{L(f + g, \mathcal{P}) | \mathcal{P} \text{ partitions } R\}
        \end{align*}
        So the supremum of this set is less than or equal to the aforementioned LHS, leading to:
        \begin{align*}
            \overline{\int_{R}}\left(  f\left(  \boldsymbol{x}\right)  +g\left(
\boldsymbol{x}\right)  \right)  \,d\boldsymbol{x}  & \leq\overline{\int_{R}%
}f\left(  \boldsymbol{x}\right)  \,d\boldsymbol{x}+\overline{\int_{R}}g\left(
\boldsymbol{x}\right)  \,d\boldsymbol{x}.
        \end{align*}
        As such, we have shown both inequalities.
    \end{enumerate}
    \item Recall the definition of a norm:

    Let $V$ be a vector space with an inner product $(\cdot, \cdot)$. Then the \emph{norm} of $x \in X$ is defined as $||\cdot||: X \to [0, \infty)$ such that:
    \begin{itemize}
        \item $||x|| = 0 \iff x = 0$
        \item $||tx|| = |t|||x||$ for all $x \in X$
        \item $||x + y|| \leq ||x|| + ||y||$ for all $x, y \in X$
    \end{itemize}
    \begin{enumerate}
        \item First we show that $\Vert\cdot\Vert_{1}$ is a norm.
        \begin{itemize}
            \item If we look at:
            \begin{align*}
                \Vert x \Vert_1 = |x_1| + \cdots + |x_N|,
            \end{align*}
            we see that the RHS is strictly positive if $x_i \neq 0$, so by contrapositive, if $\Vert x \Vert_1 = 0$, then $x_i = 0$ for all $i \Rightarrow x = 0$. This satisfies the first property of a norm.
            \item If we look at:
            \begin{align*}
                tx &= t(x_1, \ldots, x_N) = (tx_1, \ldots, tx_N)\\
                \Vert tx \Vert_1 &= |tx_1| + \cdots + |tx_N| = |t||x_1| + \cdots + |t||x_N| = |t|(|x_1| + \cdots + |x_N|) = |t|\Vert x \Vert_1
            \end{align*}
            \item We know that $|x_i + y_i| \leq |x_i| + |y_i|$ for all $i$, so we can apply this to the sum of the absolute values of the components of $x + y$ to get:
            \begin{align*}
                \Vert x + y \Vert_1 &= |x_1 + y_1| + \cdots + |x_N + y_N|\\
                &\leq |x_1| + |y_1| + \cdots + |x_N| + |y_N|\\
                &= \Vert x \Vert_1 + \Vert y \Vert_1
            \end{align*}
            Now we move to $\Vert\cdot\Vert_{\infty}$.
            \begin{itemize}
                \item If we look at:
                \begin{align*}
                    \Vert x \Vert_{\infty} = \max\{|x_1|, \ldots, |x_N|\},
                \end{align*}
                we see that the RHS is strictly positive if $x_i \neq 0$, so by contrapositive, if $\Vert x \Vert_{\infty} = 0$, then $x_i = 0$ for all $i \Rightarrow x = 0$. This satisfies the first property of a norm.
                \item We look at:
                \begin{align*}
                    |t|\Vert x \Vert_{\infty} = t\max\{|x_1|, \ldots, |x_N|\} = \max\{|tx_1|, \ldots, |tx_N|\} = \Vert tx \Vert_{\infty}
                \end{align*}
                \item We have:
                \begin{align*}
                    \Vert x + y \Vert_{\infty} &= \max\{|x_1 + y_1|, \ldots, |x_N + y_N|\} \\
                    &\leq \max\{|x_1| + |y_1|, \ldots, |x_N| + |y_N|\} \\
                    &\leq \max\{|x_1|, \ldots, |x_N|\} +  \max\{|y_1|, \ldots, |y_N|\}\\
                    &= \Vert x \Vert_{\infty} + \Vert y \Vert_{\infty}
                \end{align*}
            \end{itemize}
        \end{itemize}
        \item Turns out $\Vert\cdot\Vert = \Vert\cdot\Vert_\infty$. We can see this by the following:
        \begin{align*}
            \Vert y \Vert_1 + \Vert z \Vert_\infty &\geq \Vert y \Vert_\infty + \Vert z \Vert_\infty \\
            &\geq \Vert y + z \Vert_\infty \\
            &\geq \Vert x \Vert_\infty
        \end{align*}
        So $\Vert x \Vert_\infty$ is a lower bound of our set. But we can see that it is also in the set, because we can take $y = 0$ and $z = x$. Since a lower bound that is in the set is the infimum, we have that $\Vert x \Vert = \Vert x \Vert_\infty$. As such, $\Vert \cdot \Vert$ is a norm. 
    \end{enumerate}
    \item Recall the definitions of liminf and limsup:
    \begin{align*}
        \liminf_{n\to\infty}x_n &= \sup_{n \in \mathbb{N}}\left(\inf_{k\geq n}x_k\right)\\
        \limsup_{n\to\infty}x_n &= \inf_{n \in \mathbb{N}}\left(\sup_{k\geq n}x_k\right)
    \end{align*}
    Additionally, we showed during recitation that liminf and limsup have the following properties:
    \begin{itemize}
        \item Let $\{x_n\}$ be a sequence bounded above in $\mathbb{R}$. Then $L \in \mathbb{R}$ is the limit superior of $\{x_n\}$ if for every $\epsilon > 0$, there exists $n_\epsilon \in \mathbb{N}$ such that:
        \begin{itemize}
            \item $x_n < L + \epsilon$ for all $n \geq n_\epsilon$.
            \item $ x_n > L - \epsilon$ for infinitely many $n$.
        \end{itemize}
        \item Let $\{x_n\}$ be a sequence bounded below in $\mathbb{R}$. Then $L \in \mathbb{R}$ is the limit inferior of $\{x_n\}$ if for every $\epsilon > 0$, there exists $n_\epsilon \in \mathbb{N}$ such that:
        \begin{itemize}
            \item $x_n < L + \epsilon$ for infinitely many $n$.
            \item $x_n > L - \epsilon$ for all $n \geq n_\epsilon$.
        \end{itemize}
    \end{itemize}
    \begin{enumerate}
        \item I claim that the liminf and limsup of the sequence $x_n = \frac{1+(-1)^n}{n}$ are both 0. I will prove this by using the conditions above:
        \begin{itemize}
            \item \textbf{limsup}: Let $L = 0$. Then for every $\epsilon > 0$, there exists $n_\epsilon \in \mathbb{N}$ such that:
            \begin{itemize}
                \item $x_n < L + \epsilon = \epsilon$ for all $n \geq n_\epsilon$.
                \item $x_n > L - \epsilon = -\epsilon$ for infinitely many $n$.
            \end{itemize}
            For the first condition, we are essentially asking if there exists $n_\epsilon \in \mathbb{N}$ such that $\dfrac{2}{n_\epsilon} < \epsilon$ for any positive $\epsilon$. 
            \begin{align*}
                \dfrac{2}{n_\epsilon} &< \epsilon\\
                n_\epsilon &> \dfrac{2}{\epsilon}
            \end{align*}
            By the Archimedean property, we know that this $n_\epsilon$ exists for any $\epsilon > 0$. As such, this condition is sufficed.

            For the second property, we are asking if $x_n > -\epsilon$ for infinitely many $n$. Since $x_n$ is always positive and $\epsilon < 0$, this condition is also sufficed.
            \item \textbf{liminf}: Let $L = 0$. Then for every $\epsilon > 0$, there exists $n_\epsilon \in \mathbb{N}$ such that:
            \begin{itemize}
                \item $x_n < L + \epsilon = \epsilon$ for infinitely many $n$.
                \item $x_n > L - \epsilon = -\epsilon$ for all $n \geq n_\epsilon$.
            \end{itemize}
            For the first condition, we are asking if there exists $n_\epsilon \in \mathbb{N}$ such that $\dfrac{2}{n_\epsilon} < \epsilon$ for infinitely many $n$. This is clearly true since in the limsup case, we showed that $x_n < \epsilon$ for all $n \geq n_\epsilon$ for any $\epsilon > 0$, which is an infinite number.

            For the second property, we are asking if $x_n > -\epsilon$ for all $n \geq n_\epsilon$. Since $x_n$ is always positive and $\epsilon < 0$, this condition is also sufficed if we even choose $n_\epsilon = 1$.
        \end{itemize}
        \item I claim the limsup is 1 and the liminf is -1. 
        \begin{itemize}
            \item \textbf{limsup}: Let $L = 1$. Then for every $\epsilon > 0$, there exists $n_\epsilon \in \mathbb{N}$ such that:
            \begin{itemize}
                \item $x_n < L + \epsilon = 1 + \epsilon$ for all $n \geq n_\epsilon$.
                \item $x_n > L - \epsilon = 1 - \epsilon$ for infinitely many $n$.
            \end{itemize}
            If we analyze $x_n$, we see the following:
            \begin{align*}
                x_n = \begin{cases} 1 & n \equiv 0 \pmod{4}\\ -1 & n \equiv 2 \pmod{4}\\ 0 & \text{ otherwise}\end{cases}
            \end{align*}
            If we take any $\epsilon > 1$, we see that $x_n < 1 + \epsilon$ for all $n \geq 1$, so we can take $n_\epsilon = 1$. 
            
            Additionally, we know there are infinite many $n$ such that $x_n > 1 - \epsilon$ since $x_n = 1$ for all $n \equiv 0 \pmod{4}$.
            \item \textbf{liminf}: Let $L = -1$. Then for every $\epsilon > 0$, there exists $n_\epsilon \in \mathbb{N}$ such that:
            \begin{itemize}
                \item $x_n < L + \epsilon = -1 + \epsilon$ for infinitely many $n$.
                \item $x_n > L - \epsilon = -1 - \epsilon$ for all $n \geq n_\epsilon$.
            \end{itemize}
            Since $x_n = -1$ for infinitely many $n$, we know that $x_n < -1 + \epsilon$ for infinitely many $n$. Additionally, we know that $x_n > -1 - \epsilon$ for all $n \geq 1$.
        \end{itemize}
        \item For small-angle approximation, we have that:
        \begin{align*}
            \sin(\theta) \approx \theta \tag{for small $\theta$}
        \end{align*}
        As such, we have that for large $n$, \begin{align*}
            x_n = (-1)^n n \sin\left(\frac{1}{n}\right) \approx (-1)^n n \left(\frac{1}{n}\right) = (-1)^n
        \end{align*}
        We then have:
        \begin{align*}
            \lim_{m \to \infty} x_{2m} &= 1\\
            \lim_{m \to \infty} x_{2m + 1} &= -1
        \end{align*}
        As such, we have that $\liminf_{n\to\infty}x_n = -1$ and $\limsup_{n\to\infty}x_n = 1$ because the liminf is the infimum of the set of all subsequential limits and the limsup is the supremum of the set of all subsequential limits.
        \item There are 7 cases to consider here depending on the value of $x$. 
        \begin{itemize}
            \item If $x = 0$, then $x_n = 0$ for all $n$ and so $\liminf_{n\to\infty}x_n = \limsup_{n\to\infty}x_n = 0$.
            \item If $x = 1$, then $x_n = 1$ for all $n$ and so $\liminf_{n\to\infty}x_n = \limsup_{n\to\infty}x_n = 1$.
            \item If $x = -1$, then $x_n = (-1)^n$ and so $\liminf_{n\to\infty}x_n = -1$ and $\limsup_{n\to\infty}x_n = 1$. We know this because it's basically the same as problem (b) because is has an infinite number of $-1$s and $1$s, and our solution in (b) did not rely at all on the fact that $x_n = 0$ for some values of $n$.
            \item If $-1 > x > 0$, then the limit superior is 0 and the limit inferior is 0. For the limit superior, we know that there exists $x_n < \epsilon$ for all $n \geq n_\epsilon$. This is because the subsequence $x_n$ for even $n$ is strictly decreasing and positive, it will eventually be less than $\epsilon$. Additionally, we have that $x_n > -\epsilon$ for infinitely many $n$. This is because the subsequence $x_n$ for odd $n$ is strictly increasing and negative, it will eventually be greater than $-\epsilon$, and the even $n$ will be positive, so obviously greater than $-\epsilon$.
            
            For the limit inferior, we said that $x_n < L + \epsilon$ for any $n > n_\epsilon$, which is an infinite number of $n$. Now again since $x_n$ is positive, we have that $x_n > L - \epsilon$ for all $n \geq 1$. This is because the subsequence $x_n$ for odd $n$ is strictly increasing and negative, it will eventually be greater than $-\epsilon$, and the even $n$ will be positive, so obviously greater than $-\epsilon$.
            \item If $0 < x < 1$ then the limit superior is 0 and the limit inferior is 0. $x_n$ is always positive and decreasing since $a^x$ is decreasing for $0 < a < 1$. As such, we have that $x_n < \epsilon$ for all $n \geq n_\epsilon$ is the first number such that $x^{n_\epsilon} < \epsilon$. Since $x^n$ is positive, we also have $x_n > -\epsilon$ for infinitely many $n$. 
            
            For the limit inferior, we said that $x_n < L + \epsilon$ for any $n > n_\epsilon$, which is an infinite number of $n$. Now again since $x_n$ is positive, we have that $x_n > L - \epsilon$ for all $n \geq 1$.
            \item It's worth noting the following:
            \begin{itemize}
                \item If $x_n$ isn't bounded from above, then \[\limsup_{n\to\infty}x_n = \infty\]
                \begin{proof}
                    Suppose $\{x_n\}$ is not bounded from above. Then for any $k \in \mathbb{N}$, we also have that $\{x_n : n \geq k\}$ is also not bounded from above. Thus $s_k = \sup\{x_n : n \geq k\} = \infty$ for any $k$. As such, $\limsup_{n\to\infty}x_n = \infty$.
                \end{proof}
                \item If $x_n$ isn't bounded from below, then \[\liminf_{n\to\infty}x_n = -\infty\]
                \begin{proof}
                    Suppose $\{x_n\}$ is not bounded from below. Then for any $k \in \mathbb{N}$, we also have that $\{x_n : n \geq k\}$ is also not bounded from below. Thus $s_k = \inf\{x_n : n \geq k\} = -\infty$ for any $k$. As such, $\liminf_{n\to\infty}x_n = -\infty$.
                \end{proof}
            \end{itemize}
            If $x > 1$, we have an unbounded strictly increasing sequence. In other words, $x_n \to \infty $ as $n \to \infty$. The above proof shows that $\limsup_{n\to\infty}x_n = \infty$.
            \item If $x < -1$, then we have that $x_n$ isn't bounded from top or bottom. As such, we have that $\liminf_{n\to\infty}x_n = -\infty$ and $\limsup_{n\to\infty}x_n = \infty$.
        \end{itemize}
    \end{enumerate}
\end{enumerate}


\end{document}