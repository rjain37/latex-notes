\documentclass{report}

\input{preamble}
\input{macros}
\input{letterfonts}
\usepackage{fancyhdr}
\pagestyle{fancy}

\lhead{\bf Rohan Jain}
\cfoot{}
\rhead{\bf
Abstract Algebra \\
Assignment 15}

\begin{document}

\qs{11}{Let $\mathbb Z[ \sqrt{2} ] = \{ a + b \sqrt{2} : a, b \in {\mathbb Z} \}\text{.}$
\begin{enumerate}[label=\alph*.]
	\item Prove that ${\mathbb Z}[ \sqrt{2}\, ]$ is an integral domain. 
	\item Find all the units of  ${\mathbb Z}[ \sqrt{2}\, ]$.
	\item Determine the field of fractions of ${\mathbb Z}[ \sqrt{2}\, ]$.
	\item Prove that ${\mathbb Z}[ \sqrt{2i}\, ]$ is a Euclidean domain under the Euclidean valuation $\nu( a + b \sqrt{2}\, i) = a^2 + 2b^2\text{.}$
\end{enumerate}}
\sol
\begin{enumerate}[label=\alph*.]
	\item Consider $(a+b\sqrt{2})(c+d\sqrt{2})=0$. This means that $(a+b\sqrt{2})(a-b\sqrt{2})(c+d\sqrt{2})(c-d\sqrt{2}) =(a^2-2b^2)(c^2-2d^2) = 0$. The irrationality of $\sqrt{2}$ and that fact that $a,b,c,d$ are all integers tells us that either $a=b=0$ or $c=d=0$. $\qed$
	\item If $u$ is a unit, then so are $-u, 1/u$, and $-1/u$. At least one of these has to be greater than 1 if $u\neq 0, 1$. As such, it is enough to show that if $u<1$, then $u$ is a power of $1+\sqrt{2}$. So, we can write that $(1+\sqrt{2})^k < u < (1+\sqrt{2})^{k+1}$. Dividing by $(1+\sqrt{2})^k$ gives us $1 < u(1+\sqrt{2})^{-k} < 1+\sqrt{2}$. Since $1+\sqrt{2}$ is the smallest unit not equal to 0$, u(1+\sqrt{2})^{-k} = 1 \Rightarrow u = (1+\sqrt{2})^k$. Since norm is multiplicative, we have that all powers of $1+\sqrt{2}$ are units.
	\item $$Q = \left\{\frac{a+b\sqrt{2}}{c+d\sqrt{2}} : a,b,c,d \in \ZZ, c+d\sqrt{2}\neq 0\right\}$$
	\item We first need to show that $\nu(xy) = \nu(x)\nu(y)$ for $x,y \in \ZZ\,[\sqrt{2}i]$. 
	\begin{align*}
		\nu((a+b\sqrt{2i})(c+d\sqrt{2i})) &= \nu((ac-2bd)+(ad+bc)\sqrt{2i}) \\
		&= (ac-2bd)^2+2(ad+bc)^2 \\
		&= a^2c^2 + 2a^2d^2 + 2b^2c^2 + 4b^2d^2 \\
		\nu(a+b\sqrt{2i})\nu(c+d\sqrt{2i}) &= (a^2+2b^2)(c^2+2d^2) \\
		&= a^2c^2 + 2a^2d^2 + 2b^2c^2 + 4b^2d^2 
	\end{align*}
	For nonzero values, we have that $\nu(x) \geq 1$ and it follows that $\nu(x) \leq \nu(xy)$. 

	Next, let $a+b\sqrt{2i}, c+d\sqrt{2i} \in \ZZ\,[\sqrt{2i}]$ with nonzero $c+d\sqrt{2i}$. Now define $q_1$ to be the closest integer to $\frac{ac}{c^2}$ and $q_2$ as the closest integer to $\frac{bc}{c^2}$. Define $s_1, s_2, r_1, r_2$ as follows:
	$$s_1 + s_2\sqrt{2i} = \left(\frac{ac+2bd}{c^2 + d^2} + \frac{bc-ad}{c^2+2d^2}\sqrt{2i}\right) - (q_1 + q_2\sqrt{2i})$$
	$$r_1 + r_2\sqrt{2i} = (a+b\sqrt{2i}) - \left(q_1+q_2\sqrt{2i}\right)(c+d\sqrt{2i})$$
	From this, we need to show that $r_1^2+2r_2^2 < c^2 + 2d^2$. Start by noting that $|s_1|, |s_2| \leq \frac{1}{2}$ by definition of $q_1$ and $q_2$. So, 
	$$(s_1+s_2\sqrt{2i})(c+d\sqrt{2i}) = (a+b\sqrt{2i}) - (q_1 + q_2\sqrt{2i})(c+d\sqrt{2i}) = r_1+r_2\sqrt{2i}$$
	Thus,
	$$\nu(r_1+r_2\sqrt{2i}) = \nu(s_1+s_2\sqrt{2i})\nu(c+d\sqrt{2i}) \leq \left(\frac{1}{4} + 2\cdot\frac{1}{4}\right)\nu(c+d\sqrt{2i}) < \nu(c+d\sqrt{2i})$$
	Therefore, the original statement was proved and also we proved that $\ZZ\,[\sqrt{2i}]$ is a Euclidean domain. $\qed$
\end{enumerate}

\qs{17}{Prove or disprove: Every subdomain of a UFD is also a UFD.}
\sol $\ZZ\,[3i] \subseteq \CC$ is a subdomain of a UFD, but is not a UFD. $\qed$


\qs{18}{An ideal of a commutative ring $R$ is said to be \emph{\textbf{finitely generated}} if there exist elements $a_1, \ldots, a_n$ in $R$ such that every element $r$ in the ideal can be written as $a_1 r_1 + \cdots + a_n r_n$ for some $r_1, \ldots, r_n$ in $R$. Prove that $R$ satisfies the ascending chain condition if and only if every ideal of $R$ is finitely generated.}
\sol We start by proving that if $R$ satisifes ACC, then its ideals are finitely generated. Let $I$ be a nonzero ideal and $a_1$ a nonzero value of $I$. If $I = \langle a_1 \rangle$, then $I$ is finitely generated. If not, then $I_1 = \langle a_1 \rangle$ is a subset of $I$. Now consider $a_2 \in I \backslash I_1$. Let, $I_2 = \langle a_1, a_2 \rangle$. If $I=I_2$, we are done. If not, we have an $a_3 \in I \backslash I_2$ and we continue the process to have that $I_1 \subseteq I_2 \subseteq I_3 \cdots$. By ACC, there exists an $N$ such that $I_n = I_N$ for all $n \geq N$. But if $I_{N+1} = I_N$, then there are no elements in $I$ that aren't in $I_N$. Therefore, $I = \langle a_1, \ldots, a_N \rangle$ and $I$ is finitely generated.

For the converse, we supposed the ideals of $R$ are finitely generated. We have that $I = \bigcup_{n=1}^{\infty}I_n$ is an ideal. But since every ideal is finitely generated, we have that $I = \langle a_1, \ldots, a_k \rangle$ for some $k$. But then for $n = 1,2,3,\cdots k$, $a_i \in I_{b_i}$ for some integer $b_i$. Let $N = \max(b_1, \ldots, b_k)$. Then $a_i \in I_{b_i} \subseteq I_N$. Therefore, $I \subseteq I_N$. So, $I_n = I_N$ for all $n \geq N$ and $R$ satisfies ACC. $\qed$

\qs{19}{Let $D$ be an integral domain with a descending chain of ideals $I_1 \supset I_2 \supset I_3 \supset \cdots\text{.}$ Suppose that there exists $N$ such that $I_k = I_N$ for all $k \geq N\text{.}$ A ring satisfying this condition is said to satisfy the \textbf{\emph{descending chain condition}}, or DCC. Rings satisfying the DCC are called \textbf{\emph{Artinian rings}}, after Emil Artin. Show that if $D$ satisfies the descending chain condition, it must satisfy the ascending chain condition.}
\sol Consider $I_i = \langle a^k \rangle$ for some $a \in D$. Since $a^{n+1}d = a^n(ad)$, we have that $a^{n+1} \subseteq a^nD$. So, we have a descending chain of ideals as follows:
$$aD \supseteq a^2D \supseteq \ldots \supseteq a^nD \supseteq a^{n + 1}D \supseteq \ldots$$
which stabilizes since $D$ is Artinian. So, we can say that 
$$a^{m + 1}D = a^mD$$
for some positive integer $m$. Since $a^m = a^m 1_D \in D$, there exists $b$ such that $a^{m+1} = a^mb$, or that $a^m(1_D - ab) = 0$. This yields that $ab = 1_D$ as $a, a^m \neq 0$ since $D$ is an integral domain. We have shown that $a \neq 0 \in D$ has an inverse and as such, $D$ is a field which satisifies the ascending chain condition. $\qed$

\end{document}