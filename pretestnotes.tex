\documentclass[12pt]{report}

\input{preamble}
\input{macros}
\input{letterfonts}
\usepackage{fancyhdr}
\pagestyle{fancy}

\lhead{\bf Rohan Jain}
\cfoot{}
\rhead{\bf
Abstract Algebra \\
Pre-test Notes}

\begin{document}

Now, to write the class equation for the direct product of $A_4$ and $Z_2$, we need to find the conjugacy classes of the elements in $A_4 \times Z_2$.

First, note that $Z_2$ is the cyclic group of order $2$ and has only two elements: the identity element and a non-identity element.

Next, recall that $A_4$ is the alternating group of degree $4$ and has $12$ elements. We can list its elements as:

\begin{align*}
A_4 = \{&(1), (12)(34), (13)(24), (14)(23), \\
&(123), (132), (124), (142), (134), (143), (234), (243)\}
\end{align*}

Now, we can form the elements of $A_4 \times Z_2$ by taking the direct product of each element in $A_4$ with both the identity element and the non-identity element of $Z_2$. This gives us $24$ elements, which we can list as:
\begin{align*}
&(1,0), (1,1), \\
&((12)(34),0), ((12)(34),1), \\
&((13)(24),0), ((13)(24),1), \\
&((14)(23),0), ((14)(23),1), \\
&((123),0), ((123),1), \\
&((243),0), ((243),1) \\
&((142),0), ((142),1), \\
&((134),0), ((134),1), \\
&((132),0), ((132),1), \\
&((124),0), ((124),1), \\
&((143),0), ((143),1), \\
&((234),0), ((234),1),
\end{align*}
To find the conjugacy classes of $A_4 \times Z_2$, we need to determine which elements are conjugate to each other under the group operation. Two elements $(g,h)$ and $(g',h')$ are conjugate if there exists an element $(x,y) \in A_4 \times Z_2$ such that $(g,h) = (x,y)(g',h')(x^{-1},y^{-1})$.
Since the group operation in $A_4 \times Z_2$ is defined componentwise, we can write this condition as:
$$(gxg^{-1}, hyh^{-1}) = (g',h')(x,y)(g'^{-1},h'^{-1})$$
This gives us two conditions: $gxg^{-1} = g'$ and $hyh^{-1} = h'$.
Using these conditions, we can form the following conjugacy classes:

$$\{0, \text{conj classes of } A_4\}, \{1, \text{conj classes of } A4\}$$
Therefore, the class equation for $A_4 \times Z_2$ is:

$$|A_4 \times Z_2| = 1 + 3 + 4 + 4 + 1 + 3 + 4 + 4 = 24$$

Here, we write the normal subgroups of $A_4$:
\begin{itemize}
    \item $\{e\}$
    \item $K_4$
    \item $A_4$
\end{itemize}
Then, the normal subgroups of $\ZZ_2$:
\begin{itemize}
    \item $\{e\}$
    \item $\ZZ_2$
\end{itemize}
We can combine any of the subgroups (one from each subgroup list) to get a normal subgroup of $A_4 \times \ZZ_2$. This gives us the following list:
\begin{itemize}
    \item $\{e\} \times \{e\} = \{e\}$
    \item $\{e\} \times \ZZ_2 = \ZZ_2$
    \item $K_4 \times \{e\} = K_4$
    \item $K_4 \times \ZZ_2$
    \item $A_4 \times \{e\} = A_4$
    \item $A_4 \times \ZZ_2$
\end{itemize}
Using these, we can find the quotient groups:
\begin{itemize}
    \item $(A_4 \times \ZZ_2) / \{e\} = A_4 \times \ZZ_2$
    \item $(A_4 \times \ZZ_2) / \ZZ_2 = A_4 $
    \item $(A_4 \times \ZZ_2) / K_4 = \ZZ_6$
    \item $(A_4 \times \ZZ_2) / (K_4 \times \ZZ_2) = \ZZ_3$
    \item $(A_4 \times \ZZ_2) / A_4 = \ZZ_2$
    \item $(A_4 \times \ZZ_2) / (A_4 \times \ZZ_2) = \{e\}$
\end{itemize}
\newpage
Additionally, here are the elements and conjugacy classes of $A_4 \times \ZZ_2$ but written as generated by $r$ and $s$.

\includegraphics*[width=0.6\textwidth]{conjugacyclasses.png}

As practice, we write the closed form formulas for various Burnside lemma problems. We start with the counting how many ways you can paint the seats of the ferris wheel with $n$ colors. 
$$\frac{n^6 + 3n^5 + 3n^4 + n^3 + 2n^2 + 14n}{24}$$
Next, we count how many ways you can color the carts of the ferris wheel with $n$ colors.
$$\frac{n^3 + 2n}{3}$$


\end{document}