\documentclass[12pt]{report}

\input{preamble}
\input{macros}
\input{letterfonts}
\usepackage{fancyhdr}
\pagestyle{fancy}

\lhead{\bf Rohan Jain}
\cfoot{}
\rhead{\bf
Abstract Algebra \\
Assignment 7}

\begin{document}

\qs{1}{For each of the following groups $G$, determine whether $H$ is a normal subgroup of $G$. If $H$ is a normal subgroup, write out a Cayley table for the factor group $G/H$.
\begin{enumerate}[label=\alph*.]
    \item $G = S_4$ and $H = A_4$
    \item $G = A_5$ and $H = \{ (1), (1 \, 2 \, 3), (1 \, 3 \, 2) \}$
    \item $G = S_4$ and $H = D_4$
    \item $G = Q_8$ and $H = \{ 1, -1, i, -i\}$
    \item $G = \ZZ$ and $H = 5\ZZ$
\end{enumerate}}
\sol
\begin{enumerate}[label=\alph*.]
    \item $|G/H| = 4/12$ which is not an integer. Therefore, $H$ can't be normal in $G$.
    \item Counterexample: $(1 \; 5)(2 \; 3)(1 \; 2 \; 3) = (1 \; 3 \; 5) \neq (1 \; 2 \; 3)(1 \; 5)(2 \; 3) = (1 \; 5 \; 2)$. 
    \item $|G/H| = 4/8$ which is not an integer. Therefore, $H$ can't be normal in $G$.
    \item Since $|G| / |H| = 2$, we know that $|G:H| = 2$. So if we take any $g \in G$, then if $g \in H$, we have that $gH = H = Hg$. If $g \not\in H$, then since there are  two left cosets of $H$ in $G$ and $g$ isn't in $H$, the two cosets should be $H$ and $gH$. Left cosets are disjoint, so we can determine that $gH = G - H$. Right cosets are also disjoint, though, so $Hg = G-H = gH$, so $gH = Hg$ for all $g \in G$, so $H$ is normal in $G$ because its index is 2. 
    
    Since we know there is only one subgroup of order 2, we know that this is isomorphic to $\ZZ_2$, which has the following Cayley table.
    \begin{center}
        \noindent\begin{tabular}{c | c c}
        $+$ & 0 & 1  \\
        \cline{1-3}
        0 & 0 & 1 \\
        1 & 1 & 0
        \end{tabular}
    \end{center}
    \item $H$ is normal in $G$ because $G = \ZZ$ is abelian and all subgroups of abelian groups are normal. 
    \begin{center}
        \noindent\begin{tabular}{c | c c c c c}
        $+$ & 0 & 1 & 2 & 3 & 4  \\
        \cline{1-6}
        0 & 0 & 1 & 2 & 3 & 4 \\
        1 & 1 & 2 & 3 & 4 & 0\\
        2 & 2 & 3 & 4 & 0 & 1 \\
        3 & 3 & 4 & 0 & 1 & 2\\
        4 & 4 & 0 & 1 & 2 & 3\\
        \end{tabular}
    \end{center}
\end{enumerate}

\newpage

\qs{4}{Let $T$ be the group of nonsingular upper triangular $2 \times 2$ matrices with entries in $\RR$; that is, in the form 
$$
    \begin{pmatrix}
        a & b \\
        0 & c
    \end{pmatrix},$$ where $a,b,c \in \RR$ and $ac \neq 0$. Let $U$ consist of matrices of the form $$\begin{pmatrix}
        1 & x \\
        0 & 1
        \end{pmatrix},$$ where $x \in \RR$. 
\begin{enumerate}[label=\alph*.]
    \item Show that $U$ is a subgroup of $T$.
    \item Prove that $U$ is abelian. 
    \item Prove that $U$ is normal in $T$.
    \item Show that $T/U$ is abelian.
    \item Is $T$ normal in $GL_2(\RR)$?
\end{enumerate}
    }
\sol
\begin{enumerate}[label=\alph*.]
    \item To prove that $U$ is a subgroup of $T$, we have to prove the following criteria: associativity, identity, inverse, and closure. 
    \begin{enumerate}
        \item Associativity: Let $a,b,c \in \RR$. Then $A = \begin{pmatrix}
            1 & a \\
            0 & 1
        \end{pmatrix}, B = \begin{pmatrix}
            1 & b \\
            0 & 1
        \end{pmatrix}, C = \begin{pmatrix}
            1 & c \\
            0 & 1
        \end{pmatrix}$. Then $A(BC) = \begin{pmatrix} 1 & a \\ 0 & 1 \end{pmatrix}\begin{pmatrix}
            1 & b + c \\
            0 & 1
        \end{pmatrix} = \begin{pmatrix}
            1 & a + b + c \\
            0 & 1
        \end{pmatrix} = \begin{pmatrix} 1 & a + b \\ 0 & 1 \end{pmatrix}\begin{pmatrix} 1 & c \\ 0 & 1 \end{pmatrix} = (AB)C$. Therefore, $U$ is associative.
        \item Identity: Consider $u \in U$ where $x = 0$. Then $u = I_2$ which is the identity for all $2 \times 2$ matrices.
        \item Inverse: The inverse of $\begin{pmatrix}
            1 & x \\
            0 & 1
        \end{pmatrix}$ is $\begin{pmatrix}
            1 & -x \\
            0 & 1\end{pmatrix}$. This can be checked by multiplying the two matrices together and getting $I_2$.
        \item Closure: Let $u,v \in U$. Then $u = \begin{pmatrix}
            1 & x \\
            0 & 1 \end{pmatrix}, v = \begin{pmatrix}
            1 & y \\
            0 & 1 \end{pmatrix}$. Then $uv = \begin{pmatrix}
            1 & x + y \\
            0 & 1 \end{pmatrix} \in U$. Therefore, $U$ is closed.
    \end{enumerate}
    Therefore, $U$ is a subgroup of $T$.
    \item $U$ is abelian because if we have $u = \begin{pmatrix}
        1 & x \\
        0 & 1 \end{pmatrix}, v = \begin{pmatrix}
        1 & y \\
        0 & 1 \end{pmatrix}$, then $uv = \begin{pmatrix}
        1 & x + y \\
        0 & 1 \end{pmatrix} = \begin{pmatrix}
        1 & y + x \\
        0 & 1 \end{pmatrix} = vu$. $\qed$
    \item $U$ is normal in $T$ because $T$ is abelian and all subgroups of abelian groups are normal.
    \item Consider the following:
    $$\begin{pmatrix}
        a & b \\
        0 & c
    \end{pmatrix} = \begin{pmatrix}
        a & 0 \\ 0 & c \end{pmatrix} \begin{pmatrix} 1 & b/a \\ 0 & 1 \end{pmatrix}.$$
        From this, we know that every coset in $T/U$ has a representative diagonal matrix. We know that diagonal matrices commute, so we know that $T/U$ is abelian.
    \item Counterexample:
    $$\begin{pmatrix}
        0 & n \\ n & 0
    \end{pmatrix}\begin{pmatrix}
        n & n \\ 0 & n
    \end{pmatrix}\begin{pmatrix}
        0 & n \\ n & 0
    \end{pmatrix} = \begin{pmatrix}
        n^3 & 0 \\ n^3 & n^3
    \end{pmatrix}.$$
    

\end{enumerate}

\qs{5}{Show that the intersection of two normal subgroups is a normal subgroup.}
\sol Let $H$ and $K$ be two normal subgroups of $G$. Then, for $h \in H$ and $k \in K$ and $g \in G$, $ghg^{-1} \in H$ and $gkg^{-1} \in K$. Now, let $T = H \cap K$. 

\begin{align*}
    \text{Let } t \in T & \Rightarrow t \in H \text{ and  }t \in K \\
    & \Rightarrow gtg^{-1} \in H \text{ and } gtg^{-1} \in K
    & \Rightarrow gtg^{-1} \in H \cap K \\
    & \Rightarrow gtg^{-1} \in T
\end{align*}

$\therefore$ for all $g \in G$, $t\in T$, $gtg^{-1} \in T$. Therefore, $T$ is a normal subgroup of $G$.


\qs{11}{If a group $G$ has exactly one subgroup $H$ of order $k$, prove that $H$ is normal in $G$.}
\sol For $g \in G$, consider the conjugate subgroup $gHg^{-1} \leq G$. We also know that the order of $gHg{-1}$ is the same as the order of $H$, which we called $k$. But, since $H$ is the only subgroup of order $k$, any subgroup that has order $k$ must be $H$. Therefore, $gHg^{-1} = H$. Therefore, $H$ is normal in $G$. $\qed$

\newpage
\qs{13}{Recall that the \textbf{center} of a group $G$ is the set $$Z(G) = \{ x \in G : xg = gx \text{ for all } g \in G \}\text{.}$$
\begin{enumerate}[label=\alph*.]
    \item Calculate the center of $S_3$.
    \item Calculate the center of $GL_2(\RR)$. 
    \item Show that the center of any group $G$ is a normal subgroup of $G$.
    \item If $G/Z(G)$ is cyclic, show that $G$ is abelian.
\end{enumerate}}
\sol
\begin{enumerate}[label=\alph*.]
    \item $Z(S_3) = \{ (e) \}$
    \item If we have $A = \begin{pmatrix}
        a & b \\
        c & d
    \end{pmatrix}$ and $B =\begin{pmatrix}
        e & f \\
        g & h
    \end{pmatrix}$, we can multiply out $AB$ and $BA$ to see the following equality:
    $$ae + bg = ae + cf \Rightarrow bg = cf$$
    However, this equality needs to hold true for all choices of $g,f$ because our $B$ was arbitrary and not related to $A$. This means that $b= c =0$. This means that the equation

    $$af + bh = be + df$$

    reduces to $af = df$ or $a=d$. This means that we can say 

    $$Z(GL_2(\RR)) = \left\{\begin{pmatrix}
        a & 0 \\
        0 & a
    \end{pmatrix} : a \in \RR \backslash \{0\} \right\}.$$
    \item By definition, for any $z \in Z(G)$, the following equation will hold true, $zG=Gz$. By the definition of a normal subgroup, $Z(G)$ is normal in $G$. 
    \item By definition $G/Z(G) = \langle xZ(G) \rangle$ for some $xZ(G) \in G/Z(G)$ where $x$ is the representative for the coset $xZ(G)$. 
    
    If we let $a \in G$, then we know that $aZ(G) = (xZ(G))^m$ for some $m$. We can also rewrite $(xZ(G))^m = x^mZ(G)$.

    If we take another $b \in G$, then we know that $bZ(G) = (xZ(G))^n$ for some $n$. We can then rewrite $(xZ(G))^n = x^nZ(G)$.

    From these two equations, we gets that $ax^{-m}, bx^{-n} \in Z(G)$. For shorthand purposes, we can say $p = ax^{-m}$ and $q = bx^{-n}$. Then, we get that $a = px^m$ and $b = qx^n$. Multiplying both gives us $ab = (px^m)(qx^n) = pqx^{m+n}$. The last step was done beacuse we know that $Z(G)$ is abelian. 

    If we multiply the other way, we know see that $ba = (qx^n)(px^m) = pqx^{m+n}$. This means that $ab = ba$. Therefore, $G/Z(G)$ is abelian. $\qed$

\end{enumerate}

\qs{14}{Let $G$ be a group and let $G' = \langle aba^{- 1} b^{-1} \rangle\text{;}$ that is, $G'$ is the subgroup of all finite products of elements in $G$ of the form $aba^{-1}b^{-1}\text{.}$ The subgroup $G'$ is called the \textbf{commutator subgroup} of $G$.
\begin{enumerate}[label=\alph*.]
    \item Show that $G'$ is a normal subgroup of $G$.
    \item Let $N$ be a normal subgroup of $G$. Prove that $G/N$ is abelian iff $N$ contains the commutator subgroup of $G$.
\end{enumerate}}
\sol
\begin{enumerate}[label=\alph*.]
    \item Let $s = aba^{-1}b^{-1}$ be the generator of $G'$. We can say that for any $g \in G$, $gsg^{-1} = (gag^{-1})(gbg^{-1})(gag^{-1})^{-1}(gbg^{-1})^{-1}$. By this structure, we can see that $gsg^{-1} \in G'$. Effectively, conjugating by $g$ is a homomoprhism $G'$ is normal in $G$.
    \item If $a,b \in G$ and we assume $G/N$ is abelian, then we have $(aN)(bN) = (bN)(aN) \Leftrightarrow Nab = Nba \Leftrightarrow Naba^{-1}b^{-1} = N \Leftrightarrow aba^{-1}b^{-1} \in N$.
    
    Now, if we assume that $aba^{-1}b^{-1} \in N$, this is the same as $ab(ba)^{-1} \in N$. This means that $Nab = Nba$, or as we showed before, $(aN)(bN) = (bN)(aN)$. Therefore, $G/N$ is abelian. $\qed$.
\end{enumerate}

\end{document}