\documentclass{report}

\input{preamble}
\input{macros}
\input{letterfonts}

\title{\Huge{Abstract Algebra}}
\author{\huge{Rohan Jain}}
\date{}

\begin{document}

\maketitle
\newpage% or \cleardoublepage
% \pdfbookmark[<level>]{<title>}{<dest>}
\pdfbookmark[section]{\contentsname}{toc}
\tableofcontents

\pagebreak

\chapter{}
\section{Introductory Notes}

\subsection{Things to Remember}
\nt{
	\begin{itemize}
		\item Definitions will usually be stated as ``if" even though they mean ``if and only if".
		\item Any form of proof is valid. Avoid proofs by contradiction because of disbelief in the law of excluded middle.
		\item When you define an object, you can \emph{only} utilize its definition to prove anything about it.
	\end{itemize}
}

\subsection{Set Review}

\dfn{Set}{In mathematics, a set is an undefined term. Basically, ``everyone knows what it is.'' A few examples of sets are:

\begin{itemize}
	\item The empty set is the set with no elements. It is denoted by $\phi$ or $\emptyset$.
	\item $\NN$ is the set of natural numbers.
	\item $\ZZ$ is the set of integers.
	\item $\QQ$ is the set of rational numbers.
	\item $\RR$ is the set of real numbers.
	\item $\CC$ is the set of complex numbers.
\end{itemize}
}
\nt{
	\begin{itemize}
		\item A set is a well-defined collection of objects. The objects in a set are called elements of the set.
		\item A set is generally defined as a capital letter.
		\item $(A = B) \iff (\forall x : x \in A \iff x \in B)$
		\item $(A \subset B) \iff (\forall x \in A : x \in B)$
		\item $A$ is a proper subset of $B$ if $A \subset B$ and $A \neq B$.
	\end{itemize}
}

\thm{}{$A = B \iff A \subset B \land B \subset A$}

\nt{
	\begin{itemize}
		\item $A \cup B = {x:x\in A \lor x\in B}$
		\item $A \cap B = {x:x\in A \land x\in B}$
		\item $A$\textbackslash $B = {x:x\in A \land x\not\in B}$
		\item $C$\textbackslash $(A \cup B) = (C$\textbackslash $A) \cap (C$\textbackslash $B)$
	\end{itemize}
}

\subsection{Cartesian Products and Functions}
\nt{
	\begin{itemize}
		\item $A \times B = \{(a,b) : a \in A \land b \in B\}$
	\end{itemize}
}
\ex{Cartesian Product of two sets}{
	Let $A = \{1, 2, \Delta\}$ and $B = \{0, \pi\}$
	\begin{itemize}
		\item $(1, 0)$
		\item $(2, 0)$
		\item $(\Delta, 0)$
		\item $(1, \pi)$
		\item $(2, \pi)$
		\item $(\Delta, \pi)$
	\end{itemize}
}

\nt{Relations are subsets of Cartesian Products. For example, we can say that $<$ is a relation on the subset of $\RR \times \RR$ consisting of all ordered pairs of real numbers such that the first element is less than the second.}

\dfn{Function}{A function $f$ from a set $A$ to a set $B$ is a subset of $A \times B$ such that for every $a \in A$, there is exactly one $b \in B$ such that $(a, b) \in f$.}

\nt{Let $R$ be a relation from $A$ to $B$.
\begin{itemize}
	\item $A$ is the domain
	\item $B$ is the codomain
	\item $\{b: aRb\}$ is the image
	\item $R$ is injective (one-to-one) if $a_1Rb \land a_2Rb \implies a_1 = a_2$
	\item $R$ is surjective (onto) if $\forall b \in B : \exists a \in A : aRb$. Basically if the image is the entire codomain.
	\item $R$ is bijective if it is injective and surjective
\end{itemize}}

\nt{$A \xrightarrow{\text{R}} B$ \\ $B \xrightarrow{\text{S}} C$ \\ Define the composition as $S \circ R = \{(a, c) :$ there is some $b$ such that $(a, b) \in R$ and $(b, c) \in S$\} }

\thm{}{Let $f: A \to B$, $g: B \to C$, and $h: C \to D$. Then
\begin{itemize}
	\item $h \circ (g \circ f) = (h \circ g) \circ f$
	\item If $f$ and $g$ are injective, so is $g \circ f$
	\item If $f$ and $g$ are surjective, so is $g \circ f$
	\item If $f$ and $g$ are bijective, so is $g \circ f$
\end{itemize}}

\subsection{Equivalence Relations}
\dfn{Equivalence Relation}{An equivalence relation is a relation that has the following special properties:
\begin{itemize}
	\item Reflexivity: $aRa$ for all $a \in A$
	\item Symmetry: $aRb \implies bRa$
	\item Transitivity: $aRb \land bRc \implies aRc$
\end{itemize}}

\dfn{Partition}{Given a set $S$, a partition of $S$ is a collection of subsets of $S$ such that their union is $S$.}

\nt{Equivalence relations go hand in hand with partitions.}

\nt{If $\sim$ is an equivalence relation $a\sim b$, then $\sim$ partiations a set $X$ into chunks. $X / \sim$ is the set of chunks.

Addition is \emph{well-defined} as an operation on $\ZZ/x\ZZ$ for $x \in \ZZ$.}

\subsection{Complex Numbers and Matrices}
\dfn{Complex Number}{A complex number is a number of the form $a + bi$, where $a$ and $b$ are real numbers and $i$ is the imaginary unit. $i^2 = -1$.}

\nt{Complex numbers generally take the from $z = a + bi$. 

$\bar z = a - bi$ is the complex conjugate of $z$.

Divide complex numbers by multiplying by the complex conjugate of the denominator}

\dfn{Matrix}{A matrix is a rectangular array of numbers. A $m \times n$ matrix is an array of $m$ rows and $n$ columns. Define the group of $m \times n$ matrices over a field $\FF$ as $\FF^{m \times n}$.}

\nt{Multiplication by an $m \times n$ matrix is a function from $\FF^n$ to $\FF^m$. It is associative because all functions are associative.}

\ex{$2 \times 2$ matrix exercise}{Consider $\ZZ^{2 \times 2}$. Define a relation $A \sim B$ if there is an integer matrix $P$ whose determinant is one and $B = P^{-1}AP$. Note that if an integer matrix has a determinant 1 it is invertible and its inverse is also an integer matrix with determinant 1.

\begin{enumerate}
	\item Show that this is an equivalence relation.
	\item Show that two matrices with different determinants cannot be similar.
	\item Determine whether $\displaystyle \begin{bmatrix} 6 & 0 \\ 0 & 1\end{bmatrix}$ is similar to $\displaystyle \begin{bmatrix} 2 & 0 \\ 0 & 3\end{bmatrix}$. 
	\item Determine whether $\displaystyle \begin{bmatrix} 6 & 0 \\ 0 & 1\end{bmatrix}$ is similar to $\displaystyle \begin{bmatrix} 1 & 0 \\ 0 & 6\end{bmatrix}$. 
\end{enumerate}
\sol
\begin{enumerate}
	\item Reflexive: $A = P^{-1}AP$ for $P = I_2$. 
	
	Symmetric: $P^{-1}AP = P^{-1}BP$ for some $P$ with determinant 1. 
	
	Transitive: $B = P^{-1}_1AP_1 \land C = P^{-1}_2BP_2 \Rightarrow C = P^{-1}_2P^{-1}_1AP_1P_2 $
	\item Determinants are a multiplicative property. If $B = P^{-1}AP$ and $\det(B) \neq \det(A)$, then $\det(B) \neq 1 * \det(A) * 1$. 
	\item No, different JCF.
	\item Yes, same JCF.
\end{enumerate}
}



\subsection{Number Theory}
\nt{Know induction, division algorithm, GCD and Bezout's lemma, and Primes and the Fundmental Theorem of Arithmetic.}

\ex{Weak Induction}{Prove that $5 | n^5 -n$ for all $n$.

\begin{myproof} Proof by induction. 
	\begin{enumerate}
		\item $n = 1$ is true, $5 | 0$. 
		\item If it is true then $n=k$, show that it is true when $n = k+1$.
		
		$(k+1)^5 - (k+1) = k^5 + 5k^4 + 10k^3 + 10k^3 + 5k + 1 - (k+1) = (k^5 - k) + (5k^4 + 10k^3 + 10k^2 + 5k)$. 

		Both quantities are divisible by 5.
	\end{enumerate}
	Therefore, $5 | n^5 - n$ for all $n$. 
\end{myproof}}

\ex{Strong Induction}{Prove that every integer $n$ can be written as $n = d_11! + d_22! + \cdots + d_kk!$ for some $d_1, \ldots, d_k \leq k \in \ZZ$ and $k \geq 1$.

\begin{myproof}
	Strong induction. 
	
	Given $n$, chose $s$ s.t. $s! \leq n < (s+1)!$. Then we can write $n = q \cdot s! + r$. 
	\begin{enumerate}
		\item $q \leq s$ (if $q \geq s+1$, then $n \geq (s+1)!$, which goes against our claim)
		\item $r < s!$
	\end{enumerate}

	Assume that this is true for any $k < n$. Then we can write $n = q \cdot s! + r$ for some $r < s!$. Then we can write $r$ in the same format since it is true for all $k < n$. 
\end{myproof}}

\ex{Well-ordering}{Prove that given $a,b,b \neq 0$, there exists unique $q,r$ such that $a = qb + r$ and $0 \leq r < |b|$. 
\begin{myproof}
	Well-ordering.

	Consider all the integeres of the form $a - xb$ for $x \in \ZZ$. At least one of these is nonnegative. If $a > 0$, choose $x = 0$. If $a \leq 0$, then choose $x = -ab|b|$. So let the set of all negative $a-xb$ be nonempty. Let $q=x$ be the smallest. Define $r = a - qb$ so that $a = qb + r$ and $r < |b|$. 

	To prove uniqueness, consider two sets: $qr$ and $q'r'$. Then $qb + r = q'b + r'$ and $r < |b|$. Or, $(q-q')b=r'-r$. The absolute value of the RHS has to be between $1 - |b|$ and $|b| - 1$. This has to be 0 since its the only multipe of $b$ in that range. So $q-q' = 0$ and $q = q'$ and $r = r'$.
\end{myproof}}

\mlenma{Bezout's Lemma}{Given integers $a,b \neq 0$, their GCD can be written in the form $ra + sb$ for some $r,s$. }

\dfn{}{An integer is prime if it only has $1$ and itself as positive divisors.}

\nt{1 is not a prime.}

\mlenma{}{If $p$ is prime and $p|ab$, then either $p|a$ or $p|b$.}

\thm{Fundamental Theorem of Arithmetic}{Every integer greater than 1 is either a prime or can be written as a product of primes in a unique way.}

\section{Group Theory}
\subsection{Introduction to Groups}
\dfn{Binary Operation}{Given a set $S$, a \emph{binary operation} on $S$ is a function $S \times S \to S$. }

\dfn{Group}{A \emph{group} is a set $G$ with a binary operation $*$ such that for all $a,b,c \in G$, the following hold:
	\begin{enumerate}
		\item $(a*b)*c = a*(b*c)$ (associativity)
		\item $e*a = a*e = a$ (identity)
		\item $a*a^{-1} = e$ (inverse)
		\item $*$ is closed under $G$. 
	\end{enumerate}
}

\nt{A set that only has associativity and identity is called a \emph{monoid}.}

\nt{Examples of groups
\begin{itemize}
	\item $\ZZ, \RR, \RR^{3 \times 3}, \CC, \QQ$ with addition.
	\item $z \in \CC : |z| = 1$ with multiplication.
	\item $GL(2, \RR)$ with matrix multiplication. However, this is not abelian. 
	\item $D_4 =$ symmetries of a square.
	\item $D_2 =$ symmetries of a triangle.
	\item $U(n)$ with multiplication modulo $n$. 
\end{itemize}}

\noindent If we take a random group, say $U(5)$, then we can create a table for how the multiplication works:

\begin{center}
	\noindent\begin{tabular}{c | c c c c}
	$\cdot$ & 1 & 2 & 3 & 4  \\
	\cline{1-5}
	1 & 1 & 2 & 3 & 4 \\
	2 & 2 & 4 & 1 & 3  \\
	3 & 3 & 1 & 4 & 2 \\
	4 & 4 & 3 & 2 & 1 \\
	\end{tabular}
\end{center}

A table like this is called a \emph{Cayley Table}. Notice that this table is actually symmetric. This means that the group is \emph{commutative}, but more properly, \emph{abelian}.

\dfn{Abelian Group}{An \emph{abelian group}, $G$, is a group where $a*b = b*a$ for all $a,b \in G$.}

\subsection{Properties of Groups}
\thm{}{The identity element of a group is unique.}
\begin{myproof}
	Let $e_1$ and $e_2$ be the identity elements. Then $e_1*e_2 = e_2*e_1 = e_1$. So $e_1 = e_2$.
\end{myproof}

\thm{}{Each element has a unique inverse.}
\begin{myproof}
	Let $a^{-1}$ and $b$ both be inverses of $a$ then consider the product $baa^{-1}$. Then $b = be = b(aa^{-1}) = (ba)a^{-1} = ea^{-1}= a^{-1}$. So $b = a^{-1}$.
\end{myproof}

\cor{}{$(ab)^{-1} = b^{-1}a^{-1}$}
\begin{myproof}
	$abb^{-1}a^{-1} = aea^{-1} = aa^{-1} = e$.
\end{myproof}

\cor{}{$(a_1a_2a_3\ldots a_n)^{-1} = a_n^{-1}a_{n-1}^{-1}a_{n-2}^{-1}\ldots a_1$}
\begin{myproof}
	Induction from 1.2.1. 
\end{myproof}

\cor{}{$(a^{-1})^{-1} = a$}
\begin{myproof}
	$(a^{-1})^{-1}a^{-1} = e = aa^{-1}$, so by uniqueness of inverses...
\end{myproof}

\thm{}{Given any $a,b \in G$, the equations $ax = b$ and $ya = b$ have unique solutions, though not necessary equal.}
\begin{myproof}
	Let $x = a^{-1}b$ and $y = ba^{-1}$. Then $ax = a(a^{-1}b) = eb = b$ and $ya = ba^{-1}a = be = b$. To show uniqueness, consider $ax_1 = ax_2$ then left multiply by $a^{-1}$.
\end{myproof}

\cor{Cancellation Laws}{In any group $G$, if $ac=bc$, then $a=b$. And if $ca=cb$, then $a=b$. }
\begin{myproof}
	Right or left multiply by $c^{-1}$ for appropriate equation.
\end{myproof}

\nt{Proving that a group is associative from its Cayley digram takes too long. It is easier to show an isomorphism to a well-established group.}

\nt{Groups of order $n$:
	
\begin{itemize}
	\item 1: $\ZZ_1$
	\item 2: $\ZZ_2$
	\item 3: $\ZZ_3$
	\item 4: $\ZZ_4, V$
	\item 5: $\ZZ_5$
	\item 6: $D_3, \ZZ_6$
	\item 7: $\ZZ_7$
	\item 8: $\ZZ_8, \ZZ_2 \times \ZZ_4, \ZZ_2 \times \ZZ_2 \times \ZZ_2, D_4, H$
	\item 9: $\ZZ_9, \ZZ_3 \times \ZZ_3$
\end{itemize}}

\nt{A note on notation:

$a\cdot a = a^2, a \cdot a \cdot a = a^3 \ldots$ }

\dfn{Direct Product}{Given $G_1, G_2$ groups, then the direct proudct $G_1 \times G_2$ is the group of ordered pairs $(g_1, g_2)$ where $g_1 \in G_1$ and $g_2 \in G_2$. The operation is $(g_1, g_2) \cdot (h_1, h_2) = (g_1 \cdot h_1, g_2 \cdot h_2)$.}

\ex{}{$\{e\} \times G \cong G$}

\ex{}{$\ZZ_2 \times \ZZ_2 \cong V$}

\ex{}{$\ZZ_2 \times \ZZ_3 \cong Z_6$}

\thm{}{Let $(G, \circ, e)$ be a set with the binary operation $\circ$ and left identity $e$. Then assume each $x \in G$ has a left inverse such that $x^{-1} \circ x = e$. Then $G$ is a group.}

\mpf{Proof}{what is $xe=$?

Let $y = xe$. Then $x^{-1}y = x^{-1}(xe) = (x^{-1}x)e = e$. So $x^{-1}y  = e = x^{-1}x$. Multiply by $x^{-1^{-1}}$ to get $y = x$. Therefore, $e$ is a two-sided identity.

To show that $x^{-1}$, consider $z = x \circ x^{-1}$. Left multiply by $x^{-1}$ to get $x^{-1} \circ z = x^{-1} \circ (x \circ x^{-1}) = (x^{-1} \circ x) \circ x^{-1} = x^{-1}$. Left multiply both sides by $x^{-1^{-1}}$ to see that $e \circ z = z = e$. Therefore, $x^{-1}$ is a left inverse and $G$ is a group.}

\subsection{Subgroups}
\dfn{Subgroups}{Let $(G, \circ, e)$ be a group and let $H \subset G$. If $H$ is a group under the same operation $\circ$, then $H$ is a \emph{subgroup} of $G$. This is denoted as $H < G$. }

\nt{Having the same operation is critical. For example $GL(2) \subset \RR^{2\times 2}$, but $GL(2)$ is not a subgroup of $\RR^{2\times 2}$ because the operation is matrix multiplication, not addition.}

\mlenma{}{If $H \subset G$ and for any $h_1, h_2 \in H$, $h_1h_2^{-1} \in H$, then $H$ is a subgroup.}

\mpf{Proof}{ Following:
	
\begin{itemize}
	\item Choose $h_2 = h_1$, then $H \supset h_1h_1^{-1} = e$.
	\item Let $h_1 = e, h_2 = h$. Then $eh^{-1} = h^{-1} \in H$.
	\item $h_1h_2 = h_1(h_2^{-1})^{-1}$.
\end{itemize}}

\ex{Quarternion Units}{Let $Q_8 = \{ \pm 1, \pm i, \pm j, \pm k\}$. These function such that $i^2 = j^2 = k^2 = ijk = -1$. All the two element subgroups are $\{ \pm 1 \}$.}

\dfn{Cyclic Subgroup}{Given $a \in G$, the \emph{cyclic subgroup generated by a}, denoted $\langle a \rangle$, is the set $\{a^n : n \in  \ZZ\}$. The element $a$ is called the \emph{generator}. }

\ex{Cylic Subgroups}{
\begin{itemize}
	\item $\ZZ = \langle 1 \rangle$
	\item $\ZZ_7 = \langle 1 \rangle, \langle 5 \rangle$
	\item $\ZZ_{10} = \langle 1 \rangle, \langle 7 \rangle$
\end{itemize}}

\mprop{}{Every subgroup of $\ZZ$ is cyclic.}

\mpf{Addendum}{Any subgroup of any cyclic subgroup is itself cyclic.}

\nt{Some $U(n)$ groups are cyclic while others are not. They are cyclic if $n$ has primitive roots.}

\mlenma{}{Let $a \in G$, order of $a=n$. Then order of $a^k = \frac{n}{\text{gcd}(a,k)}$}

\mpf{Proof}{Let $b = a^k$. Order is the smallest number we can find such that $b^s = e$. Note that $b^s = a^{ks}$, so we need $n|ks$. Let $d = \text{gcd}(n, k)$. Then $n = dn'$ and $k =dk'$. Then we need $dn'$ to be a divisor of $sdk'$. So, $n'|sk'$. Since $n'$ and $k'$ are coprime, $n'|s$. Therefore, the smallest possible $s$ is $n' = n/\text{gcd}(a, k)$.}

\thm{}{A group has no proper nontrivial subgroups is and only if it is a cyclic group of prime order.}

\mpf{Proof}{Let $G = \langle a \rangle$ for any $a \in G$. What is the oder of $a$? If $a$ isn't prime, $a=xy$ and $y \neq 1$. Then $a^x$ has order $y$.}

\subsection{Permutations}

\dfn{Permutation}{A permutation is a bijection from a set $S$ to itself.}

\nt{All permutations of a set $A$ forms a group called $S_A$. This can be called either ``permutation on A'' or ``symmetric group of A''.

$|S_n| = n!$.}

\ex{Compositions and Cycles}{Given two permutations, it is not hard to multiply then. For example:

$$\begin{pmatrix} 1&2&3&4&5&6 \\ 2&4&3&6&5&1\end{pmatrix}
\begin{pmatrix} 1&2&3&4&5&6 \\ 4&3&5&1&6&2\end{pmatrix} = \begin{pmatrix} 1&2&3&4&5&6 \\ 4&5&3&2&6&1\end{pmatrix}$$}

\nt{This notation can be seen as quite cumbersome and redundant given the fact that the first row is always the same. To simplify this, we can use the following \emph{cycle} notation:

$$\begin{pmatrix} 1&2&3&4&5&6 \\ 4&5&3&2&6&1\end{pmatrix} = \begin{pmatrix}1&4&2&5&6 \end{pmatrix} \begin{pmatrix} 3 \end{pmatrix}$$

This is read as the permutation that sends 1 to 4 to 2 to 5 to 6 and 3 to 3.

The identity permutation is $\begin{pmatrix} 1&2&3&4&5&6 \end{pmatrix}$, which is annoying so mathematicians just say $e$. }

\mlenma{}{Disjoint cycles commute.}

\thm{}{Every permutation can be written as a product of disjoint cycles.}

\mpf{Proof}{Strong Induction:

Assume any permutation that moves $<n$ elements can be written. Consider $\sigma$ which has $n$ elements. Consider the set, which is called the orbit, of $\sigma$: $1, \sigma(1), \sigma^2(1) \ldots$. By the pigeonhole principle, this repeats. Cut off this set at the repeat of 1 and removed the curly braces and commas to get a cycle that 1 belongs to. }

\nt{The inverse of a cyclic is just the cycle backwards.}

\dfn{Transposition}{A transposition is a permutation that swaps just two elements. Also known as a ``swap'' or ``2-cycle''}

\mlenma{}{Any permutation may be written as a product of not disjoint transpositions.}

\mpf{Proof}{The cycle $\begin{pmatrix}A & B & C & \ldots & Y & Z \end{pmatrix} = (AZ)(AY)\ldots(AC)(AB)$.}

\mlenma{}{The following are true:

\begin{enumerate}
	\item $(AB) = (BA)$.
	\item $(AB)(AC) = (A \; B \; C)$
	\item $(AB)(CD) = (CD)(AB)$
	\item $(\ldots X \; Y \; Z \ldots)(AY) = (\ldots X \; Y \; A \; Z)$
	\item $(AY)(\ldots X \; Y \; Z \ldots) = (\ldots X \; A \; Y \; Z)$
	\item $(\ldots P \; Q  \; R \ldots X \; Y \; Z)(QY) = (\ldots P \; Q \; Z \ldots)(\; R \ldots X \; Y)$
	\item $\begin{pmatrix}A & B & C & \ldots & Y & Z \end{pmatrix} = (AZ)(AY)\ldots(AC)(AB)$
\end{enumerate}
}

\thm{}{Let $\sigma$ be a permutation. Then $\sigma$ can be written as a product of transpositions. Say $\sigma = \tau_n \tau_{n-1} \ldots \tau_1$. This permutation is not unique, but if we say that $\sigma = \tau_k \tau_{k-1} \ldots \tau_1$, then $k = n \pmod{2}$.}
\dfn{Parity}{Parity of $\sigma$ is even or odd as $k$ is.}

% \mpf{Proof}{Write sigma in two-row form. 
% $$\begin{pmatrix} 1 & 2 & \cdots & m \\ A & B & \cdots & M \end{pmatrix}$$ Create a function $D: S_m \to \ZZ^o$ which is defined as:

% $$d(\sigma) = \text{ add up the number of backward entries in the bottom row.}$$

% For example, $\displaystyle d\begin{pmatrix} 1 & 2 & 3 & 4 & 5 \\ 4 & 2 & 5 & 3 & 1 \end{pmatrix} = 3 +1 + 2 + 1 + 0 = 7$. $d$ is a well-defined function. 

% \textbf{Claim}: multiplying $\sigma$ by a transposition $\tau$ swaps two numbers in the bottom row changes $d$ by an even number.

% We do not have to consider the numbers before or after the numbers that were swapped because everything in front of them remains the same. We only have to consider the numbers in the middle chunk that are between the swapped numbers because those will affect $d$. 

% If a number is in between the swapped numbers positionally and numerically, $d$ will change by $\pm 2$ depending on if the larger swapped number is on the left or right. The swap itself will change $d$ by $\pm 1$ depending on if the larger swapped number is on the left or right.

% }

\thm{}{There are $n!/2$ odd permutations and $n!/2$ even permutations.}

\thm{}{The even permutations form a subgroup of $S_n$, called the \emph{alternating group}, denoted $A_n$.}

\nt{An \emph{alternating polynomial} is one that flips sign when you switch two of its elements. For example, $x^2 - y^2$ is alternating while $xy + yz + xz$ is not. The alternating group is the group of permutations that leave alternating polynomials invariant.}

\subsection{Generators}

\ex{Motivating Example}{The dihedral group, $D_4$, can be generated by two elements: $r_{90}$ and $f_v$. All rotations are certainly powers of $r_{90}$ and the other flips can be constructure by $f_v$ and $r_{90}$. Therefore, $r_{90}$ and $f_v$ are the \emph{generators} of $D_4$.}

\dfn{Generator}{A generator of a group is an element that generates the group.}

\mlenma{}{The transpositions $\{(1 \; 2), (2 \; 3), \ldots, (n-1 \; n)\}$ generate $S_n$. }

\thm{}{The transposition $\tau = (1 \; 2)$ and the cycle $\sigma = (1 \; 2 \; \ldots \; n)$ generate $S_n$. }

\dfn{Group Presentation}{A group presentation, $\langle g_1, g_2 \ldots, g_j | r_1, r_2, \ldots, r_k \rangle$ is a set of generators and relations. Each relation, $r_i$ is meant to simplify to $e$. }

\ex{Group Presentations}{
\begin{itemize}
	\item $\ZZ_6 = \langle a | a^6 \rangle$
	\item $D_4 = \langle r_{90}, f_v | r_{90}^4, f_v^2, r_{90}f_vr_{90}f_v\rangle$
\end{itemize}}

\subsection{Cosets}

\dfn{Cosets}{Let $H < G$ and $g \in G$. The \emph{left coset} of $H$ with representative $g$ is the set $gH = \{gh: h \in H\}$. The \emph{right coset} of $H$ with representative $g$ is the set $Hg = \{hg: h \in H\}$.}

\ex{Cosets}{
	Let $G = D_4$ and $H = \{e, f_1\}$. Then there are eight left cosets and eight right cosets of $H$, according to the eight elements of $D_4$, that could be the representative. They are listed out below:
	\begin{center}
		\includegraphics[width=8cm]{coset.png}
	\end{center}
}

\mlenma{}{Let $H<G$, then $H$ is a subgroup of $G$ and let $g_1,g_2$ be arbitrary elements of $G$. Then the following are equivalent:
\begin{enumerate}
	\item $g_1H = g_2H$
	\item $Hg_1^{-1} = Hg_2^{-1}$
	\item $g_1H \subset g_2H$
	\item $g_1 \in g_2H$
	\item $g_1^{-1}g_2 \in H$
\end{enumerate}}

\mpf{Proof}{We will prove that $1 \Longrightarrow 2 \Longrightarrow 3 \Longrightarrow 4 \Longrightarrow 5 \Longrightarrow 1$ so that the statements prove each other in a circular manner, so if any is true the rest become true.

$(1 \Longrightarrow 2)$ Consider a typical element $h g_1^{-1}$ of $H g_1^{-1}$. Its inverse is $g_1 h^{-1}$. Since $h \in H$ and $H$ is a subgroup, $h^{-1} \in H$, so $g_1 h^{-1} \in g_1 H$. Thus it is also in $g_2 H$, so can be written in the form $g_2 h^{\prime}$. So we have $\left(h g_1^{-1}\right)^{-1}=g_2 h^{\prime}$. Take the inverse on both sides, to find $h g_1^{-1}=h^{-1} g_2^{-1}$. Since $h^{\prime} \in H$ we also have $h^{\prime-1} \in H$, so this is a member of $H g_2^{-1}$. In other words, any member of $H g_1^{-1}$ is in $H g_2^{-1}$. The reverse inclusion is proven the same way, so the two sets must be equal to each other.

$(2 \Longrightarrow 3)$ Consider a typical element $g_1 h$ of $g_1 H$. Its inverse is $h^{-1} g_1^{-1} \in H g_1^{-1}=H g_2^{-1}$. So the inverse can be written as $h^{\prime} g_2^{-1}$. Then, reinverting both of these, $g_1 h=g_2 h^{\prime-1} \in g_2 H$. 

$(3 \Longrightarrow 4)$ Since $H$ is a subgroup, $e \in H$, so $g_1 e=g_1 \in g_1 H$. By subsets, it must be in $g_2 H$. 

$(4 \Longrightarrow 5)$ Since $g_1 \in g_2 H$ we know that we can write $g_1=g_2 h$. Rearranging this gives $g_1^{-1} g_2 h=e$ or $g_1^{-1} g_2=h^{-1}$. Since $H$ is a subgroups and $h \in H$, of course $h^{-1} \in H$.

$(5 \Longrightarrow 1)$ Let $g_2 h \in g_2 H$ be a typical element. Since $g_1^{-1} g_2 \in H$ we can choose $k \in H$ so that $g_1^{-1} g_2=k$. Then $g_1^{-1} g_2 h=k h$, or $g_2 h=g_1(k h) . H$ is a subgroup so contains product of its elements, and thus $g_1(k h) \in g_1 H$. Thus any element of $g_2 H$ is in $g_1 H$, or $g_2 H \subset g_1 H$.
Since $g_1^{-1} g_2$ is in $H$, so is its inverse $g_2^{-1} g_1$ so the argument of the previous paragraph may be repeated to show $g_1 H \subset g_2 H$.}

\thm{}{Left cosets $g_1H$ and $g_2H$ are either identical or disjoint. Also true for right cosets.}

\mpf{Proof}{Let $x \in g_1H \cap g_2H$. Then $x \in g_1H$ so therefore $xH = g_1H$. Same argument for $xH = g_2H$. }

\mlenma{}{There is a one-to-one correspondence between left and right cosets.}

\mpf{Proof}{Consider the map $gH \to Hg^{-1}$. It is a well-defined map by statements 1 and 2 of the lemma which also show why this map is one-to-one and onto.}

\nt{$xH = yH \not\Leftrightarrow Hx = Hy$}

\dfn{Index}{The number of cosets of H in G (right or left, since these numbers are the same by the lemma) is called the index of $H$ in $G$ and is denoted by $[G:H]$.}

\mlenma{}{The function $f_g :H \to gH$ given by $f_g(x) = gx$ is a bijection between the elements of $H$ and the elements of $gH$.}

\thm{Lagrange's Theorem}{If $G$ is a finite group and $H$ is a subgroup of $G$, then the following equation is satisfied: $$|G| = [G : H] |H|.$$ }

\mpf{Proof}{Cosets are equinumerous with $H$ and either identical or disjoint, we're done!}

\cor{}{$|H|$ divides $|G|$. }

\cor{}{Groups of prime order are necessarily cyclic, and each non-identity elements are the generators.}

\thm{}{Let $K<H<G$. Then $K < G$, and $[G:K] = [G:H][H:K]$.}

\thm{}{If you have an abelian group $G$ whose order is the product $mn$ where $m$ and $n$ are relativity prime, then $G$ is cyclic. Its generator is $ab$ where $a$ is an element with order $m$ and $b$ is an element with order $n$. }

\thm{Euler}{If $a$ is relatively prime to $n$, then $a^{\phi(n)} \equiv 1 \pmod{n}.$}

\mpf{Proof}{$|U(n)| = \phi(n)$ so the order of every element is a divisor of $\phi(n)$.}

\thm{Fermat's Little Theorem}{If $p$ is a prime number, then $a^p \equiv a \pmod{p}$.}

\mpf{Proof}{If $p$ is a divisor of $a$ then both sides are congruent to zero modulo $p$. Otherwise $\phi(p) = p-1$ and the result obtains by multiplying both sides of the result of Euler's Theorem by $a$.}

\nt{While Lagrange eliminates subgroups of certain orders (order that is relatively prime to the order of the parent group), it does not guarantee the existence of any order. }

\ex{$A_4$}{$A_4$ has 12 elements, but does not have any subgroups of size six. For assume there were such a subgroup $H$. Now $H$ would have only two left cosets-itself and $g H$ for some $g$ not in $H$. But it also only has two right cosets. Since cosets are either disjoint or identical, the right coset of $H$ other then $H$ itself must also be the left coset. That is, $g H=H g$. So for any $h \in H$, there is an $h^{\prime}$ so that $g h=h^{\prime} g$. Another way of saying this is that $g h g^{-1}=h^{\prime} \in H$ for any $h \in H$ and any $g \in G$.

Now consider the three-cycles in $A_4$. There are eight of them. So by the pigeonhole principle, there must be a three-cycle in $H$. Without loss of generality assume $(123) \in H$. By the result of the previous paragraph, $(124)(123)(142)=(243) \in H$. Also, (234) $(123)(243)=$ $(134) \in H$. In fact, all three-cycles must be in $H$. But then $H$ has more than just six elements!}

\thm{}{If $\sigma \in S_n$ is a cycle of length $k$, then $\tau \in S_n$ is also a cycle of length $k$ iff $\tau = g\sigma g^{-1}$ for some $g \in S_n$. }

\cor{}{Two permutations have the same cycle structure if and only if they are conjugates.}

\section{Group Theory}
\subsection{Isomorphisms}
\dfn{Isomorphic}{Let $(G, \cdot)$ and $(H, \circ)$ be groups. We say that $G$ and $H$ are isomorphic if there is a bijection $f:G \to H$ such that $f(g_1 \cdot g_2) = f(g_1) \circ f(g_2)$ for all $g_1, g_2 \in G$.}

\ex{}{$\phi: \ZZ_4 \to \{i, -1, -i, 1\}$ defined by $\phi(n) = i^n$. This is obviously a one-to-one and onto mapping, and trades addition in $\ZZ_4$ for multiplication in $\CC$. }

\thm{}{If $\phi$ is an isomorphism, then so is $\phi^{-1}$.}

\cor{}{If $\phi: G \to H$ is an isomorphism, then:
\begin{itemize}
	\item $\phi(e_G) = e_H$
	\item $\phi(g^{-1}) = \phi(g)^{-1}$
	\item $\phi(g^k) = (\phi(g))^k$
\end{itemize}}

\thm{Cayley's Theorem}{Every group is isomorphic to a permutation group.}

\dfn{Direct Product}{Let $(G, \cdot, e)$ and $(H, \circ, i)$ be groups. The \emph{direct product} of $G$ and $H$, denoted as $G \times H$, is the group whose elements take the form $(g, h)$ for $g \in G$ and $h \in H$. The operation is defined as $(g_1, h_1)(g_2, h_2) = (g_1 \cdot g_2, h_1 \circ h_2)$. The identity element is $(e, i)$.}

\mlenma{}{If $g$ has order $m$ in $G$ and $h$ has order $n$ in $H$, then $(g, h)$ has order lcm($m,n$) in $G \times H$.}

\cor{}{If $m$ and $n$ are relatively prime, then $\ZZ_m \times \ZZ_n$ is isomorphic to $\ZZ_{mn}$.}

\dfn{Internal Direct Product}{Let $G$ be a group with subgroups $H$ and $K$ that fit together as follows:
\begin{itemize}
	\item $H \cap K = \{e\}$
	\item $G = HK = \{hk : h \in H, k \in K\}$
	\item $hk = kh$ for any $h \in H$ and $k \in K$
\end{itemize}
Then $G$ is called the \emph{internal direct product} of $H$ and $K$, and is isomorphic to $H \times K$.
}

% \section{Random Examples}
% \begin{myproof}By openness of $V$, $x\in B_r(u)\subset V$
% 	\begin{center}
% 		\begin{tikzpicture}
% 			\draw[red] (0,0) circle [x radius=3.5cm, y radius=2cm] ;
% 			\draw (3,1.6) node[red]{$V$};
% 			\draw [blue] (1,0) circle (1.45cm) ;
% 			\filldraw[blue] (1,0) circle (1pt) node[anchor=north]{$u$};
% 			\draw (2.9,0.4) node[blue]{$B_r(u)$};
% 			\draw [green!40!black] (1.7,0) circle (0.5cm) node [yshift=0.7cm]{$B_{\delta}(x)$} ;
% 			\filldraw[green!40!black] (1.7,0) circle (1pt) node[anchor=west]{$x$};
% 		\end{tikzpicture}
% 	\end{center}
% \end{myproof}

% \cor{}{By the result of the proof, we can then show...}

% \mprop{}{$1 + 1 = 2$.}

% \section{Random}
% \dfn{Normed Linear Space and Norm $\boldsymbol{\|\cdot\|}$}{Let $V$ be a vector space over $\bbR$ (or $\bbC$). A norm on $V$ is function $\|\cdot\|\ V\to \bbR_{\geq 0}$ satisfying \begin{enumerate}[label=\bfseries\tiny\protect\circled{\small\arabic*}]
% 		\item \label{n:1}$\|x\|=0 \iff x=0$ $\forall$ $x\in V$
% 		\item \label{n:2}	$\|\lambda x\|=|\lambda|\|x\|$ $\forall$ $\lambda\in\bbR$(or $\bbC$), $x\in V$
% 		\item \label{n:3} $\|x+y\| \leq \|x\|+\|y\|$ $\forall$ $x,y\in V$ (Triangle Inequality/Subadditivity)
% 	\end{enumerate}And $V$ is called a normed linear space.

% 	$\bullet $ Same definition works with $V$ a vector space over $\bbC$ (again $\|\cdot\|\to\bbR_{\geq 0}$) where \ref{n:2} becomes $\|\lambda x\|=|\lambda|\|x\|$ $\forall$ $\lambda\in\bbC$, $x\in V$, where for $\lambda=a+ib$, $|\lambda|=\sqrt{a^2+b^2}$ }




% \section{Algorithms}
% \begin{algorithm}[H]
% \KwIn{This is some input}
% \KwOut{This is some output}
% \SetAlgoLined
% \SetNoFillComment
% \tcc{This is a comment}
% \vspace{3mm}
% some code here\;
% $x \leftarrow 0$\;
% $y \leftarrow 0$\;
% \uIf{$ x > 5$} {
%     x is greater than 5 \tcp*{This is also a comment}
% }
% \Else {
%     x is less than or equal to 5\;
% }
% \ForEach{y in 0..5} {
%     $y \leftarrow y + 1$\;
% }
% \For{$y$ in $0..5$} {
%     $y \leftarrow y - 1$\;
% }
% \While{$x > 5$} {
%     $x \leftarrow x - 1$\;
% }
% \Return Return something here\;
% \caption{what}
% \end{algorithm}

\end{document}