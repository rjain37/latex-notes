\documentclass{report}

%%%%%%%%%%%%%%%%%%%%%%%%%%%%%%%%%
% PACKAGE IMPORTS
%%%%%%%%%%%%%%%%%%%%%%%%%%%%%%%%%


\usepackage[tmargin=2cm,rmargin=1in,lmargin=1in,margin=0.85in,bmargin=2cm,footskip=.2in]{geometry}
\usepackage{amsmath,amsfonts,amsthm,amssymb,mathtools}
\usepackage[varbb]{newpxmath}
\usepackage{xfrac}
\usepackage[makeroom]{cancel}
\usepackage{mathtools}
\usepackage{bookmark}
\usepackage{enumitem}
\usepackage{hyperref,theoremref}
\hypersetup{
	pdftitle={Assignment},
	colorlinks=true, linkcolor=doc!90,
	bookmarksnumbered=true,
	bookmarksopen=true
}
\usepackage[most,many,breakable]{tcolorbox}
\usepackage{xcolor}
\usepackage{varwidth}
\usepackage{varwidth}
\usepackage{etoolbox}
%\usepackage{authblk}
\usepackage{nameref}
\usepackage{multicol,array}
\usepackage{tikz-cd}
\usepackage[ruled,vlined,linesnumbered]{algorithm2e}
\usepackage{comment} % enables the use of multi-line comments (\ifx \fi) 
\usepackage{import}
\usepackage{xifthen}
\usepackage{pdfpages}
\usepackage{transparent}

% \usepackage{extsizes}

\newcommand\mycommfont[1]{\footnotesize\ttfamily\textcolor{blue}{#1}}
\SetCommentSty{mycommfont}
\newcommand{\incfig}[1]{%
    \def\svgwidth{\columnwidth}
    \import{./figures/}{#1.pdf_tex}
}

\usepackage{tikzsymbols}
\renewcommand\qedsymbol{$\Laughey$}


%\usepackage{import}
%\usepackage{xifthen}
%\usepackage{pdfpages}
%\usepackage{transparent}


%%%%%%%%%%%%%%%%%%%%%%%%%%%%%%
% SELF MADE COLORS
%%%%%%%%%%%%%%%%%%%%%%%%%%%%%%



\definecolor{myg}{RGB}{56, 140, 70}
\definecolor{myb}{RGB}{45, 111, 177}
\definecolor{myr}{RGB}{199, 68, 64}
\definecolor{mytheorembg}{HTML}{F2F2F9}
\definecolor{mytheoremfr}{HTML}{00007B}
\definecolor{mylenmabg}{HTML}{FFFAF8}
\definecolor{mylenmafr}{HTML}{983b0f}
\definecolor{mypropbg}{HTML}{f2fbfc}
\definecolor{mypropfr}{HTML}{191971}
\definecolor{myexamplebg}{HTML}{F2FBF8}
\definecolor{myexamplefr}{HTML}{88D6D1}
\definecolor{myexampleti}{HTML}{2A7F7F}
\definecolor{mydefinitbg}{HTML}{E5E5FF}
\definecolor{mydefinitfr}{HTML}{3F3FA3}
\definecolor{notesgreen}{RGB}{0,162,0}
\definecolor{myp}{RGB}{197, 92, 212}
\definecolor{mygr}{HTML}{2C3338}
\definecolor{myred}{RGB}{127,0,0}
\definecolor{myyellow}{RGB}{169,121,69}
\definecolor{myexercisebg}{HTML}{F2FBF8}
\definecolor{myexercisefg}{HTML}{88D6D1}


%%%%%%%%%%%%%%%%%%%%%%%%%%%%
% TCOLORBOX SETUPS
%%%%%%%%%%%%%%%%%%%%%%%%%%%%

\setlength{\parindent}{1cm}
%================================
% THEOREM BOX
%================================

\tcbuselibrary{theorems,skins,hooks}
\newtcbtheorem[number within=section]{Theorem}{Theorem}
{%
	enhanced,
	breakable,
	colback = mytheorembg,
	frame hidden,
	boxrule = 0sp,
	borderline west = {2pt}{0pt}{mytheoremfr},
	sharp corners,
	detach title,
	before upper = \tcbtitle\par\smallskip,
	coltitle = mytheoremfr,
	fonttitle = \bfseries\sffamily,
	description font = \mdseries,
	separator sign none,
	segmentation style={solid, mytheoremfr},
}
{th}

\tcbuselibrary{theorems,skins,hooks}
\newtcbtheorem[number within=chapter]{theorem}{Theorem}
{%
	enhanced,
	breakable,
	colback = mytheorembg,
	frame hidden,
	boxrule = 0sp,
	borderline west = {2pt}{0pt}{mytheoremfr},
	sharp corners,
	detach title,
	before upper = \tcbtitle\par\smallskip,
	coltitle = mytheoremfr,
	fonttitle = \bfseries\sffamily,
	description font = \mdseries,
	separator sign none,
	segmentation style={solid, mytheoremfr},
}
{th}


\tcbuselibrary{theorems,skins,hooks}
\newtcolorbox{Theoremcon}
{%
	enhanced
	,breakable
	,colback = mytheorembg
	,frame hidden
	,boxrule = 0sp
	,borderline west = {2pt}{0pt}{mytheoremfr}
	,sharp corners
	,description font = \mdseries
	,separator sign none
}

%================================
% Corollery
%================================
\tcbuselibrary{theorems,skins,hooks}
\newtcbtheorem[number within=section]{Corollary}{Corollary}
{%
	enhanced
	,breakable
	,colback = myp!10
	,frame hidden
	,boxrule = 0sp
	,borderline west = {2pt}{0pt}{myp!85!black}
	,sharp corners
	,detach title
	,before upper = \tcbtitle\par\smallskip
	,coltitle = myp!85!black
	,fonttitle = \bfseries\sffamily
	,description font = \mdseries
	,separator sign none
	,segmentation style={solid, myp!85!black}
}
{th}
\tcbuselibrary{theorems,skins,hooks}
\newtcbtheorem[number within=chapter]{corollary}{Corollary}
{%
	enhanced
	,breakable
	,colback = myp!10
	,frame hidden
	,boxrule = 0sp
	,borderline west = {2pt}{0pt}{myp!85!black}
	,sharp corners
	,detach title
	,before upper = \tcbtitle\par\smallskip
	,coltitle = myp!85!black
	,fonttitle = \bfseries\sffamily
	,description font = \mdseries
	,separator sign none
	,segmentation style={solid, myp!85!black}
}
{th}


%================================
% LENMA
%================================

\tcbuselibrary{theorems,skins,hooks}
\newtcbtheorem[number within=section]{Lenma}{Lenma}
{%
	enhanced,
	breakable,
	colback = mylenmabg,
	frame hidden,
	boxrule = 0sp,
	borderline west = {2pt}{0pt}{mylenmafr},
	sharp corners,
	detach title,
	before upper = \tcbtitle\par\smallskip,
	coltitle = mylenmafr,
	fonttitle = \bfseries\sffamily,
	description font = \mdseries,
	separator sign none,
	segmentation style={solid, mylenmafr},
}
{th}

\tcbuselibrary{theorems,skins,hooks}
\newtcbtheorem[number within=chapter]{lenma}{Lenma}
{%
	enhanced,
	breakable,
	colback = mylenmabg,
	frame hidden,
	boxrule = 0sp,
	borderline west = {2pt}{0pt}{mylenmafr},
	sharp corners,
	detach title,
	before upper = \tcbtitle\par\smallskip,
	coltitle = mylenmafr,
	fonttitle = \bfseries\sffamily,
	description font = \mdseries,
	separator sign none,
	segmentation style={solid, mylenmafr},
}
{th}


%================================
% PROPOSITION
%================================

\tcbuselibrary{theorems,skins,hooks}
\newtcbtheorem[number within=section]{Prop}{Proposition}
{%
	enhanced,
	breakable,
	colback = mypropbg,
	frame hidden,
	boxrule = 0sp,
	borderline west = {2pt}{0pt}{mypropfr},
	sharp corners,
	detach title,
	before upper = \tcbtitle\par\smallskip,
	coltitle = mypropfr,
	fonttitle = \bfseries\sffamily,
	description font = \mdseries,
	separator sign none,
	segmentation style={solid, mypropfr},
}
{th}

\tcbuselibrary{theorems,skins,hooks}
\newtcbtheorem[number within=chapter]{prop}{Proposition}
{%
	enhanced,
	breakable,
	colback = mypropbg,
	frame hidden,
	boxrule = 0sp,
	borderline west = {2pt}{0pt}{mypropfr},
	sharp corners,
	detach title,
	before upper = \tcbtitle\par\smallskip,
	coltitle = mypropfr,
	fonttitle = \bfseries\sffamily,
	description font = \mdseries,
	separator sign none,
	segmentation style={solid, mypropfr},
}
{th}


%================================
% CLAIM
%================================

\tcbuselibrary{theorems,skins,hooks}
\newtcbtheorem[number within=section]{claim}{Claim}
{%
	enhanced
	,breakable
	,colback = myg!10
	,frame hidden
	,boxrule = 0sp
	,borderline west = {2pt}{0pt}{myg}
	,sharp corners
	,detach title
	,before upper = \tcbtitle\par\smallskip
	,coltitle = myg!85!black
	,fonttitle = \bfseries\sffamily
	,description font = \mdseries
	,separator sign none
	,segmentation style={solid, myg!85!black}
}
{th}



%================================
% Exercise
%================================

\tcbuselibrary{theorems,skins,hooks}
\newtcbtheorem[number within=section]{Exercise}{Exercise}
{%
	enhanced,
	breakable,
	colback = myexercisebg,
	frame hidden,
	boxrule = 0sp,
	borderline west = {2pt}{0pt}{myexercisefg},
	sharp corners,
	detach title,
	before upper = \tcbtitle\par\smallskip,
	coltitle = myexercisefg,
	fonttitle = \bfseries\sffamily,
	description font = \mdseries,
	separator sign none,
	segmentation style={solid, myexercisefg},
}
{th}

\tcbuselibrary{theorems,skins,hooks}
\newtcbtheorem[number within=chapter]{exercise}{Exercise}
{%
	enhanced,
	breakable,
	colback = myexercisebg,
	frame hidden,
	boxrule = 0sp,
	borderline west = {2pt}{0pt}{myexercisefg},
	sharp corners,
	detach title,
	before upper = \tcbtitle\par\smallskip,
	coltitle = myexercisefg,
	fonttitle = \bfseries\sffamily,
	description font = \mdseries,
	separator sign none,
	segmentation style={solid, myexercisefg},
}
{th}

%================================
% EXAMPLE BOX
%================================

\newtcbtheorem[number within=section]{Example}{Example}
{%
	colback = myexamplebg
	,breakable
	,colframe = myexamplefr
	,coltitle = myexampleti
	,boxrule = 1pt
	,sharp corners
	,detach title
	,before upper=\tcbtitle\par\smallskip
	,fonttitle = \bfseries
	,description font = \mdseries
	,separator sign none
	,description delimiters parenthesis
}
{ex}

\newtcbtheorem[number within=chapter]{example}{Example}
{%
	colback = myexamplebg
	,breakable
	,colframe = myexamplefr
	,coltitle = myexampleti
	,boxrule = 1pt
	,sharp corners
	,detach title
	,before upper=\tcbtitle\par\smallskip
	,fonttitle = \bfseries
	,description font = \mdseries
	,separator sign none
	,description delimiters parenthesis
}
{ex}

%================================
% DEFINITION BOX
%================================

\newtcbtheorem[number within=section]{Definition}{Definition}{enhanced,
	before skip=2mm,after skip=2mm, colback=red!5,colframe=red!80!black,boxrule=0.5mm,
	attach boxed title to top left={xshift=1cm,yshift*=1mm-\tcboxedtitleheight}, varwidth boxed title*=-3cm,
	boxed title style={frame code={
					\path[fill=tcbcolback]
					([yshift=-1mm,xshift=-1mm]frame.north west)
					arc[start angle=0,end angle=180,radius=1mm]
					([yshift=-1mm,xshift=1mm]frame.north east)
					arc[start angle=180,end angle=0,radius=1mm];
					\path[left color=tcbcolback!60!black,right color=tcbcolback!60!black,
						middle color=tcbcolback!80!black]
					([xshift=-2mm]frame.north west) -- ([xshift=2mm]frame.north east)
					[rounded corners=1mm]-- ([xshift=1mm,yshift=-1mm]frame.north east)
					-- (frame.south east) -- (frame.south west)
					-- ([xshift=-1mm,yshift=-1mm]frame.north west)
					[sharp corners]-- cycle;
				},interior engine=empty,
		},
	fonttitle=\bfseries,
	title={#2},#1}{def}
\newtcbtheorem[number within=chapter]{definition}{Definition}{enhanced,
	before skip=2mm,after skip=2mm, colback=red!5,colframe=red!80!black,boxrule=0.5mm,
	attach boxed title to top left={xshift=1cm,yshift*=1mm-\tcboxedtitleheight}, varwidth boxed title*=-3cm,
	boxed title style={frame code={
					\path[fill=tcbcolback]
					([yshift=-1mm,xshift=-1mm]frame.north west)
					arc[start angle=0,end angle=180,radius=1mm]
					([yshift=-1mm,xshift=1mm]frame.north east)
					arc[start angle=180,end angle=0,radius=1mm];
					\path[left color=tcbcolback!60!black,right color=tcbcolback!60!black,
						middle color=tcbcolback!80!black]
					([xshift=-2mm]frame.north west) -- ([xshift=2mm]frame.north east)
					[rounded corners=1mm]-- ([xshift=1mm,yshift=-1mm]frame.north east)
					-- (frame.south east) -- (frame.south west)
					-- ([xshift=-1mm,yshift=-1mm]frame.north west)
					[sharp corners]-- cycle;
				},interior engine=empty,
		},
	fonttitle=\bfseries,
	title={#2},#1}{def}



%================================
% Solution BOX
%================================

\makeatletter
\newtcbtheorem{question}{Question}{enhanced,
	breakable,
	colback=white,
	colframe=myb!80!black,
	attach boxed title to top left={yshift*=-\tcboxedtitleheight},
	fonttitle=\bfseries,
	title={#2},
	boxed title size=title,
	boxed title style={%
			sharp corners,
			rounded corners=northwest,
			colback=tcbcolframe,
			boxrule=0pt,
		},
	underlay boxed title={%
			\path[fill=tcbcolframe] (title.south west)--(title.south east)
			to[out=0, in=180] ([xshift=5mm]title.east)--
			(title.center-|frame.east)
			[rounded corners=\kvtcb@arc] |-
			(frame.north) -| cycle;
		},
	#1
}{def}
\makeatother

%================================
% SOLUTION BOX
%================================

\makeatletter
\newtcolorbox{solution}{enhanced,
	breakable,
	colback=white,
	colframe=myg!80!black,
	attach boxed title to top left={yshift*=-\tcboxedtitleheight},
	title=Solution,
	boxed title size=title,
	boxed title style={%
			sharp corners,
			rounded corners=northwest,
			colback=tcbcolframe,
			boxrule=0pt,
		},
	underlay boxed title={%
			\path[fill=tcbcolframe] (title.south west)--(title.south east)
			to[out=0, in=180] ([xshift=5mm]title.east)--
			(title.center-|frame.east)
			[rounded corners=\kvtcb@arc] |-
			(frame.north) -| cycle;
		},
}
\makeatother

%================================
% Question BOX
%================================

\makeatletter
\newtcbtheorem{qstion}{Question}{enhanced,
	breakable,
	colback=white,
	colframe=mygr,
	attach boxed title to top left={yshift*=-\tcboxedtitleheight},
	fonttitle=\bfseries,
	title={#2},
	boxed title size=title,
	boxed title style={%
			sharp corners,
			rounded corners=northwest,
			colback=tcbcolframe,
			boxrule=0pt,
		},
	underlay boxed title={%
			\path[fill=tcbcolframe] (title.south west)--(title.south east)
			to[out=0, in=180] ([xshift=5mm]title.east)--
			(title.center-|frame.east)
			[rounded corners=\kvtcb@arc] |-
			(frame.north) -| cycle;
		},
	#1
}{def}
\makeatother

\newtcbtheorem[number within=chapter]{wconc}{Wrong Concept}{
	breakable,
	enhanced,
	colback=white,
	colframe=myr,
	arc=0pt,
	outer arc=0pt,
	fonttitle=\bfseries\sffamily\large,
	colbacktitle=myr,
	attach boxed title to top left={},
	boxed title style={
			enhanced,
			skin=enhancedfirst jigsaw,
			arc=3pt,
			bottom=0pt,
			interior style={fill=myr}
		},
	#1
}{def}



%================================
% NOTE BOX
%================================

\usetikzlibrary{arrows,calc,shadows.blur}
\tcbuselibrary{skins}
\newtcolorbox{note}[1][]{%
	enhanced jigsaw,
	colback=gray!20!white,%
	colframe=gray!80!black,
	size=small,
	boxrule=1pt,
	title=\textbf{Note:},
	halign title=flush center,
	coltitle=black,
	breakable,
	drop shadow=black!50!white,
	attach boxed title to top left={xshift=1cm,yshift=-\tcboxedtitleheight/2,yshifttext=-\tcboxedtitleheight/2},
	minipage boxed title=1.5cm,
	boxed title style={%
			colback=white,
			size=fbox,
			boxrule=1pt,
			boxsep=2pt,
			underlay={%
					\coordinate (dotA) at ($(interior.west) + (-0.5pt,0)$);
					\coordinate (dotB) at ($(interior.east) + (0.5pt,0)$);
					\begin{scope}
						\clip (interior.north west) rectangle ([xshift=3ex]interior.east);
						\filldraw [white, blur shadow={shadow opacity=60, shadow yshift=-.75ex}, rounded corners=2pt] (interior.north west) rectangle (interior.south east);
					\end{scope}
					\begin{scope}[gray!80!black]
						\fill (dotA) circle (2pt);
						\fill (dotB) circle (2pt);
					\end{scope}
				},
		},
	#1,
}

%%%%%%%%%%%%%%%%%%%%%%%%%%%%%%
% SELF MADE COMMANDS
%%%%%%%%%%%%%%%%%%%%%%%%%%%%%%


\newcommand{\thm}[2]{\begin{Theorem}{#1}{}#2\end{Theorem}}
\newcommand{\cor}[2]{\begin{Corollary}{#1}{}#2\end{Corollary}}
\newcommand{\mlenma}[2]{\begin{Lenma}{#1}{}#2\end{Lenma}}
\newcommand{\mprop}[2]{\begin{Prop}{#1}{}#2\end{Prop}}
\newcommand{\clm}[3]{\begin{claim}{#1}{#2}#3\end{claim}}
\newcommand{\wc}[2]{\begin{wconc}{#1}{}\setlength{\parindent}{1cm}#2\end{wconc}}
\newcommand{\thmcon}[1]{\begin{Theoremcon}{#1}\end{Theoremcon}}
\newcommand{\ex}[2]{\begin{Example}{#1}{}#2\end{Example}}
\newcommand{\dfn}[2]{\begin{Definition}[colbacktitle=red!75!black]{#1}{}#2\end{Definition}}
\newcommand{\dfnc}[2]{\begin{definition}[colbacktitle=red!75!black]{#1}{}#2\end{definition}}
\newcommand{\qs}[2]{\begin{question*}{#1}{}#2\end{question*}}
\newcommand{\mpf}[2]{\begin{myproof}[#1]#2\end{myproof}}
\newcommand{\nt}[1]{\begin{note}#1\end{note}}

\newcommand*\circled[1]{\tikz[baseline=(char.base)]{
		\node[shape=circle,draw,inner sep=1pt] (char) {#1};}}
\newcommand\getcurrentref[1]{%
	\ifnumequal{\value{#1}}{0}
	{??}
	{\the\value{#1}}%
}
\newcommand{\getCurrentSectionNumber}{\getcurrentref{section}}
\newenvironment{myproof}[1][\proofname]{%
	\proof[\bfseries #1: ]%
}{\endproof}

\newcommand{\mclm}[2]{\begin{myclaim}[#1]#2\end{myclaim}}
\newenvironment{myclaim}[1][\claimname]{\proof[\bfseries #1: ]}{}

\newcounter{mylabelcounter}

\makeatletter
\newcommand{\setword}[2]{%
	\phantomsection
	#1\def\@currentlabel{\unexpanded{#1}}\label{#2}%
}
\makeatother




\tikzset{
	symbol/.style={
			draw=none,
			every to/.append style={
					edge node={node [sloped, allow upside down, auto=false]{$#1$}}}
		}
}


% deliminators
\DeclarePairedDelimiter{\abs}{\lvert}{\rvert}
\DeclarePairedDelimiter{\norm}{\lVert}{\rVert}

\DeclarePairedDelimiter{\ceil}{\lceil}{\rceil}
\DeclarePairedDelimiter{\floor}{\lfloor}{\rfloor}
\DeclarePairedDelimiter{\round}{\lfloor}{\rceil}

\newsavebox\diffdbox
\newcommand{\slantedromand}{{\mathpalette\makesl{d}}}
\newcommand{\makesl}[2]{%
\begingroup
\sbox{\diffdbox}{$\mathsurround=0pt#1\mathrm{#2}$}%
\pdfsave
\pdfsetmatrix{1 0 0.2 1}%
\rlap{\usebox{\diffdbox}}%
\pdfrestore
\hskip\wd\diffdbox
\endgroup
}
\newcommand{\dd}[1][]{\ensuremath{\mathop{}\!\ifstrempty{#1}{%
\slantedromand\@ifnextchar^{\hspace{0.2ex}}{\hspace{0.1ex}}}%
{\slantedromand\hspace{0.2ex}^{#1}}}}
\ProvideDocumentCommand\dv{o m g}{%
  \ensuremath{%
    \IfValueTF{#3}{%
      \IfNoValueTF{#1}{%
        \frac{\dd #2}{\dd #3}%
      }{%
        \frac{\dd^{#1} #2}{\dd #3^{#1}}%
      }%
    }{%
      \IfNoValueTF{#1}{%
        \frac{\dd}{\dd #2}%
      }{%
        \frac{\dd^{#1}}{\dd #2^{#1}}%
      }%
    }%
  }%
}
\providecommand*{\pdv}[3][]{\frac{\partial^{#1}#2}{\partial#3^{#1}}}
%  - others
\DeclareMathOperator{\Lap}{\mathcal{L}}
\DeclareMathOperator{\Var}{Var} % varience
\DeclareMathOperator{\Cov}{Cov} % covarience
\DeclareMathOperator{\E}{E} % expected

% Since the amsthm package isn't loaded

% I prefer the slanted \leq
\let\oldleq\leq % save them in case they're every wanted
\let\oldgeq\geq
\renewcommand{\leq}{\leqslant}
\renewcommand{\geq}{\geqslant}

% % redefine matrix env to allow for alignment, use r as default
% \renewcommand*\env@matrix[1][r]{\hskip -\arraycolsep
%     \let\@ifnextchar\new@ifnextchar
%     \array{*\c@MaxMatrixCols #1}}


%\usepackage{framed}
%\usepackage{titletoc}
%\usepackage{etoolbox}
%\usepackage{lmodern}


%\patchcmd{\tableofcontents}{\contentsname}{\sffamily\contentsname}{}{}

%\renewenvironment{leftbar}
%{\def\FrameCommand{\hspace{6em}%
%		{\color{myyellow}\vrule width 2pt depth 6pt}\hspace{1em}}%
%	\MakeFramed{\parshape 1 0cm \dimexpr\textwidth-6em\relax\FrameRestore}\vskip2pt%
%}
%{\endMakeFramed}

%\titlecontents{chapter}
%[0em]{\vspace*{2\baselineskip}}
%{\parbox{4.5em}{%
%		\hfill\Huge\sffamily\bfseries\color{myred}\thecontentspage}%
%	\vspace*{-2.3\baselineskip}\leftbar\textsc{\small\chaptername~\thecontentslabel}\\\sffamily}
%{}{\endleftbar}
%\titlecontents{section}
%[8.4em]
%{\sffamily\contentslabel{3em}}{}{}
%{\hspace{0.5em}\nobreak\itshape\color{myred}\contentspage}
%\titlecontents{subsection}
%[8.4em]
%{\sffamily\contentslabel{3em}}{}{}  
%{\hspace{0.5em}\nobreak\itshape\color{myred}\contentspage}



%%%%%%%%%%%%%%%%%%%%%%%%%%%%%%%%%%%%%%%%%%%
% TABLE OF CONTENTS
%%%%%%%%%%%%%%%%%%%%%%%%%%%%%%%%%%%%%%%%%%%

\usepackage{tikz}
\definecolor{doc}{RGB}{0,60,110}
\usepackage{titletoc}
\contentsmargin{0cm}
\titlecontents{chapter}[3.7pc]
{\addvspace{30pt}%
	\begin{tikzpicture}[remember picture, overlay]%
		\draw[fill=doc!60,draw=doc!60] (-7,-.1) rectangle (-0.9,.5);%
		\pgftext[left,x=-3.5cm,y=0.2cm]{\color{white}\Large\sc\bfseries Chapter\ \thecontentslabel};%
	\end{tikzpicture}\color{doc!60}\large\sc\bfseries}%
{}
{}
{\;\titlerule\;\large\sc\bfseries Page \thecontentspage
	\begin{tikzpicture}[remember picture, overlay]
		\draw[fill=doc!60,draw=doc!60] (2pt,0) rectangle (4,0.1pt);
	\end{tikzpicture}}%
\titlecontents{section}[3.7pc]
{\addvspace{2pt}}
{\contentslabel[\thecontentslabel]{2pc}}
{}
{\hfill\small \thecontentspage}
[]
\titlecontents*{subsection}[3.7pc]
{\addvspace{-1pt}\small}
{}
{}
{\ --- \small\thecontentspage}
[ \textbullet\ ][]

\makeatletter
\renewcommand{\tableofcontents}{%
	\chapter*{%
	  \vspace*{-20\p@}%
	  \begin{tikzpicture}[remember picture, overlay]%
		  \pgftext[right,x=15cm,y=0.2cm]{\color{doc!60}\Huge\sc\bfseries \contentsname};%
		  \draw[fill=doc!60,draw=doc!60] (13,-.75) rectangle (20,1);%
		  \clip (13,-.75) rectangle (20,1);
		  \pgftext[right,x=15cm,y=0.2cm]{\color{white}\Huge\sc\bfseries \contentsname};%
	  \end{tikzpicture}}%
	\@starttoc{toc}}
\makeatother
\newcommand{\id}{\mathrm{id}}
\newcommand{\taking}[1]{\xrightarrow{#1}}
\newcommand{\inv}{^{-1}}

%From M170 "Introduction to Graph Theory" at SJSU
\DeclareMathOperator{\diam}{diam}
\DeclareMathOperator{\ord}{ord}
\newcommand{\defeq}{\overset{\mathrm{def}}{=}}

%From the USAMO .tex files
\newcommand{\ts}{\textsuperscript}
\newcommand{\dg}{^\circ}
\newcommand{\ii}{\item}

% % From Math 55 and Math 145 at Harvard
% \newenvironment{subproof}[1][Proof]{%
% \begin{proof}[#1] \renewcommand{\qedsymbol}{$\blacksquare$}}%
% {\end{proof}}

\newcommand{\liff}{\leftrightarrow}
\newcommand{\lthen}{\rightarrow}
\newcommand{\opname}{\operatorname}
\newcommand{\surjto}{\twoheadrightarrow}
\newcommand{\injto}{\hookrightarrow}
\newcommand{\On}{\mathrm{On}} % ordinals
\DeclareMathOperator{\img}{im} % Image
\DeclareMathOperator{\Img}{Im} % Image
\DeclareMathOperator{\coker}{coker} % Cokernel
\DeclareMathOperator{\Coker}{Coker} % Cokernel
\DeclareMathOperator{\Ker}{Ker} % Kernel
\DeclareMathOperator{\rank}{rank}
\DeclareMathOperator{\Spec}{Spec} % spectrum
\DeclareMathOperator{\Tr}{Tr} % trace
\DeclareMathOperator{\pr}{pr} % projection
\DeclareMathOperator{\ext}{ext} % extension
\DeclareMathOperator{\pred}{pred} % predecessor
\DeclareMathOperator{\dom}{dom} % domain
\DeclareMathOperator{\ran}{ran} % range
\DeclareMathOperator{\Hom}{Hom} % homomorphism
\DeclareMathOperator{\Mor}{Mor} % morphisms
\DeclareMathOperator{\End}{End} % endomorphism

\newcommand{\eps}{\epsilon}
\newcommand{\veps}{\varepsilon}
\newcommand{\ol}{\overline}
\newcommand{\ul}{\underline}
\newcommand{\wt}{\widetilde}
\newcommand{\wh}{\widehat}
\newcommand{\vocab}[1]{\textbf{\color{blue} #1}}
\providecommand{\half}{\frac{1}{2}}
\newcommand{\dang}{\measuredangle} %% Directed angle
\newcommand{\ray}[1]{\overrightarrow{#1}}
\newcommand{\seg}[1]{\overline{#1}}
\newcommand{\arc}[1]{\wideparen{#1}}
\DeclareMathOperator{\cis}{cis}
\DeclareMathOperator*{\lcm}{lcm}
\DeclareMathOperator*{\argmin}{arg min}
\DeclareMathOperator*{\argmax}{arg max}
\newcommand{\cycsum}{\sum_{\mathrm{cyc}}}
\newcommand{\symsum}{\sum_{\mathrm{sym}}}
\newcommand{\cycprod}{\prod_{\mathrm{cyc}}}
\newcommand{\symprod}{\prod_{\mathrm{sym}}}
\newcommand{\Qed}{\begin{flushright}\qed\end{flushright}}
\newcommand{\parinn}{\setlength{\parindent}{1cm}}
\newcommand{\parinf}{\setlength{\parindent}{0cm}}
% \newcommand{\norm}{\|\cdot\|}
\newcommand{\inorm}{\norm_{\infty}}
\newcommand{\opensets}{\{V_{\alpha}\}_{\alpha\in I}}
\newcommand{\oset}{V_{\alpha}}
\newcommand{\opset}[1]{V_{\alpha_{#1}}}
\newcommand{\lub}{\text{lub}}
\newcommand{\del}[2]{\frac{\partial #1}{\partial #2}}
\newcommand{\Del}[3]{\frac{\partial^{#1} #2}{\partial^{#1} #3}}
\newcommand{\deld}[2]{\dfrac{\partial #1}{\partial #2}}
\newcommand{\Deld}[3]{\dfrac{\partial^{#1} #2}{\partial^{#1} #3}}
\newcommand{\lm}{\lambda}
\newcommand{\uin}{\mathbin{\rotatebox[origin=c]{90}{$\in$}}}
\newcommand{\usubset}{\mathbin{\rotatebox[origin=c]{90}{$\subset$}}}
\newcommand{\lt}{\left}
\newcommand{\rt}{\right}
\newcommand{\bs}[1]{\boldsymbol{#1}}
\newcommand{\exs}{\exists}
\newcommand{\st}{\strut}
\newcommand{\dps}[1]{\displaystyle{#1}}

\newcommand{\sol}{\setlength{\parindent}{0cm}\textbf{\textit{Solution:}}\setlength{\parindent}{1cm} }
\newcommand{\solve}[1]{\setlength{\parindent}{0cm}\textbf{\textit{Solution: }}\setlength{\parindent}{1cm}#1 \Qed}
% Things Lie
\newcommand{\kb}{\mathfrak b}
\newcommand{\kg}{\mathfrak g}
\newcommand{\kh}{\mathfrak h}
\newcommand{\kn}{\mathfrak n}
\newcommand{\ku}{\mathfrak u}
\newcommand{\kz}{\mathfrak z}
\DeclareMathOperator{\Ext}{Ext} % Ext functor
\DeclareMathOperator{\Tor}{Tor} % Tor functor
\newcommand{\gl}{\opname{\mathfrak{gl}}} % frak gl group
\renewcommand{\sl}{\opname{\mathfrak{sl}}} % frak sl group chktex 6

% More script letters etc.
\newcommand{\SA}{\mathcal A}
\newcommand{\SB}{\mathcal B}
\newcommand{\SC}{\mathcal C}
\newcommand{\SF}{\mathcal F}
\newcommand{\SG}{\mathcal G}
\newcommand{\SH}{\mathcal H}
\newcommand{\OO}{\mathcal O}

\newcommand{\SCA}{\mathscr A}
\newcommand{\SCB}{\mathscr B}
\newcommand{\SCC}{\mathscr C}
\newcommand{\SCD}{\mathscr D}
\newcommand{\SCE}{\mathscr E}
\newcommand{\SCF}{\mathscr F}
\newcommand{\SCG}{\mathscr G}
\newcommand{\SCH}{\mathscr H}

% Mathfrak primes
\newcommand{\km}{\mathfrak m}
\newcommand{\kp}{\mathfrak p}
\newcommand{\kq}{\mathfrak q}

% number sets
\newcommand{\RR}[1][]{\ensuremath{\ifstrempty{#1}{\mathbb{R}}{\mathbb{R}^{#1}}}}
\newcommand{\NN}[1][]{\ensuremath{\ifstrempty{#1}{\mathbb{N}}{\mathbb{N}^{#1}}}}
\newcommand{\ZZ}[1][]{\ensuremath{\ifstrempty{#1}{\mathbb{Z}}{\mathbb{Z}^{#1}}}}
\newcommand{\QQ}[1][]{\ensuremath{\ifstrempty{#1}{\mathbb{Q}}{\mathbb{Q}^{#1}}}}
\newcommand{\CC}[1][]{\ensuremath{\ifstrempty{#1}{\mathbb{C}}{\mathbb{C}^{#1}}}}
\newcommand{\PP}[1][]{\ensuremath{\ifstrempty{#1}{\mathbb{P}}{\mathbb{P}^{#1}}}}
\newcommand{\HH}[1][]{\ensuremath{\ifstrempty{#1}{\mathbb{H}}{\mathbb{H}^{#1}}}}
\newcommand{\FF}[1][]{\ensuremath{\ifstrempty{#1}{\mathbb{F}}{\mathbb{F}^{#1}}}}

% number sets without arguments
\newcommand{\R}{\ensuremath{\mathbb{R}}}
\newcommand{\N}{\ensuremath{\mathbb{N}}}
\newcommand{\Z}{\ensuremath{\mathbb{Z}}}
\newcommand{\Q}{\ensuremath{\mathbb{Q}}}
\newcommand{\C}{\ensuremath{\mathbb{C}}}
\newcommand{\F}{\ensuremath{\mathbb{F}}}

% expected value
\newcommand{\EE}{\ensuremath{\mathbb{E}}}
\newcommand{\charin}{\text{ char }}
\DeclareMathOperator{\sign}{sign}
\DeclareMathOperator{\Aut}{Aut}
\DeclareMathOperator{\Inn}{Inn}
\DeclareMathOperator{\Syl}{Syl}
\DeclareMathOperator{\Gal}{Gal}
\DeclareMathOperator{\GL}{GL} % General linear group
\DeclareMathOperator{\SL}{SL} % Special linear group

%---------------------------------------
% BlackBoard Math Fonts :-
%---------------------------------------

%Captital Letters
\newcommand{\bbA}{\mathbb{A}}	\newcommand{\bbB}{\mathbb{B}}
\newcommand{\bbC}{\mathbb{C}}	\newcommand{\bbD}{\mathbb{D}}
\newcommand{\bbE}{\mathbb{E}}	\newcommand{\bbF}{\mathbb{F}}
\newcommand{\bbG}{\mathbb{G}}	\newcommand{\bbH}{\mathbb{H}}
\newcommand{\bbI}{\mathbb{I}}	\newcommand{\bbJ}{\mathbb{J}}
\newcommand{\bbK}{\mathbb{K}}	\newcommand{\bbL}{\mathbb{L}}
\newcommand{\bbM}{\mathbb{M}}	\newcommand{\bbN}{\mathbb{N}}
\newcommand{\bbO}{\mathbb{O}}	\newcommand{\bbP}{\mathbb{P}}
\newcommand{\bbQ}{\mathbb{Q}}	\newcommand{\bbR}{\mathbb{R}}
\newcommand{\bbS}{\mathbb{S}}	\newcommand{\bbT}{\mathbb{T}}
\newcommand{\bbU}{\mathbb{U}}	\newcommand{\bbV}{\mathbb{V}}
\newcommand{\bbW}{\mathbb{W}}	\newcommand{\bbX}{\mathbb{X}}
\newcommand{\bbY}{\mathbb{Y}}	\newcommand{\bbZ}{\mathbb{Z}}

%---------------------------------------
% MathCal Fonts :-
%---------------------------------------

%Captital Letters
\newcommand{\mcA}{\mathcal{A}}	\newcommand{\mcB}{\mathcal{B}}
\newcommand{\mcC}{\mathcal{C}}	\newcommand{\mcD}{\mathcal{D}}
\newcommand{\mcE}{\mathcal{E}}	\newcommand{\mcF}{\mathcal{F}}
\newcommand{\mcG}{\mathcal{G}}	\newcommand{\mcH}{\mathcal{H}}
\newcommand{\mcI}{\mathcal{I}}	\newcommand{\mcJ}{\mathcal{J}}
\newcommand{\mcK}{\mathcal{K}}	\newcommand{\mcL}{\mathcal{L}}
\newcommand{\mcM}{\mathcal{M}}	\newcommand{\mcN}{\mathcal{N}}
\newcommand{\mcO}{\mathcal{O}}	\newcommand{\mcP}{\mathcal{P}}
\newcommand{\mcQ}{\mathcal{Q}}	\newcommand{\mcR}{\mathcal{R}}
\newcommand{\mcS}{\mathcal{S}}	\newcommand{\mcT}{\mathcal{T}}
\newcommand{\mcU}{\mathcal{U}}	\newcommand{\mcV}{\mathcal{V}}
\newcommand{\mcW}{\mathcal{W}}	\newcommand{\mcX}{\mathcal{X}}
\newcommand{\mcY}{\mathcal{Y}}	\newcommand{\mcZ}{\mathcal{Z}}


%---------------------------------------
% Bold Math Fonts :-
%---------------------------------------

%Captital Letters
\newcommand{\bmA}{\boldsymbol{A}}	\newcommand{\bmB}{\boldsymbol{B}}
\newcommand{\bmC}{\boldsymbol{C}}	\newcommand{\bmD}{\boldsymbol{D}}
\newcommand{\bmE}{\boldsymbol{E}}	\newcommand{\bmF}{\boldsymbol{F}}
\newcommand{\bmG}{\boldsymbol{G}}	\newcommand{\bmH}{\boldsymbol{H}}
\newcommand{\bmI}{\boldsymbol{I}}	\newcommand{\bmJ}{\boldsymbol{J}}
\newcommand{\bmK}{\boldsymbol{K}}	\newcommand{\bmL}{\boldsymbol{L}}
\newcommand{\bmM}{\boldsymbol{M}}	\newcommand{\bmN}{\boldsymbol{N}}
\newcommand{\bmO}{\boldsymbol{O}}	\newcommand{\bmP}{\boldsymbol{P}}
\newcommand{\bmQ}{\boldsymbol{Q}}	\newcommand{\bmR}{\boldsymbol{R}}
\newcommand{\bmS}{\boldsymbol{S}}	\newcommand{\bmT}{\boldsymbol{T}}
\newcommand{\bmU}{\boldsymbol{U}}	\newcommand{\bmV}{\boldsymbol{V}}
\newcommand{\bmW}{\boldsymbol{W}}	\newcommand{\bmX}{\boldsymbol{X}}
\newcommand{\bmY}{\boldsymbol{Y}}	\newcommand{\bmZ}{\boldsymbol{Z}}
%Small Letters
\newcommand{\bma}{\boldsymbol{a}}	\newcommand{\bmb}{\boldsymbol{b}}
\newcommand{\bmc}{\boldsymbol{c}}	\newcommand{\bmd}{\boldsymbol{d}}
\newcommand{\bme}{\boldsymbol{e}}	\newcommand{\bmf}{\boldsymbol{f}}
\newcommand{\bmg}{\boldsymbol{g}}	\newcommand{\bmh}{\boldsymbol{h}}
\newcommand{\bmi}{\boldsymbol{i}}	\newcommand{\bmj}{\boldsymbol{j}}
\newcommand{\bmk}{\boldsymbol{k}}	\newcommand{\bml}{\boldsymbol{l}}
\newcommand{\bmm}{\boldsymbol{m}}	\newcommand{\bmn}{\boldsymbol{n}}
\newcommand{\bmo}{\boldsymbol{o}}	\newcommand{\bmp}{\boldsymbol{p}}
\newcommand{\bmq}{\boldsymbol{q}}	\newcommand{\bmr}{\boldsymbol{r}}
\newcommand{\bms}{\boldsymbol{s}}	\newcommand{\bmt}{\boldsymbol{t}}
\newcommand{\bmu}{\boldsymbol{u}}	\newcommand{\bmv}{\boldsymbol{v}}
\newcommand{\bmw}{\boldsymbol{w}}	\newcommand{\bmx}{\boldsymbol{x}}
\newcommand{\bmy}{\boldsymbol{y}}	\newcommand{\bmz}{\boldsymbol{z}}

%---------------------------------------
% Scr Math Fonts :-
%---------------------------------------

\newcommand{\sA}{{\mathscr{A}}}   \newcommand{\sB}{{\mathscr{B}}}
\newcommand{\sC}{{\mathscr{C}}}   \newcommand{\sD}{{\mathscr{D}}}
\newcommand{\sE}{{\mathscr{E}}}   \newcommand{\sF}{{\mathscr{F}}}
\newcommand{\sG}{{\mathscr{G}}}   \newcommand{\sH}{{\mathscr{H}}}
\newcommand{\sI}{{\mathscr{I}}}   \newcommand{\sJ}{{\mathscr{J}}}
\newcommand{\sK}{{\mathscr{K}}}   \newcommand{\sL}{{\mathscr{L}}}
\newcommand{\sM}{{\mathscr{M}}}   \newcommand{\sN}{{\mathscr{N}}}
\newcommand{\sO}{{\mathscr{O}}}   \newcommand{\sP}{{\mathscr{P}}}
\newcommand{\sQ}{{\mathscr{Q}}}   \newcommand{\sR}{{\mathscr{R}}}
\newcommand{\sS}{{\mathscr{S}}}   \newcommand{\sT}{{\mathscr{T}}}
\newcommand{\sU}{{\mathscr{U}}}   \newcommand{\sV}{{\mathscr{V}}}
\newcommand{\sW}{{\mathscr{W}}}   \newcommand{\sX}{{\mathscr{X}}}
\newcommand{\sY}{{\mathscr{Y}}}   \newcommand{\sZ}{{\mathscr{Z}}}


%---------------------------------------
% Math Fraktur Font
%---------------------------------------

%Captital Letters
\newcommand{\mfA}{\mathfrak{A}}	\newcommand{\mfB}{\mathfrak{B}}
\newcommand{\mfC}{\mathfrak{C}}	\newcommand{\mfD}{\mathfrak{D}}
\newcommand{\mfE}{\mathfrak{E}}	\newcommand{\mfF}{\mathfrak{F}}
\newcommand{\mfG}{\mathfrak{G}}	\newcommand{\mfH}{\mathfrak{H}}
\newcommand{\mfI}{\mathfrak{I}}	\newcommand{\mfJ}{\mathfrak{J}}
\newcommand{\mfK}{\mathfrak{K}}	\newcommand{\mfL}{\mathfrak{L}}
\newcommand{\mfM}{\mathfrak{M}}	\newcommand{\mfN}{\mathfrak{N}}
\newcommand{\mfO}{\mathfrak{O}}	\newcommand{\mfP}{\mathfrak{P}}
\newcommand{\mfQ}{\mathfrak{Q}}	\newcommand{\mfR}{\mathfrak{R}}
\newcommand{\mfS}{\mathfrak{S}}	\newcommand{\mfT}{\mathfrak{T}}
\newcommand{\mfU}{\mathfrak{U}}	\newcommand{\mfV}{\mathfrak{V}}
\newcommand{\mfW}{\mathfrak{W}}	\newcommand{\mfX}{\mathfrak{X}}
\newcommand{\mfY}{\mathfrak{Y}}	\newcommand{\mfZ}{\mathfrak{Z}}
%Small Letters
\newcommand{\mfa}{\mathfrak{a}}	\newcommand{\mfb}{\mathfrak{b}}
\newcommand{\mfc}{\mathfrak{c}}	\newcommand{\mfd}{\mathfrak{d}}
\newcommand{\mfe}{\mathfrak{e}}	\newcommand{\mff}{\mathfrak{f}}
\newcommand{\mfg}{\mathfrak{g}}	\newcommand{\mfh}{\mathfrak{h}}
\newcommand{\mfi}{\mathfrak{i}}	\newcommand{\mfj}{\mathfrak{j}}
\newcommand{\mfk}{\mathfrak{k}}	\newcommand{\mfl}{\mathfrak{l}}
\newcommand{\mfm}{\mathfrak{m}}	\newcommand{\mfn}{\mathfrak{n}}
\newcommand{\mfo}{\mathfrak{o}}	\newcommand{\mfp}{\mathfrak{p}}
\newcommand{\mfq}{\mathfrak{q}}	\newcommand{\mfr}{\mathfrak{r}}
\newcommand{\mfs}{\mathfrak{s}}	\newcommand{\mft}{\mathfrak{t}}
\newcommand{\mfu}{\mathfrak{u}}	\newcommand{\mfv}{\mathfrak{v}}
\newcommand{\mfw}{\mathfrak{w}}	\newcommand{\mfx}{\mathfrak{x}}
\newcommand{\mfy}{\mathfrak{y}}	\newcommand{\mfz}{\mathfrak{z}}

\title{\Huge{Abstract Algebra}}
\author{\huge{Rohan Jain}}
\date{}

\begin{document}

\maketitle
\newpage% or \cleardoublepage
% \pdfbookmark[<level>]{<title>}{<dest>}
\pdfbookmark[section]{\contentsname}{toc}
\tableofcontents

\pagebreak

\chapter{}
\section{Introductory Notes}

\subsection{Things to Remember}
\nt{
	\begin{itemize}
		\item Definitions will usually be stated as ``if" even though they mean ``if and only if".
		\item Any form of proof is valid. Avoid proofs by contradiction because of disbelief in the law of excluded middle.
		\item When you define an object, you can \emph{only} utilize its definition to prove anything about it.
	\end{itemize}
}

\subsection{Set Review}

\dfn{Set}{In mathematics, a set is an undefined term. Basically, ``everyone knows what it is.'' A few examples of sets are:

\begin{itemize}
	\item The empty set is the set with no elements. It is denoted by $\phi$ or $\emptyset$.
	\item $\NN$ is the set of natural numbers.
	\item $\ZZ$ is the set of integers.
	\item $\QQ$ is the set of rational numbers.
	\item $\RR$ is the set of real numbers.
	\item $\CC$ is the set of complex numbers.
\end{itemize}
}
\nt{
	\begin{itemize}
		\item A set is a well-defined collection of objects. The objects in a set are called elements of the set.
		\item A set is generally defined as a capital letter.
		\item $(A = B) \iff (\forall x : x \in A \iff x \in B)$
		\item $(A \subset B) \iff (\forall x \in A : x \in B)$
		\item $A$ is a proper subset of $B$ if $A \subset B$ and $A \neq B$.
	\end{itemize}
}

\thm{}{$A = B \iff A \subset B \land B \subset A$}

\nt{
	\begin{itemize}
		\item $A \cup B = {x:x\in A \lor x\in B}$
		\item $A \cap B = {x:x\in A \land x\in B}$
		\item $A$\textbackslash $B = {x:x\in A \land x\not\in B}$
		\item $C$\textbackslash $(A \cup B) = (C$\textbackslash $A) \cap (C$\textbackslash $B)$
	\end{itemize}
}

\subsection{Cartesian Products and Functions}
\nt{
	\begin{itemize}
		\item $A \times B = \{(a,b) : a \in A \land b \in B\}$
	\end{itemize}
}
\ex{Cartesian Product of two sets}{
	Let $A = \{1, 2, \Delta\}$ and $B = \{0, \pi\}$
	\begin{itemize}
		\item $(1, 0)$
		\item $(2, 0)$
		\item $(\Delta, 0)$
		\item $(1, \pi)$
		\item $(2, \pi)$
		\item $(\Delta, \pi)$
	\end{itemize}
}

\nt{Relations are subsets of Cartesian Products. For example, we can say that $<$ is a relation on the subset of $\RR \times \RR$ consisting of all ordered pairs of real numbers such that the first element is less than the second.}

\dfn{Function}{A function $f$ from a set $A$ to a set $B$ is a subset of $A \times B$ such that for every $a \in A$, there is exactly one $b \in B$ such that $(a, b) \in f$.}

\nt{Let $R$ be a relation from $A$ to $B$.
\begin{itemize}
	\item $A$ is the domain
	\item $B$ is the codomain
	\item $\{b: aRb\}$ is the image
	\item $R$ is injective (one-to-one) if $a_1Rb \land a_2Rb \implies a_1 = a_2$
	\item $R$ is surjective (onto) if $\forall b \in B : \exists a \in A : aRb$. Basically if the image is the entire codomain.
	\item $R$ is bijective if it is injective and surjective
\end{itemize}}

\nt{$A \xrightarrow{\text{R}} B$ \\ $B \xrightarrow{\text{S}} C$ \\ Define the composition as $S \circ R = \{(a, c) :$ there is some $b$ such that $(a, b) \in R$ and $(b, c) \in S$\} }

\thm{}{Let $f: A \to B$, $g: B \to C$, and $h: C \to D$. Then
\begin{itemize}
	\item $h \circ (g \circ f) = (h \circ g) \circ f$
	\item If $f$ and $g$ are injective, so is $g \circ f$
	\item If $f$ and $g$ are surjective, so is $g \circ f$
	\item If $f$ and $g$ are bijective, so is $g \circ f$
\end{itemize}}

\subsection{Equivalence Relations}
\dfn{Equivalence Relation}{An equivalence relation is a relation that has the following special properties:
\begin{itemize}
	\item Reflexivity: $aRa$ for all $a \in A$
	\item Symmetry: $aRb \implies bRa$
	\item Transitivity: $aRb \land bRc \implies aRc$
\end{itemize}}

\dfn{Partition}{Given a set $S$, a partition of $S$ is a collection of subsets of $S$ such that their union is $S$.}

\nt{Equivalence relations go hand in hand with partitions.}

\nt{If $\sim$ is an equivalence relation $a\sim b$, then $\sim$ partiations a set $X$ into chunks. $X / \sim$ is the set of chunks.

Addition is \emph{well-defined} as an operation on $\ZZ/x\ZZ$ for $x \in \ZZ$.}

\subsection{Complex Numbers and Matrices}
\dfn{Complex Number}{A complex number is a number of the form $a + bi$, where $a$ and $b$ are real numbers and $i$ is the imaginary unit. $i^2 = -1$.}

\nt{Complex numbers generally take the from $z = a + bi$. 

$\bar z = a - bi$ is the complex conjugate of $z$.

Divide complex numbers by multiplying by the complex conjugate of the denominator}

\dfn{Matrix}{A matrix is a rectangular array of numbers. A $m \times n$ matrix is an array of $m$ rows and $n$ columns. Define the group of $m \times n$ matrices over a field $\FF$ as $\FF^{m \times n}$.}

\nt{Multiplication by an $m \times n$ matrix is a function from $\FF^n$ to $\FF^m$. It is associative because all functions are associative.}

\ex{$2 \times 2$ matrix exercise}{Consider $\ZZ^{2 \times 2}$. Define a relation $A \sim B$ if there is an integer matrix $P$ whose determinant is one and $B = P^{-1}AP$. Note that if an integer matrix has a determinant 1 it is invertible and its inverse is also an integer matrix with determinant 1.

\begin{enumerate}
	\item Show that this is an equivalence relation.
	\item Show that two matrices with different determinants cannot be similar.
	\item Determine whether $\displaystyle \begin{bmatrix} 6 & 0 \\ 0 & 1\end{bmatrix}$ is similar to $\displaystyle \begin{bmatrix} 2 & 0 \\ 0 & 3\end{bmatrix}$. 
	\item Determine whether $\displaystyle \begin{bmatrix} 6 & 0 \\ 0 & 1\end{bmatrix}$ is similar to $\displaystyle \begin{bmatrix} 1 & 0 \\ 0 & 6\end{bmatrix}$. 
\end{enumerate}
\sol
\begin{enumerate}
	\item Reflexive: $A = P^{-1}AP$ for $P = I_2$. 
	
	Symmetric: $P^{-1}AP = P^{-1}BP$ for some $P$ with determinant 1. 
	
	Transitive: $B = P^{-1}_1AP_1 \land C = P^{-1}_2BP_2 \Rightarrow C = P^{-1}_2P^{-1}_1AP_1P_2 $
	\item Determinants are a multiplicative property. If $B = P^{-1}AP$ and $\det(B) \neq \det(A)$, then $\det(B) \neq 1 * \det(A) * 1$. 
	\item No, different JCF.
	\item Yes, same JCF.
\end{enumerate}
}



\subsection{Number Theory}
\nt{Know induction, division algorithm, GCD and Bezout's lemma, and Primes and the Fundmental Theorem of Arithmetic.}

\ex{Weak Induction}{Prove that $5 | n^5 -n$ for all $n$.

\begin{myproof} Proof by induction. 
	\begin{enumerate}
		\item $n = 1$ is true, $5 | 0$. 
		\item If it is true then $n=k$, show that it is true when $n = k+1$.
		
		$(k+1)^5 - (k+1) = k^5 + 5k^4 + 10k^3 + 10k^3 + 5k + 1 - (k+1) = (k^5 - k) + (5k^4 + 10k^3 + 10k^2 + 5k)$. 

		Both quantities are divisible by 5.
	\end{enumerate}
	Therefore, $5 | n^5 - n$ for all $n$. 
\end{myproof}}

\ex{Strong Induction}{Prove that every integer $n$ can be written as $n = d_11! + d_22! + \cdots + d_kk!$ for some $d_1, \ldots, d_k \leq k \in \ZZ$ and $k \geq 1$.

\begin{myproof}
	Strong induction. 
	
	Given $n$, chose $s$ s.t. $s! \leq n < (s+1)!$. Then we can write $n = q \cdot s! + r$. 
	\begin{enumerate}
		\item $q \leq s$ (if $q \geq s+1$, then $n \geq (s+1)!$, which goes against our claim)
		\item $r < s!$
	\end{enumerate}

	Assume that this is true for any $k < n$. Then we can write $n = q \cdot s! + r$ for some $r < s!$. Then we can write $r$ in the same format since it is true for all $k < n$. 
\end{myproof}}

\ex{Well-ordering}{Prove that given $a,b,b \neq 0$, there exists unique $q,r$ such that $a = qb + r$ and $0 \leq r < |b|$. 
\begin{myproof}
	Well-ordering.

	Consider all the integeres of the form $a - xb$ for $x \in \ZZ$. At least one of these is nonnegative. If $a > 0$, choose $x = 0$. If $a \leq 0$, then choose $x = -ab|b|$. So let the set of all negative $a-xb$ be nonempty. Let $q=x$ be the smallest. Define $r = a - qb$ so that $a = qb + r$ and $r < |b|$. 

	To prove uniqueness, consider two sets: $qr$ and $q'r'$. Then $qb + r = q'b + r'$ and $r < |b|$. Or, $(q-q')b=r'-r$. The absolute value of the RHS has to be between $1 - |b|$ and $|b| - 1$. This has to be 0 since its the only multipe of $b$ in that range. So $q-q' = 0$ and $q = q'$ and $r = r'$.
\end{myproof}}

\mlenma{Bezout's Lemma}{Given integers $a,b \neq 0$, their GCD can be written in the form $ra + sb$ for some $r,s$. }

\dfn{}{An integer is prime if it only has $1$ and itself as positive divisors.}

\nt{1 is not a prime.}

\mlenma{}{If $p$ is prime and $p|ab$, then either $p|a$ or $p|b$.}

\thm{Fundamental Theorem of Arithmetic}{Every integer greater than 1 is either a prime or can be written as a product of primes in a unique way.}

\section{Group Theory}
\subsection{Introduction to Groups}
\dfn{Binary Operation}{Given a set $S$, a \emph{binary operation} on $S$ is a function $S \times S \to S$. }

\dfn{Group}{A \emph{group} is a set $G$ with a binary operation $*$ such that for all $a,b,c \in G$, the following hold:
	\begin{enumerate}
		\item $(a*b)*c = a*(b*c)$ (associativity)
		\item $e*a = a*e = a$ (identity)
		\item $a*a^{-1} = e$ (inverse)
		\item $*$ is closed under $G$. 
	\end{enumerate}
}

\nt{A set that only has associativity and identity is called a \emph{monoid}.}

\nt{Examples of groups
\begin{itemize}
	\item $\ZZ, \RR, \RR^{3 \times 3}, \CC, \QQ$ with addition.
	\item $z \in \CC : |z| = 1$ with multiplication.
	\item $GL(2, \RR)$ with matrix multiplication. However, this is not abelian. 
	\item $D_4 =$ symmetries of a square.
	\item $D_2 =$ symmetries of a triangle.
	\item $U(n)$ with multiplication modulo $n$. 
\end{itemize}}

\noindent If we take a random group, say $U(5)$, then we can create a table for how the multiplication works:

\begin{center}
	\noindent\begin{tabular}{c | c c c c}
	$\cdot$ & 1 & 2 & 3 & 4  \\
	\cline{1-5}
	1 & 1 & 2 & 3 & 4 \\
	2 & 2 & 4 & 1 & 3  \\
	3 & 3 & 1 & 4 & 2 \\
	4 & 4 & 3 & 2 & 1 \\
	\end{tabular}
\end{center}

A table like this is called a \emph{Cayley Table}. Notice that this table is actually symmetric. This means that the group is \emph{commutative}, but more properly, \emph{abelian}.

\dfn{Abelian Group}{An \emph{abelian group}, $G$, is a group where $a*b = b*a$ for all $a,b \in G$.}

\subsection{Properties of Groups}
\thm{}{The identity element of a group is unique.}
\begin{myproof}
	Let $e_1$ and $e_2$ be the identity elements. Then $e_1*e_2 = e_2*e_1 = e_1$. So $e_1 = e_2$.
\end{myproof}

\thm{}{Each element has a unique inverse.}
\begin{myproof}
	Let $a^{-1}$ and $b$ both be inverses of $a$ then consider the product $baa^{-1}$. Then $b = be = b(aa^{-1}) = (ba)a^{-1} = ea^{-1}= a^{-1}$. So $b = a^{-1}$.
\end{myproof}

\cor{}{$(ab)^{-1} = b^{-1}a^{-1}$}
\begin{myproof}
	$abb^{-1}a^{-1} = aea^{-1} = aa^{-1} = e$.
\end{myproof}

\cor{}{$(a_1a_2a_3\ldots a_n)^{-1} = a_n^{-1}a_{n-1}^{-1}a_{n-2}^{-1}\ldots a_1$}
\begin{myproof}
	Induction from 1.2.1. 
\end{myproof}

\cor{}{$(a^{-1})^{-1} = a$}
\begin{myproof}
	$(a^{-1})^{-1}a^{-1} = e = aa^{-1}$, so by uniqueness of inverses...
\end{myproof}

\thm{}{Given any $a,b \in G$, the equations $ax = b$ and $ya = b$ have unique solutions, though not necessary equal.}
\begin{myproof}
	Let $x = a^{-1}b$ and $y = ba^{-1}$. Then $ax = a(a^{-1}b) = eb = b$ and $ya = ba^{-1}a = be = b$. To show uniqueness, consider $ax_1 = ax_2$ then left multiply by $a^{-1}$.
\end{myproof}

\cor{Cancellation Laws}{In any group $G$, if $ac=bc$, then $a=b$. And if $ca=cb$, then $a=b$. }
\begin{myproof}
	Right or left multiply by $c^{-1}$ for appropriate equation.
\end{myproof}

\nt{Proving that a group is associative from its Cayley digram takes too long. It is easier to show an isomorphism to a well-established group.}

\nt{Groups of order $n$:
	
\begin{itemize}
	\item 1: $\ZZ_1$
	\item 2: $\ZZ_2$
	\item 3: $\ZZ_3$
	\item 4: $\ZZ_4, V$
	\item 5: $\ZZ_5$
	\item 6: $D_3, \ZZ_6$
	\item 7: $\ZZ_7$
	\item 8: $\ZZ_8, \ZZ_2 \times \ZZ_4, \ZZ_2 \times \ZZ_2 \times \ZZ_2, D_4, H$
	\item 9: $\ZZ_9, \ZZ_3 \times \ZZ_3$
\end{itemize}}

\nt{A note on notation:

$a\cdot a = a^2, a \cdot a \cdot a = a^3 \ldots$ }

\dfn{Direct Product}{Given $G_1, G_2$ groups, then the direct proudct $G_1 \times G_2$ is the group of ordered pairs $(g_1, g_2)$ where $g_1 \in G_1$ and $g_2 \in G_2$. The operation is $(g_1, g_2) \cdot (h_1, h_2) = (g_1 \cdot h_1, g_2 \cdot h_2)$.}

\ex{}{$\{e\} \times G \cong G$}

\ex{}{$\ZZ_2 \times \ZZ_2 \cong V$}

\ex{}{$\ZZ_2 \times \ZZ_3 \cong Z_6$}

\thm{}{Let $(G, \circ, e)$ be a set with the binary operation $\circ$ and left identity $e$. Then assume each $x \in G$ has a left inverse such that $x^{-1} \circ x = e$. Then $G$ is a group.}

\mpf{Proof}{what is $xe=$?

Let $y = xe$. Then $x^{-1}y = x^{-1}(xe) = (x^{-1}x)e = e$. So $x^{-1}y  = e = x^{-1}x$. Multiply by $x^{-1^{-1}}$ to get $y = x$. Therefore, $e$ is a two-sided identity.

To show that $x^{-1}$, consider $z = x \circ x^{-1}$. Left multiply by $x^{-1}$ to get $x^{-1} \circ z = x^{-1} \circ (x \circ x^{-1}) = (x^{-1} \circ x) \circ x^{-1} = x^{-1}$. Left multiply both sides by $x^{-1^{-1}}$ to see that $e \circ z = z = e$. Therefore, $x^{-1}$ is a left inverse and $G$ is a group.}

\dfn{Subgroups}{Let $(G, \circ, e)$ be a group and let $H \subset G$. If $H$ is a group under the same operation $\circ$, then $H$ is a \emph{subgroup} of $G$. This is denoted as $H < G$. }

\nt{Having the same operation is critical. For example $GL(2) \subset \RR^{2\times 2}$, but $GL(2)$ is not a subgroup of $\RR^{2\times 2}$ because the operation is matrix multiplication, not addition.}

\mlenma{}{If $H \subset G$ and for any $h_1, h_2 \in H$, $h_1h_2^{-1} \in H$, then $H$ is a subgroup.}

\mpf{Proof}{ Following:
	
\begin{itemize}
	\item Choose $h_2 = h_1$, then $H \supset h_1h_1^{-1} = e$.
	\item Let $h_1 = e, h_2 = h$. Then $eh^{-1} = h^{-1} \in H$.
	\item $h_1h_2 = h_1(h_2^{-1})^{-1}$.
\end{itemize}}

\ex{Quarternion Units}{Let $Q_8 = \{ \pm 1, \pm i, \pm j, \pm k\}$. These function such that $i^2 = j^2 = k^2 = ijk = -1$. All the two element subgroups are $\{ \pm 1 \}$.}

\dfn{Cyclic Subgroup}{Given $a \in G$, the \emph{cyclic subgroup generated by a}, denoted $\langle a \rangle$, is the set $\{a^n : n \in  \ZZ\}$. The element $a$ is called the \emph{generator}. }

\ex{Cylic Subgroups}{
\begin{itemize}
	\item $\ZZ = \langle 1 \rangle$
	\item $\ZZ_7 = \langle 1 \rangle, \langle 5 \rangle$
	\item $\ZZ_{10} = \langle 1 \rangle, \langle 7 \rangle$
\end{itemize}}

\mprop{}{Every subgroup of $\ZZ$ is cyclic.}

\mpf{Addendum}{Any subgroup of any cyclic subgroup is itself cyclic.}

\nt{Some $U(n)$ groups are cyclic while others are not. They are cyclic if $n$ has primitive roots.}

\mlenma{}{Let $a \in G$, order of $a=n$. Then order of $a^k = \frac{n}{\text{gcd}(a,k)}$}

\mpf{Proof}{Let $b = a^k$. Order is the smallest number we can find such that $b^s = e$. Note that $b^s = a^{ks}$, so we need $n|ks$. Let $d = \text{gcd}(n, k)$. Then $n = dn'$ and $k =dk'$. Then we need $dn'$ to be a divisor of $sdk'$. So, $n'|sk'$. Since $n'$ and $k'$ are coprime, $n'|s$. Therefore, the smallest possible $s$ is $n' = n/\text{gcd}(a, k)$.}

\thm{}{A group has no proper nontrivial subgroups is and only if it is a cyclic group of prime order.}

\mpf{Proof}{Let $G = \langle a \rangle$ for any $a \in G$. What is the oder of $a$? If $a$ isn't prime, $a=xy$ and $y \neq 1$. Then $a^x$ has order $y$.}

\subsection{Permutations}

\dfn{Permutation}{A permutation is a bijection from a set $S$ to itself.}

\nt{All permutations of a set $A$ forms a group called $S_A$. This can be called either ``permutation on A'' or ``symmetric group of A''.

$|S_n| = n!$.}

\ex{Compositions and Cycles}{Given two permutations, it is not hard to multiply then. For example:

$$\begin{pmatrix} 1&2&3&4&5&6 \\ 2&4&3&6&5&1\end{pmatrix}
\begin{pmatrix} 1&2&3&4&5&6 \\ 4&3&5&1&6&2\end{pmatrix} = \begin{pmatrix} 1&2&3&4&5&6 \\ 4&5&3&2&6&1\end{pmatrix}$$}

\nt{This notation can be seen as quite cumbersome and redundant given the fact that the first row is always the same. To simplify this, we can use the following \emph{cycle} notation:

$$\begin{pmatrix} 1&2&3&4&5&6 \\ 4&5&3&2&6&1\end{pmatrix} = \begin{pmatrix}1&4&2&5&6 \end{pmatrix} \begin{pmatrix} 3 \end{pmatrix}$$

This is read as the permutation that sends 1 to 4 to 2 to 5 to 6 and 3 to 3.

The identity permutation is $\begin{pmatrix} 1&2&3&4&5&6 \end{pmatrix}$, which is annoying so mathematicians just say $e$. }

\mlenma{}{Disjoint cycles commute.}

\thm{}{Every permutation can be written as a product of disjoint cycles.}

\mpf{Proof}{Strong Induction:

Assume any permutation that moves $<n$ elements can be written. Consider $\sigma$ which has $n$ elements. Consider the set, which is called the orbit, of $\sigma$: $1, \sigma(1), \sigma^2(1) \ldots$. By the pigeonhole principle, this repeats. Cut off this set at the repeat of 1 and removed the curly braces and commas to get a cycle that 1 belongs to. }

\nt{The inverse of a cyclic is just the cycle backwards.}

\dfn{Transposition}{A transposition is a permutation that swaps just two elements. Also known as a ``swap'' or ``2-cycle''}

\mlenma{}{Any permutation may be written as a product of not disjoint transpositions.}

\mpf{Proof}{The cycle $\begin{pmatrix}A & B & C & \ldots & Y & Z \end{pmatrix} = (AZ)(AY)\ldots(AC)(AB)$.}

\mlenma{}{The following are true:

\begin{enumerate}
	\item $(AB) = (BA)$.
	\item $(AB)(AC) = (A \; B \; C)$
	\item $(AB)(CD) = (CD)(AB)$
	\item $(\ldots X \; Y \; Z \ldots)(AY) = (\ldots X \; Y \; A \; Z)$
	\item $(AY)(\ldots X \; Y \; Z \ldots) = (\ldots X \; A \; Y \; Z)$
	\item $(\ldots P \; Q  \; R \ldots X \; Y \; Z)(QY) = (\ldots P \; Q \; Z \ldots)(\; R \ldots X \; Y)$
	\item $\begin{pmatrix}A & B & C & \ldots & Y & Z \end{pmatrix} = (AZ)(AY)\ldots(AC)(AB)$
\end{enumerate}
}



% \section{Random Examples}
% \begin{myproof}By openness of $V$, $x\in B_r(u)\subset V$
% 	\begin{center}
% 		\begin{tikzpicture}
% 			\draw[red] (0,0) circle [x radius=3.5cm, y radius=2cm] ;
% 			\draw (3,1.6) node[red]{$V$};
% 			\draw [blue] (1,0) circle (1.45cm) ;
% 			\filldraw[blue] (1,0) circle (1pt) node[anchor=north]{$u$};
% 			\draw (2.9,0.4) node[blue]{$B_r(u)$};
% 			\draw [green!40!black] (1.7,0) circle (0.5cm) node [yshift=0.7cm]{$B_{\delta}(x)$} ;
% 			\filldraw[green!40!black] (1.7,0) circle (1pt) node[anchor=west]{$x$};
% 		\end{tikzpicture}
% 	\end{center}
% \end{myproof}

% \cor{}{By the result of the proof, we can then show...}

% \mprop{}{$1 + 1 = 2$.}

% \section{Random}
% \dfn{Normed Linear Space and Norm $\boldsymbol{\|\cdot\|}$}{Let $V$ be a vector space over $\bbR$ (or $\bbC$). A norm on $V$ is function $\|\cdot\|\ V\to \bbR_{\geq 0}$ satisfying \begin{enumerate}[label=\bfseries\tiny\protect\circled{\small\arabic*}]
% 		\item \label{n:1}$\|x\|=0 \iff x=0$ $\forall$ $x\in V$
% 		\item \label{n:2}	$\|\lambda x\|=|\lambda|\|x\|$ $\forall$ $\lambda\in\bbR$(or $\bbC$), $x\in V$
% 		\item \label{n:3} $\|x+y\| \leq \|x\|+\|y\|$ $\forall$ $x,y\in V$ (Triangle Inequality/Subadditivity)
% 	\end{enumerate}And $V$ is called a normed linear space.

% 	$\bullet $ Same definition works with $V$ a vector space over $\bbC$ (again $\|\cdot\|\to\bbR_{\geq 0}$) where \ref{n:2} becomes $\|\lambda x\|=|\lambda|\|x\|$ $\forall$ $\lambda\in\bbC$, $x\in V$, where for $\lambda=a+ib$, $|\lambda|=\sqrt{a^2+b^2}$ }




% \section{Algorithms}
% \begin{algorithm}[H]
% \KwIn{This is some input}
% \KwOut{This is some output}
% \SetAlgoLined
% \SetNoFillComment
% \tcc{This is a comment}
% \vspace{3mm}
% some code here\;
% $x \leftarrow 0$\;
% $y \leftarrow 0$\;
% \uIf{$ x > 5$} {
%     x is greater than 5 \tcp*{This is also a comment}
% }
% \Else {
%     x is less than or equal to 5\;
% }
% \ForEach{y in 0..5} {
%     $y \leftarrow y + 1$\;
% }
% \For{$y$ in $0..5$} {
%     $y \leftarrow y - 1$\;
% }
% \While{$x > 5$} {
%     $x \leftarrow x - 1$\;
% }
% \Return Return something here\;
% \caption{what}
% \end{algorithm}

\end{document}