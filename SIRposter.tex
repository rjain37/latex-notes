\documentclass{beamer}

% ====================
% Packages
% ====================

\usepackage[T1]{fontenc}
\usepackage{lmodern}
\usepackage[size=custom,width=120,height=88,scale=1.0]{beamerposter}
\usetheme{gemini}
\usecolortheme{imsa}
\usepackage{graphicx}
\usepackage{booktabs}
\usepackage{tikz}
\usepackage{pgfplots}
\usepackage{fancyhdr}


% ====================
% Lengths
% ====================

% If you have N columns, choose \sepwidth and \colwidth such that
% (N+1)*\sepwidth + N*\colwidth = \paperwidth
\newlength{\sepwidth}
\newlength{\colwidth}
\setlength{\sepwidth}{0.025\paperwidth}
\setlength{\colwidth}{0.3\paperwidth}



\newcommand{\separatorcolumn}{\begin{column}{\sepwidth}\end{column}}

% ====================
% Title
% ====================


\title{Optimizing Trigger Selection for Detection of Doubly Charged Higgs Bosons at the LHC}

\author{Rohan Jain \and Dr. Peter J. Dong}

\institute[shortinst]{ {\texttt{rjain@imsa.edu} \samelineand  \texttt{pdong@imsa.edu} \\

Illinois Mathematics and Science Academy
}}
\logo{
\includegraphics[width=0.15\textwidth]{IMSAlogo.png}
\hspace{1.5cm}
\includegraphics[]{CMSlogo.png}
}

% ====================
% Body
% ====================

\pgfplotsset{compat=1.18}

\begin{document}

\begin{frame}[t]
\begin{columns}[t]
\separatorcolumn

\begin{column}{\colwidth}
  \begin{block}{Abstract}
    In this project, we aim to programmatically find the most efficient triggers for selecting H++ events for application in the Compact Muon Solenoid experiment. Highly efficient triggers are defined as those with high signal efficiency and low background efficiency to give as much signal and as little background as possible. The types of events analyzed were H++, Drell-Yan, and QCD. The initial process started with finding the most efficient triggers on the three sets of events independently, then pairwise comparing the differences, and then finally creating a new figure of merit which was the harmonic mean of all the differences. 
  \end{block}

\begin{block}{Signal}
  The signal interaction we are searching for is activity of the double charged Higgs boson:

  \begin{itemize}
    \item H++: The doubly charged Higgs boson is our signal and is a hypothetical particle that we believe exists and are therefore trying to prove its existence within Monte Carlo events.
  \end{itemize}
\end{block}

  \begin{block}{Background}

    Background interactions include the following:

    \begin{itemize}
      \item Quantum Chromodynamics (QCD). QCD interactions are a strong interaction between quarks that are done through gluons. 
      \item Drell-Yan (DY). DY interactions occur when a quark and an antiquark of distinct hadrons annilhate, form a Z-boson, then decay into oppositely charged leptons.
    \end{itemize}
  \end{block}

  \begin{block}{Process}
    We start by running Monte Carlo data through a trigger simulation, enabling all triggers individually.

    \begin{itemize}
        \item H++
        \begin{itemize}
            \item Higgs900.txt
            \begin{itemize}
                \item $m$ of at least 900 GeV
            \end{itemize}
        \end{itemize}
        \item QCD
        \begin{itemize}
            \item QCD500-700.txt
            \begin{itemize}
                \item $H_T$ between 500-700 GeV
            \end{itemize}
        \end{itemize}
        \item Drell-Yan
        \begin{itemize}
            \item DY50.txt
            \begin{itemize}
                \item $m$ of at least 50 GeV
            \end{itemize}
        \end{itemize}
    \end{itemize}
  \end{block}

  \begin{block}{Preliminary Results}
    The tables below show the results for the top 5 triggers (highest efficiencies) for each of the three types of events.

    \begin{table}[]
      \begin{center}
           \caption{\label{table:1}Trigger Results for H++}
     \begin{tabular}[t]{c|c}
          \hline
          \textbf{Trigger Name} & \textbf{Efficiency}\\
          \hline
          HLT\_AK8PFJet40 & 0.999155\\
          HLT\_HcalPhiSym & 0.999 \\
          HLT\_AK4PFJet30 & 0.99879 \\
          HLT\_AK4CaloJet30 & 0.993445 \\
          HLT\_DiPFJetAve40  & 0.991245
      \end{tabular}
    \end{center}
    \end{table}
    Best Electron Trigger: HLT\_Ele8\_CaloIdL\_TrackIdL\_IsoVL\_PFJet30 | 0.83359

    Best Muon Trigger: HLT\_Mu8\_TrkIsoVVL | 0.78888
  \end{block}

\end{column}

\separatorcolumn

\begin{column}{\colwidth}
  \begin{block}{Preliminary Results (cont.)}
    \begin{table}
      \caption{\label{table:2}Trigger Results for QCD}
          \begin{tabular}[t]{c|c}
              \hline
              \textbf{Trigger Name} & \textbf{Efficiency}\\
              \hline
              HLT\_HcalPhiSym & 0.862971\\
              HLT\_AK8PFJet40 & 0.688192 \\
              HLT\_AK4CaloJet30 & 0.619257 \\
              HLT\_AK4PFJet30 & 0.598404 \\
              HLT\_PFJet40 & 0.465758
          \end{tabular}
  \end{table}
  Best Electron Trigger: HLT\_Ele8\_CaloIdM\_TrackIdM\_PFJet30 | 0.0633013

Best Muon Trigger: HLT\_Mu3\_PFJet40 | 0.277873
\begin{table}[h!]
  \caption{\label{table:3}Trigger Results for DY}
      \begin{tabular}[t]{c|c}
          \hline
          \textbf{Trigger Name} & \textbf{Efficiency}\\
          \hline
          HLT\_HcalPhiSym & 0.884871\\
          HLT\_AK8PFJet200 & 0.882028 \\
          HLT\_HT425 & 0.872316 \\
          HLT\_PFJet200 & 0.773207 \\
          HLT\_DiPFJetAve200 & 0.546264
      \end{tabular}
\end{table}
Best Electron Trigger: HLT\_Ele20\_WPLoose\_Gsf | 0.217682

Best Muon Trigger: HLT\_L1SingleMu18 | 0.277113
  \end{block}
  \begin{block}{Redefining Efficiency}

    To account for comparing multiple efficiencies, we redefine the efficiency as follows:
    $$Eff = Eff_{Signal} - Eff_{Background}$$
  \end{block}

  \begin{block}{Intermediary Results}
    The tables below show the results for the top 5 triggers for the pairwise comparisons
    \begin{table}[h!]
      \caption{\label{table:4}Trigger Results for H++ vs. QCD}
          \begin{tabular}[t]{c|c}
              \hline
              \textbf{Trigger Name} & \textbf{Efficiency}\\
              \hline
              HLT\_DiPFJetAve320 & 0.81028139\\
              HLT\_PFJet320 & 0.8088902\\
              HLT\_PFMETNoMu110\_PFMHTNoMu110\_IDTight & 0.8054678 \\
              HLT\_Photon75 & 0.8039625 \\
              HLT\_Photon90 & 0.8035066 
          \end{tabular}
    \end{table}
    Best Electron Trigger: HLT\_Ele12\_CaloIdL\_TrackIdL\_IsoVL\_PFJet30 | 0.7971663

    Best Muon Trigger: HLT\_IsoMu20 | 0.75249405
    \begin{table}[h!]
      \caption{\label{table:5}Trigger Results for H++ vs. DY}
          \begin{tabular}[t]{c|c}
              \hline
              \textbf{Trigger Name} & \textbf{Efficiency}\\
              \hline
              HLT\_AK8PFJet140 & 0.94992018\\
              HLT\_DiPFJetAve80 & 0.9497367\\
              HLT\_PFJet140 & 0.94761912 \\
              HLT\_PFHT250 & 0.94691313 \\
              HLT\_DiPFJetAve140 & 0.94661373
          \end{tabular}
    \end{table}
    Best Electron Trigger: HLT\_Ele50\_CaloIdVT\_GsfTrkIdT\_PFJet165 | 0.79523

Best Muon Trigger: HLT\_Mu15\_IsoVVVL\_PFHT600 | 0.742399778
  \end{block}
\end{column}

\separatorcolumn

\begin{column}{\colwidth}

  \begin{block}{Redefining Efficiency (pt. 2)}
    To compare values in the two tables above, we need to redefine efficiency again. This time, we will use the following equation:

    $$Eff = Eff_{H++ vs. QCD} + Eff_{H++ vs. DY}$$
  \end{block}
  \begin{block}{Final Results}
    The tables below shows the sum of pairwise compared efficiencies.

    \begin{table}[h!]
      \caption{\label{table:6}Trigger Results for H++ vs. DY \& QCD}
          \begin{tabular}[t]{c|c}
              \hline
              \textbf{Trigger Name} & \textbf{Efficiency}\\
              \hline
              HLT\_PFJet320 & 1.6658838 \\
              HLT\_Photon75 & 1.6451807 \\
              HLT\_AK8PFJet320 & 1.6429796 \\
              HLT\_Photon90 & 1.64207 \\
              HLT\_DiPFJetAve260 & 1.632191
          \end{tabular}
  \end{table}
  Best Electron Trigger: HLT\_Ele50\_CaloIdVT\_GsfTrkIdT\_PFJet165 | 1.58410567

  Best Muon Trigger: HLT\_Mu15\_IsoVVVL\_PFHT600 | 1.4755256380000001
  \end{block}

  \begin{block}{Questions}
    \begin{itemize}
      \item Why do photon triggers perform so well?
      \item How do we optimize using multiple triggers at once?
      \item How can we account for the fact that we are using Monte Carlo events instead of real data?
      \item How can we account for high energy collisions that wrongly set off triggers?
    \end{itemize}
  \end{block}
  \begin{block}{Next Steps}
    \begin{enumerate}
      \item Optimize for multiple triggers.
      \item Use these triggers on real data instead of just Monte Carlo events.
    \end{enumerate}
  \end{block}
  \begin{block}{Conclusions}
    By the process of comparing various triggers against each other and on different types of Monte Carlo events, we have narrowed down our selection of triggers to a few of the most efficient, which can be found in the "Final Results" section. 
  \end{block}


 % \begin{block}{References}

    %\nocite{*}
    %\footnotesize{\bibliographystyle{plain}\bibliography{poster}}

  %\end{block}

\end{column}

\separatorcolumn
\end{columns}
\end{frame}

\end{document}