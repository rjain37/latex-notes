\documentclass[12pt]{report}

\input{preamble}
\input{macros}
\input{letterfonts}
\usepackage{fancyhdr}
\pagestyle{fancy}

\lhead{\bf Rohan Jain}
\cfoot{}
\rhead{\bf
Abstract Algebra \\
Assignment 12}

\begin{document}

\qs{3}{Use the division algorithm to find $q(x)$ and $r(x)$ such that $a(x) = q(x)b(x) + r(x)$ with $\operatorname{deg}r(x) < \operatorname{deg}b(x)$ for each of the following pairs of polynomials.
\begin{enumerate}[label=\alph*.]
    \item $a(x) = 5 x^3 + 6x^2 - 3 x + 4$ and $b(x) = x -2$ in $\ZZ_7[x]$
    \item $a(x) = 6 x^4 - 2 x^3 + x^2 - 3 x + 1$ and $b(x) = x^2 + x -2$ in $\ZZ_7[x]$
    \item $a(x) = 4 x^5 - x^3 + x^2 + 4$ and $b(x) = x^3 - 2$ in $\ZZ_5[x]$
    \item $a(x) = x^5 + x^3 -x^2 - x$ and $b(x) = x^3 + x$ in $\ZZ_2[x]$
\end{enumerate}}
\sol
\begin{enumerate}[label=\alph*.]
    \item $5 x^3 + 6x^2 - 3 x + 4 = (5x^2 + 4x + 6)(x-2) + (2x + 5)\pmod{7}$.
    \item $ 6 x^4 - 2 x^3 + x^2 - 3 x + 1 = (6x^2 -8x + 21)(x^2+x-2) + (-40x +43) = (6x^2 - x)(x^2+x-2) + (2x+1) \pmod{7}$.
    \item $ 4 x^5 - x^3 + x^2 + 4 = (4x^2 - 1)(x^3 - 2) + (2) \pmod{5}$.
    \item $ x^5 + x^3 -x^2 - x = (x^2)(x^3+x) - (x^2+x) \pmod{2}$
\end{enumerate}

\qs{4}{Find the greatest common divisor of each of the following pairs $p(x)$ and $q(x)$ of polynomials. If $d(x) = \operatorname{gcd}(p(x), q(x))$, find two polynomials $a(x)$ and $b(x)$ such that $d(x)=a(x)p(x) + b(x)q(x)$.
\begin{enumerate}[label=\alph*.]
    \item $p(x) = x^3 - 6x^2 + 14x - 15$ and $q(x) = x^3 - 8x^2 + 21x - 18\text{,}$ where $p(x), q(x) \in \QQ[x]$. 
    \item $p(x) = x^3 + x^2 - x + 1$ and $q(x) = x^3 + x - 1$, where $p(x),q(x) \in \ZZ_2[x]$. 
    \item $p(x) = x^3 + x^2 - 4x + 4$ and $q(x) = x^3 + 3 x -2\text{,}$ where $p(x),q(x) \in \ZZ_5[x]$. 
    \item $p(x) = x^3 - 2 x + 4$ and $q(x) = 4x^3 + x + 3$, where $p(x), q(x) \in \QQ[x]$. 
\end{enumerate} }
\sol
\begin{enumerate}[label=\alph*.]
    \item Finding $d(x)$:\begin{align*}
        x^3 - 8x^2 + 21x - 18 &= (1)( x^3 - 6x^2 + 14x - 15) + (2x^2 - 7x + 3) \\
        x^3 - 6x^2 + 14x - 15 &= (1)\left(\frac{1}{2}x - \frac{9}{4}\right) + \left(\frac{15}{4}x - \frac{45}{4}\right)\\
        \frac{1}{2}x - \frac{9}{4} &= \left(\frac{15}{4}x - \frac{45}{4}\right)\left(\frac{8}{15}x - \frac{4}{15}\right) \\
        \operatorname{gcd}(p(x), q(x)) &= x - 3
    \end{align*} 
    what is this called lmao:
    \begin{align*}
        \left(\frac{15}{4}x - \frac{45}{4}\right) &= (x^3 - 8x^2 + 21x - 18) - (2x^2 - 7x + 3)\left(\frac{1}{2}x - \frac{9}{4}\right) \\
        &=  (x^3 - 8x^2 + 21x - 18) - ((x^3 - 6x^2 + 14x - 15) - (x^3 - 8x^2 + 21x - 18))\left(\frac{1}{2}x - \frac{9}{4}\right) \\
        &= (x^3 - 8x^2 + 21x - 18)\left(\frac{1}{2}x - \frac{5}{4}\right) + (x^3 - 6x^2 + 14x - 15)\left(-\frac{1}{2}x + \frac{9}{4}\right) \\
        (x-3) &= (x^3 - 8x^2 + 21x - 18)\left(\frac{2}{15}x - \frac{1}{3}\right) + (x^3 - 6x^2 + 14x - 15)\left(-\frac{2}{15}x + \frac{3}{5}\right) 
    \end{align*}
    \item Finding $d(x)$:\begin{align*}
        x^3 + x^2 -x + 1 &= (1)(x^3 + x -1) + (x^2)\pmod{2}\\
        x^3 + x -1 &= (x)(x^2) + (x-1)\pmod{2} \\
        x^2 &= (x)(x-1) + (1) \pmod{2}\\
        x-1 &= (x)(1) + (1)\pmod{2} \\
        1 &= (1)(1) \pmod{2} \\
        \operatorname{gcd}(p(x), q(x)) &= 1
    \end{align*} 
    what is this called!!:
    \begin{align*}
        1 &= (x-1) - (x)(1) \pmod{2} \\
        &= (x-1) - ((x^2) - (x)(x-1))\pmod{2} \\
        &= (x-1)(x+1) - (x^2) \pmod{2} \\
        &= ((x^3+x-1)-(x^2)(x))(x+1) - (x^2) \pmod{2} \\
        &= (x^3+x-1)(x+1) - (x^2)(x^2+x-1) \pmod{2} \\
        &= (x^3+x-1)(x+1) - ((x^3+x-1) -(x^2+x-1)) (x^2+x+1) \pmod{2} \\
        &= (x^3+x-1)(x^2) - (x^3+x^2-x+1)(x^2+x+1)\pmod{2} \\
    \end{align*}
    \item Finding $d(x)$:\begin{align*}
        x^3 + x^2 -4x + 4 &= (1)(x^3+3x-2)+(x^2+3x+1) \pmod{5} \\
        x^3 + 3x -2 &= (x)(x^2+3x+1) + (2x^2+2x-2) \pmod{5} \\
    \end{align*} 
    \item \begin{align*}
        4x^3 + x + 3 &= (4)(x^3 - 2 x + 4) + (9x-13) \\
        x^3 - 2 x + 4 &= \left(\frac{1}{9}x^2 + \frac{13}{81}x + \frac{7}{729}\right)(9x-13) + \left(\frac{3007}{729}\right) \\
    \end{align*} 
\end{enumerate}

\qs{5}{Find all of the zeros for each of the following polynomials.
\begin{enumerate}[label=\alph*.]
    \item $5x^3 + 4x^2 - x + 9$ in $\ZZ_{12}[x]$
    \item $3x^3 - 4x^2 - x + 4$ in $\ZZ_5[x]$
    \item $5x^4 + 2x^2 - 3$ in $\ZZ_7[x]$
    \item $x^3 + x + 1$ in $\ZZ_2[x]$
\end{enumerate}}
\sol You just have to plug all the numbers from $0$ to $n-1$ in $\ZZ_n$ to see if there are any zeros $\pmod{n}$.
\begin{enumerate}[label=\alph*.]
    \item There are no zeroes.
    \item $x \cong 2 \pmod{5}$.
    \item $x \cong 3,4 \pmod{7}$.
    \item There are no zeroes.
\end{enumerate}

\qs{6}{Find all of the units in $\ZZ[x]$.}
\sol If $pq = 1$, then $\operatorname{deg}(pq) = 0 \Rightarrow\operatorname{deg}(p) + \operatorname{deg}(q) = 0 \Rightarrow \operatorname{deg}(p) = \operatorname{deg}(q) = 0 $. Therefore, $p$ and $q$ can only be $\pm 1$. 

\qs{7}{Find a unit $p(x)$ in $\ZZ_4[x]$ such that $\operatorname{deg}p(x) > 1$. }
\sol If we look at $p(x) = (2x+1)^2 = 4x^2 + 4x + 1 = 1 \pmod{4}$, we have found at degree 2 unit.

\qs{10}{Give two different factorizations of $x^2 + x + 8$ in $\ZZ_{10}[x]$. }
\sol Testing all values of $x$ from $0$ to $9$, we see that the zeroes $ x \cong 1, 3, 6, 8$. Pairing these numbers to add up to $-1$, we see that they can't. Therefore, we have to make two of these values negative. So, we can say that $x \cong 1, 3, -4, -2$ and pair as follows:
\begin{align*}
    x^2 + x + 8 &= (x-1)(x+2) \\
    &= (x-3)(x+4). \\
\end{align*}
\qs{25}{Let $F$ be a field and $f(x) = a_0  +a_1x + \cdots + a^nx^n$ be in $F[x]$. Define $f'(x) = a_1 + 2a_2x + \cdots + na_nx^{n-1}$ to be the \textbf{\emph{derivative}} of $f(x)$.
\begin{enumerate}[label=\alph*.]
    \item Prove that $$(f+g)'(x) = f'(x) + g'(x)$$ Conclude that we can define a homomorphism of abelian groups $D : F[x] \rightarrow F[x]$ by $D(f(x)) = f'(x)$.
    \item Calculate the kernel of $D$ if $\operatorname{char}F = 0$.
    \item Calculate the kernel of $D$ if $\operatorname{char}F = p$.
    \item Prove that $$(fg)'(x) = f'(x)g(x) + f(x) g'(x)\text{.}$$
    \item Suppose that we can factor a polynomial $f(x) \in F[x]$ into linear factors, say $$f(x) = a(x - a_1)(x-a_2) \cdots (x-a_n).$$ Prove that $f(x)$ has no repeated factors if and only if $f(x)$ and $f'(x)$ are relatively prime.
\end{enumerate}}
\sol
\begin{enumerate}[label=\alph*.]
    \item \begin{align*}
        (f+g)'(x) &= [f(x) + g(x)]' \\
        &= \lim_{h\to 0}\frac{f(x+h) + g(x+h) - f(x) - g(x)}{h} \\
        &= \lim_{h\to 0}\frac{f(x+h) - f(x)}{h} + \lim_{h\to 0}\frac{g(x+h) - g(x)}{h} \\
        &= f'(x) + g'(x) \qed
    \end{align*}
    This is a homomorphism because we essentially have that $D(f(x) + g(x)) = D(f(x)) + D(g(x))$.
    \item $\operatorname{ker}(D) = \{f(x) \in F[x] : f'(x) = 0\}$
    
    $\operatorname{ker}(D) = \{f(x) \in F[x] : f(x) = a, a\in F\}$

    $\operatorname{ker}(D) = F$
    \item If $\operatorname{char}F = p$, then $f'(x) = 0$ under a few conditions: 
    \begin{enumerate}
        \item Is constant
        \item If not constant, coefficients are multiples of $p$
        \item If coefficients aren't multiples of $p$, then exponents are multiples of $p$
        \item Both
    \end{enumerate}
    So, the kernel can be described as polynomials that fall under the above criteria.
    \item Let us have $f(x) = \sum_{i=0}^n a_ix^i$ and $g(x) = \sum_{i=0}^n b_ix^i$. Then,
    \begin{align*}
        (fg)(x) &= \sum _{k=0}^{2n}\left\{\sum_{i = max(k-n, 0)}^{min(n, k)}a_ib_{k-i}\right\}x^k \\ 
        (fg)'(x) &= \sum _{k=0}^{2n}\left\{\sum_{i = max(k-n, 0)}^{min(n, k)}a_ib_{k-i}\right\}kx^{k-1} \\
    \end{align*}
    We also have the following:
    $$f'(x)g(x) = \sum_{k=0}^{2n}\left\{\sum_{i = max(k-n, 0)}^{min(k,n)}ia_ib_{k-i}\right\}x^{k-1}$$
    $$f(x)g'(x) = \sum_{k=0}^{2n}\left\{\sum_{i = max(k-n, 0)}^{min(k,n)}(k-i)a_ib_{k-i}\right\}x^{k-1}$$
    So, we have that 
    $$f'(x)g(x) + f(x)g'(x) = \sum_{k=0}^{2n}\left\{\sum_{i = max(k-n, 0)}^{min(k,n)}a_ib_{k-i}\right\}kx^{k-1}
    = (fg)'(x) \qed$$
    \item If $f(x)$ has no repeated factors, then we have that all $\alpha_i$ are distinct in $$f(x) = \alpha_0(x - \alpha_1)(x-\alpha_2)\cdots(x-\alpha_n).$$ 
    Now, let us define $g_i(x) = \alpha_0(x - \alpha_1)\cdots(x-\alpha_{i-1})(x - \alpha_{i+1})\cdots (x - \alpha_n)$. Then we have that $f(x) = (x - \alpha_i)g_i(x)$. 

    By the product rule, we have that 
    $$f'(x) = (x - \alpha_i)g_i'(x) + g_i(x)$$
    $$f'(\alpha_i) = g_i(\alpha_i) = \alpha_0(\alpha_i - \alpha_1) \cdots (\alpha_i - \alpha_{i-1})(\alpha_i - \alpha_{i+1})\cdots(\alpha_i - \alpha_n)$$
    Since all the zeroes are distinct, this does not evaluate to 0. So, $(x-\alpha_i)$ is not a factor of $f'(x)$. $f(x)$ only has linear factors in the form $(x-\alpha_k)$ for $1 \leq k \leq n$. Therefore, $f(x)$ and $f'(x)$ are relatively prime.

    Now suppose that $f(x)$ has a repeated factor that is $(x -\alpha_i)$ such that $f(x) = (x  - \alpha_i)^kg_i(x)$ for some $k \geq 2$ and $\displaystyle g_i(x) = \frac{f(x)}{(x - \alpha_i)^k}$. Then we have that $f'(x) = k(x - \alpha_i)^{k-1}g_i(x) + (x - \alpha_i)^kg_i'(x)$. So, $f'(x-\alpha_i) = 0$ and as such, is a factor of both $f(x)$ and $f'(x)$, meaning that they are not relatively prime.

    Since both directions have been proven, the statement is true. $\qed$
\end{enumerate}

\end{document}