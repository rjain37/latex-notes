\documentclass{report}

\input{../preamble}
\input{../macros}
\input{../letterfonts}

\title{\Huge{21603 Model Theory I}}
\author{\huge{Rohan Jain}}
\date{}
% \usepackage{lmodern}
\begin{document}

\maketitle
\newpage% or \cleardoublepage
% \pdfbookmark[<level>]{<title>}{<dest>}
\pdfbookmark[section]{\contentsname}{toc}
\tableofcontents

\pagebreak

\chapter{}
\section{random info}
rami@cmu.edu \\
username: MOAB \\
password: APLATON \\
MOAB1.pdf
\begin{enumerate}
    \item Set Theory
    \item Model Theory
    \item Recursion Theory
    \item Proof Theory
\end{enumerate}

\noindent 1973 book by Chang and Keisler - Model Theory - Highly recommended for elementary model theory. 

\noindent What is model thoery? Model Theory = logic + universal algebra

\noindent 1984 - W. Hodges - Shorter Model Theory

\noindent model theory = algebraic geometry - field theory

\noindent Algebraic structures:
\begin{enumerate}
    \item groups
    \item rings
    \item vector spaces 
    \item fields
    \item graphs - $(V, E)$
    \item ordered structures
\end{enumerate}

Around 1870, mathematicians started to layout the foundations for mathematics. One of the ideas was axiomatization. One example was Euclidean axioms for plane geometry. 

\section{Logic}

\dfn{Language}{$L$ is a language if $L = F \cup R \cup C$ are parameter disjoint. }

\dfn{$L$-structure}{Let $L$ be a language (similarity type/signature). Then $\mathcal M$ is an $L$-structure provided:
\begin{align*}
    \mathcal M = (U, \{ f^{\mathcal M} \mid f \in F \}, \{ r^{\mathcal M} \mid r \in R \}, \{ c^{\mathcal M} \mid c \in C \})
\end{align*}
where $U$ is a nonempty set. $U$ is also called the universe of $\mathcal M$. 

For any $f \in F$ there is $U(f)$ natural number such that $f^{\mathcal M} : U^{n(F)} \to U$, $R^{\mathcal M} \subseteq U^{n(R)}$, $C^{\mathcal M} \subseteq U$, $\forall c \in C$. }

Notation: $| \mathcal M | = U$. The cardinal of $\mathcal M$ is $|U|$. $\Vert \mathcal M \Vert$ denotes the cardinality of $\mathcal M$. 

\dfn{Theory}{Let $L$ be a language. A theory $T$ is a set of sentences in $L$. A sentence is a finite set of symbols from $L$.}

\ex{Sentences}{
$L_{\text{gr}} = \{ e, \cdot\}$. $e \in C$, $\cdot \in F$. $T_{\text{gr}} = \{ \forall x \forall y \forall z (x \cdot (y \cdot z) = (x \cdot y) \cdot z), \forall x (x \cdot e = x, e \cdot x = x), \forall x \exists y (x \cdot y = e, y \cdot x = e) \}$. These are the group axioms (associativity, identity, existence of inverse). }

\dfn{Term}{Let $L$ be a language. A term is:
\begin{enumerate}
    \item $c$ is a term for any $c \in C$.
    \item $x$ when $x$ is a variable.
    \item $\tau_1,\ldots, \tau_k$ terms, $f \in F$, $n(f) = k$, then $f(\tau_1, \ldots, \tau_k)$ is a term.
\end{enumerate}}

\dfn{Term}{$\text{Term}(L)$ is a minimal set of finite strings of symbols from $L \cup \{ (, ) \} \cup X$ that contains $C \cup x$ and closed under the following rule:
\begin{align*}
    \tau_1, \ldots, \tau_k \in \text{Term}(L), f k-\text{place function symbol, then} f(\tau_1, \ldots, \tau_k) \in \text{Term}(L)
\end{align*}
}
\ex{$L_r$}{
    $L_r = \{0, 1, +, -\}$. $\text{Term}(L_r) \supseteq \{\sum a_j x_1^{n_j} \mid a_j \in \mathbb Z, n_j \in \mathbb N\}$.
}
\ex{$L_{\text{gr}}$}{
    $\text{Term}(L_{\text{gr}}) \supseteq \{x_1 \cdot x_n \cdots x_n \mid x_i \in X, n \in \omega\}$. 
}

\newcommand{\term}{\text{Term}}
\newcommand{\Fml}{\text{Fml}}

\dfn{$\Fml$}{$\Fml (L)$ is the set of first order formulas in $L$.}



\end{document}