\documentclass{report}

\input{../preamble}
\input{../macros}
\input{../letterfonts}

\title{\Huge{21603 Model Theory I}}
\author{\huge{Rohan Jain}}
\date{}
% \usepackage{lmodern}
\begin{document}

\newcommand{\term}{\text{Term}}
\newcommand{\fml}{\text{Fml}}
\newcommand{\afml}{\text{AFml}}
\newcommand{\fv}{\text{FV}}
\newcommand{\sent}{\text{Sent}}
\newcommand{\val}{\text{Val}}
\newcommand{\Mod}{\text{Mod}}

\maketitle
\newpage% or \cleardoublepage
% \pdfbookmark[<level>]{<title>}{<dest>}
\pdfbookmark[section]{\contentsname}{toc}
\tableofcontents

\pagebreak

\chapter{}
\section{random info}
rami@cmu.edu
\begin{enumerate}
    \item Set Theory
    \item Model Theory
    \item Recursion Theory
    \item Proof Theory
\end{enumerate}

\noindent 1973 book by Chang and Keisler - Model Theory - Highly recommended for elementary model theory. 

\noindent What is model thoery? Model Theory = logic + universal algebra

\noindent 1984 - W. Hodges - Shorter Model Theory

\noindent model theory = algebraic geometry - field theory

\noindent Algebraic structures:
\begin{enumerate}
    \item groups
    \item rings
    \item vector spaces 
    \item fields
    \item graphs - $(V, E)$
    \item ordered structures
\end{enumerate}

Around 1870, mathematicians started to layout the foundations for mathematics. One of the ideas was axiomatization. One example was Euclidean axioms for plane geometry. 

\section{Structures and Languages}

\dfn{Language}{$L$ is a language if $L = F \cup R \cup C$ are parameter disjoint. }

\dfn{$L$-structure}{Let $L$ be a language (similarity type/signature). Then $\mathcal M$ is an $L$-structure provided:
\begin{align*}
    \mathcal M = (U, \{ f^{\mathcal M} \mid f \in F \}, \{ r^{\mathcal M} \mid r \in R \}, \{ c^{\mathcal M} \mid c \in C \})
\end{align*}
where $U$ is a nonempty set. $U$ is also called the universe of $\mathcal M$. 

For any $f \in F$ there is $U(f)$ natural number such that $f^{\mathcal M} : U^{n(F)} \to U$, $R^{\mathcal M} \subseteq U^{n(R)}$, $C^{\mathcal M} \subseteq U$, $\forall c \in C$. }

Notation: $| \mathcal M | = U$. The cardinal of $\mathcal M$ is $|U|$. $\Vert \mathcal M \Vert$ denotes the cardinality of $\mathcal M$. 

\dfn{Theory}{Let $L$ be a language. A theory $T$ is a set of sentences in $L$. A sentence is a finite set of symbols from $L$.}

\ex{Sentences}{
$L_{\text{gr}} = \{ e, \cdot\}$. $e \in C$, $\cdot \in F$. $T_{\text{gr}} = \{ \forall x \forall y \forall z (x \cdot (y \cdot z) = (x \cdot y) \cdot z), \forall x (x \cdot e = x, e \cdot x = x), \forall x \exists y (x \cdot y = e, y \cdot x = e) \}$. These are the group axioms (associativity, identity, existence of inverse). }

\dfn{Term}{Let $L$ be a language. A term is:
\begin{enumerate}
    \item $c$ is a term for any $c \in C$.
    \item $x$ when $x$ is a variable.
    \item $\tau_1,\ldots, \tau_k$ terms, $f \in F$, $n(f) = k$, then $f(\tau_1, \ldots, \tau_k)$ is a term.
\end{enumerate}}

\dfn{Term}{$\text{Term}(L)$ is a minimal set of finite strings of symbols from $L \cup \{ (, ) \} \cup X$ that contains $C \cup x$ and closed under the following rule:
\begin{align*}
    \tau_1, \ldots, \tau_k \in \text{Term}(L), f k-\text{place function symbol, then} f(\tau_1, \ldots, \tau_k) \in \text{Term}(L)
\end{align*}
}
\ex{$L_r$}{
    $L_r = \{0, 1, +, -\}$. $\text{Term}(L_r) \supseteq \{\sum a_j x_1^{n_j} \mid a_j \in \mathbb Z, n_j \in \mathbb N\}$.
}
\ex{$L_{\text{gr}}$}{
    $\text{Term}(L_{\text{gr}}) \supseteq \{x_1 \cdot x_n \cdots x_n \mid x_i \in X, n \in \omega\}$. 
}



\dfn{$\afml$}{Let $L$ be a language. The set of atomic fomrulas denotes by $\afml(L)$ is the smallest set of formulas in $L$ that contains $L \cup \{(, ), =\} \cup X$ such that:
\begin{enumerate}
    \item If $\tau_1, \tau_2 \in \term (L)$, then $\tau_1 = \tau_2 \in \afml(L)$.
    \item Given $R(x_1, \ldots, x_n)$ relation symbol and $\tau_1, \ldots, \tau_n \in \term(L)$, then $R(\tau_1, \ldots, \tau_n) \in \afml(L)$.
\end{enumerate}
}

\dfn{$\fml$}{$\fml (L)$ is the set of (first order) formulas in $L$. Which is the mininmal set of finite strings of symbols from $L \cup \{ (, ), =, \neg, \vee, \wedge, \implies, \impliedby, \iff, \forall, \exists \} \cup X$ such that:
\begin{enumerate}
    \item $\fml(L) \supseteq \afml(L)$.
    \item If $\varphi$ is a formula, then $\neg \varphi$ is a formula.
    \item If $x \in \{\wedge, \vee, \implies, \iff\}$ and $\varphi, \psi \in \fml(L)$, then $(\varphi x \psi) \in \fml(L)$.
    \item If $\varphi \in \fml(L)$, $Q \in \{\forall, \exists\}$, and $x \in X$, then $Q x \varphi \in \fml(L)$.
    \item If $\varphi \in \fml(L)$, $\fv (\varphi)$ is the set of free variables in $\varphi$ defined by induction on the structure of $\varphi$. 
    
    Case 1: $\varphi \in \afml (L)$. 
    \begin{enumerate}
        \item $\varphi$ is $\tau_1 = \tau_2$. $\fv(\varphi) = \fv(\tau_1) \cup \fv(\tau_2)$.
        \item $\varphi$ is $R(\tau_1, \ldots, \tau_n)$. $\fv(\varphi) = \fv(\tau_1) \cup \ldots \cup \fv(\tau_n)$.
    \end{enumerate}
    Case 2: 
    \begin{enumerate}
        \item if $\varphi$ is $\neg \psi$, then $\fv(\varphi) = \fv(\psi)$.
        \item if $\varphi = \psi_1 * \psi_2$ for $* \in \{\wedge, \vee, \implies, \iff\}$, then $\fv(\varphi) = \fv(\psi_1) \cup \fv(\psi_2)$.
    \end{enumerate}
    Case 3: $\varphi$ is $Q x \psi$, $Q \in \{\forall, \exists\}$. Then $\fv(\varphi) = \fv(\psi) \setminus \{x\}$.
    \item $\sent(L)$ are the sentences in $L$. $\sent(L) = \{ \varphi \in \fml(L) \mid \fv(\varphi) = \emptyset \}$.
\end{enumerate}
}

\ex{}{If $L_f = \{+, \cdot, 0, 1\}$, then $T_f = \{$
    \begin{itemize}
        \item $\forall x \forall y \forall z (x \cdot (y \cdot z) = (x \cdot y) \cdot z)$,
        \item $\forall x \forall y \forall z (x + (y + z) = (x + y) + z),$
        \item $\forall x \forall y (x + y = y + x),$
        \item $\forall x \forall y (x \cdot y = y \cdot x)$,
        \item $\forall x (x \cdot 1 = x, 1 \cdot x = x)$, 
        \item $\forall x (x + 0 = x, 0 + x = x)$,
        \item $\forall x \exists y (x \cdot y = 1, y \cdot x = 1)$,
        \item $ \forall x \exists y (x + y = 0, y + x = 0)$,
        \item $\forall x \forall y \forall z (x \cdot (y + z) = (x \cdot y) + (x \cdot z))$
    \end{itemize}$\}$.}

\dfn{$L$-theory}{$T$ is an $L$-theory if $T \subseteq \sent(L)$.}
\noindent The example above is ``field theory".
\newpage
\dfn{}{Let $M$ be an $L$-structure. $\tau(\bar x)$ is a term, $\bar a \in |M|^{\ell(n)}$. T

Case 1: $\tau(\bar x) = c$ for some constant symbol. Then $\tau^M(\bar a) = c^M$.

Case 2: $\tau(\bar x) = x_i$. Then $\tau^M(\bar a) = a_i$.

Case 3: $\tau(\bar x) = f(\tau_1, \ldots, \tau_k)$. Then $\tau^M(\bar a) = f^M(\tau_1^M(\bar a), \ldots, \tau_k^M(\bar a))$.
}

\dfn{$\models$}{Let $L$ be a language, $\varphi \in \fml(L)$, $ M$ and $L$-structure, $n = \ell(\bar x)$, $\bar a \in |  M |^n$. Define $ M \models \varphi(\bar a)$ at $\bar a$ by induction on the structure of $\varphi$:

\begin{itemize}
    \item If $\varphi$ is atomic, 
    \begin{itemize}
        \item when $\varphi(x)$ is $\tau_1 = \tau_2$, then $M \models \varphi(\bar a)$ iff $\tau_1(\bar a) = \tau_2(\bar a)$.
        \item when $\varphi(x)$ is $R(\tau_1, \ldots, \tau_k)$, then $M \models \varphi(\bar a)$ iff $(\tau_1(\bar a), \ldots, \tau_k(\bar a)) \in R^M$.
    \end{itemize}
    \item If $\varphi$ is not atomic, then:
    \begin{itemize}
        \item if $\varphi$ is $\neg \psi$, then $M \models \varphi(\bar a)$ iff $M \models \psi(\bar a)$ is false.
        \item if $\varphi$ is $\psi_1 * \psi_2$ for $* \in \{\wedge, \vee, \implies, \iff\}$, then $M \models \varphi(\bar a)$ iff $M \models \psi_1(\bar a)$ and $M \models \psi_2(\bar a)$.
        \item if $\varphi$ is $\exists y \psi(y, \bar x)$, then $M \models \varphi(\bar a)$ iff there is $b \in |M|$ such that $M \models \psi(b, \bar a)$.
        \item if $\varphi$ is $\forall y \psi(y, \bar x)$, then $M \models \varphi(\bar a)$ iff for all $b \in |M|$, $M \models \psi(b, \bar a)$.
    \end{itemize}
\end{itemize}
}
\dfn{}{Let $M$ be an $L$-structure and $T$ an $L$-theory. $M \models T$ iff for every $\varphi \in T$, $M \models \varphi$. We say $T$ ``satisfies'' $M$.}
\ex{Models}{$M \models T_f \iff (|M|, +^M, \cdot^M, 0^M, 1^M)$ is a field.}

\dfn{$\Mod$}{$\Mod(T) = \{ M\, L \text{-structure} \mid M \models T \}$.}
\ex{}{$\Mod(T_f)$ is the class of all fields and $\Mod(T_{\text{gr}})$ is the class of all groups.}

\dfn{Structure Isomorphism}{Let $M, N$ both be $L$-structures. $f$ is an isomorphism from $M$ onto $N$ if $f: |M| \to |N|$ is a bijection such that:
\begin{itemize}
    \item $f(c^M) = c^N$ for all $c \in C$.
    \item $G(x_1, \ldots, x_k)$ function symbol. $a_1, \ldots, a_k \in |M|$, then $f(G^M(a_1, \ldots, a_k)) = G^N(f(a_1), \ldots, f(a_k))$.
    \item $R(x_1, \ldots, x_k)$ predicate symbol. $a_1, \ldots, a_k \in |M|$, then $(a_1, \ldots, a_k) \in R^M$ iff $(f(a_1), \ldots, f(a_k)) \in R^N$.
\end{itemize}
We write $f: M \cong N$. Also $M \cong N \iff \exists f: M \cong N$.}

\dfn{}{Let $\lambda \geq \aleph_0$, $T$ an $L$-theory. $T$ is $\lambda$-categorical if for all $M, N \models T$ of cardinality $\lambda$, $M \cong N$.}

\thm{Los Conjecture (1954)}{Let $L$ be a language, $T$ a first order $L$-theory, in an at most countable language. If $\exists \lambda > \aleph_0$ such that $T$ is $\lambda$-categorical, then for all $\mu > \aleph_0$, $T$ is $\mu$-categorical.}
\noindent Somewhere around 1961-1965, Morley proved this conjecture.

\end{document}