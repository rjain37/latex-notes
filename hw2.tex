\documentclass{report}

\input{preamble}
\input{macros}
\input{letterfonts}
\usepackage{fancyhdr}
\pagestyle{fancy}

\lhead{\bf Rohan Jain}
\cfoot{}
\rhead{\bf
Abstract Algebra \\
Assignment 2}

\begin{document}

\qs{2}{Which of the following multiplication tables defined on the set $G = \{a, b, c, d\}$ form a group? Support your answer in each case.

\begin{center}
    \includegraphics[width=15cm]{hw2q2.png}
\end{center}}

\sol

\begin{enumerate}[label=\alph*.]
    \item This is not a group because it doesn't have an identity element, $e$, such that $g \circ e = e \circ g = g$ for all $g \in G$.
    \item This table does have an identity element, $e = a$. Now we check for invertibility. In this table, each element is it's own inverse because $g \circ g = e = a$ for all $g \in G$. 
    \item d
    \item 
\end{enumerate}

\qs{7}{Let $S = \RR$ \textbackslash $\{-1\}$ and define a binary operation on $S$ by $a * b = a + b + ab$. Prove that $(S, *)$ is an abelian group.}
\sol

\qs{17}{Give an example of three different groups with eight elements. Why are the groups
different?}
\sol

\qs{25}{Let $a$ and $b$ be elements in a group $G$. Prove that $ab^na^{-1} = (aba^{-1})^n$ for $n \in \ZZ$.}
\sol

\qs{31}{Show that if $a^2 = e$ for all elements a in a group $G$, then $G$ must be abelian.}
\sol

\qs{32}{Show that if $G$ is a finite group of even order, then there is an $a \in G$ such that $a$ is not the identity and $a^2 = e$.}
\sol

\qs{33}{Let $G$ be a group and suppose that $(ab)^2 = a^2b^2$ for all $a$ and $b$ in $G$. Prove that $G$ is an abelian group.}
\sol

\end{document}