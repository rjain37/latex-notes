\documentclass{report}

\input{preamble}
\input{macros}
\input{letterfonts}
\usepackage{fancyhdr}
\pagestyle{fancy}

\lhead{\bf Rohan Jain}
\cfoot{}
\rhead{\bf
Abstract Algebra \\
Assignment 2}

\begin{document}

\qs{2}{Which of the following multiplication tables defined on the set $G = \{a, b, c, d\}$ form a group? Support your answer in each case.

\begin{center}
    \includegraphics[width=15cm]{hw2q2.png}
\end{center}}

\sol

\begin{enumerate}[label=\alph*.]
    \item This is not a group because it doesn't have an identity element, $e$, such that $g \circ e = e \circ g = g$ for all $g \in G$.
    \item This table does have an identity element, $e = a$. Now we check for invertibility. In this table, each element is it's own inverse because $g \circ g = e = a$ for all $g \in G$. Since the table only contains $a,b,c,d$, we know that the operation is closed. Since we cannot prove associativity from the Cayley diagram, we will attempt to show an isomorphism to one of the groups of order 4. Since we know the identity is $a$ and that the diagonal is all $a$, it is fair to assume that this is the Klein-4 group ($K_4$), a group such that $a^2 = e$ for all $a$ in the group. This property means that the corresponding operation is mulitplication. Calculating the rest of the Cayley table for $K_4$ shows that this is true.
    \item The identity element of this table is $a$ and each element is its own inverse. Since the table only contains $a,b,c,d$, we know that the operation is closed. With the same approach as the last problem, we will assume an isomorphism to the other group of order 4, $\ZZ_4$, under addition. Calculating the rest of the Cayley table for $\ZZ_4$ shows that this is true.
    \item The identity here is $a$ again, but $d$ doesn't have an inverse so this is not a group. 
\end{enumerate}

\qs{7}{Let $S = \RR$ \textbackslash $\{-1\}$ and define a binary operation on $S$ by $a * b = a + b + ab$. Prove that $(S, *)$ is an abelian group.}
\sol 

To show that the $*$ is closed, assume $a*b = -1$. Therefore $a + b + ab = -1$ where $a,b \neq -1$. So, $a(b+1) + b = -1 \Rightarrow a(b+1) = -(b+1)$ and since $b \neq -1$, we can divide through by $b+1$ to show that $a = -1$. $\rightarrow \leftarrow$

$(S, *)$ also has an identity and it is $0$. To show this, consider $a*0 = a + 0 + a\cdot 0 = a$. Also consider $0 * b = 0 + b + 0\cdot b = b$. Therefore, $0$ is the identity element.

$*$ is associative. Consider 
\begin{equation} 
    \begin{split}
    a * (b * c) & = a  * (b + c + bc) \\
    & = a + (b + c + bc) + a(b + c + bc) \\
    & = a + b + c + bc + ab + ac + abc \\
    & = a + b + ab + c + c(a + b + ab) \\
    & = (a + b + ab) * c \\
    & = (a * b) * c
    \end{split}
\end{equation}

To find an inverse, consider $a * a^{-1} = a + a^{-1} + a\cdot a^{-1} = 0$. Grouping shows that $a^{-1} = \displaystyle\frac{-a}{1 + a}$, so this is our right inverse for $a$. But we can also say that it is our left inverse because $a * b = a + b + ab = b*a$ because addition and multiplication are commutative in $\RR$.

Therefore $S$ is an abelian group under the operation $*$. $\qed$

\qs{17}{Give an example of three different groups with eight elements. Why are the groups
different?}
\sol In class, we went over the 5 groups of order 8 which are as follows:

\begin{itemize}
    \item $\ZZ_8$
    \item $\ZZ_4 \times \ZZ_2$
    \item $\ZZ_2 \times \ZZ_2 \times \ZZ_2$
    \item $D_4$
    \item $H$
\end{itemize}

We can pick any combination of these three groups to answer this question. For simplicity, I will pick the first three groups. These groups differ in that the highest order of any element is different in all three. 

\dfn{Order}{The order of an element $g \in G$ is the smallest positive integer $n$ such that $g^n = e$.}

The highest order of each of the first three groups is 8, 4, and 2 respectively. This is not hard to show for all of the groups. Therefore, the groups are different.

\qs{25}{Let $a$ and $b$ be elements in a group $G$. Prove that $ab^na^{-1} = (aba^{-1})^n$ for $n \in \ZZ$.}
\sol To start off, when $n=0$, we have $ab^0a^{-1} = e$ and $(aba^{-1})^0 = e$. Therefore, the equation holds for $n=0$.

We will prove the rest by induction on $n > 0$. Assume that the equation holds for $n-1$. So, we have $ab^{n-1}a^{-1} = (aba^{-1})^{n-1}$. Then we have $(aba^{-1})^{n} = (aba^{-1})^{n-1} (aba^{-1}) = ab^{n-1}a^{-1}aba^{-1} = ab^{n}a^{-1}$. Therefore, the equation holds for $n > 0$.

We can repeat this process for $n < 0$ by defining $m = -n$ and inducting on $m$ instead. $\qed$

\qs{31}{Show that if $a^2 = e$ for all elements a in a group $G$, then $G$ must be abelian.}
\sol Since $a^2 = e$ for all $a \in G$, $a = a^{-1}$ for all $a$ as well. For $a,b \in G$, consider $ab = (ab)^{-1} = b^{-1}a^{-1} = ba$. Therefore, $G$ is abelian if $a^2 = e$ for all $a$.

\qs{32}{Show that if $G$ is a finite group of even order, then there is an $a \in G$ such that $a$ is not the identity and $a^2 = e$.}
\sol

\qs{33}{Let $G$ be a group and suppose that $(ab)^2 = a^2b^2$ for all $a$ and $b$ in $G$. Prove that $G$ is an abelian group.}
\sol For $a,b \in G$, consider the expanded equation $abab = aabb$. Left multiplying by $a^{-1}$ and right multiplying by $b^{-1}$ reveals our desired symmetry, $ba = ab$. Therefore, $G$ is abelian. $\qed$

\end{document}