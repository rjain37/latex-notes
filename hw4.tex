\documentclass[12pt]{report}

\input{preamble}
\input{macros}
\input{letterfonts}
\usepackage{fancyhdr}
\pagestyle{fancy}

\lhead{\bf Rohan Jain}
\cfoot{}
\rhead{\bf
Abstract Algebra \\
Assignment 4}

\begin{document}

\qs{5}{List the subgroups of $S_4$. Find each of the following sets:
\begin{enumerate}[label=\alph*.]
    \item $\displaystyle \{ \sigma \in S_4 : \sigma(1) = 3 \}$
    \item $\displaystyle \{ \sigma \in S_4 : \sigma(2) = 2 \}$
    \item $\displaystyle \{ \sigma \in S_4 : \sigma(1) = 3$ and $\sigma(2) = 2 \}$
\end{enumerate}

Are any of these sets subgroups of $S_4$?}
\sol The subgroups of $S_4$ are:
\begin{itemize}
    \item $\langle e \rangle$
    \item $\langle (1 \; 2) \rangle$
    \item $\langle(1 \; 3)\rangle$
    \item $\langle(1 \; 4)\rangle$
    \item $\langle(2 \; 3)\rangle$
    \item $\langle(2 \; 4)\rangle$
    \item $\langle(3 \; 4)\rangle$
    \item $\langle(1 \; 2)(3 \; 4)\rangle$
    \item $\langle(1 \; 3)(2 \; 4)\rangle$
    \item $\langle(1 \; 4)(2 \; 3)\rangle$
    \item $\langle (1 \; 2 \; 3)\rangle$
    \item $\langle (1 \; 2 \; 4)\rangle$
    \item $\langle (1 \; 3 \; 4)\rangle$
    \item $\langle (2 \; 3 \; 4)\rangle$
    \item $\langle(1 \; 2)(3 \; 4), (1 \; 3)(2 \; 4)\rangle$
    \item $\langle(1 \; 2), (3 \; 4)\rangle$
    \item $\langle(1 \; 3), (2 \; 4)\rangle$
    \item $\langle(1 \; 4), (2 \; 3)\rangle$
    \item $\langle(1 \; 2 \; 3 \; 4)\rangle$
    \item $\langle(1 \; 2 \; 4 \; 3)\rangle$
    \item $\langle(1 \; 3 \; 2 \; 4)\rangle$
    \item $\langle (1 \; 2 \; 3), (1 \; 2) \rangle$
    \item $\langle (1 \; 2 \; 4), (1 \; 2) \rangle$
    \item $\langle (1 \; 3 \; 4), (1 \; 3) \rangle$
    \item $\langle (2 \; 3 \; 4), (2 \; 3) \rangle$
    \item $\langle (1 \; 2 \; 3 \; 4), (1 \; 3) \rangle$
    \item $\langle (1 \; 2 \; 4 \; 3), (1 \; 4) \rangle$
    \item $\langle (1 \; 3 \; 2 \; 4), (1 \; 2) \rangle$
    \item $A_4$
    \item $S_4$
\end{itemize}

\begin{enumerate}[label=\alph*.]
    \item $\{(1 \; 3), (1 \; 3)(2 \; 4), (1 \; 3 \; 4), (1 \; 3 \; 2), (1 \; 3 \; 4 \; 2), (1 \; 3 \; 2 \; 4)\}$
    \item $\{e, (1 \; 3), (1\; 4), (3\; 4), (1 \; 3 \; 4), (1 \; 4 \; 3)\}$
    \item $\{(1 \; 3), (1 \; 3 \; 4)\}$
\end{enumerate}

The set given by part b. is a subgroup of $S_4$. It is the only one you would even consider because it is the only one with the identity in it. It is  isomorphic to the subgroup $\langle (1 \; 3 \; 4), (1 \; 3) \rangle$.

\qs{6}{Find all of the subgroups in $A_4$. What is the order of each subgroup?}
\sol 
\begin{itemize}
    \item $A_4$, order of 12
    \item $\langle (1) \rangle = \{(1)\}$, order of 1
    \item $\langle (1 \; 2)(3 \; 4) \rangle = \{(1), (1 \; 2), (3 \; 4)\}$, order of 3
    \item $\langle (1 \; 3)(2 \; 4) \rangle = \{(1), (1 \; 3), (2 \; 4)\}$, order of 3
    \item $\langle (1 \; 4)(2 \; 3) \rangle = \{(1), (1 \; 4), (2 \; 3)\}$, order of 3
    \item $\langle (1 \; 2 \; 3) \rangle = \langle (1 \; 3 \; 2) \rangle = \{(1), (1 \; 2 \; 3), (1 \; 3 \; 2)\}$, order of 3
    \item $\langle (1 \; 2 \; 4) \rangle = \langle (1 \; 4 \; 2) \rangle = \{(1), (1 \; 2 \; 4), (1 \; 4 \; 2)\}$, order of 3
    \item $\langle (1 \; 3 \; 4) \rangle = \langle (1 \; 4 \; 3) \rangle = \{(1), (1 \; 3 \; 4), (1 \; 4 \; 3)\}$, order of 3
    \item $\langle (2 \; 3 \; 4) \rangle = \langle (2 \; 4 \; 3) \rangle = \{(1), (2 \; 3 \; 4), (2 \; 4 \; 3)\}$, order of 3
    \item $\{(1), (1 \; 2)(3 \; 4), (1 \; 3) (2 \; 4), (1 \; 4)(2 \; 3)\}$, order of 4
\end{itemize}

\qs{7}{Find all possible orders of elements in $S_7$ and $A_7$.}
\sol We start by proving that the order of permutations is defined by the least common multiple of lengths of disjoint cycles. For $S_7$, this corresponds to partitions of 7. This lemma is not hard to prove by splitting any permutation into disjoint cycles and then raising them to the power of the least common multiple of the lengths of the cycles.

Listing them all out shows that elements in $S_7$ can have order of $1, 2, 3, 4, 5, 6, 7, 10, 12$. $A_7$ has the same list except any elements with odd parity are disregarded, so the order of the elements would be $1, 2, 3, 4, 5, 6, 7$. 

\qs{23}{If $\sigma$ is a cycle of odd length, prove that $\sigma^2$ is also a cycle.}
\sol Let's say that $\sigma$ has length $k$ and that it can be written in cycle notation as $(a_1, a_2, \ldots, a_k)$. Then $\sigma^2$ is the permutation $(a_1, a_3, \ldots, a_k, a_2, a_4, \ldots a_{k-1})$. The term $a_{k-1}$ would be sent to $a_1$ which would complete the cycle if $k$ is odd since $k-1$ would then be even $\qed$. 

\qs{25}{Prove that in $A_n$ for $n\geq 3$, any permutation is a product of cycles of length 3. }
\sol First we need to prove that $|A_n|$ is even. This is true because $|A_n| = n!/2$ and that value will be even for $n \geq 3$. 

Now, consider $\sigma=\underbrace{(a_1 \; a_2)\dots(a_{4n-1} \; a_{4n})}_{2n}$. 

If we consider any pair of transpositions, $(a_\alpha \; a_\beta)(a_\gamma \; a_\delta)$, we can do the following:

$$(a_\alpha \; a_\beta)(a_\beta \; a_\gamma)(a_\beta \; a_\gamma)(a_\gamma \; a_\delta)$$
$$(a_\alpha \; a_\beta)[(a_\beta \; a_\gamma)(a_\beta \; a_\gamma)](a_\gamma \; a_\delta)$$
$$(a_\alpha \; a_\beta \; a_\gamma)(a_\beta \; a_\gamma \; a_\delta)$$

Therefore, in $A_n$, any permutation is a product of cycles of length $3$. $\qed$

\qs{29}{Recall that the the \textbf{\emph{center}} of a group $G$ is
\begin{equation*}
    Z(G) = \{ g \in G : gx = xg \text{ for all } x \in G \}\text{.}
    \end{equation*}
    Find the center of $D_8$. What about the center of $D_{10}$? What is the center of $D_n$?
}
\sol It is important to start by noting that rotations in $D_n$ commute. If we define $r$ as an action of $D_n$ that rotations the $n$-gon by $2\pi/n$ radians, then the set of rotations in $D_n$ are $r^i$ for $i \in [1, n]$ where $r^n = e$. If we also define a flip $f$ that flips the $n$-gon over the vertical axis. Then the set of flips in $D_n$ is $fr^j$ for $j \in [0, n-1]$.

If we take any $i,j$, then we can see that $r^i(sr^j) = r^i(sr^jss) = r^i(srs)^js = r^i(r^{-1})^js = r^ir^{-j}s = r^{i-j}s$. If we reverse the order, then we can see that $(sr^j)r^i = sr^{j+i}ss = r^{-i-j}s$. Therefore, we can conclude that $r^i(sr^j) = (sr^j)r^i$  only when $i - j \equiv -i-j \pmod{n} \Longleftrightarrow 2i \equiv 0 \pmod{n}$. As such, the center of $D_n$, $Z(D_n) = \{e, r^{n/2}\}$. This means that $Z(D_8) = \{e, r^4\}$ and $Z(D_{10}) = \{e, r^5\}$.



\end{document}